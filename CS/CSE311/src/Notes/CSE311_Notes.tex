% Preamble
\documentclass{article}

% Packages
\usepackage{../../../mypackages}


% Macros
\usepackage{../../../mymacros}


% File info
\title{CSE311 Notes}
\author{Eric Xia}
\date{Last Updated 29 March 2021}


% Document
\begin{document}

    \maketitle
    \tableofcontents
    \pagebreak



    \textbf{Proposition:} a "well-formed" statement that is either true or false \\
    $\bullet$ Propositional variables: $q,r,s$ \\

    Example of a compound proposition: "Garfield has black stripes if he is an orange cat and likes lasagna, and he is an orange cat or does not like lasagna" \\

    We can break this proposition into the simplest (\textbf{atomic propositions}): \\
    $q$: "Garfield has black stripes" \\
    $r$: "Garfield is an orange cat" \\
    $s$: "Garfield likes lasagna" \\

    We rewrite the original proposition like so: \\
    (q if (r and s)) and (r or (not s))

    \begin{center}
        \textbf{Logical connectives}: \\
        \begin{tabular}{|c|c|}
            \hline
            Negation (not)  & $\neg q$ \\
            \hline
            Conjunction (and) &  $q\land r$ \\
            \hline
            Disjunction (or)   & $q \lor r$  \\
            \hline
            Exclusive Or       & $q \oplus r$ \\
            \hline
            Implication        & $q \implies r$ \\
            \hline
            Biconditional      & $q \iff r$ \\
            \hline
        \end{tabular}
    \end{center}

    We can then replace the compound proposition about Garfield with logical symbols like so:
    (q if (r $\and$ s)) $\land$ (r$\lor\neg $s)

    Implication truth table:

    \begin{center}
        \begin{tabular}{|c|c|c|}
            \hline
            $q$ &   $r$ & $q\implies r$ \\
            \hline
            T   &   T   & T \\
            \hline
            T   &   F   & F \\
            \hline
            F   &   T   & T \\
            \hline
            F   &   F   & T \\
            \hline
        \end{tabular}
    \end{center}

    \textit{\blue{Example:}} $2+2=4\implies \text{the earth is a planet}$, this is a true implication because both $2+2=4$ and "the earth is a planet" are true \\
    \textit{\blue{Example 2:}} $2+2 = 5 \implies 26$ is a prime number, this is a true implication because both $2+2=5$ and "26 is a prime number" are false \\

    Analyzing the Garfield Sentence with a truth table:

    \begin{figure*}[hbt!]
        \centering
        \includegraphics[scale = 0.75]{Assets/Garfield}
    \end{figure*}

    Implication: $q\implies r$ \\
    Converse: $r\implies q$ \\
    Contrapositive: $\neg r \implies \neg q$ \\
    Inverse: $\neg q \implies \neg r$ \\

    In the truth table below, notice that an implication and its contrapositive have the same truth value, and that the converse and inverse have the same truth value

    \begin{figure*}[hbt!]
        \centering
        \includegraphics[scale = 0.75]{Assets/Contrapositive}
    \end{figure*}

    A compound proposition is a \\
    \textbf{Tautology} if it is always true \\
    $\bullet$ $q\lor \neg q$ is a tautology and is called "\textit{the law of the excluded middle}". If $q$ is true then $q\lor \neg q$ is true. If $q$ is false then $q\lor \neg q$ is true.
    \textbf{Contradiction} if it is always false \\
    $q \oplus q$ is a contradiction because it's always false no matter what truth value $q$ takes on
    \textbf{Contingency} if it can be either true or false \\
    $(q\implies r) \land q$ is a contingency. When $q=T,r=T,(T\implies T)\land T$ is true. When $q=T,r=F,(T\implies F)\land T$ is false. \\

    Equivalence versus equal: \\

    A = B means A and B are identical "strings": \\
    $\bullet$ $q\land r = q\land r$, These are equal because they are character-for-character identical
    $\bullet$ $q \land r \not = r \land q$, These are NOT equal because they are different sequences of characters. They "mean" the same thing though \\

    $A \equiv B$ means A and B have identical truth values: \\
    $\bullet$ $q\lor r \equiv q\lor r$ \\
    $\bullet$ $q\lor r \equiv r\lor q$ \\
    $\bullet$ $q\lor r \not \equiv r\land q$ \\

    Biconditional versus equivalence:

    $A \equiv B$ is an \textbf{assertion over all possible truth values} that A and B always have the same truth values \\
    $A \iff B$ is a \textbf{proposition} that may be true or false depending on the truth values of the variables in A and B \\
    $A\equiv B$ and $(A\iff B) \equiv T$ have the same meaning

    \begin{axiom}{De Morgan's Laws}
        \begin{align*}
            \neg (q\land r) &\equiv \neg q \land \neg r \\
            \neg(q\lor r)   &\equiv \neg q \lor \neg r
        \end{align*}
    \end{axiom}

    Some equivalences related to implication:

    \begin{align*}
        q \implies r    &= \neg q \lor r \\
        q \implies r    &= \neg r \implies \neg q \\
        q \iff r        &= (q\implies r) \land (r\implies q) \\
        q \iff r        &= \neg q \iff \neg r
    \end{align*}

    \begin{figure*}[hbt!]
        \centering
        \caption*{Properties of logical connectives:}
        \includegraphics[scale = 0.75]{Assets/Connectives}
    \end{figure*}

    \textbf{Boolean algebra:} another notation for propositional logic consisting of: \\
    $\bullet$ a set of elements $B = \{0,1\}$ \\
    $\bullet$ binary operations $\{+,\cdot\}$ (OR, AND) \\
    $\bullet$ unary operation $\{'\}$ (NOT) \\

    Example: $q\lor (r\land \neg s)$ written as $q+(r\cdot s')$ \\
    $\bullet$ notation used in circuit design \\





\end{document}