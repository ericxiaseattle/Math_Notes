\section{Introduction}

    \textbf{Rational:} maximally achieving pre-defined goals; maximizing your expected utility \\
    \textbf{Rationality}: only concerns what decisions are made and not the thought process behind them \\

    Artificial intelligence is often observed in two dimensions, human vs. rational and thought (internal reasoning) vs. behavior (external characterization) \\

    \textbf{Turing Test:} designed as a thought experiment to sidestep the philosophical vagueness of the question "can a machine think?", would need the following capabilities: \\
    $\bullet$ \textbf{Natural language processing} to communicate successfully in a human language \\
    $\bullet$ \textbf{Knowledge representation} to store what it knows or hears \\
    $\bullet$ \textbf{Automated reasoning} to answer questions and to draw new conclusions \\
    $\bullet$ \textbf{Machine learning} to adapt to new circumstances and to detect and extrapolate patterns \\

    \textbf{Total Turing Test:} requires interaction with objects and people in the real world, proposed because Turing viewed the \textit{physical} simulation of a person as unnecessary to demonstrate intelligence,
    the robot would need: \\
    $\bullet$ \textbf{Computer vision:} and speech recognition to perceive the world \\
    $\bullet$ \textbf{Robotics} to manipulate objects and move about \\
    \textbf{Agent:} an entity that operates autonomously, perceives its environment, persists over a prolonged time period, adapts to change, and creates and pursues goals \\
    \textbf{Rational agent:} selects actions that maximize its (expected) utility \\
    \textbf{Limited rationality:} acting appropriately when there is not enough time to do all the computations one might like \\
    \textbf{Value alignment problem:} the values or objectives put into the machine must be aligned with those of the human \\
    \textbf{Incompleteness theorem:} in any formal theory as strong as Peano arithmetic (the elementary theory of natural numbers), there are necessarily true statements that have no proof within the theory \\
    \textbf{Computability:} capable of being computed by an effective procedure \\
    \textbf{Tractability:} a problem is intractable if the time required to solve instances of the problem grows exponentially with the size of the instances \\
    \textbf{NP-completeness:} provides a basis for analyzing the tractability of problems: any problem class to which the class of NP-complete problems can be reduced is likely to be intractable \\
    \textbf{Decision theory:} combines probability theory with utility theory and provides a formal and complete framework for individual decisions made under uncertainty \\
    \textbf{Satisficing:} making decisions that are "good enough" rather than laboriously calculating an exact optimal decision \\

