\section{Integration}

    \subsection{Riemann Sums and Definite Integrals}
        \textbf{Riemann Sums} are approximations of a region's area, obtained by summing the
        areas of numerous simplified slices of said region. The area approximation gets better
        if more slices are used to simplify the area. As you can see from the figures below,
        the figure with more and smaller rectangles takes up more of the area we are trying to
        approximate.

        \begin{figure} [hbt!]
            \centering
            \begin{subfigure}[b]{.45\textwidth}
                \includegraphics[scale=0.8]{Resources/Unit4Integration/Riemann1}
            \end{subfigure}
            \begin{subfigure}[b]{.45\textwidth}
                \includegraphics[scale=0.8]{Resources/Unit4Integration/Riemann2}
            \end{subfigure}
        \end{figure}

        \noindent \textbf{Left Riemann Sums} have rectangles that touch the curve with their
        top-left corners, while \textbf{Right Riemann Sums} have rectangles that touch the curve
        with their top-right corners. \textbf{Midpoint Riemann Sums} have rectangles that touch
        the curve with the middle of their top edges. If the graph is increasing on the interval,
        then the left sum is an underestimate and the right sum is an overestimate. If the graph
        is decreasing on the interval, then the left sum is an overestimate and the left sum is
        an underestimate. Conventionally, we use $i=0$ for left and midpoint sums and $i=1$ for
        right sums. \\

        \noindent Subdivisions of Riemann Sums can be \textbf{uniform}, meaning they are of
        equal width, or \textbf{nonuniform}, meaning they are not of equal length. \\

        \noindent \textbf{Trapezoidal Sums} have subdivisions which touch the curve with both of
        its top vertices. The figures below display left, right, midpoint, and trapezoidal sums.

        \begin{figure}[hbt!]
            \centering
            \begin{subfigure}[b]{.45\textwidth}
                \includegraphics[scale=0.8]{Resources/Unit4Integration/Riemann_Left}
            \end{subfigure}
            \begin{subfigure}[b]{.45\textwidth}
                \includegraphics[scale=0.8]{Resources/Unit4Integration/Riemann_Right}
            \end{subfigure}
        \end{figure}

        \begin{figure}[hbt!]
            \centering
            \begin{subfigure}[b]{.45\textwidth}
                \includegraphics[scale=0.8]{Resources/Unit4Integration/Riemann_Mid}
            \end{subfigure}
            \begin{subfigure}[b]{.45\textwidth}
                \includegraphics[scale=0.8]{Resources/Unit4Integration/Riemann_Trapezoid}
            \end{subfigure}
        \end{figure}

        \noindent If $f(x)$ is defined on the closed interval $[a,b]$ and $c_k$ is any point in
        $[x_{k-1}, x_k]$, then a Riemann Sum is defined as

        \begin{equation*}
            \sum_{i=1}^n f(x_i)\Delta x
        \end{equation*}

        \noindent The Riemman Sum of a function is related to the \textbf{definite integral} as
        follows:

        \begin{equation*}
            \lim_{n\rightarrow\infty}\sum^n_{i=1}f(x_i)\Delta x_i = \int^b_a f(x)dx
        \end{equation*}

        \noindent \color{blue} \textit{Example 1: Approximate the area between $f(x)=|x+3|$ and
        the $x$-axis on the interval $[-5,5]$ using a left Riemann sum with 10 equal subdivisions.}
        \color{black} \\

        \noindent The width of each rectangle, $\Delta x$, is given by

        \begin{equation*}
            \Delta x = \frac{5-(-5)}{10} = 1
        \end{equation*}

        \noindent Since we are using left sums, let $i=0$. Since we have 10 equal subdivisions, the other
        indice of our summation will be 9 such that our area is given by $\sum^9_{i=0}$. We now
        want to find $f(x_i)$. Each time $i$ increases by 1, the value of $x_i$ also increases
        by 1. Its initial value is the left endpoint of the interval, which is -5.

        \begin{equation*}
            x_i = -5 + 1i = -5 + i
        \end{equation*}

        \noindent To confirm that $x_i$ gives us the correct first left endpoint, we substitute 0 for $i$:

        \begin{equation*}
            x_0=-5+0=-5
        \end{equation*}

        \noindent Thus,

        \begin{align*}
            f(x)    &= |x+3| \\
            f(x_i)  &= |(-5+i) +3| \\
            A       &= \sum_{i=0}^9 |(-5+i) + 3| \cdot 1 \\
            A       &= \sum_{i=0}^9 |i-2|
        \end{align*}

        \noindent \color{blue} \textit{Example 2: Approximate the area between $f(x)=\cos{(x)}$
        and the $x$-axis on the interval $\left[-\frac{3}{10}\pi, \frac{1}{2}\pi\right]$ using
        a right Riemann sum with 8 equal subdivisions.} \color{black}

        \begin{align*}
            \Delta x    &= \frac{\frac{\pi}{2}-\left(-\frac{3\pi}{10}\right)}{8} = \frac{\pi}{10} \\
            x_8         &= \frac{\pi}{2} \\
            x_i - x_8   &= \frac{\pi}{10} (i-8) \\
            x_i         &= \frac{\pi i}{10} - \frac{3\pi}{10} \\
            f(x)        &= \cos{(x)} \\
            f(x_i)      &= \cos{\left(\frac{\pi i}{10}-\frac{3\pi}{10}\right)} \\
            A           &= \sum^8_{i=1}\cos{\left(\frac{\pi i}{10}-\frac{3\pi}{10}\right)}
            \cdot\frac{\pi}{10}
        \end{align*}

        \noindent \color{blue} \textit{Example 3: Write the definite integral $\int^3_0 e^x dx$
        as a limit of a left Riemann sum.} \color{black} \\

        \begin{align*}
            \Delta x                                    &= \frac{b-a}{n} = \frac{3-0}{n} = \frac{3}{n} \\
            \sum^{n-1}_{i=0}f(a+i\Delta x)\cdot\Delta x &= \sum^{n-1}_{i=0} f\left(0+\frac{3i}{n}\right)
            \cdot \frac{3}{n} \\
            &= \sum^{n-1}_{i=0} e^{\frac{3i}{n}}\cdot\frac{3}{n}
        \end{align*}


    \subsection{Fundamental Theorem of Calculus}
        If $f(x)$ is continous over an interval $[a,b]$ and the function $F(x)$ is defined by

        \begin{equation*}
            F(x) = \int^x_a f(t)dt,
        \end{equation*}

        \noindent then $F'(x)=f(x)$ over $[a,b]$. Additionally, \\

        \noindent If $f(x)$ is continuous over the interval $[a,b]$ and $F(x)$ is any
        antiderivative of $f(x)$ then

        \begin{equation*}
            \int^b_a f(x)dx = F(b) - F(a)
        \end{equation*}

        \pagebreak
        \noindent \color{blue} \textit{Example 1: Let $F(x)=\int_1^{\sqrt{x}}\sin{t} dt$.
        Find $F'(x)$ using the FTC.} \color{black} \\

        \noindent Let $u(x)=\sqrt{x}$. Then $F(x)=\int_1^{u(x)}\sin{t}dt$. Thus, by the FTC,

        \begin{align*}
            F'(x) &= \sin{(u(x))}\frac{du}{dx} \\
            &= \sin{(u(x))}\cdot\left(\frac{1}{2}x^{-\frac{1}{2}}\right) \\
            &= \frac{\sin{\sqrt{x}}}{2\sqrt{x}}
        \end{align*}

        \noindent \color{blue} \textit{Example 2: Let $F(x)=\int^{2x}_x t^3 dt$. Find $F'(x).$}
        \color{black} \\

        \begin{align*}
            F(x)    &= \int_x^{2x} t^3 dt \\
            &= \int_x^0 t^3 dt + \int_0^{2x} t^3 dt \\
            &= -\int_0^x t^3 dt + \int_0^{2x} t^3 dt \\
            F'(x)   &= \frac{d}{dx}\left[-\int_0^x t^3 dt\right] +
            \frac{d}{dx}\left[\int_0^2x t^3 dt\right] \\
            &= -x^3 + 16x^3 \\
            &= 15x^3
        \end{align*}



    \subsection{Indefinite Integrals}
        Given a function, $f(x)$, an \textbf{antiderivative} of $f(x)$ is any function $F(x)$ such that

        \begin{equation*}
            F'(x) = f(x)
        \end{equation*}

        \noindent If $F(x)$ is any antiderivative of $f(x)$ then the most general antiderivative
        of $f(x)$ is called an \textbf{indefinite integral} and denoted

        \begin{equation*}
            \int f(x)dx = F(x) + k,\text{    where $k$ is any constant}
        \end{equation*}

        \noindent Here, $\int$ is the \textbf{integral symbol}, $f(x)$ is the \textbf{integrand},
        $x$ is the integration variable, and $k$ is the constant of integration. \\



    \subsection{Basic Integral Rules}
        \begin{center}
            \begin{tabular}{|c|c|}
                \hline
                $\int^a_a f(x) dx = 0$ & Integral with Equal Bounds \\
                \hline
                $\int^b_a f(x)dx = -\int^a_b f(x)dx$ & Opposite of Integral \\
                \hline
                $\int k dx = kx + C$ & Constant \\
                \hline
                $\int kf(x)dx = k\int f(x)dx$ & Constant Multiple \\
                \hline
                $\int^b_a[f(x)\pm g(x)]dx = \int^b_a f(x)dx\pm\int^b_a g(x)dx$ & Sum/Difference \\
                \hline
                $\int x^n dx = \frac{x^{n+1}}{n+1}+C, n\not =-1$ & Reverse Power Rule \\
                \hline
            \end{tabular}
        \end{center} \\

        \begin{center}
            \begin{tabular}{|c|}
                \hline
                If $f(x)\geq 0$ on $[a,b]$, then $\int^b_a f(x)dx\geq 0$  \\
                \hline
                If $f(x)\leq 0$ on $[a,b]$, then $\int^b_a f(x)dx\leq 0$  \\
                \hline
                If $f(x)\geq g(x)$ on $[a,b]$, then $\int^b_a f(x)dx\pm\int^b_a g(x)dx$  \\
                \hline
            \end{tabular}
        \end{center}


    \subsection{Integrals of Exponential and Logarithmic Functions}
        \begin{center}
            \begin{tabular} {|c|c|}
                \hline
                $\int \frac{1}{x} dx = \ln{|x|}+C$ & Reciprocal \\
                \hline
                $\int \frac{1}{ax+b}dx = \frac{1}{a}\ln{|ax+b|}+C$ & \\
                \hline
                $\int \frac{u'(x)}{u(x)}dx = \ln{|u(x)|+C}$ & \\
                \hline
                $\int e^x dx = e^x+C$ & Exponential \\
                \hline
                $\int a^x dx = \frac{a^x}{\ln{a}}+C$ & \\
                \hline
                $\int \ln{x} dx = x\ln{x}-x + C$ & \\
                \hline
            \end{tabular}
        \end{center}

    \pagebreak
    \subsection{Integrals of the Trig Functions}
        \begin{center}
            \begin{tabular} {|c|c|}
                \hline
                $\int \sin{x} dx = -\cos{x}+C$ & Sine \\
                \hline
                $\int \cos{x} dx = \sin{x}+C$ & Cosine \\
                \hline
                $\int \tan{x} dx = -\ln{|\cos{x}|}+C=\ln{|\sec{x}|}+C$ & Tangent \\
                \hline
                $\int \cot{x} dx = -\ln{|\sin{x}|}+C = \ln{|\cos{x}|}+C$ & Cotangent \\
                \hline
                $\int \sec{x} dx = -\ln{|\sec{x}+\tan{x}|}+C=$ & Secant \\
                \hline
                $\int \csc{x} dx = -\ln{|\csc{x}+\cot{x}|}+C$ & Cosecant \\
                \hline
            \end{tabular}
        \end{center}

        \begin{center}
            \begin{tabular}{|c|c|}
                \hline
                $\int \sec^2{x}dx = \tan{x}+C$ & \\
                \hline
                $\int \csc^2{x}dx = -\cot{x}+C$ & \\
                \hline
                $\int \sec{x}\tan{x}dx = \sec{x}+C$ & \\
                \hline
                $\int \csc{x}\cot{x}dx = -\csc{x}+C$ & \\
                \hline
                %%%
                $\int \frac{1}{1-x^2}, x\not =\pm 1 = \arcsin{x}$ & \\
                \hline
                $\int -\frac{1}{1-x^2}, x\not =\pm 1 = \arccos{x}$ & \\
                \hline
                $\int \frac{1}{1+x^2}=\arctan{x}$ & \\
                \hline
                $\int -\frac{1}{1+x^2}=\arccot{x}$ & \\
                \hline
                $\int \frac{1}{|x|\sqrt{x^2-1}}, x\not = \pm 1,0=\arcsec{x}$ & \\
                \hline
                $\int -\frac{1}{|x|\sqrt{x^2-1}}, x\not = \pm 1,0=\arccsc{x}$ & \\
                \hline
            \end{tabular}
        \end{center}


    \subsection{Integration Using Substitution}
        To use integration by substitution, we must first be able to write our integral in the
        form $\int f(g(x))g'(x)dx=\int f(u)du$, where $u=g(x)$. \\

        \noindent \color{blue} \texit{Example 1 Compute $\int \cos{(x^2)}2xdx$} \color{black}

        \begin{align*}
            u                   &= x^2 \\
            du                  &= 2xdx \\
            \int \cos{(u)}du    &= \sin{(u)}+C \\
            &= \sin{(x^2)} + C
        \end{align*}

        \noindent \color{blue} \textit{Example 2: Compute $\int x\sqrt{3x^2-1}dx$} \color{black}

        \begin{align*}
            u                                   &= 3x^2-1 \\
            du                                  &= 6x \\
            \int \frac{1}{6}(6x)\sqrt{3x^2-1}dx &= \frac{1}{6}\int\sqrt{u}du \\
            &= \frac{1}{9}u^{\frac{3}{2}}+C \\
            &= \frac{1}{9}(3x^2-1)^{\frac{3}{2}} + C
        \end{align*}

        \noindent \color{blue} \texit{Example 3: Compute $\int \sin^2{\theta}\cos^2{\theta}d\theta$} \color{black}

        \begin{align*}
            \sin^2{\theta}\cos^2{\theta}                     &= (\sin{\theta}\cos{\theta})^2 \\
            &= \left(\frac{1}{2}\sin{(2\theta)}\right)^2 \\
            &= \frac{1}{4}\sin^2{(2\theta)} \\
            \int\sin^2{\theta}\cos^2{\theta}d\theta          &= \frac{1}{4}\int\sin^2{2\theta}d\theta \\
            \because \cos{2x}                                &= 1-2\sin^2{x} \\
            \therefore \sin^2{x}                             &=\frac{1}{2}(1-\cos{2x}) \\
            \implies \int\sin^2{\theta}\cos^2{\theta}d\theta &= \frac{1}{8}\int(1-\cos{4\theta})d\theta \\
            &= \frac{1}{8}\left(\theta-\frac{1}{4}\sin{4\theta}\right)+C \\
            &= \frac{\theta}{8}-\frac{\sin{4\theta}}{32}+C
        \end{align*}



    \pagebreak
    \subsection{Integrating Using Long Division and Completing the Square}
        When we have a rational function of the form $\frac{P(x)}{Q(x)}$, we can use long
        division and completing the squre to simplify the integral prior to integration. \\

        \noindent \color{blue} \textit{Example 1: Compute $\int\frac{x^3+2x^2+9x-17}{x+4}dx$} \color{black} \\

        \begin{align*}
            \frac{x^3+2x^2+9x-17}{x+4}          &= x^2-2x+17-\frac{85}{x+4} \\
            \int\frac{x^3+2x^2+9x-17}{x+4}dx    &= \int x^2-2x+17-\frac{85}{x+4}dx \\
            &= \frac{x^3}{3}-x^2+17x-85\ln{|x+4|}+C
        \end{align*}

        \noindent \color{blue} \textit{Example 2: Compute $\int\frac{dx}{\sqrt{1-2x-x^2}}$} \color{black}

        \begin{align*}
            1-2x-x^2                        &= 1-(x^2+2x) \\
            &= 2-(x^2+2x+1) \\
            &= 2-(x+1)^2 \\
            &= (\sqrt{2})^2-(x+1)^2 \\
            u                               &= x+1 \\
            du                              &= dx \\
            \int\frac{dx}{\sqrt{1-2x-x^2}}  &= \int\frac{dx}{\sqrt{(\sqrt{2})^2-(x+1)^2}} \\
            &= \int\frac{du}{\sqrt{(\sqrt{2})^2-u^2}} \\
            &= \arcsin{\left(\frac{u}{\sqrt{2}}\right)}+C \\
            &= \arcsin{\left(\frac{x+1}{\sqrt{2}}\right)}+C
        \end{align*}



    \subsection{Integration by Parts}
        Recall the differentiation product rule, $(uv)'=uv'+u'v$. Rearranging the equation, we
        get $uv'=(uv)'-u'v$. Hence,

        \begin{align*}
            \int uv'dx  &= \int ((uv)'-u'v)dx \\
            &= uv - \int u'vdx \\
            du          &= u'dx \\
            dv          &= v'dx
        \end{align*}

        \noindent \color{purple} \textbf{Integration by Parts:} \color{black} \\

        \begin{equation*}
            \int udv = uv - \int v du
        \end{equation*}

        \noindent \textbf{Guidelines for Choosing $u$ and $dv$:} \\
        \noindent "LIATE" (Choose $u$ in the following order): \\
        \noindent L: Logarithmic Functions \\
        \noindent I: Inverse Trig Functions \\
        \noindent A: Algebraic Functions \\
        \noindent T: Trig Functions \\
        \noindent E: Exponential Functions \\

        \noindent We want $u$ to be an expression whose derivative $du$ is a simpler function
        than $u$ itself. We want $dv$ to be the most complicated part of the integrand that
        can be easily integrated.

        \pagebreak
        \noindent \color{blue} \textit{Example 1: Compute $\int x^3\ln{x}dx$} \color{black}

        \begin{align}
            u                   &= \ln{x} \\
            dv                  &= x^3 dx \\
            du                  &= \frac{dx}{x} \\
            v                   &= \int x^3 dx = \frac{x^4}{4} \\
            \int x^3\ln{x}dx    &= uv - \int vdu \\
            &= (\ln{x})\frac{x^4}{4}-\int\frac{x^4}{4}\frac{1}{x}dx \\
            &= \frac{x^4\ln{x}}{4}-\frac{1}{4}\int x^3 dx \\
            &= \frac{x^4\ln{x}}{4}-\frac{x^4}{16}+C
        \end{align*}

        \noindent \color{blue} \textit{Example 2: Compute $\int x^3\sqrt{4-x^2}dx$} \color{black}

        \begin{align*}
            dv                      &= x\sqrt{4-x^2}dx \\
            u                       &= x^2 \\
            du                      &= 2xdx \\
            v                       &= \int x\sqrt{4-x^2}dx = -\frac{1}{3}(4-x^2)^{\frac{3}{2}} \\
            \int x^3\sqrt{4-x^2}dx  &= uv - \int vdu \\
            &= x^2\left(-\frac{1}{3}(4-x^2)^\frac{3}{2}\right)
            - \int -\frac{1}{3}(4-x^2)^\frac{3}{2}(2x)dx \\
            &= -\frac{x^2}{3}(4-x^2)^\frac{3}{2}
            -\frac{2}{15}(4-x^2)^\frac{5}{2}+C
        \end{align*}



    \subsection{Tabular Integration}
        \textbf{Tabular Integration} is a shortcut for performing repeated integration by parts.
        We want to create a table like so, until one of the derivatives/integrals reaches 0:

        \begin{center}
            \begin{tabular}{|c|c|c|}
                \hline
                & u         & dv \\
                \hline
                1   & $f(x)$    & $g(x)$ \\
                \hline
                2   & $f'(x)$   & $\int g(x)$ \\
                \hline
                3   & $f"(x)$   & $\int \int g(x)$ \\
                \hline
                n   & $f^{(n)}(x)$  & $\int\dots\int g(x), \text{where there is one $\int$ for each $n$}$ \\
                \hline
            \end{tabular}
        \end{center}

        \noindent Then, we multiply $f(x)$ by $\int g(x)$, $f'(x)$ by $\int \int g(x)$, and so
        on until we reach the last antiderivative. We then alternatively add and subtract these
        products, starting with addition. \\

        \noindent \color{blue} \textit{Example 1: Compute $\int (x^3+2x-1)\cos{(4x)}$} \color{black} \\

        \noindent Let $f(x)=x^3+2x-1$ and $g(x)=\cos{(4x)}$. Then we can make a table like so:

        \begin{figure}[hbt!]
            \centering
            \includegraphics[scale=0.75]{Resources/Unit4Integration/Tabular}
        \end{figure}

        \pagebreak
        \noindent Thus, the antiderivative becomes \\

        \begin{align*}
            &\frac{1}{4}(x^3+2x-1)\sin{(4x)}+\frac{1}{16}(3x^2+2)\cos{(4x)}
            -\frac{3x}{32}\sin{(4x)}-\frac{3}{128}\cos{(4x)} \\
            &=\sin{(4x)}\left(\frac{x^3}{4}+\frac{13x}{32}-\frac{1}{4}\right)
            +\cos{(4x)}\left(\frac{3x^2}{16}+\frac{13}{128}\right) + C
        \end{align*}



    \subsection{Partial Fraction Decomposition}
        \color{blue} \textit{Example 1: $\int\frac{3x+11}{x^2-x-6}dx$} \color{black}

        \begin{align*}
            \frac{3x+11}{(x-3)(x+2)}    &= \frac{A}{x-3}+\frac{B}{x+2} \\
            &= \frac{A(x+2)+B(x-3)}{(x-3)(x+2)} \\
            3x+11                       &= A(x+2)+B(x-3) \\
            x                           &= -2\implies B=-1 \\
            x                           &= 3\implies A=4 \\
            \int\frac{3x+11}{x^2-x-6}dx &= \int \frac{4}{x-3}dx-\frac{1}{x+2}dx \\
            &= 4\ln{|x-3|}-\ln{|x+2|}+C
        \end{align*}

        \noindent \color{blue} \textit{Example 2: Compute $\int\frac{x^3+10x^2+3x+36}{(x-1)(x^2+4)}dx$} \color{black}

        \begin{align*}
            \frac{x^3+10x^2+3x+36}{(x-1)(x^2+4)}    &= \frac{A}{x-1} + \frac{Bx+C}{x^2+4}
            + \frac{Dx+E}{(x^2+4)^2} \\
            x^3+10x^2+3x+36                         &= A(x^2+4)^2+(Bx+C)(x-1)(x^2+4)+(Dx+E)(x-1) \\
            &= x^4(A-B)+x^3(C-B)+x^2(8A+4B-C+D)
            + x(-4B+4C-D+E)+16A-4C-E
        \end{align*}

        \begin{align*}
            x^4: & A+B=0 \\
            x^3: & C-B=1 \\
            x^2: & 8A+4B-C+D=10 \\
            x^1: & -4B+4C-D+E=3 \\
            x^0: & 16A-4C-E=36
        \end{align*}

        \noindent $\implies A=2,B=-2,C=-1,D=1,E=0$

        \begin{align*}
            \int\frac{x^3+10x^2+3x+36}{(x-1)(x^2+4)}dx  &= \int\frac{2}{x-1}-\frac{2x}{x^2+4}
            -\frac{1}{x^2+4}+\frac{x}{(x^2+4)^2}dx \\
            &= 2\ln{|x-1|}-\ln{|x^2+4|}-\frac{1}{2}
            \arctan{\left(\frac{x}{2}\right)}
            -\frac{1}{2(x^2+4)}+C
        \end{align*}


    \subsection{Improper Integrals}
        \color{purple} \textbf{The 3 Common Methods to Compute Improper Integrals} \color{black} \\

        \noindent 1. If $\int^t_a f(x)dx$ exists for every $t>a$ then,
        \begin{equation*}
            \int^\infty_a f(x)dx = \lim_{t\rightarrow\infty}\int^t_a f(x)dx
        \end{equation*}
        \noindent as long as the limit exists and is finite. \\

        \noindent 2. If $\int^b_t f(x)dx$ exists for every $t<b$ then,
        \begin{equation*}
            \int^b_{-\infty}f(x)dx = \lim_{t\rightarrow-\infty}\int^b_t f(x)dx
        \end{equation*}
        \noindent as long as the limit exists and is finite. \\

        \noindent 3. if $\int^c_{-\infty} f(x)dx$ and $\int^\infty_c f(x)dx$ are both convergent then,
        \begin{equation*}
            \int^\infty_{-\infty}f(x)dx = \int^c_{-\infty}f(x)dx+\int^\infty_c f(x)dx
        \end{equation*}
        \noindent where $c$ is any constant. If either of the two integrals are divergent then so
        is this integral.

        \noindent \color{blue} \textit{Example 1: Determine if the integral $\int^0_{-\infty}
        \frac{1}{\sqrt{3-x}}dx$ is convergent or divergent and find its value if it is
        convergent.} \color{black}

        \begin{align*}
            \int^0_{-\infty}\frac{dx}{\sqrt{3-x}}   &= \lim_{t\rightarrow-\infty}\int^0_t
            \frac{dx}{\sqrt{3-x}} \\
            &= \lim_{t\rightarrow-\infty}-2\sqrt{3-x}
            \big|^0_t \\
            &= \lim_{t\rightarrow-\infty}
            (-2\sqrt{3}+2\sqrt{3-t}) \\
            &= -2\sqrt{3}+\infty \\
            &= \infty
        \end{align*}

        \noindent Since the limit is infinite, this integral is divergent. \\

        \noindent \color{blue} \textit{Example 2: Determine if the integral $\int^\infty_{-\infty}
        xe^{-x^2}dx$ is convergent or divergent and find its value if it is convergent.} \color{black}

        \begin{align*}
            \int^\infty_{-\infty}xe^{-x^2}dx    &= \lim^0_{-\infty}xe^{-x^2}dx+\int^\infty_0 xe^{-x^2}dx \\
            &= \lim_{t\rightarrow-\infty}\int^0_t xe^{-x^2}dx
            + \lim_{t\rightarrow\infty}\int^t_0 xe^{-x^2}dx \\
            &= -\frac{1}{2} + \frac{1}{2} \\
            &= 0
        \end{align*}



    \subsection{Integration Strategy}
        1. See if you can simplify the integrand \\
        \noindent   2. See if a simple substitution will work \\
        \noindent   3. Identify the type of integrand: \\
        \textbullet    If the integrand is a rational expression, partial fractions may work \\
        \textbullet     If the integrand is a polynomial times a trig, exponential, or logarithmic
        function, then try integration by parts \\
        \textbullet     If the integrand is a product of trig functions then try rewriting the
        integrand in terms of other trig function or try integration by parts
        or substitution \\
        \textbullet     Look for trig substitutions if the integrand contains some form of
        $\sqrt{b^2x^2\pm a^2}$ or $\sqrt{a^2-b^2x^2}$ \\
        \textbullet     If the integrand contains a quadratic then try completing the square \\
        \noindent   4. Remember that you can use multiple techniques on the same integral. \\
        \noindent   5. If it doesn't work then try another method.