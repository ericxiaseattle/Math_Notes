\section{Infinite Sequences and Series}

    \subsection{Sequences and Series Basics}
        \color{purple} \textbf{Sequences} \color{black} are a list of terms in a definite order,
        whereas \color{purple} \textbf{series} \color{black} are the sum of terms of an infinite
        sequence. \\

        \noindent Given any sequence $\{a_n\}$, the sequence is: \\
        \noindent \color{purple} \textbf{increasing} \color{black} if $a_n<a_{n+1}$ for every $n$ \\
        \noindent \color{purple} \textbf{decreasing} \color{black} if $a_n>a_{n+1}$ for every $n$ \\
        \noindent \color{purple} \textbf{monotonic} \color{black} if $\{a_n\}$ is strictly
        increasing or decreasing along its entire domain \\
        \noindent \color{purple} \textbf{bounded below} \color{black} if there exists a number
        $m$ such that $m\leq a_n$ for every $n$. $m$ is called the \color{purple} \textbf{lower bound}
        \color{black} of $\{a_n\}$ \\
        \noindent \color{purple} \textbf{bounded above} \color{black} if there exists a number
        $m$ such that $m\geq a_n$ for every $n$. $m$ is called the \color{purple} \textbf{upper bound}
        \color{black} of $\{a_n\}$ \\
        \noindent \color{purple} \textbf{bounded} \color{black} if $a_n$ is both bounded below
        and bounded above \\

        \noindent A series is \color{purple} \textbf{convergent} \color{black} if a given partial
        sum of the sequence has a finite limit. If $\sum a_n$ converges then $\lim_{n\rightarrow\infty}a_n=0$.
        A series is \color{purple} \textbf{divergent} \color{black} if it is not convergent. \\

        \noindent \color{purple} \textbf{Properties of Sequences:} \color{black}
        \noindent If $\{a_n\}$ and $\{b_n\}$ are both convergent sequences then

        \begin{center}
            \begin{tabular}{|c|c|}
                \hline
                $\lim_{n\rightarrow\infty}(a_n\pm b_n)=\lim_{n\rightarrow\infty}a_n
                \pm\lim_{n\rightarrow\infty}b_n$                                     & Sum/Difference    \\
                \hline
                $\lim_{n\rightarrow\infty}ca_n=c\lim_{n\rightarrow\infty}a_n$        & Constant Multiple \\
                \hline
                $\lim_{n\rightarrow\infty}(a_n b_n)=\left(\lim_{n\rightarrow\infty a_n}\right)
                \left(\lim_{n\rightarrow\infty b_n}\right)$                          & Product           \\
                \hline
                $\lim_{n\rightarrow\infty}\frac{a_n}{b_n}=\frac{\lim_{n\rightarrow\infty}a_n}
                {\lim_{n\rightarrow\infty}b_n}, \lim_{n\rightarrow\infty}b_n\not =0$ & Quotient          \\
                \hline
                $\lim_{n\rightarrow\infty} a^p_n=\left[\lim_{n\rightarrow\infty}a_n\right]^p,
                a_n\geq 0$                                                           & Power             \\
                \hline
            \end{tabular}
        \end{center}

        \noindent \color{blue} \textit{Example 1: Determine if the series $\sum^\infty_{n=2}
        \frac{1}{n^2-1}$ converges or diverges and find its sum if it converges.} \color{black} \\

        \noindent The partial sum, $s_n$, is given by
        \begin{align*}
            s_n &= \sum^n_{i=2}\frac{1}{i^2-1} \\
            &= \frac{3}{4}-\frac{1}{2n}-\frac{1}{2(n+1)}
        \end{align*}

        \noindent Then

        \begin{align*}
            \lim_{n\rightarrow\infty}s_n &= \lim_{n\rightarrow\infty}
            \left(\frac{3}{4}-\frac{1}{2n}-\frac{1}{2(n+1)}\right) \\
            &= \frac{3}{4}
        \end{align*}

        \noindent Hence, the series converges and its sum is $\frac{3}{4}$. \\

        \noindent \color{blue} \textit{Example 2: Determine if the series $\sum^\infty_{n=0}
        \frac{4n^2-n^3}{10+2n^3}$ is convergent or divergent.} \color{black}

        \begin{align*}
            \lim_{n\rightarrow\infty} \frac{4n^2-n^3}{10+2n^3} &= -\frac{1}{2} \not =0
        \end{align*}

        \noindent Since the limit of the series is not 0, the series diverges.

    \subsection{Arithmetic and Geometric Sequences and Series}
        \color{purple} \textbf{Arithmetic Sequences} \color{black} have terms that are found by
        adding a constant to the the previous term. The general formula for arithmetic sequences
        is

        \begin{equation*}
            a_n = a + d(n-1)
        \end{equation*}

        \noindent where $a_n$ is the $n$-th term, $a$ is the first term, $d$ is the common
        difference, and $n$ is the term. Since the limit of any arithmetic sequence approaches
        $\pm\infty$, all arithmetic sequences diverge. \\

        \noindent \color{purple} \textbf{Geometric Series} \color{black} are any series that
        can be written in one of the forms below.

        \begin{align*}
            \sum^\infty_{n=1} ar^{n-1} \text{ or } \sum^\infty_{n=0} ar^n
        \end{align*}

        \noindent A geometric series converges if $|r|<1$ and its sum is $\frac{a}{1-r}$. \\

        \pagebreak
        \noindent \color{blue} \textit{Example: Determine if the series $\{a_n\}=\sum^\infty_{n=0}
        \frac{(-4)^{3n}}{5^{n-1}}$ converges or diverges and find the value of the series if
        it converges.} \color{black} \\

        \noindent Let us first rewrite the series so we can clearly see the value of $r$.

        \begin{align*}
            \sum^\infty_{n=0} \frac{(-4)^{3n}}{5^{n-1}} &= \sum^\infty_{n=0}
            \frac{\left((-4)^3\right)^n}{5^n 5^{-1}} \\
            &= \sum^\infty_{n=0} 5\frac{(-64)^n}{5^n} \\
            &= \sum^\infty_{n=0} 5 \left(\frac{-64}{5}\right)^n \\
            \implies r &= -\frac{64}{5} \\
            \because |r| & \geq 1 \\
            \therefore \{a_n\} & \text{ diverges}
        \end{align*}

    \subsection{Binomial Series}

        \begin{align*}
            (1+x)^\alpha                        &= \sum^\infty_{n=0}\left(^\alpha_n\right)x^n \\
            \text{where }\left(^\alpha_n\right) &= \frac{\alpha!}{(\alpha-n)!n!},n,k\in\mathbb{N}
        \end{align*}

        \noindent Binomial Series converge when:

        \begin{align*}
            -1<x<1,         & \alpha<-1 \\
            -1<x\leq1,      & -1<n<0 \\
            -1\leq x\leq 1, & n>0
        \end{align*}

        \noindent \color{blue} \textit{Example: Find the first four terms in the binomial series
        for $\sqrt{9-x}$} \color{black} \\

        \noindent In this case, $k=\frac{1}{2}$ and we must rewrite the given expression to put it
        into binomial form. Then,

        \begin{align*}
            \sqrt{9-x}  &= 3\left(1+\left(-\frac{x}{9}\right)\right)^\frac{1}{2} \\
                        &= 3\sum^\infty_{n=0}\left(^{\frac{1}{2}}_n\right)\left(-\frac{x}{9}\right)^n \\
                        &= 3\left[1+\left(\frac{1}{2}\right)\left(-\frac{x}{9}\right)
                         + \frac{\frac{1}{2}\left(-\frac{1}{2}\right)}{2}\left(-\frac{x}{9}\right)^2
                         + \frac{\frac{1}{2}\left(-\frac{1}{2}\right)\left(-\frac{3}{2}\right)}{6}
                          \left(-\frac{x}{9}\right)^3+\dots\right] \\
                        &= 3-\frac{x}{6}-\frac{x^2}{216}-\frac{x^3}{3888}-\dots
        \end{align*}


    \subsection{The $n$th Term Test for Divergence}
        \color{purple} \textbf{The $n$th Term Test:} \color{black}

        \begin{align*}
            \text{If } \lim_{n\to\infty}a_n\not = 0 \text{ or the limit does not exist, then }
            a_n \text{ diverges}
        \end{align*}

        \noindent \color{blue} \textit{Example: Does the series $a_n=\sum^\infty_{n=1}\frac{(-1)^{n+1}}{n}$
        converge?} \color{black}

        \begin{align*}
            \lim_{n\to\infty}\frac{(-1)^{n+1}}{n} &= 0
        \end{align*}

        \noindent Hence, the $n$th term test is inconclusive for $a_n$.

    \subsection{Integral Test for Convergence}
        \color{purple} \textbf{Integral Test:} \color{black} \\

        \noindent Suppose that $f(x)$ is a continuous, positive, and decreasing function on the
        interval $[k,\infty)$ and that $f(n)=a_n$. Then,

        \begin{align*}
            \text{If }\int^\infty_k f(x)dx & \text{ is convergent then so is } \sum^\infty_{n=k}a_n \\
            \text{If }\int^\infty_k f(x)dx & \text{ is divergent then so is} \sum^\infty_{n=k}a_n
        \end{align*}

        \noindent \color{blue} \textit{Example: Determine if the series $a_n=\sum^\infty_{n=2}
        \frac{1}{n\ln{n}}$ is convergent or divergent.} \color{black} \\

        \noindent Let $f(x)=\frac{1}{x\ln{x}}$. This function is clearly positive and as $x$ grows
        larger, the denominator also gets larger and so the function is decreasing.

        \begin{align*}
            \int^\infty_{2} \frac{1}{x\ln{x}}dx     &= \lim_{t\to\infty}\int^t_2 \frac{1}{x\ln{x}}dx,
                                                    u=\ln{x} \\
                                                    &= \lim_{t\to\infty}(\ln(\ln{x}))\Big|^t_2 \\
                                                    &= \lim_{t\to\infty}(\ln(\ln{t})-\ln(\ln{2})) \\
                                                    &= \infty
        \end{align*}

        \noindent Since the integral diverges, so does $a_n$.


    \subsection{Harmonic and $p$-Series}
        \color{purple} \textbf{$p$-Series} \color{black} have the form

        \begin{align*}
            \sum^\infty_{n=1} \frac{1}{n^p} &= \frac{1}{1^p}+\frac{1}{2^p}+\frac{1}{3^p}+\dots
        \end{align*}

        \noindent for any real-valued number $p, p>0$. \\

        \noindent \color{purple} \textbf{Harmonic Series} \color{black} are $p$-Series with $p=1$
        such that

        \begin{align}
            \sum^\infty_{n=1} \frac{1}{n} &= 1+\frac{1}{2}+\frac{1}{3}+\frac{1}{4}+\frac{1}{5}+\dots
        \end{align}

        \noindent For any $p$-Series,

        \begin{align*}
            \text{If } p>1, & \text{ then the series converges} \\
            \text{If } 0<p\leq 1, & \text{ then the series diverges}
        \end{align*}


    \subsection{Comparison Tests for Convergence}
        \color{purple} \textbf{Comparison Test:} \color{black} \\
        \noindent Suppose we have two series $\sum a_n$ and $\sum b_n$ with $a_n,b_n\geq 0\forall n$
        and $a_n\leq b_n\forall n$. Then,

        \begin{align*}
            \text{If }\sum b_n \text{ is convergent then so is } \sum a_n \\
            \text{If }\sum a_n \text{ is divergent then so is } \sum b_n
        \end{align*}

        \noindent \color{blue} \textit{Example 1: Determine if the series $a_n=\sum^\infty_{n=1}
        \frac{e^{-n}}{n+\cos^2{(n)}}$ converges or diverges.} \color{black}

        \begin{equation*}
            \frac{e^{-n}}{n+\cos^2{(n)}}    \leq \frac{e^{-n}}{n}
        \end{equation*}

        \noindent We know that $n\geq 1$ so we can replace the $n$ in the denominator with its
        smallest possible value (1) as the term will get larger anyway. Then

        \begin{align*}
            \frac{e^{-n}}{n+\cos^2{(n)}}  \leq \frac{e^{-n}}{n} \leq \frac{e^{-n}}{1}=e^{-n}
        \end{align*}

        \noindent Let us now determine if $\sum^\infty_{n=1}e^{-n}$ converges or diverges. Notice
        how $f(x)=e^{-x}$ is always positive and is also decreasing, hence we can use the integral
        test:

        \begin{align*}
            \int^\infty_1 e^{-x}dx  &= \lim_{t\to\infty}\int^t_1 e^{-x}dx \\
                                    &= \lim_{t\to\infty}\left(-e^{-x}\right)\Big|^t_1 \\
                                    &= \lim_{t\to\infty}\left(-e^{-t}+e^{-1}\right) \\
                                    &= e^{-1}
        \end{align*}

        \noindent Since the integral converges, the series $\sum^\infty_{n=1}e^{-n}$ also must
        converge. Furtheremore, because $\sum^\infty_{n=1}e^{-n}$ is larger than $a_n$ we know that
        $a_n$ must also converge. \\

        \pagebreak
        \noindent \color{purple} \textbf{Limit Comparison Test:} \color{black} \\
        \noindent Suppose we have two series $\sum a_n$ and $\sum b_n$ with $a_n\geq 0,b_n\geq0\forall n$.
        Let us define $c$ such that

        \begin{equation*}
            c = \lim_{n\to\infty} \frac{a_n}{b_n}
        \end{equation*}

        \noindent If $0<c<\infty$ then either both series converge or both series diverge. \\

        \noindent \color{blue} \textit{Example 2: Determine if the series $a_n=\sum^\infty_{n=0}
        \frac{1}{3^n-n}$ converges or diverges.} \color{black} \\

        \noindent Let us use $b_n=\sum^\infty_{n=0}$ for our second series because we know that $b_n$
        converges and it is possible that since both series contain $3^n$ the limit won't be too
        complex. Then

        \begin{align*}
            c   &= \lim_{n\to\infty} \frac{1}{3^n}\frac{3^n-n}{1} \\
                &= \lim_{n\to\infty} 1-\frac{n}{3^n} \\
                &= 1-\lim_{n\to\infty} \frac{1}{3^n\ln{(3)}} \\
                &= 1
        \end{align*}

        \noindent Since $c>0,c\not=\infty$, both series must converge since $b_n$ converges.


    \subsection{Alternating Series Test for Convergence}
        \color{purple} \textbf{Alternating Series Test:} \color{black} \\
        If $a_n$ is a decreasing sequence of positive integers such that $\lim_{n\to\infty}a_n=0$,
        then $\sum^\infty_{n=1}(-1)^n a_n$ and $\sum^\infty_{n=1} (-1)^{n+1}a_n$ converge. \\

        \noindent \color{blue} \textit{Example 1: What are all of the positive values of $p$ such
        that $\sum^\infty_{n=1}(-1)^{n-1}\left(\frac{2}{p}\right)^n$ converges?}. \color{black} \\

        \noindent We can write the series as $\sum^\infty_{n=1}(-1)^{n+1}a_n$, where
        $a_n=\left(\frac{2}{p}\right)^n$. If $p>0$ then $a_n$ is always positive, forming an
        alternating series. For the sequence $a_n$ to be decreasing and for the limit
        $\lim_{n\to\infty}a_n=0$, we must have $0<\frac{2}{p}<1$. Solving for $p$, we get $p>2$.

    \subsection{Absolute and Conditional Convergence}
        A series $\sum a_n$ is \color{purple} \textbf{absolutely convergent} \color{black} if
        $\sum |a_n|$ is convergent. If $\sum a_n$ is convergent and $\sum |a_n|$ is divergent
        then the series is \color{purple} \textbf{conditionally convergent} \color{black}. \\

        \noindent \color{blue} \textit{Example: Determine if the series $\sum^\infty_{n=1}
        \frac{\sin{n}}{n^3}$ is absolutely convergent, conditionally convergent, or divergent.}
        \color{black}

        \begin{align*}
            \sum^\infty_{n=1}\Bigg|\frac{\sin{n}}{n^3}\Bigg| &= \sum^\infty_{n=1}\frac{|\sin{n}|}{n^3} \\
            -1\leq\sin{n}\leq 1     \implies                 &  |\sin{n}|\leq 1 \\
            \frac{|\sin{n}|}{n^3}                            &\leq \frac{1}{n^3}
        \end{align*}

        \noindent Hence, $\sum^\infty_{n=1}\frac{1}{n^3}$ converges by the $p$-series test and so by
        the comparison test we know that $\sum^\infty_{n=1}\frac{|\sin{n}|}{n^3}$ converges.
        Therefore, the original series is absolutely convergent.


    \subsection{Ratio Test for Convergence}
        Suppose we have the series $\sum a_n$. For

        \begin{align*}
            L    &= \lim_{n\to\infty}\Bigg|\frac{a_{n+1}}{a_n}\Bigg|, \\
            \text{If } L<1 & \text{ the series is absolutely convergent} \\
            \text{If } L>1 & \text{ the series diverges} \\
            \text{If } L=1 & \text{ the test is inconclusive}
        \end{align*}

        \noindent \color{blue} \textit{Example: Determine if the series $\sum^\infty_{n=0}
        \frac{n!}{5^n}$ is convergent or divergent.} \color{black}

        \begin{align*}
            L   &= \lim_{n\to\infty}\Bigg|\frac{(n+1)!}{5^{n+1}}\frac{5^n}{n!}\Bigg| \\
                &= \lim_{n\to\infty}\frac{(n+1)!}{5n!}
        \end{align*}

        \pagebreak
        \noindent Recall that we can always take out terms from a factorial. Doing that with the
        numerator, we get

        \begin{align*}
            L   &= \lim_{n\to\infty}\frac{(n+1)!n!}{5n!} \\
                &= \lim_{n\to\infty}\frac{(n+1)}{5} \\
                &= \infty \\
                &> 1
        \end{align*}

        \noindent Hence, this series diverges.


    \subsection{Alternating Series Error Bound}
        \color{purple} \textbf{Alternating Series Estimation Theorem}: \color{black} \\
        \noindent Let $\sum^\infty_{n=1}a_n$ be a series satisfying all conditions of the
        Alternating Series Test. The error estimation between the sum $S$ and the $n$th partial sum
        $S_n$ can be evaluated by

        \begin{equation*}
            |s-s_n|\leq|a_{n+1}|=|s_{n+1}-s_n|
        \end{equation*}

        \noindent \color{blue} \textit{Example: Determine the number of terms of the series
        $\sum^\infty_{n=1}\frac{2(-1)^n}{n}$ needed to be computed in order for the sum of the series
        to have an error less than 0.01.} \color{black} \\

        \noindent We can verify that this series satisfies all conditions of the alternating series
        test and so we need to find a value of $n$ such that

        \begin{equation*}
            |s-s_n|\leq|a_{n+1}|=\left|\frac{2(-1)^{n+1}}{n+1}\right|=\frac{2}{n+1}<0.01
        \end{equation*}

        \noindent Note that the above equality holds if the following inequality holds.

        \begin{equation*}
            \frac{n+1}{2} > 100\equiv n>199
        \end{equation*}

        \noindent Hence, if $n\geq 200$ then $|s-s_n|\leq0.01$ and so the error between the partial
        sum $s_n$ and the actual sum $s$ is less than 0.01.


    \subsection{Taylor and Maclaurin Series of a Function}
        \color{purple} \textbf{Taylor's Theorem} \color{black} states that any differentiable
        function may be represented by a Taylor Series such that

        \begin{align*}
            f(x) &= \sum^\infty_{n=0}\frac{f^{(n)}(a)}{n!}(x-a)^n \\
                 &= f(a)+f'(a)(x-a)+\frac{f"(a)(x-a)^2}{2!}+\frac{f'''(a)(x-a)^3}{3!}+\dots
        \end{align*}

        \noindent A \color{purple} \textbf{Maclaurin Series} \color{black} is a special case of a
        Taylor series where $a=0$ such that

        \begin{align*}
            f(x) &= \sum^\infty_{n=0} \frac{f^{(n)}(0)}{n!}x^n \\
                 &= f(0) + f'(0)x + \frac{f"(0)}{2!}x^2 + \frac{f'''(0)}{3!}x^3+\dots
        \end{align*}

        \noindent \color{purple} \textbf{Important Taylor Expansions to Know:} \color{black}

        \begin{align*}
            e^x     &= \sum^\infty_{n=0} \frac{x^n}{n!} \\
            \cos{x} &= \sum^\infty_{n=0} \frac{(-1)^n x^{2n}}{(2n)!} \\
            \sin{x} &= \sum^\infty_{n=0} \frac{(-1)^n x^{2n+1}}{(2n+1)!}
        \end{align*}


    \subsection{Taylor Polynomial Approximations}
        A \color{purple} \textbf{Taylor Polynomial Approximation} \color{black} uses a Taylor series
        to represent a number as a polynomial that has a very similar value to a number near
        a particular $x$ value:

        \begin{equation*}
            f(x)    = f(a) + \frac{f'(a)(x-a)}{1!} + \frac{f"(a)(x-a)^2}{2!}
                    + \frac{f^{(3)}(a)(x-a)^3}{3!} + \dots
        \end{equation*]}

        \pagebreak
        \noindent \color{blue} \textit{Example 1: Use the first three terms of the Taylor series
        expansion of $f(x)=\sqrt[3]{x}$ centered at $x=8$ to approximate $\sqrt[3]{8.1}$.} \color{black}

        \begin{align*}
            f(x)    &= \sqrt[3]{x} \\
                    &\approx 2+\frac{(x-8)}{12}-\frac{(x-8)^2}{288} \\
                    &= 2.008298611111\dots
        \end{align*}

        \noindent \color{blue} \textit{Example 2: Use the quadratic Taylor polynomial for
        $f(x)=\frac{1}{x^2}$ to approximate the value of $\frac{1}{4.41}$.} \color{black} \\

        \noindent The quadratic Taylor polynomial is given by

        \begin{align*}
            T_2(x)  &= f(a) + \frac{f'(a)(x-a)}{1!} + \frac{f"(a)(x-a)^2}{2!}
        \end{align*}

        \noindent Let us rewrite the approximated value as

        \begin{align*}
            4.41        &= (2+0.1)^2 \\
            \implies    & a=2,x=2.1
        \end{align*}

        \noindent Then

        \begin{align*}
            T_2(2.1)    &= f(2) + \frac{f'(2)(2.1-2)}{1!} + \frac{f"(2)(2.1-2)^2}{2!} \\
                        &= 0.226875
        \end{align*}


    \subsection{Lagrange Error Bound}
        The \color{purple} \textbf{Lagrange Error Bound} \color{black} gives us an interval of the
        magntiude of error in a Taylor series expansion of a particular function:

        \begin{align*}
            |R_n| \leq \frac{M|x-a|^{n+1}}{(n+1)!}
        \end{align*}

        \noindent where \\
        $R_n$ is the remainder (error) \\
        $x$ is the given $x$-value \\
        $a$ is where the polynomial is centered \\
        $n$ is the degree of the polynomial \\
        $M$ is the maximum absolute value of the $(n+1)$-order derivative on the interval between
        $c$ and $x$.

        \noindent \color{blue} \textit{Example: Use the Lagrange error bound to estimate the error
        in using a 4th degree Maclaurin polynomial to approximate $\cos{\left(\frac{\pi}{4}\right)}$}.
        \color{black}

        \begin{align*}
            T(x)    &= 1-\frac{x^2}{2} + \frac{x^4}{24}
        \end{align*}

        \noindent For the error bound, we will need to know what the 5th-derivative of $f(x)=\cos{x}$
        is:

        \begin{align*}
            f(x)        &= \cos{x} \\
            f'(x)       &= -\sin{x} \\
            f"(x)       &= -\cos{x} \\
            f'''(x)     &= \sin{x} \\
            f^{(4)}(x)  &= \cos{x} \\
            f^{(5)}(x)  &= -\sin{x}
        \end{align*}

        \noindent Note how the largest that $|-\sin{x}|$ could possibly be is 1, so let's use $M=1$.
        Then

        \begin{align*}
            \text{Error }   &\leq \frac{M}{(n+1)!}(x-c)^{n+1} \\
                            &\leq \frac{1}{5!}\left(\frac{\pi}{4}-0\right)^5 \\
                            &= \frac{\left(\frac{\pi}{4}\right)^5}{120} \\
                            &\approx 0.00249
        \end{align*}


    \subsection{Power Series}
        A \color{purple} \textbf{Power Series} \color{black} is any series that can be written in
        the form

        \begin{equation*}
            \sum^\infty_{n=0} c_n (x-a)^n = a_0 + a_1(x-c)^1 + a_2(x-c)^2+\dots
        \end{equation*}

        \noindent where $a$ and $c_n$ are numbers and $c_n$ is often called the \textit{coefficients
        of the series}. \\

        \noindent For a number $R$ called the \color{purple} \textbf{Radius of Convergence}
        \color{black}, any power series will:

        \begin{align*}
            \text{converge if } & a-R<x<a+R \\
            \text{diverge if }  & x<a-R \text{ and } x>a+R
        \end{align*}

        \noindent The \color{purple} \textbf{Interval of Convergence} \color{black} of a power
        series is the interval of all $x$'s, endpoints inclusive if need be, for which the series
        converges. \\

        \noindent \color{blue} \textit{Example: Determine the radius of convergence and interval
        of convergence for the power series $\sum^\infty_{n=1} \frac{(-1)^nn}{4^n}(x+3)^n$}
        \color{black} \\

        \noindent So far, we know that this power series will converge for $x=-3$. To determine the
        remainder of the $x$'s we will get convergence we will use the ratio test:

        \begin{align*}
            L   &= \lim_{n\to\infty} \left|\frac{(-1)^{n+1}(n+1)(x+3)^{n+1}}{4^{n+1}}\cdot
                   \frac{4^n}{(-1)^n(n)(x+3)^n}\right| \\
                &= \lim_{n\to\infty}\left|\frac{-(n+1)(x+3)}{4n}\right|
        \end{align*}

        \noindent Notice that we can remove $x$ from the limit as $x$ does not depend on the limit:

        \begin{align*}
            L   &= |x+3|\lim_{n\to\infty}\frac{n+1}{4n} \\
                &= \frac{|x+3|}{4}
        \end{align*}

        \noindent The ratio test tells us that if $L<1$ then the series will converge, if $L>1$ then
        the series will diverge, and if $L=1$ then the test is inconclusive. So,

        \begin{align*}
            \frac{|x+3|}{4}<1   &\implies |x+3|<4   & \text{ series converges} \\
            \frac{|x+3|}{4}>1   &\implies |x+3|>4   & \text{ series diverges}
        \end{align*}

        \noindent Hence, the radius of convergence for this series is $R=4$. Now we can find the
        interval of convergence. We can get most of the interval by solving the inequality above:

        \begin{align*}
            -4 < x &+ 3 < 4 \\
            -7 < x & < 1
        \end{align*}

        \noindent So, most of the interval of validity is given by $-7<x<1$ and now we only need to
        determine if the power series will converge or diverge at the interval's endpoints. Such
        values of $x$ correspond to the value of $x$ that gives $L=1$. Now we determine convergence
        at these points by plugging them into the original power series and checking the series
        that we get for convergence/divergence. Let $x=-7$. Then the series is:

        \begin{align*}
            \sum^\infty_{n=1} \frac{(-1)^nn}{4n}(-4)^n  &= \sum^\infty_{n=1}\frac{(-1)^nn}{4^n}(-1)^n 4^n \\
                                                        &= \sum^\infty_{n=1}(-1)^n(-1)^n n \\
                                                        &= \sum^\infty_{n=1} (-1)^{2n}n \\
                                                        &= \sum^\infty_{n=1} 1n \\
                                                        &= \sum^\infty_{n=1} n
        \end{align*}

        \pagebreak
        \noindent Since $\lim_{n\to\infty}=\infty\not =0$ the series is divergent. Now let $x=1$.
        The series we get is

        \begin{align*}
            \sum^\infty_{n=1} \frac{(-1)^nn}{4^n}(4)^n &= \sum^\infty_{n=1}(-1)^n n
        \end{align*}

        \noindent Since $\lim_{n\to\infty}(-1)^n n$ does not exist this series is also divergent. \\

        \noindent Hence, our power series will not converge for either endpoint. Then the interval
        of convergence is $-7<x<1$.


    \subsection{Series Strategy}
        1. Does the series look like its terms don't converge to zero? if so, use the $n$th term
        test for divergence. \\
        2. Is the series a $p$-series $\left(\sum\frac{1}{n^p}\right)$ or a geometric series
        $\left(\sum^\infty_{n=0}ar^n\text{ or }\sum^\infty_{n=1}ar^{n-1}\right)$? If so, recall that
        $p$-series will only converge if $p>1$ and geometric series will only converge if $|r|<1$. \\
        3. Is the series similar to a $p$-series or geometric series? Try the Comparison Test. \\
        4. Is the series a rational expression containing only polynomials or radical polynomials?
        Try the Comparison Test and/or the Limit Comparison Test. Note that in order to use the
        Comparison Tests the series' terms all need to be positive. \\
        5. Does the series contain factorials or constants raised to powers involving $n$? Try the
        Ratio Test. \\
        6. Can the series' terms be written in the form $a_n=(-1)^n b_n$ or $a_n=(-1)^{n+1}b_n$?
        Try the Alternating Series Test. \\
        7. Can the series' terms be written in the form $a_n=(b_n)^n$? Try the Root Test. \\
        8. if $a_n=f(n)$ for some positive, decreasing function and $\int^\infty_a f(x)dx$ is
        relatively easy to evaluate then try the Integral Test.