\section{Differentiation}

    \subsection{Rates of Change and Derivatives}
        An \textbf{average rate of change} describes the overall
        $\frac{\text{rise}}{\text{run}}$-value over an interval $[a,b]$. Geometrically,
        this can be represented by the slope of a secant line through $f(a)$ and $f(b)$ of a
        function $f(x)$. In the graph below, the average rate of change between $x=0$ and
        $x=2$ is given by \\

        \begin{equation*}
            m_{\text{avg}} = \frac{f(b)-f(a)}{b-a} = \frac{4-0}{2-0} = 2
        \end{equation*}

        \begin{center}
            \begin{tikzpicture}
                \begin{axis} [
                    axis lines = center,
                    xmin = -5,
                    xmax = 5,
                    ymin = -5,
                    ymax = 5
                ]
                %f(x)
                \addplot[
                    color = red,
                    samples = 100,
                ]
                {-(x-2)^2+4};
                \addlegendentry{$f(x)$}
                %f(a)
                \fill[red] (0,0) circle [radius = 0.1];
                \node [
                    above left = 10pt of {(0,0)}
                ]
                {$f(a)$};
                %f(b)
                \fill[red] (2,4) circle [radius = 0.1];
                \node[
                    left = 10pt of
                    {(2,4)}
                ]
                {$f(b)$};
                %secant
                \draw [
                    dashed,
                    color = blue
                ]
                (0,0) -- (2,4);
                \end{axis}
            \end{tikzpicture}
        \end{center}

        \noindent A derivative is an \textbf{instantaneous rate of change}, or the slope of the
        tangent to a function at a particular point, $a$. Derivatives are always taken
        \textit{with respect} to a variable. For example, a derivative representing the
        instantaneous rate of change between distance and time is a derivative of distance
        with respect to time, the independent variable. For reference, the derivative of a
        function $f(x)$ at $x=1$ is shown in the graph below. \\

        \begin{center}
            \begin{tikzpicture}
                \begin{axis} [
                    axis lines = center,
                    xmin = -5,
                    xmax = 5,
                    ymin = -5,
                    ymax = 5
                ]
                %f(x)
                \addplot[
                    color = red,
                    samples = 100,
                ]
                {-(x-2)^2+4};
                \addlegendentry{$f(x)$}
                %f(1)
                \fill[red] (1,3) circle [radius = 0.1];
                \node [
                    above left = 10pt of {(0,2)}
                ]
                {$f'(1)$};
                %tangent
                \draw [
                    dashed,
                    color = blue
                ]
                (5,11) -- (-5,-9);
                %rise
                \draw[
                    dashed,
                    color = purple
                ]
                (1,0) -- (1,3);
                %run
                \draw[
                    dashed,
                    color = purple
                ]
                (-0.5,0) -- (1,0);
                \end{axis}
            \end{tikzpicture}
        \end{center}

        \noindent Since the derivative is an instantaneous rate of change, we can define the
        derivative as below. \\
        \color{purple} \textbf{$1^{st}$ Limit Definition of the Derivative} \color{black} \\
        The derivative of $f(x)$ with respect to $x$ is the function $f'(x)$, defined as \\

        \begin{equation*}
            f'(x) = \lim_{x\to a} \frac{f(x)-f(a)}{x-a}
        \end{equation*}

        \noindent Then, if we let $x=a+h$ and change the variable of inspection from $a$ to $x$,
        we can realize the most commonly used definition of the derivative, given below. This
        definition is preferred to the above definition, as finding the derivative only requires
        one value of $x$, rather than 2.  \\
        \color{purple} \textbf{$2^{nd}$ Limit Definition of the Derivative} \color{black} \\
        The derivative of $f(x)$ with respect to $x$ is the function $f'(x)$, defined as \\

        \begin{equation*}
            f'(x) = \lim_{h\to 0} \frac{f(a+h)-f(x)}{h}
        \end{equation*}

        \noindent The expression consisting the right side of the equation above is known as the
        \textbf{difference quotient}. \\

        \noindent A function is $f(x)$ is \textbf{differentiable} at $x=a
        $ if $f'(a)$ exists. Since derivatives are defined by limits, for a derivative to exist,
        its limit must also exist. Hence, \\

        \begin{equation*}
            \text{If $f(x)$ is differentiable at $x=a$ then $f(x)$ is continuous at $x=a$}
        \end{equation*}

        \noindent By this, we can also say that differentiability implies continuity. It is
        important to note that its converse is not always true, as not all continuous functions
        are differentiable. \\


        \noindent A few examples of contexts where a derivative may not exist are at vertical
        tangents, corners, and cusps. Take, for example, the graph below, which does not have a
        derivative at $x=0$ since the respective slopes immediately right and left of $x=0$ are
        different. \\

        \begin{center}
            \begin{tikzpicture}
                \begin{axis}[
                    axis lines = center,
                    axis equal image
                ]
                %f(x)
                \addplot [
                    samples = 50,
                    color = red,
                ]
                {abs(x)};
                \addlegendentry{$f(x)=|x|$}
                \end{axis}
            \end{tikzpicture}
        \end{center}

        \noindent \color{purple} \textbf{Derivative Notations:} \\
        \noindent \textbf{Leibniz's Notation:} \color{black} \\
        The Leibniz notation is popular throughout mathematics, most commonly used when the
        equation $y=f(x)$ is regarded as a functional relationship between dependent and
        independent variables $y$ and $x$. Leibniz's notation makes this relationship explicit
        by representing the derivative as \\

        \begin{equation*}
            \frac{dy}{dx}=\frac{d}{dx}y
        \end{equation*}

        \noindent The Leibniz notation is also called \textbf{differential notation}, where
        $dy$ and $dx$ are \textbf{differentials}. Where the function $y=f(x)$, we can write \\

        \begin{equation*}
            \frac{df}{dx}(x)=\frac{df(x)}{dx}=\frac{d}{dx}f(x)
        \end{equation*}

        \noindent A \textbf{differential equation} is an equation relating one or more functions
        and their derivatives. As long as the precondition ($f(x)$ is satisfied), a function can
        be differentiated indefinite times. For example, in physics, acceleration is the
        derivative of velocity, the derivative of position. The \textbf{order} of a differential
        equation is the highest derivative of said equation. Higher-order derivatives can be
        written in the Leibniz's notation as \\

        \begin{equation*}
            \frac{d^2y}{dx^2}, \frac{dy^3}{dx^3}, \frac{dy^4}{dx^4},\dots,\frac{d^ny}{dx^n}
        \end{equation*}

        \noindent With Leibniz's notation, the value of a derivative at a particular point, $a$,
        can be expressed as \\

        \begin{equation*}
            \frac{dy}{dx}\Bigr|_{x=a}
        \end{equation*}

        \noindent This notation is particularly useful in expressing partial derivatives
        (covered later) and making the chain rule intuitive: \\

        \begin{equation*}
            \frac{dy}{dx} = \frac{dy}{du}\cdot\frac{du}{dx}
        \end{equation*}

        \noindent \color{purple} \textbf{Lagrange's Notation:} \color{black} \\
        In Lagrange's notation, each prime mark denotes a derivative. For higher-order derivatives,
        the order is enclosed by parantheses in front and above the function. \\

        \begin{equation*}
            f'(x), f''(x), f'''(x), f^{(4)}(x),f^{(5)}(x),f^{(6)}(x),f^{(n)}(n)
        \end{equation*}

        \noindent \color{purple} \textbf{Newton's Notation:} \color{black} \\

        This notation is often applied to physics contexts, where time ($t$) is the independent
        variable. The number of dots over the dependent variable represent the order the
        derivative is in. If $y$ is a function of $t$, then the first $n$ derivatives of
        the function are as below. \\

        \begin{equation*}
            \dot{y}, \ddot{y}, \dddot{y}, _{\dot{y}}^4, _{\dot{y}}^5, _{\dot{y}}^n
        \end{equation*}

        \noindent \color{purple} \textbf{Euler's Notation:} \color{black} \\
        This notation is quite inconvenient, as it leaves the variable being differentiated with
        respect to entirely implicit. However, we can modify the notation to explicitly write
        said variable. This notation is defined by \\

        \begin{equation*}
            (Df)(x) = \frac{df(x)}{dx}
        \end{equation*}

        \pagebreak
        \noindent Higher-order Derivatives are expressed by \\

        \begin{equation*}
            D^2f=D^2_xf, D^3f=D^3_xf, D^nf=D^n_xf
        \end{equation*}

        \noindent \color{blue} \textit{Example 1: Find the derivative, $g'(t)$, given the function} \\

        \begin{equation}
            g(t) = \frac{t}{t+1}
        \end{equation} \color{black}

        \begin{align*}
            g'(t) &= \lim_{h\to 0} \frac{g(t+h)-g(t)}{h} \\
            &= \lim_{h\to 0} \frac{1}{h}\left(\frac{t+h}{t+h+1}-\frac{t}{t+1}\right) \\
            &= \lim_{h\to 0}\left(\frac{(t+h)(t+1)-t(t+h+1)}{(t+h+1)(t+1)}\right) \\
            &= \lim_{h\to 0} \frac{1}{h}\left(\frac{t^2+t+th+h-(t^2+th+t)}{(t+h+1)(t+1)}\right) \\
            &= \lim_{h\to 0} \frac{1}{h} \left(\frac{h}{(t+h+1)(t+1)}\right) \\
            &= \frac{1}{(t+1)(t+1)} \\
            g'(t) &= \frac{1}{(t+1)^2}
        \end{align*}

        \noindent \color{blue} \textit{Example 2: Find the derivative, $f'(x)$, of } \\

        \begin{equation*}
            f(x) = 2x^2-16x+35
        \end{equation*} \color{black}

        \begin{align*}
            f'(x) &= \lim_{h\to 0} \frac{2(x+h)^2-16(x+h)+35-(2x^2-16x+35)}{h} \\
            &= \lim_{h\to 0} \frac{4xh+2h^2-16h}{h} \\
            &= \lim_{h\to 0} (4x+2h-16) \\
            f'(x) &= 4x-16
        \end{align*}


    \subsection{Basic Derivative Rules}
        \begin{center}
            \begin{tabular}{|c|c|}
                \hline
                $\frac{d}{dx}a = 0$ & \textbf{Constant} \\
                \hline
                $\frac{d}{dx}au = a\frac{du}{dx}$ & \textbf{Constant Multiple} \\
                \hline
                $\frac{d}{dx}x^n=nx^{n-1}$ & \color{purple} \textbf{The Power Rule} \color{black} \\
                \hline
                $\frac{d}{dx}(u\pm v) = \frac{d}{dx}u \pm \frac{d}{dx}v$ & \textbf{Sum/Difference}\\
                \hline
                $\frac{d}{dx}(uv) = u\frac{dv}{dx}+v\frac{du}{dx}$ & \color{purple}
                \textbf{The Product Rule} \color{black} \\
                \hline
                $\frac{d}{dx}\left(\frac{u}{v}\right)=\frac{v\frac{du}{dx}-u\frac{dv}{dx}}{v^2},
                v\not = 0$ & \color{purple} \textbf{The Quotient Rule} \color{black} \\
                \hline
                $[f^{-1}]'(b)=\frac{1}{f'(a)}$ & \textbf{Inverse Functions} \\
                \hline
                $\frac{d}{dx}[f^{-1}(x)]=\frac{1}{f'(f^{-1}(x))}=\frac{1}{\frac{dy}{dx}}$ &
                \textbf{Alternate Inverse Functions} \\
                \hline
                $\frac{d(1/f)}{dx}=-\frac{1}{f^2}\frac{df}{dx}$ & \textbf{Reciprocal} \\
                \hline
            \end{tabular}
        \end{center}


    \subsection{Derivatives of Exponential and Logarithmic Functions}
        \begin{center}
            \begin{tabular}{|c|c|}
                \hline
                $\frac{d}{dx}e^x=e^x$ & \textbf{Natural Exponent} \\
                \hline
                $\frac{d}{dx}\ln{x}=\frac{1}{x}$ & \textbf{Natural Logarithm} \\
                \hline
                $\frac{d}{dx}a^x=a^x\ln{a}$ & \textbf{Exponent} \\
                \hline
                $\frac{d}{dx}\log_b{x}=\frac{1}{x\ln{b}}$ & \textbf{Logarithm} \\
                \hline
                $\frac{d}{dx}x^x=x^x(1+\ln{x})$ & \\
                \hline
            \end{tabular}
        \end{center}


    \pagebreak
    \subsection{Derivatives of the Trigonometric Functions}
        The derivatives of $\sin{x}$ and $\cos{x}$ help us derive the others through applying
        the quotient and reciprocal differentiation rules. \\

        \begin{center}
            \begin{tabular}{|c|}
                \hline
                $\frac{d}{dx}\sin{x}=\cos{x}$ \\
                \hline
                $\frac{d}{dx}\cos{x}=-\sin{x}$ \\
                \hline
                $\frac{d}{dx}\tan{x}=\sec^2{x}$ \\
                \hline
                $\frac{d}{dx}\csc{x}=-\cot{x}\csc{x}$ \\
                \hline
                $\frac{d}{dx}\sec{x}=\tan{x}\sec{x}$ \\
                \hline
                $\frac{d}{dx}\cot{x}=-\csc^2{x}$ \\
                \hline
            \end{tabular}
        \end{center}

        \noindent The derivatives of the inverse trigonometric functions can be derived through
        applying the inverse differentiation rule. \\

        \begin{center}
            \begin{tabular}{|c|}
                \hline
                $\frac{d}{dx}\arcsin{x} = \frac{1}{\sqrt{1-x^2}}$ \\
                \hline
                $\frac{d}{dx}\arccos{x} = -\frac{1}{\sqrt{1-x^2}}$ \\
                \hline
                $\frac{d}{dx}\arctan{x} = \frac{1}{1+x^2}$ \\
                \hline
                $\frac{d}{dx}\arccsc{x} = -\frac{1}{|x|\sqrt{x^2-1}}$ \\
                \hline
                $\frac{d}{dx}\arcsec{x} = \frac{1}{|x|\sqrt{x^2-1}}$ \\
                \hline
                $\frac{d}{dx}\arccot{x} = -\frac{1}{1+x^2}$ \\
                \hline
            \end{tabular}
        \end{center}


        \noindent \color{purple} \textbf{Derivation of the Inverse Sine Derivative:} \color{black} \\
        \begin{align*}
            \frac{d(\arcsin{x})}{dx} &= \frac{1}{\frac{d(\sin{y})}{dy}} \\
            &= \frac{1}{\cos{y}} \\
        \end{align*}
        \noindent From the Pythagorean Identity, \\
        \begin{align*}
            \cos^2{y} + \sin^2{y} &= 1 \\
            \cos^2{y} &= 1 - \sin^2{y} \\
            \cos{y} &= \sqrt{1-\sin^2{y}}
        \end{align*}
        \noindent Substituting this back into the original equation, we get \\
        \begin{align*}
            \frac{d(\arcsin{x})}{dx} &= \frac{1}{\sqrt{1-\sin^2{y}}}
        \end{align*}
        \noindent By the definition of the inverse sine function, we can express the equation as \\
        \begin{align*}
            \frac{d}{dx}\arcsin{x} &= \frac{1}{\sqrt{1-x^2}}
        \end{align*}

        \noindent \color{purple} \textbf{Derivation of the Inverse Cosine Derivative} \color{black} \\
        The derivatives of the inverse trigonometric functions are the \textit{negatives} of the
        derivatives of their cofunctions. This is because \\

        \begin{align*}
            \arccos{x} &= \frac{\pi}{2} - \arcsin{x} \\
            \frac{d}{dx}\arccos{x} &= -\frac{d}{dx}\arcsin{x} \\
            &= -\frac{1}{\sqrt{1-x^2}}
        \end{align*}

        \noindent \color{purple} \textbf{Derivation of the Inverse Tangent Derivative} \color{black} \\

        \begin{align*}
            \frac{d}{dx}\arctan{x} &= \frac{1}{\frac{d(\tan{y})}{dy}} \\
            &= \frac{1}{\sec^2{y}} \\
            &= \frac{1}{1+\tan^2{y}} \\
            &= \frac{1}{1+x^2}
        \end{align*}

    \pagebreak
    \subsection{The Chain Rule}
        The Chain Rule is incredibly useful for finding the derivative of composite functions.
        If $y=f(u)$ and $u=g(x)$ then \\

        \begin{align*}
            (f(g(x)))' &= f'(g(x)) \cdot g'(x) \\
            &= f'(u) \cdot g'(x) \\
            \frac{dy}{dx} &= \frac{dy}{du} \cdot \frac{du}{dx}
        \end{align*}

        \noindent \color{blue} \textit{Example 1:} \color{black} \\

        \begin{align*}
            f(x) &= \sqrt{5x-8} \\
        \end{align*}

        \noindent We let the inside function be expressed as $u=5x-8$. Then \\

        \begin{align*}
            f'(x) &= \frac{du}{dx}u^{\frac{1}{2}}\cdot\frac{dy}{du}(5x-8) \\
            &= \frac{1}{2\sqrt{u}}\cdot 5 \\
            &= \frac{5}{2\sqrt{5x-8}}
        \end{align*}

        \noindent \color{blue} \textit{Example 2:} \color{black} \\
        \noindent Let $u=\cos{t}$ and $v=t^4$

        \begin{align*}
            g(t) &= \cos^4{t} + \cos{(t^4)} \\
            g'(t) &= \frac{d(u^4)}{du} \cdot \frac{d}{dt}\cos{t}
            +
            \frac{d(t^4)}{dt} \cdot \frac{d(\cos{v})}{dv} \\
            &= 4u^3(-\sin{t}) + 4t^3(-\sin{v}) \\
            &= -4\cos^3{t}\sin{t} - 4t^3\sin{(t^4)}
        \end{align*}

    \subsection{Implicit Differentiation}
        \color{purple} \textbf{Implicit Differentiation} \color{black} is an approach to
        differentiating a function that may not be in the explicit form, $y=f(x)$. It
        considers all other variables as functions of one of its variables, then uses the
        chain rule to find the derivative. \\

        \noindent \color{blue} \textit{Example 1: Given $x^2+x+y^2=15$, what is $\frac{dy}{dx}$
        at the point $(2,3)$?} \color{black} \\

        \begin{align*}
            \frac{dy}{dx}(x^2+x+y^2) &= \frac{dy}{dx}15 \\
            2x + 1 + 2y\left(\frac{dy}{dx}\right) &= 0 \\
            \frac{dy}{dx} &= -\frac{2x+1}{2y} \\
            \frac{dy}{dx}\Bigr|_{(2,3)} &= -\frac{2(2)+1}{2(3)} \\
            &= -\frac{5}{6}
        \end{align*}

        \noindent \color{blue} \textit{Example 2: Find the derivative of
        $\ln{y}+e^y=\sin{y^2}-3\cos{x}$} \color{black} \\

        \begin{align*}
            \frac{dy}{dx}\ln{y} + \frac{dy}{dx}e^y &= \frac{d}{dx}\sin{y^2}-\frac{d}{dx}3\cos{x} \\
            \frac{d}{dy}\ln{y}\cdot\frac{dy}{dx}+\frac{d}{dy}e^y\cdot\frac{dy}{dx}
            &= \frac{d}{dy}\sin{y^2}\cdot\frac{dy}{dx}+3\sin{x} \\
            \frac{dy}{dx}\left(\frac{1}{y}+e^y-2y\cos{y^2}\right) &= 3\sin{x} \\
            \frac{dy}{dx} &= \frac{3\sin{x}}{\frac{1}{y}+e^y-2y\cos{y^2}}
        \end{align*}

        \noindent \color{blue} \textit{Example 3: Find $\frac{dy}{dx}$ if $y^2 &= x^2 + \sin{(xy)}$}
        \color{black} \\

        \begin{align*}
            \frac{d}{dx}(y^2) &= \frac{d}{dx}(x^2) + \frac{d}{dx}(\sin{(xy)} \\
            2y\frac{d}{dx} &= 2x + (\cos{(xy)})\left(y+x\frac{d}{dx}\right) \\
            (2y-x\cos{(xy)})\frac{d}{dx} &= 2x + y\cos{(xy)} \\
            \frac{dy}{dx} &= \frac{2x+y\cos{(xy)}}{2y-x\cos{(xy)}}
        \end{align*}


    \subsection{Logarithmic Differentiation}
        This method of finding derivatives uses the basic properties of logarithms outside
        of calculus to simply a function prior to differentiating it. \\

        \noindent \color{blue} \textit{Example 1:} \color{black} \\

        \begin{align*}
            y &= \frac{x^5}{(1-10x)\sqrt{x^2+2}} \\
            \ln{y} &= \ln{\left(\frac{x^5}{(1-10x)\sqrt{x^2+2}}\right)} \\
            &= \ln{(x^5)} - \ln{\left((1-10x)\sqrt{x^2+2}\right)} \\
            &= \ln{(x^5)} - \ln{(1-10x)} - \ln{(\sqrt{x^2+2})}
        \end{align*}

        \noindent By Implicit Differentiation, \\

        \begin{align*}
            \fra{y'}{y} &= \frac{5x^4}{x^5}- \frac{-10}{1-10x} - \frac{\frac{1}{2}(x^2+2)^{-\frac{1}{2}}(2x)}{(x^2+2)^{\frac{1}{2}}} \\
            &= \frac{5}{x} + \frac{10}{1-10x} - \frac{x}{x^2+2} \\
            \frac{dy}{dx} &= \frac{x^5}{(1-10x)\sqrt{x^2+2}}\left(\frac{5}{x}+\frac{10}{1-10x}-\frac{x}{x^2+2}\right)
        \end{align*}

        \noindent \color{blue} \textit{Example 2:} \color{black} \\

        \begin{align*}
            y &= \frac{x+3}{(x+4)^3} \\
            \ln{y} &= \ln{\left(\frac{x+3}{(x+4)^3}\right)} \\
            &= \ln{(x+3)} - \ln{(x+4)^3} \\
            &= \ln{(x+3)}  - 3\ln{(x+4)} \\
            \frac{1}{y}\cdot\frac{dy}{dx} &= \frac{1}{x+3}-3\cdot\frac{1}{x+4} \\
            &= \frac{1}{x+3}-\frac{3}{x+4} \\
            \frac{dy}{dx} &= y\left(\frac{1}{x+3}-\frac{3}{x+4}\right)
        \end{align*}


    \subsection{Derivatives of Parametric Functions}
        If $x=f(t)$ and $y=g(t)$ are differentiable functions of parameter $t$, then \\

        \begin{equation*}
            \frac{dy}{dx} = \frac{\frac{dy}{dt}}{\frac{dx}{dt}}
        \end{equation*}

        \noindent and

        \begin{equation*}
            \frac{d^2y}{dx^2} = \frac{d}{dx}\left(\frac{dy}{dx}\right)
            = \frac{\frac{d}{dt}\left(\frac{dy}{dx}\right)}{\frac{dx}{dt}}
        \end{equation*}

        \noindent \color{blue} \textit{Example: If $x=2\sin{\theta}$ and $y=\cos{2\theta}$,
        find $\frac{d^2y}{dx^2}$} \color{black} \\

        \begin{align*}
            \frac{dy}{dx} &= \frac{\frac{dy}{d\theta}}{\frac{dx}{d\theta}} \\
            &= \frac{-2\sin{2\theta}}{2\cos{\theta}} \\
            &= -\frac{2\sin\theta\cos\theta}{\cos\theta} \\
            &= -2\sin\theta
        \end{align*}

        \begin{align*}
            \frac{d^2y}{dx^2} &= \frac{\frac{d}{d\theta}\left(\frac{dy}{dx}\right)}{\frac{dx}{d\theta}} \\
            &= \frac{-2\cos\theta}{2\cos\theta} \\
            &= -1
        \end{align*}



    \subsection{Derivatives of Polar Functions}
        We know that $x=r\cos\theta$ and $y=r\sin\theta$. Then, by using the parametric derivative
        formula, we can determine the rule for differentiating polar functions. \\

        \begin{align*}
            \frac{dy}{dx} &= \frac{\frac{dy}{d\theta}}{\frac{dx}{d\theta}} \\
            &= \frac{f'(\theta)\sin\theta+f(\theta)\cos\theta}{f'(\theta)\cos\theta-f(\theta)\sin\theta} \\
            &= \frac{r'\sin\theta+r\cos\theta}{r'\cos\theta-r\sin\theta}
        \end{align*}

        \noindent \color{blue} \textit{Example: Find the slope of the cardioid
        $r=2(1+\cos\theta)$ at $\theta=\frac{\pi}{6}$} \color{black} \\

        \begin{align*}
            r=2(1+\cos\theta) \\
            r' = -2\sin\theta
        \end{align*}

        \noindent Then \\

        \begin{align*}
            \frac{dy}{dx} &= \frac{\frac{dy}{d\theta}}{\frac{dx}{d\theta}} \\
            &= \frac{(-2\sin\theta)\sin\theta+2(1+\cos\theta)(\cos\theta)}{(-2\sin\theta)\cos\theta-2(1+\cos\theta)(\sin\theta)} \\
            \frac{dy}{dx}\Bigr|_{\frac{\pi}{6}} &= -1
        \end{align*}



    \subsection{Derivatives of Vector-Valued Functions}
        If a point moves along a curve defined parametrically by $P(t)=\langle x(t), y(t)\rangle$,
        where $t$ represents time, then the vector from the origin to $P$ is called the
        \textbf{position vector}. Then the derivative of the position vector is called the
        \textbf{velocity vector}, and its derivative is called the \textbf{acceleration vector}.  \\

        \noindent \color{purple} \textbf{The Velocity Vector:} \color{black} \\

        \begin{equation*}
            \overrightarrow{v}(t) = \left\langle\frac{dx}{dt},\frac{dy}{dt}\right\rangle
        \end{equation*}

        \noindent The slope of $\overrightarrow{v}$ is given by \\

        \begin{equation*}
            \frac{dy}{dx} = \frac{\frac{dy}{dt}}{\frac{dx}{dt}}
        \end{equation*}

        \noindent \color{purple} \textbf{The Magnitude of a Velocity Vector:} \color{black} \\

        \begin{equation*}
            |v| = \sqrt{\left(\frac{dx}{dt}\right)^2+\left(\frac{dy}{dt}\right)^2} = \sqrt{v^2_x+v^2_y}
        \end{equation*}

        \noindent \color{purple} \textbf{The Acceleration Vector:} \color{black} \\

        \begin{equation*}
            \overrightarrow{a}(t) = \left\langle\frac{d^2x}{dt^2},\frac{d^2y}{dt^2}\right\rangle
        \end{equation*}

        \noindent \color{purple} \textbf{The Magnitude of an Acceleration Vector:} \color{black} \\

        \begin{equation*}
            |a| = \sqrt{\left(\frac{d^2x}{dt^2}\right)^2+\left(\frac{d^2y}{dt^2}\right)^2} = \sqrt{a^2_x+a^2_y}
        \end{equation*}



    \subsection{Rolle's Theorem}
        \color{purple} \textbf{Rolle's Theorem:} \color{black} \\
        \noindent Suppose $f(x)$ is a function that is both continuous on the closed interval
        $[a,b]$ and differentiable on the open interval $(a,b)$, and that $f(a)=f(b)$.
        Then there is a number $c$ such that $a<c<b$ and $f'(c)=0$. Or rather, $f(x)$ has a
        critical point in $(a,b)$. \\



    \subsection{The Mean Value Theorem}
        \color{purple} \textbf{The Mean Value Theorem:} \color{black} \\
        \noindent Suppose $f(x)$ is a function that is both continuous on the closed interval
        $[a,b]$ and differentiable on the open interval $(a,b)$. Then there is a number
        $c$ such that $a<c<b$ and \\

        \begin{equation*}
            f'(c) = \frac{f(b)-f(a)}{b-a}
        \end{equation*}

        \noindent Or, \\

        \begin{equation*}
            f(b) - f(a) = f'(c) (b-a)
        \end{equation*}

        \noindent \color{blue} \textit{Example 1: Determine all the numbers $c$ satisfying
        $f(x)=x^3+2x^2-x$ on $[-1,2]$} \color{black} \\

        \begin{align*}
            f'(x) &= 3x^2+4x-1 \\
            f'(c) &= \frac{f(2)-f(-1)}{2-(-1)} \\
            3c^2+4c-1 &= \frac{14-2}{3} \\
            3c^2+4c-1 &= 4 \\
            3c^2+4c-5 &= 0 \\
            c &= \frac{-4\pm\sqrt{16-4(3)(-5)}}{6} \\
            c &= 0.7863, -2.1196 \\
            c &= 0.7863
        \end{align*}

        \noindent Notice we were able to remove the other $c$, as it is outside of the desired interval.

        \noindent \color{blue} \textit{Example 2: Suppose we know that $f(x)$ is continuous and
        differentiable on $[6,15]$ and that $f(6)=-2$ and $f'(x)\leq10$. Find the
        largest possible value for $f(15)$.} \color{black} \\

        \noindent The MVT tells us that \\

        \begin{equation*}
            f(15) - f(6) = f'(c)(15-6)
        \end{equation*}

        \noindent Plugging in known quantities and simplifying, we get \\

        \begin{equation*}
            f(15) = -2 + 9f'(c)
        \end{equation*}

        \noindent Since we are given $f'(x)\leq 10$, we know that $f'(c)\leq 10$. This gives us \\

        \begin{align*}
            f(15) &= -2 + 9f'(c) \\
            &\leq -2 + (9)10 \\
            &\leq 88
        \end{align*}

        \noindent Hence, the largest possible value of $f(15)=88$.