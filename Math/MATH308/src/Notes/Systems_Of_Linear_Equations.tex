\section{Systems of Linear Equations}

    \subsection{Lines and Linear Equations}     % 1.1

        \textbf{Linear equations}: have the form

        \[
            a_1 x_1 + a_2 x_2 + a_3 x_3 + \dots + a_n x_n = b
        \]

        where $a_1, a_2, \dots, a_n$ and $b$ are constants. \\

        \textbf{Solution}: the solution $(s_1, s_2, \dots, s_n)$ to a linear equation is an ordered set of $n$ numbers ($n$-tuple) such that if we set $x_1 = s_1, x_2 = s_2, \dots, x_n = s_n$, then the linear equation
        is satisfied \\
        \textbf{Solution set:} the set of all solutions to an equation \\
        $\bullet$ The graph of the solution set to an equation with two variables is a line \\
        $\bullet$ The graph of the solution set to an equation with three variables is a plane \\

        \textbf{Hyperplane}: the solution set of all $n$-tuples that satisfy a linear equation where $n\geq 4$ \\
        \textbf{Consistent}: a linear system with at least one solution \\
        \textbf{Inconsistent}: a linear system that is not consistent \\

        \textit{\blue{Example:}}: Find all solutions to the system of linear equations

        \begin{align*}
            4x_1 + 10x_2    &= 14 \\
            -6x_1 - 15x_2   &= -21
        \end{align*}

        Multiplying the first equation by $\frac{3}{2}$ and then adding,

        \begin{align*}
            6x_1 - 15x_2    &= 21 \\
            +(-6x_1 + 15x_2 &= -21) \\
            \implies 0      &= 0
        \end{align*}

        The equation $0=0$ is satisfied by any choices of $x_1$ and $x_2$, thus we can select any one of the equations and solve for $x_1$ in terms of $x_2$, giving

        \[
            x_1 = \frac{7}{2} - \frac{5}{2}x_2
        \]

        There are infinitely many solutions to this and so to avoid confusing variables with values satisfying the linear system, we can describe the solutions by

        \begin{align*}
            x_1 &= \frac{7}{2} - \frac{5}{2}s_1 \\
            x_2 &= s_1
        \end{align*}

        where $s_1$ is called a \textbf{free parameter} and can be any real number. This is known as the \textbf{general solution} because it gives all solutions to the system of equations. \\

        \textbf{Back Substitution}: example to solve system is given below

        \begin{align*}
            x_1 - 2x_2 - 5x_3 + 3x_4    &= 2 \\
            x_2 + 3x_3 - 4x_4           &= 7 \\
            x_3 + 2x_4                  &= -4 \\
            x_4                         &= 5
        \end{align*}

        We can solve the system like so:

        \begin{align*}
            x_3 + 2(5)  = -4  &\implies x_3 = -14 \\
            x_2 + 3(-14) - 4(5) = 7 &\implies x_2 = 69 \\
            x_1 - 2(69) - 5(-14) + 3(5) = 2 &\implies x_1 = 55
        \end{align*}

        Thus,

        \[
            x_1 = 55, x_2 = 69, x_3 = -14, x_4 = 5
        \]

        \textbf{Leading variable:} a variable that appears as the first term in at least one equation \\
        $\bullet$ in the example above, $x_1, x_2, x_3$, and $x_4$ are leading variables \\
        $\bullet$ in the system below, $x_1, x_4, x_5$ are leading variables and $x_2,x_3$ are not

        \begin{align*}
            -4x_1 + 2x_2 - x_3 + 3x_5   &= 7 \\
            -3x_4 + 4x_5                &= -7 \\
            x_4 - 2x_5                  &= 1 \\
            7x_5                        &= 2
        \end{align*}

        \textbf{Triangular Form, Triangular System}: has the following three properties \\
        $\bullet$ every variable is the leading variable of exactly one equation \\
        $\bullet$ there are the same number of equations as variables \\
        $\bullet$ there is exactly one solution

        \begin{figure*}[hbt!]
            \centering
            \caption*{Example of a triangular system:}
            \includegraphics[scale = 0.75]{Assets/1_Triangular_System}
        \end{figure*}

        \textbf{Echelon Form, Echelon System}: a linear system organized in a descending "stair step" pattern from left to right, so that the indices of the leading variables are strictly increasing from top to bottom \\
        $\bullet$ equations without variables of the form $0=c$ are at the bottom, with those where $c\not = 0$ above those where $c=0$ \\
        $\bullet$ all triangular systems are in echelon form \\

        Properties of echelon systems: \\
        $\bullet$ every variable is the leading variable of \textit{at most one} equation \\
        $\bullet$ there are no solutions, exactly one solution, or infinitely many solutions \\
        $\bullet$ back substitution may be used starting with the last variable

        \textbf{Free variable} a variable that is not a leading variable in a system in echelon form \\

        To find the general solution to an echelon system: \\
        1. Set each free variable equal to a free parameter \\
        2. Back substitute to solve for the leading variables

    \subsection{Linear Systems and Matrices}    % 1.2

        It is nice to convert a linear system into echelon form so that back-substitution may be applied. Consider the projectile motion problem: \textit{Suppose that a cannon sits on a hill and fires a ball across a
        flat field below. The path of the ball is known to be approximately parabolic and so can be modeled by a quadratic function $E(x)=ax^2 + bx + c$, where $E$ is the elevation in feet over position $x$, and
        $a,b,c$ are constants.}. \\

        Since every point on the cannonball's path is given by $(x, E(x))$, the data can be converted into three linear equations:

        \begin{align*}
            100a + 10b + c      &= 117 \\
            2900a + 30b + c     &= 171 \\
            2500a + 50b + c     &= 145
        \end{align*}

        As this system is not in echelon form, back substitution is not easy to use here.  \\

        \textbf{Equivalent:} linear systems that have the same set of solutions \\
        \textbf{Elementary operations:} there are three types, given below \\

        \textbf{1. Interchange the position of two equations:} In the example below, the places of the first and second equations are exchanged and thus the systems are equal \\
        $\bullet$ the tilda, ~, indicates the \textit{transformation} from one linear system to an equivalent linear system

        \begin{figure*}[hbt!]
            \centering
            \includegraphics[scale = 0.75]{Assets/1_Transformation}
        \end{figure*}

        \textbf{2. Multiply an equation by a nonzero constant:} In the example below, the third equation is multiplied by -2

        \begin{figure*}[hbt!]
            \centering
            \includegraphics[scale = 0.75]{Assets/1_Transformation2}
        \end{figure*}

        \textbf{Add a multiple of one equation to another:} In the example below the top equation is multiplied by -4 and then added to the bottom equation, replacing the bottom equation with the result.

        \begin{figure*}[hbt!]
            \centering
            \includegraphics[scale = 0.75]{Assets/1_Transformation3}
        \end{figure*}

        \textit{\blue{Example:}} Find the set of solutions to the system of linear equations

        \begin{align*}
            x_1 - 3x_2 + 2x_3   &= -1 \\
            2x_1 - 5x_2 - x_3   &= 2 \\
            -4x_1 + 13x_2 - 12x_3   &= 11
        \end{align*}

        Our goal is to transform the system to echelon form so we want to eliminate the $x_1$ terms in the second and third equations. This will leave $x_1$ as the leading variable in only the top equation. \\

        \textit{Add a multiple of one equation to another}, focusing on $x_1$. Recall the generic system of equations as follows.

        \begin{figure*}[hbt!]
            \centering
            \includegraphics[scale = 0.75]{Assets/1_Generic_Linear_System}
        \end{figure*}

        We need to transform $a_{21}$ and $a_{31}$ to 0. Since $a_{21} = 2$, if we take -2 times the first equation and add it to the second, then the resulting coefficient on $x_1$ will be $(-2) \cdot 1 + 2 = 0$.

        \begin{figure*}[hbt!]
            \centering
            \includegraphics[scale = 0.75]{Assets/1_Transformation_Example}
        \end{figure*}

        Similarly, for the second part, $(4)\cdot 1-4=0$ and so we multiply the first equation by 4 and add it to the third:

        \begin{figure*}[hbt!]
            \centering
            \includegraphics[scale = 0.75]{Assets/1_Transformation_Example2}
        \end{figure*}

        Now the $x_1$ terms in the second and third equations are gone like we wanted and we can focus on the $x_2$ coefficients. Since our goal is to reach echelon form, we only care about the $x_2$ in the second and
        third equations. \textit{Add a multiple of one equation to another}, focusing on $x_2$. Here we need to transform $a_{32}$ to 0. Since $(-1)\cdot 1+1=0$, we multiply the second equation by -1 and add the result
        to the third equation.

        \begin{figure*}[hbt!]
            \centering
            \includegraphics[scale = 0.75]{Assets/1_Transformation_Example3}
        \end{figure*}

        The system is now in echelon form and using back substitution it can be easily shown that the solution is $x_1=50, x_2 = 19, x_3 = 3$. \\

        \textbf{Augmented matrix:} a matrix that contains all the coefficients of a linear system including the constant terms on the right side of each equation \\
        $\bullet$ the vertical line in augmented matrices separates the left and right sides of equations

        \begin{figure*}[hbt!]
            \centering
            \includegraphics[scale = 0.75]{Assets/1_Augmented_Matrix}
        \end{figure*}

        \textbf{Elementary Row Operations:} correspond to the elementary operations performed on equations \\
        1. Interchange two rows \\
        2. Multiply a row by a nonzero constant \\
        3. Replace a row with the sum of that row and the scalar multiple of another row \\

        Matrices have rows numbered from top to bottom and columns numbered from left to right. \\
        \textbf{Zero row:} row consisting entirely of zeros \\

        \textit{\blue{Example 2:}} Find all solutions to the system of linear equations

        \begin{align*}
            2x_1 - 3x_2 + 10x_3 &= -2 \\
            x_2 - 2x_2 + 3x_3   &= -2 \\
            -x_1 + 3x_3 + x_3   &= 4
        \end{align*}

        Begin by converting the system to an augmented matrix:

        \[
            \begin{bmatrix}[*2cr@{\quad}|@{\quad}>{\color{red}}r]
                2   & -3    & 10    & -2 \\
                1   & -2    & 3     & -2 \\
                -1  & 3     & 1     & 4
            \end{bmatrix}
        \]

        Interchange rows, focusing on column 1, which contains the coefficients of $x_1$. Exchanging rows 1 and 2 will move a 1 into the upper left position and avoid the early introduction of fractions

        \[
            \begin{bmatrix}[*2cr@{\quad}|@{\quad}>{\color{red}}r]
                2   & -3    & 10    & -2 \\
                1   & -2    & 3     & -2 \\
                -1  & 3     & 1     & 4
            \end{bmatrix}
            \approx
            \begin{bmatrix}[*2cr@{\quad}|@{\quad}>{\color{red}}r]
                1   & -2    & 3     & -2 \\
                2   & -3    & 10    & -2 \\
                -1  & 3     & 1     & 4
            \end{bmatrix}
        \]

        The compact notation for this operation is $R_1 \iff R_2$. Then we add a multiple of one row to another. To transform the system to echelon form, we need to introduce zeros in the first column below Row 1. This
        requires two operations. Focusing first on row 2 wince $(-2)(1) + 2 = 0$, we add -2 times row 1 to row 2 and replace row 2 with the result.

        \[
            \begin{bmatrix}[*2cr@{\quad}|@{\quad}>{\color{red}}r]
                1   & -2    & 3     & -2 \\
                2   & -3    & 10    & -2 \\
                -1  & 3     & 1     & 4
            \end{bmatrix}
            \approx
            \begin{bmatrix}[*2cr@{\quad}|@{\quad}>{\color{red}}r]
                1   & -2    & 3     & -2 \\
                0   & 1     & 4     & 2 \\
                -1  & 3     & 1     & 4
            \end{bmatrix}
        \]

        The compact notation for this operation is $-2R_1 + R_2 \implies R_2$. Focusing now on row 3 since $(1)(1)+(-1)=0$, we add 1 times row 1 to row 3 and replace row 3 with the reuslt.

        \[
            \begin{bmatrix}[*2cr@{\quad}|@{\quad}>{\color{red}}r]
                1   & -2    & 3     & -2 \\
                0   & 1     & 4     & 2 \\
                -1  & 3     & 1     & 4
            \end{bmatrix}
            \approx
            \begin{bmatrix}[*2cr@{\quad}|@{\quad}>{\color{red}}r]
                1   & -2    & 3     & -2 \\
                0   & 1     & 4     & 2 \\
                0   & 1     & 4     & 2
            \end{bmatrix}
        \]

        The compact notation for this operation is $R_2 + R_3 \implies R_3$. Add a multiple of one row to another, focusing on column 2. With the first column complete, we move down to the second row and to the right
        to the second column. Since $(-1)(1)+(1)=0$, we add -1 times row 2 to row 3 and replace row 3 with the result.

        \[
            \begin{bmatrix}[*2cr@{\quad}|@{\quad}>{\color{red}}r]
                1   & -2    & 3     & -2 \\
                0   & 1     & 4     & 2 \\
                0   & 1     & 4     & 2
            \end{bmatrix}
            \approx
            \begin{bmatrix}[*2cr@{\quad}|@{\quad}>{\color{red}}r]
                1   & -2    & 3     & -2 \\
                0   & 1     & 4     & 2 \\
                0   & 0     & 0     & 0
            \end{bmatrix}
        \]

        The compact notation for this operation is $-R_2 + R_3 \implies R_3$. Now we extract the transformed system of equations from the matrix:

        \begin{align*}
            x_1 - 2x_2 + 3x_3   &= -2 \\
            x_2 + 4x_3          &= 2
        \end{align*}

        Back substitution can be used to show that the general solution is

        \[
            x_1 = 2-11s_1, x_2 = 2-4s_1, x_3 = s_1
        \]

        where $s_1\in \mathbb{R}$. \\

        The procedure used in the example above is known as \textbf{Gaussian elimination}. The resulting matrix is said to be in \textbf{echelon form} or \textbf{row echelon form}. A matrix is in echelon form if every
        leading term is in a column to the left of the leading term of the row below it and any zero rows are at the bottom of the matrix. \\

        \textbf{Gauss-Jordan elimination:} an algorithm used to solve systems of linear equations and to find inverses of matrices \\
        $\bullet$ purpose is to use elementary row operations to convert a matrix into a \textbf{reduced-row echelon form (row canonical form)}, which has the conditions: \\
        1. it is in ecehelon form \\
        2. all pivot positions contain a 1 \\
        3. the only nonzero trm in a pivot column is in the pivot position \\

        A given matrix can be equivalent to many different echelon form matrices, however, the same is not true of reduced row echelon form matrices. \\

        \textbf{Homogeneous linear equation:} has the form

        \[
            a_1 x_1 + a_2 x_2 + \dots + a_n x_n = 0
        \]

        \textbf{Homogenous linear systems:} systems made up of homogeneous linear systems \\
        $\bullet$ all homogeneous linear systems are consistent because there is always one easy solution, namely:

        \[
            x_1 = 0, x_2 = 0, \dots, x_n = 0
        \]

        which is called the \textbf{trivial solution}. If there are additional solutions they are called \textbf{nontrivial solutions}.

    \subsection{Applications of Linear Systems} % 1.3
    \subsection{Numerical Solutions}            % 1.4