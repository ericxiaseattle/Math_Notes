\section{Euclidean Space}

    \subsection{Vectors}

        \textbf{Vector:} an ordered list of real numbers $u_1, u_2, \dots, u_n$ expressed as

        \[
            \mathbf{u} =
            \begin{bmatrix}
                u_1 \\
                u_2 \\
                \vdots \\
                u_n
            \end{bmatrix}
        \]

        or as $\mathbf{u} = (u_1, u_2, \dots, u_n)$. The set of all vectors with $n$ entries is denoted by $\mathbf{R}^n$. Each of the entries $u_1, u_2, \dots, u_n$ is called a \textbf{component} of the vector. A vector
        expressed in the vertical form is also called a \textbf{column vector} and a vector expressed in horizontal form is also called a \textbf{row vector}. \\

        Let $\mathbf{u}$ and $\mathbf{v}$ be vectors in $\mathbf{R}^n$ given by

        \[
            \mathbf{u} =
            \begin{bmatrix}
                u_1 \\
                u_2 \\
                \vdots \\
                u_n
            \end{bmatrix}
            \text{ and }
            \mathbf{v} =
            \begin{bmatrix}
                v_1 \\
                v_2 \\
                \vdots \\
                v_n
            \end{bmatrix}
        \]

        Suppose that $c$ is a real number, called  \textbf{scalar}. \\
        \textbf{Equality}: $\mathbf{u} = \mathbf{v}$ if and only if $u_1 = v_1, u_2 = v_2, \dots, u_n = v_n$. \\
        \textbf{Addition:}

        \[
            \mathbf{u} + \mathbf{v} =
            \begin{bmatrix}
                u_1 \\
                u_2 \\
                \vdots \\
                u_n
            \end{bmatrix}
            +
            \begin{bmatrix}
                v_1 \\
                v_2 \\
                \vdots \\
                v_n
            \end{bmatrix}
            =
            \begin{bmatrix}
                u_1 + v_1 \\
                u_2 + v_2 \\
                \vdots \\
                u_n + v_n
            \end{bmatrix}
        \]

        \textbf{Scalar multiplication:}

        \[
            c\mathbf{u} = c
            \begin{bmatrix}
                u_1 \\
                u_2 \\
                \vdots \\
                u_n
            \end{bmatrix}
            =
            \begin{bmatrix}
                c\cdot u_1 \\
                c\cdot u_2 \\
                \vdots \\
                c\cdot u_n
            \end{bmatrix}
        \]

        The set of all vectors in $\mathbf{R}^n$, taken together with these definitions of addition and scalar multiplication, is called \textbf{Euclidean space}. \\

        \textbf{Algebraic properties of vectors:} let $a$ and $b$ be scalars, and \textbf{u, v}, and \textbf{w} be vectors in $\mathbf{R}^n$. Then
        (a) $\mathbf{u+v=v+u}$ \\
        (b) $a(\mathbf{u+v})=a\mathbf{u} + a\mathbf{v}$ \\
        (c) $(a+b)\mathbf{u}=a\mathbf{u} + b\mathbf{u}$ \\
        (d) $(\mathbf{u+v}) + \mathbf{w} = \mathbf{u} + (\mathbf{v+w})$ \\
        (e) $a(b\mathbf{u}) = (ab)\mathbf{u}$ \\
        (f) $\mathbf{u} + -\mathbf{(u)}=\mathbf{0}$ \\
        (g) $\mathbf{u+0=0+u=u}$ \\
        (h) $1\mathbf{u=u}$ \\

        The \textbf{zero vector} is given y

        \[
            \mathbf{0} =
            \begin{bmatrix}
                0 \\
                0 \\
                \vdots \\
                0
            \end{bmatrix}
        \]

        If $u_1, u_2, \dots, u_m$ are vectors and $c_1, c_2, \dots, c_m$ are scalars, then

        \[
            c_1 u_1 + c_2 u_2 + \dots + c_m u_m
        \]

        is a \textbf{linear combination} of $u_1, u_2, \dots, u_m$. Linear combinations provide an alternate way to express a system of linear equations. Consider the \textbf{vector equation} below.

        \[
            x_1
            \begin{bmatrix}
                5 \\
                -1 \\
                6
            \end{bmatrix}
            + x_2
            \begin{bmatrix}
                -3 \\
                2 \\
                1
            \end{bmatrix}
            +x_3
            \begin{bmatrix}
                7 \\
                5 \\
                0
            \end{bmatrix}
            =
            \begin{bmatrix}
                12 \\
                -4 \\
                11
            \end{bmatrix}
        \]

        and the corresponding system

        \begin{align*}
            5x_1 - 3x_2 + 7x_3  &= 12 \\
            -x_1 + 2x_2 + 5x_3  &= -4 \\
            6x_1 + x_2          &= 11
        \end{align*}

        \textbf{Vector form} of general solution example:

        \[
            \mathbf{x} =
            \begin{bmatrix}
                x_1 \\
                x_2 \\
                x_3
            \end{bmatrix}
            =
            \begin{bmatrix}
                2 \\
                2 \\
                0
            \end{bmatrix}
            + s_1
            \begin{bmatrix}
                -11 \\
                -4 \\
                1
            \end{bmatrix}
        \]

        where $s_1$ can be any real number.

    \subsection{Span}

        Let $\{\mathbf{u_1, u_2, \dots, u_m\}}$ be a set of vectors in $\mathbf{R}^n$. The \textbf{span} of this set is denoted span$\{\mathbf{u_1, u_2, \dots, u_m\}}$ and is defined as the set of all linear combinations

        \[
            x_1 \mathbf{u_1} + x_2 \mathbf{u_2} + \dots + x_m \mathbf{u_m}
        \]

        where $x_1, x_2, \dots, x_m$ can be any real numbers. If span$\{\mathbf{u_1, u_2, \dots, u_m}\} = \mathbf{R}^n$, then we say that the set $\{\mathbf{u_1, u_2, \dots, u_m}\}$ spans $\mathbf{R}^n$. \\

        \textit{\blue{Example:}} Show that $v_1 = \begin{bmatrix}-1 \\ 4 \\ 7\end{bmatrix}$ is in $S = \text{span}\{u_1, u_2\}$ and that $v_2 = \begin{bmatrix}8\\ 2 \\ 1\end{bmatrix}$ is not. \\

        To show that $v_1$ is in $S$, we need to find scalars $x_1$ and $x_2$ that satisfy the equation

        \[
            x_1 = \begin{bmatrix} 2 \\ 1 \\ 1 \end{bmatrix}
            + x_2 \begin{bmatrix} 1 \\ 2 \\ 3 \end{bmatrix}
            =     \begin{bmatrix}-1 \\ 4 \\ 7 \end{bmatrix}
        \]

        which is equivalent to the linear system

        \begin{align*}
            2x_1 + x_2  &= -1 \\
            x_1 + 2x_2  &= 4 \\
            x_1 + 3x_2  &= 7
        \end{align*}

        Transferring to an augmented matrix and performing row operations gives us

        \[
            \begin{bmatrix}[cc|c]
                2 & 1 & -1 \\
                1 & 2 & 4 \\
                1 & 3 & 7
            \end{bmatrix}
            \sim
            \begin{bmatrix}[cc|c]
                1 & 2 & 4 \\
                0 & 1 & 3 \\
                0 & 0 & 0
            \end{bmatrix}
        \]

        Extracting the echelon system and back substituting gives $x_1 = -2, x_2 = 3$. Thus $v_1 = -2u_1 + 3u_2$, so $v_1$ is in $S = \text{span}\{u_1, u_2\}$. Similarly, the approach for $v_2$ involves determining
        if there exists scalars $x_1$ and $x_2$ s.t.

        \[
            x_1 \begin{bmatrix} 2 \\ 1 \\ 1 \end{bmatrix} + x_2 \begin{bmatrix} 1 \\ 2 \\ 3 \end{bmatrix} = \begin{bmatrix} 8 \\ 2 \\ 1 \end{bmatrix}
        \]

        which is equivalent to the linear system

        \begin{align*}
            2x_1 + x_2  &= 8 \\
            x_1 + 2x_2  &= 2 \\
            x_1 + 3x_2  &= 1
        \end{align*}

        The augmented matrix and corresponding echelon form are

        \[
            \begin{bmatrix}[cc|c]
                2 & 1 & 8 \\
                1 & 2 & 2 \\
                1 & 3 & 1
            \end{bmatrix}
            \sim
            \begin{bmatrix}[cc|c]
                1 & 2 & 2 \\
                0 & 1 & -1 \\
                0 & 0 & 1
            \end{bmatrix}
        \]

        The third row of the echelon matrix corresponds to the equation $0 = 1$. Thus the system has no solutions and $v_2$ is not in $S$.

        \begin{theorem}{Theorem}
            Let $u_1, u_2, \dots, u_m$ and $v$ be vectors in $\mathbf{R}^n$. Then $v$ is an element of span$\{u_1, u_2, \dots, u_m\}$ iff the linear system with augmented matrix

            \[
                \begin{bmatrix}[cccc|c]
                    u_1 & u_2 & \dots & u_m & v
                \end{bmatrix}
            \]
        \end{theorem}

        \begin{theorem}{Theorem}
            Let $u_1, u_2, \dots, u_m$ and $u$ be vectors in $\mathbf{R}^n$. If $u$ is in span$\{u_1, u_2, \dots, u_m\}$, then

            \[
                \text{span}\{u, u_1, u_2, \dots, u_m\} = \text{span}\{u_1, u_2, \dots, u_m\}
            \]
        \end{theorem}

        \begin{theorem}{Theorem}
            Suppose that $u_1, \dots, u_m$ are in $\mathbf{R}^n$, and let

            \[
                A =
                \begin{bmatrix}[cccc]
                    u_1 \\
                    u_2 \\
                    \dots \\
                    u_m
                \end{bmatrix}
                \sim B
            \]

            where $B$ is in echelon form. Then span $\{u_1, \dots, u_m\} = \mathbf{R}^n$ exactly when $B$ has a pivot position in every row.
        \end{theorem}

        \begin{theorem}{Theorem}
            Let $\{u_1, u_2, \dots, u_m\}$ be a set of vectors in $\mathbf{R}^n$. If $m < n$, then this set does not span $\mathbf{R}^n$. If $m \geq n$, then the set might span $\mathbf{R}^n$ or it might not. In this
            case, we cannot say more without additional information about the vectors.
        \end{theorem}

        Let $a_1, a_2, \dots, a_m$ be vectors in $\mathbf{R}^n$. If

        \[
            A =
            \begin{bmatrix}
                a_1 a_2 \dots a_m
            \end{bmatrix}
            \text{ and }
            x =
            \begin{bmatrix}
                x_1 \\
                x_2 \\
                \vdots \\
                x_m
            \end{bmatrix}
        \]

        then $Ax = x_1 a_1 + x_2 a_1 + \dots + x_m a_m$. Note that the product $Ax$ is only defined when the number of columns of $A$ equals the number of entries of $x$.

        \begin{theorem}{Theorem}
            Let $a_1, a_2, \dots, a_m$ and $b$ be vectors in $\mathbf{R}^n$. Then the following statements are equivalent. That is, if one is true, then so are the others, and if one is false, then so are the others. \\
            (a) $b$ is in span$\{a_1, a_2, \dots, a_m\}$. \\
            (b) The vector equation $x_1 a_1 + x_2 a_2 + \dots + x_m a_m = b$ has at least one solution \\
            (c) The linear system corresponding to $\begin{bmatrix}[cccc|c]a_1 & a_2 & \dots & a_m & b\end{bmatrix}$ has at least one solution \\
            (d) The equation $Ax = b$ with $A$ and $x$ given as in the definition above, has at least one solution.
        \end{theorem}

    \subsection{Linear Independence}