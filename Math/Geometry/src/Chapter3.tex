\section{Polygons}

    \subsection{Polygon Basics}
        \textbf{Polygons} are closed 2D figures composed of straight lines. By this definition,
        any figure that has a curve or is open is not a polygon. A \textbf{regular polygon} has
        all angles and sides equal. An \textbf{irregular polgyon} is any polygon that is not
        regular. A \textbf{convex polygon} has no inwardly-pointing (greater than $180^\circ$)
        angles. A \textbf{concave polygon} is any polygon with at least one internal angle
        greater than $180^\circ$. A \textbf{simple polygon} does not self-intersect, whereas a
        \textbf{complex polygon} self-intersects. \\

        \noindent \color{purple} \textbf{Common Regular Polygons:} \color{black} \\

        \begin{center}
            \begin{tabular}{|c|c|}
                \hline
                \textbf{Name} & \textbf{Sides} \\
                \hline
                Triangle (Trigon) & 3 \\
                \hline
                Quadrilateral (Tetragon) & 4 \\
                \hline
                Pentagon & 5 \\
                \hline
                Hexagon & 6 \\
                \hline
                Heptagon (Septagon) & 7 \\
                \hline
                Octagon & 8 \\
                \hline
                Nonagon (Enneagon) & 9 \\
                \hline
                Decagon & 10 \\
                \hline
                n-gon & n \\
                \hline
            \end{tabular}
        \end{center}

        \noindent \color{purple} \textbf{Naming Prefixes and Suffixes: } \color{black} \\

        \begin{center}
            \begin{tabular}{|c|c|}
                \hline
                \textbf{Sides} & \textbf{Prefixes/Suffixes} \\
                \hline
                20 & Icosi- \\
                \hline
                30 & Triaconta- \\
                \hline
                40 & Tetraconta- \\
                \hline
                50 & Pentaconta- \\
                \hline
                60 & Hexaconta- \\
                \hline
                70 & Heptaconta- \\
                \hline
                80 & Octaconta- \\
                \hline
                90 & Enneaconta-/Nonaconta- \\
                \hline
                100 & Hecta- \\
                \hline
                +1 & -henagon \\
                \hline
                +2 & -digon \\
                \hline
                +3 & -trigon \\
                \hline
                +4 & -tetragon \\
                \hline
                +5 & -pentagon \\
                \hline
                +6 & -hexagon \\
                \hline
                +7 & -heptagon \\
                \hline
                +8 & -octagon \\
                \hline
                +9 & -nonagon/-enneagon \\
                \hline
            \end{tabular}
        \end{center}

        \noindent Given a convex polygon with $n$ sides, we can find the sum of the interior
        angles, $S$, such that \\

        \begin{equation*}
            S = 180(n-2)
        \end{equation*}

        \noindent The sum of all the exterior angles for a convex polygon always add to
        $360^\circ$.



    \subsection{Quadrilaterals}
        For quadrilaterals, the sum of all interior angles always add to $360^\circ$.

        \begin{figure} [hbt!]
            \centering
            \caption*{\color{purple}Types of Quadrilaterals:\color{black}}
            \includegraphics[scale = 0.8] {Resources/Unit3Polygons/quad.PNG}
        \end{figure}

        \noindent The diagonals of a rhombus bisect each other. Therefore, if a parallelogram
        has diagonals that bisect each other, then the parallelogram is a rhombus. For kites,
        the angles where the two pairs meet are equal, the diagonals meet at right angles, and
        one of the diagonals bisects the other. \\

        \noindent The square is the only regular quadrilaterals and all other quadrilaterals are
        irregular. \\

        \pagebreak
        \noindent \color{purple} \textbf{Quadrilateral Area Formulas:} \color{black} \\

        \begin{center}
            \begin{tabular} {|c|c|}
                \hline
                $A=s^2$, $s$ = side & Square \\
                \hline
                $A=bh$, $b$ = base, $h$ = height & Rectangle \\
                \hline
                $A=\frac{pq}{2}$, $p$ and $q$ are diagonals & Rhombus \\
                \hline
                $A=bh$, $b$ = base, $h$ = height & Parallelogram \\
                \hline
                $A=h(\frac{b_1+b_2}{2})$, $b_1$ and $b_2$ are bases, $h$ = height & Trapezoid \\
                \hline
                $A=\frac{pq}{2}$, $p$ and $q$ are diagonals & Kite \\
                \hline
            \end{tabular}
        \end{center}