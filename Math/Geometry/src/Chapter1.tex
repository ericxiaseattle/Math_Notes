\section{Geometry Basics}

    \subsection{Common Geometrical Notations}
        \begin{center}
            \begin{tabular}{|c|c|}
                \hline
                \textbf{Notation} & \textbf{Definition} \\
                \hline
                $\angle$ & Angle \\
                \hline
                $\triangle$ & Triangle \\
                \hline
                $|F|$ & Area of Figure F \\
                \hline
                $\parallel$ & Parallel \\
                \hline
                $\perp$ & Perpendicular \\
                \hline
                $\sim$ & Similar to \\
                \hline
                $\equiv$ & Congruent to \\
                \hline
                $\frown \atop AB$ & Arc AB \\
                \hline
                $\overleftrightarrow{AB}$ & Line AB \\
                \hline
                $[AB],\overline{AB}$ & Line Segment AB \\
                \hline
                $[AB)$ & Ray AB \\
                \hline
                $\overrightarrow{AB}$ & Vector AB \\
                \hline
                $^{\circ}$ & Degrees \\
                \hline
                rad & Radians \\
                \hline
                $\pi$ & Pi Constant \\
                \hline
            \end{tabular}
        \end{center}



    \subsection{Fundamentals of Geometry}
        A \textbf{point} has no dimensions, only position. A point is depicted by a dot. \\

        \begin{figure} [hbt!]
            \centering
            \includegraphics[scale = 0.5] {Resources/Unit1Basics/point.PNG}
            \caption*{Point P}
        \end{figure}

        \noindent A \textbf{line} is a straight one-dimensional figure with no thickness and
        extending to infinity in both directions. \textbf{Collinear points} are points existing
        on the same line. A line can be defined by two points. \textbf{Intersecting lines} are
        two lines that meet at a point. A \textbf{line segment} is a part of a line with defined
        endpoints. A line that has one defined endpoint and extends infinitely in only one
        direction is called a ray. \\

        \begin{figure} [hbt!]
            \centering
            \includegraphics[scale = 0.5] {Resources/Unit1Basics/line.PNG}
            \caption*{$\overleftrightarrow{AB}$}
        \end{figure}

        \noindent A \textbf{plane} extends infinitely in two dimensions and it has no thickness.
        A plane is defined by three non-collinear points. \\

        \begin{figure} [hbt!]
            \centering
            \includegraphics[scale = 0.3] {Resources/Unit1Basics/plane.PNG}
            \caption*{Plane ABC}
        \end{figure}

        \noindent A \textbf{space} extends infinitely in three dimensions and is a set of all
        points in three dimensions. A space can be thought of the inside of an infinitely large
        box. \\

        \begin{figure} [hbt!]
            \centering
            \includegraphics[scale = 0.3] {Resources/Unit1Basics/space.PNG}
            \caption*{A Space}
        \end{figure}

        \noindent Two figures are \textbf{congruent} if they have the same shape and size,
        whereas two figures are \textbf{similar} if they have the same shape (not necessarily
        same size). Angle and line congruency are depicted by a tick mark on the congruent
        figures. \\

        \begin{figure} [hbt!]
            \centering
            \includegraphics[scale = 0.75] {Resources/Unit1Basics/congruent.PNG}
            \caption*{$\overline{AB}\equiv\overline{CD}$}
        \end{figure}

        \noindent An \textbf{angle} is formed between two rays that share an endpoint, called
        the \textbf{vertex}. An angle is also a fraction of a circle, where the whole circle is
        $360^\circ$ or $2\pi$ radians. \textbf{Types of Angles} (Where A is an Angle): \\

        \begin{center}
            \begin{tabular}{|c|c|c|}
                \hline
                $A<90^\circ$ & $A<\frac{\pi}{2}$ & \textbf{Acute} Angle \\
                \hline
                $A=90^\circ$ & $A=\frac{\pi}{2}$ & \textbf{Right} Angle \\
                \hline
                $90^\circ < A < 180^\circ$ & $\frac{\pi}{2} < A < \pi$ &\textbf{Obtuse} Angle \\
                \hline
                $A=180^\circ$ & $\pi$ & \textbf{Straight} Angle \\
                \hline
            \end{tabular}
        \end{center}

        \noindent Two angles are \textbf{complementary} if their measures add to $90^\circ$.
        Two angles are \textbf{supplementary} if their measures add to $180^\circ$.

        \noindent Let the endpoints of a line segment be $A(x_1, y_1)$ and $B(x_2, y_2)$. Then, \\

        \noindent \color{purple} \textbf{The Distance Formula:} \color{black}

        \begin{equation*}
            d=\sqrt{(x_2-x_1)^2+(y_2-y_1)^2}
        \end{equation*}

        \noindent \color{purple} \textbf{The Midpoint Formula:} \color{black}

        \begin{equation*}
            \text{mid}=\left(\frac{x_1+x_2}{2},\frac{y_1+y_2}{2}\right)
        \end{equation*}



    \subsection{Perpendicular and Parallel}
        Lines are \textbf{perpendicular} if they are positioned at right angles to each other.
        Perpendicular lines are depicted by a box. \\

        \begin{figure} [hbt!]
            \centering
            \includegraphics[scale = 0.6] {Resources/Unit1Basics/perpendicular.PNG}
        \end{figure}

        \noindent Lines are \textbf{parallel} if they are \textbf{equidistant}, or always the
        same distance apart, and will never meet. \\

        \begin{figure} [hbt!]
            \centering
            \includegraphics[scale = 0.4] {Resources/Unit1Basics/parallel.PNG}
            \caption*{Two parallel lines}
        \end{figure}

        \noindent \textbf{Parallel curves} are curves that are equidistant. For example, see the
        below curves \\

        \begin{figure} [hbt!]
            \centering
            \includegraphics[scale = 0.4] {Resources/Unit1Basics/parallel_curves.PNG}
        \end{figure}

        \noindent A \textbf{transversal line} is a line that crosses two other lines. When the
        two lines being crossed are parallel, \textbf{corresponding angles} are made.

        \begin{figure} [hbt!]
            \centering
            \includegraphics[scale = 0.6] {Resources/Unit1Basics/corresponding.PNG}
        \end{figure}

        \begin{center}
            \begin{tabular} {|c|c|}
                \hline
                $\angle A$, $\angle F$, $\angle G$, $\angle D$ & \textbf{Exterior} Angles \\
                \hline
                $\angle B$, $\angle E$, $\angle H$, $\angle C$ & \textbf{Interior} Angles \\
                \hline
                $\angle B$ and $\angle E$, $\angle H$ and $\angle C$ &\textbf{Consecutive Interior} Angles \\
                \hline
                $\angle A$ and $\angle G$, $\angle F$ and $\angle D$ & \textbf{Alternate Exterior} Angles \\
                \hline
                $\angle E$ and $\angle C$, $\angle H$ and $\angle B$ & \textbf{Alternate Interior} Angles \\
                \hline
                $\angle A$ and $\angle E$, $\angle C$ and $\angle G$ & \textbf{Corresponding} Angles \\
                $\angle D$ and $\angle H$, $\angle F$ and $\angle B$ & \\
                \hline
            \end{tabular}
        \end{center}

        \noindent The \textbf{perpendicular bisector} of a line is a line segment perpendicular
        to and passing through the midpoint of said line. The \textbf{angle bisector} is a line
        that splits an angle into two equal angles.