\section{Complex Numbers}

    \subsection{Introduction to Complex Numbers}
        A \textbf{complex number} is any number that can be written in the form $a+bi$, where
        $a$ and $b$ are real numbers and $i$ is the imaginary unit defined by $i=\sqrt{-1}$.
        Complex numbers appear often in trigonometry and polar coordinates, making them
        particularly applicable in physics and engineering. \\

        \noindent Complex numbers can be geometrically represented by graphing them on the
        \textbf{complex plane}, where $a+bi$ is graphed just as the ordered pair $(a,b)$ would
        be graphed on Cartesian coordinates. The real axis corresponds to the $x$-axis and the
        imaginary axis corresponds to the $y$-axis.\\

        \begin{figure} [hbt!]
            \centering
            \includegraphics[scale=0.75]{Resources/Unit4Complex/complex.PNG}
            \caption*{Complex Numbers Graphed on the Complex Plane}
        \end{figure}

        \noindent \textbf{Properties of $i$:} \\
        $i=\sqrt{-1}$ \\
        $i^2=-1$ \\
        $i^3=i\cdot i^2=i(-1)=-i$ \\
        $i^4=i^2\cdot i^2=(-1)(-1)=1$ \\
        To simplify larger powers of $i$, take the last two digits of the power and divide it by
        4. Find the remainder, $k$. Then the value is $i^k$. \\
        $i^k+i^{k+1}+i^{k+2}+i^{k+3}=0$ where $k$ is an integer. (The sum of four consecutive
        powers of $i$ equals 0)


    \subsection{Complex Arithmetic and Conjugates}
        \textbf{Addition:} \\
        Given complex numbers $a+bi$ and $c+di$, their sum is \\
        \begin{equation*}
            (a+c) + (b+d)i
        \end{equation*}

        \noindent \textbf{Multiplication:} \\
        Given complex numbers $a+bi$ and $c+di$, their product is

        \begin{align*}
            (a+bi) \times (c+di) &= a(c+di)+bi(c+di) \\
            &= (ac) + (ad)i + (bc)i + (bd)i^2 \\
            &= (ac) + (ad+bc)i + (bd)(-1) \\
            &= (ac-bd) + (ad+bc)i
        \end{align*}

        \noindent \textbf{Division:} \\
        Given complex numbers $a+bi$ and $c+di$, the division of these two numbers is done by
        rationalizing the complex number or multiplying and dividing by the conjugate of the
        denominator. \\

        \noindent The \textbf{complex conjugate} of a complex number $a+bi$ is $a-bi$.
        Complex conjugates are highly useful for rationalizing denominators containing complex
        numbers. \\

        \noindent \textit{Example 1: Rationalize the denominator and write in standard form for
        $\frac{3+2i}{5-2i}$} \\

        \begin{align*}
            \frac{3+2i}{5-2i} &= \frac{(3+2i)(5+2i)}{(5-2i)(5+2i)} \\
            &= \frac{11+16i}{29} \\
            &= \frac{11}{29} + \frac{11}{29}i
        \end{align*}

        \noindent \textbf{Complex Conjugate Root Theorem:} If $a+bi$ is a root of a polynomial
        with rational coefficients, then $a-bi$ is also a root of that polynomial. \\
        \textit{Example 2:} The quadratic $x^2+bx+c$ has $1+i$ as a root, where $b$ and $c$ are
        integers. What is $b+c$? \\
        We are given that $b$ and $c$ are integers, hence the other root must be the conjugate
        of $1+i=1-i$. Writing the polynomial in factored form gives \\

        \begin{align*}
            x^2+bx+c &= (x-1-i)(x-1+i) \\
            &= x^2-2x+2
        \end{align*}

        \noindent $\implies b=-2,c=2$ \\
        $\therefore b+c=0$



    \subsection{Gaussian Integers}
        A \textbf{Gaussian Integer} is a complex number $a+bi$, where both $a$ and $b$ are integers.
        Note that Gaussian Integers are not actually integers unless the imaginary component equals 0. \\

        \noindent \textbf{Examples of Gaussian Integers:} \\
        $3+2i$ \\
        $7-8i$ \\
        $14$ \\
        $-92i$ \\
        \noindent \textbf{Examples of Non-Gaussian Integers:} \\
        $\frac{1}{2}+\frac{\sqrt{3}}{2}i$ \\
        $7-\frac{i}{3}$



    \subsection{Complex Modulus and Argument}

        The absolute value of a real number is defined as the positive distance from 0 to that
        number. The absolute value of a complex number is defined in the same way, except its
        distance oftentimes referred to as \textbf{modulus}, is measured on the complex plane.
        Since the segment between 0 and the complex number is a hypotenuse of a right triangle \\

        \noindent \textbf{Modulus of a Complex Number $a+bi$}: \\

        \begin{equation*}
            |a+bi| = \sqrt{a^2+b^2}
        \end{equation*}

        \begin{figure} [hbt!]
            \centering
            \includegraphics[scale=0.75]{Resources/Unit4Complex/complex.PNG}
        \end{figure}

        \noindent The angle that the positive real axis makes with the ray connecting 0 to a
        complex number is called the \textbf{argument} of a complex number. Using triangle
        relationships, we can determine \\

        \noindent \textbf{Argument of a Complex Number $a+bi$}: \\
        \begin{equation*}
            \tan\theta=\frac{b}{a}
        \end{equation*}

        \noindent When attempting to determine $\theta$, one should take into account which
        quadrant the complex number is located in. \\
        \textit{Example: Determine the argument of $-3+4i$} \\
        We have \\

        \begin{equation*}
            \tan\theta=-\frac{4}{3}
        \end{equation*}

        \noindent Recall that the inverse tangent function has a range of $(-\frac{\pi}{2},\frac{\pi}{2})$.
        Taking the inverse tangent will give the angle that is in $Q4.$ The complex number,
        however, is located in $Q2$, so we add $\pi$ to give the correct argument. \\

        \begin{equation*}
            \therefore \theta = \arctan{-\frac{4}{3}}+\pi
        \end{equation*}



    \subsection{Complex Roots}
        By the \textbf{Fundamental Theorem of Algebra}, every polynomial of degree $n$ has exactly
        $n$ roots, counting for multiplicity. Occasionally, these roots are complex numbers.
        For example, $x^2+1=0\implies x^2=-1\implies x=\pm i$. \\

        \noindent \textit{Example 1: Find the roots of $2x^2+1=0$} \\
        \begin{align*}
            2x^2 + 1 &= 0 \\
            x^2 &= -\frac{1}{2} \\
            x &= \pm \sqrt{-\frac{1}{2}} \\
            &= \pm \sqrt{-1} \sqrt{\frac{1}{2}} \\
            &= \pm i\sqrt{\frac{1}{2}} \\
            &= \pm \frac{i}{\sqrt{2}}
        \end{align*}

        \noindent \textit{Example 2: Factor $x^2+6x+10$} \\
        Computing the discriminant $D$, we get \\

        \begin{equation*}
            D = b^2-4ac = 6^2-4(1)(10) = -4
        \end{equation*}

        \noindent Since $D<0$, we conclude that the quadratic has a pair of complex roots with
        imaginary components. To find the roots, we use the quadratic formula. \\

        \begin{align*}
            x &= \frac{-b\pm\sqrt{D}}{2a} \\
            &= \frac{-6\pm\sqrt{-4}}{2(1)} \\
            &= \frac{-6\pm 2\sqrt{-1}}{2} \\
            &= -3 \pm i
        \end{align*}

        \noindent Factoring the quadratic, we get \\

        \begin{align*}
            x^2+6x+10 &= (x-(-3+i))(x-(-3-i)) \\
            &= (x+3-i)(x+3+i)
        \end{align*}



    \subsection{Euler's Formula}
        \textbf{Euler's Formula} allows us to express a complex number in exponential form. \\
        \noindent Given a complex number $z$ with modulus $r$ and argument $\theta$, \\
        \begin{equation*}
            z=re^{i\theta}=r(\cos\theta+i\sin\theta)
        \end{equation*}
        \noindent This is usually wrote in simpler terms as given below. \\
        \begin{equation*}
            e^{i\theta}=\cos\theta+i\sin\theta
        \end{equation*}
        \noindent Oftentimes a shorthand notation is used where the $cis$ function refers to
        $cis=\cos\theta+i\sin\theta$.

        \noindent \textit{Example 1: Express $3e^{\frac{\pi i}{2}}$ in standard form} \\
        \begin{align*}
            3e^{\frac{\pi i}{2}} &= 3(\cos{\frac{\pi}{2}}+i\sin{\frac{\pi}{2}}) \\
            &= 3i
        \end{align*}

        \noindent \textit{Example 2: Express $\frac{1}{2}-i\frac{\sqrt{3}}{2}$ in exponential form} \\
        The modulus is given by \\
        \begin{align*}
            \left| \frac{1}{2}-i\frac{\sqrt{3}}{2}\right|
            &= (\frac{1}{2})^2+(\frac{\sqrt{3}}{2})^2 \\
            &= 1
        \end{align*}
        \noindent The argument is then \\
        \begin{equation*}
            \tan\theta=-\sqrt{3} \\
        \end{equation*}
        The complex number is in $Q4$ so \\
        \begin{equation*}
            \theta &= -\frac{\pi}{3}
        \end{equation*}

        \noindent With the modulus and argument known, we can directly stick them into
        Euler's Formula, giving us \\
        \begin{equation*}
            \frac{1}{2}-i\frac{\sqrt{3}}{2} = e^{\frac{-\pi i}{3}}
        \end{equation*}


        \noindent \textbf{Euler's Identity} is a special case of Euler's Formula in which
        $r=1$ and $\theta=\pi$. It is incredibly notable because it combines several important
        mathematical constants ($0,1,\pi,e,$ and $i$) into one equation. It is given by \\

        \begin{equation*}
            e^{\pi i}+1=0
        \end{equation*}

        \noindent \textbf{Finding Trig Identities with Euler's Formula}: \\
        For example, let's find the $\cos$ and $\sin$ functions' sum formulas. \\
        \begin{align*}
            e^{i\theta} &= \cos\theta + i\sin\theta \\
            e^{i(\alpha+\beta)} &= \cos{(\alpha+\beta)}+i\sin{(\alpha+\beta)} \\
        \end{align*}
        Using exponent laws, we can rewrite the left side. \\
        \begin{align*}
            e^{i\alpha}e^{\i\beta} &= (\cos\alpha+i\sin\alpha)(\cos\beta+i\sin\beta) \\
            &= \cos\alpha\cos\beta-\sin\alpha\sin\beta+i(\cos\alpha\sin\beta+\sin\alpha\cos\beta)
        \end{align*}
        Setting equal the real and imaginary parts of the second and third equations, we get \\
        \begin{equation*}
            \cos(\alpha+\beta) &= \cos\alpha\cos\beta-\sin\alpha\sin\beta
        \end{equation*}
        \noindent and \\
        \begin{equation*}
            \sin(\alpha+\beta) &= \cos\alpha\sin\beta + \sin\alpha\cos\beta
        \end{equation*}

        \noindent Euler's Formula is also useful for writing trigonometric functions in terms
        of exponentials. Knowing that $\cos{(-\theta)}=\cos{\theta}$ and $\sin{(-\theta)}=-\sin{(\theta)}$,
        \begin{equation*}
            e^{i\theta}=\cos\theta+i\sin\theta
        \end{equation*}
        and
        \begin{equation*}
            e^{-i\theta}=\cos\theta-i\sin\theta
        \end{equation*}
        $\implies$
        \begin{equation*}
            \cos\theta=\frac{e^{i\theta}+e^{-i\theta}}{2}
        \end{equation*}
        and
        \begin{equation*}
            \sin\theta=\frac{e^{i\theta}-e^{-i\theta}}{2i}
        \end{equation*}

        \noindent We can apply these identities to Calculus, for example, we can rewrite the
        following integrand as follows. \\
        \begin{align*}
            \sin^2{x} &= \left(\frac{e^{ix}-e^{-ix}}{2i}\right)^2 \\
            &= -\frac{1}{4}(e^{i2x}+e^{-i2x}-2e^0) \\
            &= -\frac{1}{4}(2\cos{(2x)}-2)
        \end{align*}



    \subsection{De Moivre's Theorem}

        Following Euler's Formula directly, De Moivre's Theorem allows us to raise a complex number
        to any real number power. \\

        \noindent Given a complex number $z=re^{i\theta}$ and a real number $n$, \\
        \begin{equation*}
            z^n = (re^{i\theta})^n=r^n[\cos{(n\theta)}+i\sin{(n\theta)}]
        \end{equation*}

        \noindent \textit{Example: Compute $(\sqrt{2}+i\sqrt{2})^4$} in standard form \\
        The modulus of the base complex number is \\
        \begin{equation*}
            |\sqrt{2}+i\sqrt{2}| = \sqrt{(\sqrt(2))^2+(\sqrt{2})^2} = 2
        \end{equation*}



    \subsection{Roots of Unity}

        The \textbf{$n^{th}$ roots of unity} are the complex solutions to an equation of the form
        $x^n=1$, where $n$ is a positive integer. Such equations can be solved by applying
        Euler's Formula and De Moivre'S Theorem. \\

        \noindent \textit{Example 1: Find the complex solutions to the equation $x^6=1$} \\
        We have \\
        \begin{align*}
            1 &= e^{2k\pi i},k\in\mathbb{Z} \\
            x^6 &= e^{2k\pi i} \\
            x &= e^{\frac{k\pi i}{3}}
        \end{align*}
        \noindent Below are 6 values of $k$ that give distinct complex numbers.
        All other values of $k$ give arguments that are co-terminal with these solutions,
        called the \textit{$6^{th}$ roots of unity}. \\

        \begin{equation*}
            k=0: x=e^0 = 1
        \end{equation*}
        \begin{equation*}
            k=1: x=e^{\frac{\pi i}{3}} = \frac{1}{2}+i\frac{\sqrt{3}}{2}
        \end{equation*}
        \begin{equation*}
            k=2: x=e^{\frac{2\pi i}{3}} = -\frac{1}{2}+i\frac{\sqrt{3}}{2}
        \end{equation*}
        \begin{equation*}
            k=3: x=e^{\pi} = -1
        \end{equation*}
        \begin{equation*}
            k=4: x=e^{\frac{4\pi i}{3}} = -\frac{1}{2}-i\frac{\sqrt{3}}{2}
        \end{equation*}
        \begin{equation*}
            k=5: x=e^{\frac{5\pi i}{3}} = \frac{1}{2} - i\frac{\sqrt{3}}{2}
        \end{equation*}



    \subsection{Complex Numbers in Geometry}
        Because of the circular relationships associated with complex numbers, they are useful
        for many geometrical problems. For example, the rotation of a point or rigid figure
        can be performed with complex numbers much more simply than it can be done with trigonometry. \\

        \noindent To rotate a point $\theta$ radians anticlockwise about the origin, \\
        1. Convert the ordered pair to the corresponding complex number \\
        2. Multiply the complex number by $e^{i\theta}$ \\
        3. Convert the result to the corresponding ordered pair \\

        \noindent \textit{Example: Find the image of rotation when the point (2,5) is rotated
        $30^\circ$ anticlockwise about the origin} \\
        \begin{equation*}
            (2,5)\implies 2+5i
        \end{equation*}
        \noindent The angle of rotation in radians is $\frac{\pi}{6}$, giving the complex number \\
        \begin{equation*}
            e^{\frac{\pi i}{6}}=\frac{\sqrt{3}}{2}+\frac{i}{2}
        \end{equation*}
        \noindent Multiplying these complex numbers together, we obtain the image of rotation \\
        \begin{equation*}
            \left ( 2+5i\right )\left (\frac{\sqrt{3}}{2}+\frac{i}{2}\right)
            =
            \sqrt{3}-\frac{5}{2}+i \left (1+\frac{5\sqrt{3}}{2} \right )
        \end{equation*}

        \noindent The corresponding ordered pair is then
        $\left(\sqrt{3}-\frac{5}{2},1+\frac{5\sqrt{3}}{2}\right)$