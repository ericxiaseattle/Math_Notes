\section{Introductory Set Theory}

    \subsection{Introduction to Sets}
        Forget what you think math is. Forget even what numbers are. Instead of numbers, think
        about math in terms of "things". What is a set? Simply put, a \textbf{set} is a
        \textit{collection} of "things" with a shared defining characteristic. This is
        exactly like the array data structure in Computer Science. For example, a set of
        clothing may include shirts, pants, hats, jackets, socks, etc. and a set of colors
        may include red, blue, green, purple, brown, etc. \\

        \noindent In their purest form, sets are pretty useless. However, they are the
        foundation of mathematics and can be seen in the many branches of mathematics,
        including graph theory, algebra, real analysis, complex analysis, number theory,
        and so on. Set theory is important because it is all about using logic to connect
        numbers, models, axioms, and more. Without set theory, mathematics would not have
        meaning, and may as well be a bunch of scribbles.\\

        \noindent The \textbf{universal set} is one that contains everything relevant to the
        focus. For example, in number theory the universal set is all of the integers.
        In Calculus the universal set is generally the real numbers and in complex analysis
        the universal set is the complex numbers.\\

        \noindent Sets are composed of many \textbf{elements}, separated by commas and enclosed
        by curly braces. For example, a set of clothes may be defined as \{shirts, pants, socks,
        jackets,\dots\}. We add the \textbf{ellipsis} (dots) after all the definite elements to
        indicate that the set can keep going on forever, as there are many more types of clothing
        that we don't need to bother defining.



    \subsection{Order and Equality}
        For sets, the arrangement of the elements does not matter. For example, the set
        $\{1,2,3,4\}$ is the same set as $\{4,1,3,2\}$. In set theory, \textbf{order} is not the
        arrangement of the elements, rather it is \textit{the size of the set}. The preferred
        term for this is \textbf{cardinality}, the \textit{number of elements a set has}. \\

        \noindent In general, mathematicians use capital letters to represent sets and lowercase
        letters to represent elements in that set. For example, in the set $A=\{a,\dots\}$, $a$
        is an element of the set $A$. If an element $a$ is in the set $A$, then we can write this
        with the symbol $\in$. If $a$ is not in $A$ then we denote this with $\notin$. For
        example, if $A=\{1,2,3\}$ then it is valid to say that $1\in A$ and $5\notin A$. \\

        \noindent Two sets are equal if they have the same elements. We use the equals sign to
        show equality. For example, if $A$ is the set defined by the first four positive whole
        numbers and $B=\{4,2,1,3\}$, then $A=B$. Remember that the arrangement of the elements
        in a set do not matter.



    \pagebreak
    \subsection{Notation}
        Some of these symbols and their corresponding topics will be covered in later sections.
        It's just convenient to put everything on set notation in one section.

        \begin{center}
            \begin{tabular}{|c|c|c|}
                \hline
                \textbf{Symbol} & \textbf{Meaning} & \textbf{Example} \\\hline
                $\{ \}$ & \textbf{Set} & $\{1,2,3,4\}$ \\ \hline
                $A \cup B$ & \textbf{Union}: in A or B or both & $C\cup D=\{1,2,3,4,5\}$ \\ \hline
                $A\cap B$ & \textbf{Intersection}: in both A and B & $C\cap D=\{3,4\}$ \\ \hline
                $A\subseteq B$ & \textbf{Subset}: A has some or all elements of B & $\{3,4,5\}\subseteq D$ \\ \hline
                $A\subset B$ & \textbf{Proper Subset}: A has some elements of B & $\{3,5\}\subset D$ \\ \hline
                $A\not\subset B$ & \textbf{Not Subset}: A is not a subset of B & $\{ 1,6 \} \not\subset C$ \\ \hline
                $A\supseteq B$ & \textbf{Superset}: A has the same elements of B or more & $\{1,2,3\}\supseteq \{1,2,3\}$ \\ \hline
                $A \supset B$ & \textbf{Proper Superset}: A has all of B's elements and more & $\{1,2,3,4\} \supset \{1,2,3\}$ \\ \hline
                $A \not\supset B$ & \textbf{Not Superset}: A is not a superset of B & $\{1,2,6\} \not\supset \{1,9\}$ \\ \hline
                %%%
                $A^c$ & \textbf{Complement}: elements not in A & $D^c=\{1,2,6,7\}$  \\
                & & when $\mathbb{U}=\{1,2,3,4,5,6,7\}$ \\ \hline
                $A-B$ & \textbf{Difference}: in A but not in B & $\{1,2,3,4\}-\{3,4\}=\{1,2\}$ \\ \hline
                $a \in A$ & \textbf{Element}: $a$ is in $A$ & $3 \in \{1,2,3,4\}$ \\ \hline
                $b \notin A$ & \textbf{Not Element}: $b$ is not in $A$ & $6 \notin \{1,2,3,4\}$ \\ \hline
                $\emptyset$ & \textbf{Empty Set}: $\{\}$ & $\{1,2\} \cap\{3,4\}=\emptyset$ \\ \hline
                $\mathbb{U}$ & \textbf{Universal Set}: set of all possible values in the area of interest &   \\ \hline
                $P(A)$ & \textbf{Power Set}: all subsets of A & $P(\{1,2\})=\{\{\},\{1\},\{2\},\{1,2\}\}$ \\ \hline
                %%%
                $A=B$ & \textbf{Equality}: both A and B have the same elements & $\{3,4,5\}=\{5,4,3\}$ \\ \hline
                $A\times B$ & \textbf{Cartesian Product}: set of ordered pairs from A and B & $\{1,2\}\times\{3,4\}$ \\
                & & $=\{(1,3),(1,4),(2,3),(2,4)\}$  \\ \hline
                $|A|$ & \textbf{Cardinality}: the number of elements in $A$ & $|\{3,4\}|=2$ \\ \hline
                %%%
                $|,:$ & \textbf{Such That} & $\{n|n>0\}=\{1,2,3,\dots\}$ \\ \hline
                $\forall$ & \textbf{For All} & $\forall x>1,x^2.x$ \\ \hline
                $\exists$ & \textbf{There Exists} & $\exists$  $x|x^2>x$ \\ \hline
                $\therefore$ & \textbf{Therefore} & $a=b \therefore b =a$ \\ \hline
                $\because$ & \textbf{Because} & \\ \hline
                %%%
                $\mathbb{N}$ & \textbf{Natural Numbers} & $\{1,2,3,\dots\}$ or $\{0,1,2,3,\dots\}$ \\ \hline
                $\mathbb{Z}$ & \textbf{Integers} & $\{\dots,-3,-2,-1,0,1,2,3,\dots\}$ \\ \hline
                $\mathbb{Q}$ & \textbf{Rational Numbers} & \\ \hline
                $\mathbb{A}$ & \textbf{Algebraic Numbers} & \\ \hline
                $\mathbb{R}$ & \textbf{Real Numbers} & \\ \hline
                $\mathbb{I}$ & \textbf{Imaginary Numbers} & $3i$ \\ \hline
                $\mathbb{C}$ & \textbf{Complex Numbers} & $2+5i$ \\ \hline
            \end{tabular}
        \end{center}



    \subsection{Fundamental Laws of Set Algebra}
        \begin{center}
            \begin{tabular}{|c|c|}
                \hline
                $A\cup B=B \cap A$ & \textbf{Commutative} Property \\
                $A\cap B = B\cap A$ & \\ \hline
                $(A\cup B)\cup C = A\cup (B\cup C)$ & \textbf{Associative} Property \\
                $(A\cap B)\cap C = A\cap (B\cap C)$ & \\ \hline
                $A\cup (B \cap C) = (A\cup B) \cap (A\cup C)$ & \textbf{Distributive} Property \\
                $A\cap (B \cup C) = (A\cap B) \cup (A\cap C)$ & \\ \hline
            \end{tabular}

            \begin{tabular}{|c|c|}
                \hline
                $A \cup \emptyset = A$ & \textbf{Identity} \\
                $A \cap \mathbb{U}=A$ & \\ \hline
                $A \cup A^C = \mathbb{U}$ & \textbf{Complement} \\
                $A \cap A^C = \emptyset$ & \\ \hline
            \end{tabular}
        \end{center}\\

        \noindent \color{purple} \textbf{The Principle of Duality} \color{black} states that
        for any true statement about sets, the \textit{dual }statement obtained by interchanging
        $\cup$ and $\cap$, $\cup$ and $\emptyset$, and reversing inclusions is also true. A
        statement is \textbf{self-dual} if it is equal to its dual. \\

        \begin{center}
            \begin{tabular}{|c|c|}
                \hline
                $A\cup A=A$ & \textbf{Idempotent} Laws \\
                $A \cap A = A$ & \\ \hline
                $A\cup \mathbb{U}=\mathbb{U}$ & \textbf{Domination} Laws\\
                $A\cap \emptyset = \emptyset$ & \\ \hline
                $A\cup (A \cap B) = A$  & \textbf{Absorption} Laws \\
                $A\cap (A \cup B) = A$ & \\ \hline
            \end{tabular}
        \end{center}



    \pagebreak
    \subsection{Complements}
        The \textbf{complement} of a set $A$ refers to the elements not in $A$. In the figure
        below, if $A$ is the area colored red in the left image, then the complement of $A$,
        denoted $A^c$ is everything else, as shown in the right image.

        \begin{figure}[hbt!]
            \centering
            \includegraphics[scale=0.4]{Resources/Unit1SetTheory/complement1.PNG}
            \caption*{Figure 1}
        \end{figure}

        \noindent Below are some laws about complements.

        \begin{center}
            \begin{tabular}{|c|c|}
                \hline
                $(A\cup B)^C = A^C \cap B^C$ & \textbf{De Morgan's Laws} \\
                $(A\cap B)^C = A^C \cup B^C$ & \\ \hline
                $(A^C)^C=A$ & \textbf{Involution} Law, \textit{also known as "double complement law"} \\ \hline
                $\emptyset^C=\mathbb{U}$ & \\ \hline
                $\mathbb{U}^C=\emptyset$ & \\ \hline
            \end{tabular}
        \end{center}\\

        \noindent If $A\cup B=\mathbb{U}$ and $A\cap B=\emptyset$, then $B=A^C$.



    \subsection{Inclusion}
        \noindent \textbf{Subsets} are parts of a set. For example, in the set $\{1,2,3,4,5\}$,
        one subset is $\{1,2,3\}$. Two others are $\{3,4\}$ and $\{1\}$. However, $\{1,6\}$ is
        not a subset since the element 6 is not in the parent set. We use \textbf{$A \subseteq B$}
        to denote that A is a subset of B.

        \begin{figure} [hbt!]
            \centering
            \includegraphics[scale=0.6]{Resources/Unit1SetTheory/subsets.PNG}
            \caption*{Figure 2}
        \end{figure}

        \noindent \color{blue} \textit{Example 1: Let $A$ be all the multiples of 4 and let
        $B$ be all the multiples of 2.} Is $A$ a subset of $B$? Is $B$ a subset of $A$?
        \color{black} \\

        $A=\{\dots,-8,-4,0,4,8,\dots\}$\\
        $B=\{\dots,-8,-6,-4,-2,0,2,4,6,8,\dots\}$\\
        By pairing elements from $A$ and $B$, we can see that every element of $A$ is also an
        element of $B$, but not every element of $B$ is an element of $A$.\\

        \begin{figure} [hbt!]
            \centering
            \includegraphics[scale=0.6]{Resources/Unit1SetTheory/subset2.PNG}
            \caption*{Figure 3}
        \end{figure}

        \noindent $\therefore A \subseteq B, B\not\subseteq A$ \\
        \noindent $A$ is a \textbf{proper subset} of $B$ if and only if every element in $A$ is
        also in $B$ and there exists \textit{at least one element} in $B$ that is \textit{not}
        in $A$. We use $\{1,2,3\}\subset\{1,2,3,4\}$ to denote that the first set is a proper
        subset of the second set since the element $4$ is not in the first set. In another
        example, $\{1,2,3\}\subseteq\{1,2,3\}$ but $\{1,2,3\}\not\subset\{1,2,3\}$.



    \subsection{Null Sets}
        An \textbf{empty} set, or \textbf{null} set, is one with \textit{no elements}.
        Represented by $\emptyset$, an example of this is "even numbers that are also odd".
        Obviously, no number like this exists, so the set is null. Furthermore,
        \textit{every empty set is a subset}. Intuitively, since we cannot find any elements
        in the empty set that are not in set $A$, then all elements in the empty set are in $A$.