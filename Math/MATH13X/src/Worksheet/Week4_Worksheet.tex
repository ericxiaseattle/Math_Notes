% Preamble
\documentclass{article}


% Package Imports
\usepackage{../../../../mypackages}


% Macros
\usepackage{../../../../mymacros}


% Homework Details and Basic Document Settings
\pagestyle{fancy}
\lhead{\textbf{Eric Xia}}
\chead{MATH134 (Professor Ebru Bekyel): Week 1 Assignment}
\cfoot{\thepage}

\renewcommand\headrulewidth{0.4pt}
\renewcommand\footrulewidth{0.4pt}


% Title Page
\title{
    \vspace{2in}
    \textmd{\textbf{MATH134: Week 4 Worksheet}}\\
    \normalsize\vspace{0.1in}\small{Due on October 23, 2020 at 11:59 PM}\\
    \vspace{0.1in}\large{\textit{Professor Ebru Bekyel}}
    \vspace{3in}
}

\author{\textbf{Eric Xia}}
\date{}


% Problem Headers and Footers
\fancypagestyle{2}{\rhead{Problem 1}\fancyfoot[L]{Problem 1 continued on next page\ldots}}
\fancypagestyle{3}{\rhead{Problem 1}\fancyfoot[L]{Problem 1 continued on next page\ldots}}
\fancypagestyle{4}{\rhead{Problem 1}\fancyfoot[L]{Problem 3 continued on next page\ldots}}
\fancypagestyle{5}{\rhead{Problem 3}\fancyfoot[L]{Problem 5 continued on next page\ldots}}
\fancypagestyle{6}{\rhead{Problem 5}}

%-------------------------------------------------------------------------------------------------------------------------------------------------------------------------------------------------------------------------
%-------------------------------------------------------------------------------------------------------------------------------------------------------------------------------------------------------------------------
%-------------------------------------------------------------------------------------------------------------------------------------------------------------------------------------------------------------------------
\begin{document}

    \maketitle
    \pagebreak

    \thispagestyle{2}

    \begin{tbhtheorem}{Problem 1}
        You have a piece of wire of length $L$. You cut it into two pieces and make a circle and an equilateral triangle. Where should you make the cut to maximize the sum of their areas? Where should you make the cut
        to minimize the sum of their areas?
    \end{tbhtheorem}

    Let $r$ be the radius of the circle made by a piece of the wire. Let $a$ be one of the sides of the triangle made by a piece of the wire. The circumference of the circle is given by $2\pi r$ and the perimeter of the
    triangle is $3a$. The circumference of the circle and perimeter of the triangle add up to the length of the wire, $L$. This relationship is demonstrated visually in the following figure.

    \begin{figure*}[hbt!]
        \centering
        \includegraphics[scale=0.1]{wire}
    \end{figure*}


    Thus, our "constraint" equation for this optimization problem is:

    \[
        L = 2\pi r + 3a
    \]

    The height of an equilateral triangle can be determined to be $\frac{\sqrt{3}}{2}a$. This process is shown in the figure below.
    
    \begin{figure*}[hbt!]
        \centering
        \includegraphics[scale=0.1]{triangle}
    \end{figure*}

    Because the area of a triangle is $A=\frac{1}{2}bh$, where $b$ is the triangle's base and $h$ is the triangle's height, the area of an equilateral triangle is then given by:

    \[
        A = \frac{\sqrt{3}}{4}a
    \]

    Thus, the total area made by the circular and triangular pieces of the wire is:

    \[
        A = \pi r^2 + \frac{\sqrt{3}}{4}a^2
    \]

    The side length of the triangle, $a$, can be isolated in our constraint equation to get

    \[
        a = \frac{L-2\pi r}{3}
    \]

    Plugging this value for $a$ in the area equation, we can define the combined areas made by the wire as a function of $r$:

    \begin{align*}
        A   &= \pi r^2 + \frac{\sqrt{3}}{4}\left(\frac{L-2\pi r}{3}\right)^2 \\
        A(r)&= \pi r^2 + \frac{\sqrt{3}}{36} (L-2\pi r)^2
    \end{align*}

    Function $A(r)$ is bounded on the closed interval that has endpoints equal to the areas of when the entirety of the wire is made as a circle or triangle, respectively. When all of the wire is made into a circle,

    \[
        L = 2\pi r
    \]

    and the area is

    \begin{align*}
        A   &= \pi \left(\frac{L}{2\pi}\right)^2 \\
            &= \frac{L^2}{4\pi}
    \end{align*}

    \pagebreak
    \thispagestyle{3}

    Similarly, when all of the wire is made into an equilateral triangle,

    \[
        L = 3a
    \]

    and the area is

    \begin{align*}
        A   &= \frac{\sqrt{3}}{4}\left(\frac{L}{3}\right)^2 \\
            &= \frac{L^2 \sqrt{3}}{36}
    \end{align*}

    To find the closed interval under which our area function is bounded, we must compare the previously computed areas when all of the wire is made into a circle or triangle, respectively. \\

    Suppose that the following inequality holds:

    \[
        \frac{L^2}{4\pi} > \frac{L^2 \sqrt{3}}{36}
    \]

    Then,

    \begin{align*}
        \frac{9L^2}{36\pi} &> \frac{\pi L^2 \sqrt{3}}{36 \pi} \\
        9L^2               &> \pi L^2 \sqrt{3} \\
        9                  &> \pi \sqrt{3}
    \end{align*}

    This is true, and thus we can assume that the proposed inequality holds. Then our area function, $A(r)$, is bounded on the closed interval:

    \[
        \left[\frac{L^2 \sqrt{3}}{36}, \frac{L^2}{4\pi}\right]
    \]

    Recall that $A(r)$ is as follows:

    \[
        A(r) = \pi r^2 + \frac{\sqrt{3}}{36}(L-2\pi r)^2
    \]

    The gradient at a critical point is 0 by definition. Hence, differentiating $A(r)$ with respect to $r$ and setting the result equal to 0 will give us the relative extrema (critical points) of $A(r)$. So,

    \begin{align*}
        \frac{dA}{dr}   &= 2\pi r + \frac{\sqrt{3}}{18}(L-2\pi r)(-2\pi) \\
        0               &= 2\pi r - \frac{\pi \sqrt{3}}{9}(L-2\pi r) \\
                        &= 2\pi r - \frac{\pi L\sqrt{3}}{9} + \frac{2\pi ^2 r\sqrt{3}}{9} \\
                        &= 18\pi r - \pi L\sqrt{3} + 2\pi ^2 r \sqrt{3} \\
                        &= 18r - L\sqrt{3} + 2\pi r\sqrt{3} \\
        L\sqrt{3}       &= r(18+2\pi\sqrt{3}) \\
        r               &= \frac{L\sqrt{3}}{18+2\pi\sqrt{3}}
    \end{align*}

    When $r$ is the above value, $A(r)$ is given by:

    \[
        A\left( \frac{L\sqrt{3}}{18+2\pi\sqrt{3}}\right) = \pi \left(\frac{L\sqrt{3}}{18+2\pi\sqrt{3}}\right)^2 + \frac{\sqrt{3}}{36}\left(L-2\pi\left(\frac{L\sqrt{3}}{18+2\pi\sqrt{3}}\right)\right)^2
    \]

    Because $A(r)$ is bounded on a closed interval, then by the Extreme Value Theorem, there must be an absolute maximum and absolute minimum attained either at the endpoints of the interval or at a critical point.
    This means that we must compare the values of the endpoints and the critical point we found to determine the absolute maximum and absolute minimum combined area that can be made by a wire of length $L$. Because of
    the atrociously complicated critical point, let us define the arbitrary length $L$ as $L=7$ units for the sake of computational simplicity. \\

    Plugging $L=7$ into the endpoints of the interval $A(r)$ is bounded on, the interval can be rewritten in decimal approximations to 3 digit places:

    \[
        [2.356, 3.899]
    \]

    The area attained at the critical point can be approximated as 1.469. Because $L$ is arbitrary, the relational areas given by the endpoints and the critical point when $L=7$ can be generalized to all values of $L$.
    1.469 is the smallest area in these three values and hence, the area made by the wire is minimized when

    \pagebreak
    \thispagestyle{4}

    \[
        r               &= \frac{L\sqrt{3}}{18+2\pi\sqrt{3}}.
    \]



    Similarly, 3.899 is the largest area in these three values and hence, the area made by the wire is maximized at the associated endpoint. This particular endpoint is associated with the scenario in which all of the
    wire is made into a circle. Therefore, the total areas made by the wire is maximized when the wire is not cut and instead used completely to make the circle. \\

    Recall that the length of the wire is:

    \[
        L = 2\pi r + 3a
    \]

    To minimize the areas, the wire must be cut at $L-2\pi r$.

    Plugging

    \[
        r               &= \frac{L\sqrt{3}}{18+2\pi\sqrt{3}}.
    \]

    into $L=2\pi r$, we find that, relative to either endpoint of the wire, the wire must be cut at

    \[
        L - 2\pi \left(\frac{L\sqrt{3}}{18+2\pi\sqrt{3}}\right) \text{ units}
    \]

    for the total areas made by the wire to be minimized.

    \begin{tbhtheorem}{Problem 3}
        Find the maximum area of a rectangle inscribed in an equilateral triangle of side $a$.
    \end{tbhtheorem}

    Let the base of the inscribed rectangle be $b$. As demonstrated in the figure below, the height of the inscribed triangle can be determined using the Pythagorean Theorem to be

    \[
        \text{height } = \frac{\sqrt{3}}{2}(a-b)
    \]

    \begin{figure*}[hbt!]
        \centering
        \includegraphics[scale=0.1]{3}
    \end{figure*}

    Let us define the area of the rectangle as the function $A(b)$:

    \begin{align*}
        A(b)   &= \frac{\sqrt{3}}{2}b(a-b) \\
               &= \frac{\sqrt{3}}{2}(ab-b^2)
    \end{align*}

    The relative extrema of $A(b)$ can be found by differentiating with respect to $b$ and setting the result to 0, as done below.

    \begin{align*}
        \frac{dA}{db}   &= \frac{\sqrt{3}}{2}(a-2b) \\
        0               &= a-2b \\
        b               &= \frac{a}{2}
    \end{align*}

    The area function $A(b)$ is bounded on the closed interval $[0,a]$. Thus, by the Extreme Value Theorem, $A(b)$ must attain a maximum value either at the endpoints or at a critical point. When $b=0$,

    \[
        A(0)    &= \frac{\sqrt{3}}{2}(0-0) = 0
    \]

    When $b=a$,

    \[
        A(a)    &= \frac{\sqrt{3}}{2}(a^2-a^2) =0
    \]

    \pagebreak
    \thispagestyle{5}

    When $b=\frac{a}{2}$,

    \begin{align*}
        A\left(\frac{a}{2}\right)  &= \frac{\sqrt{3}}{2}\left(a\left(\frac{a}{2}\right)-\left(\frac{a}{2}\right)^2 \right) \\
                                   &= \frac{a^2\sqrt{3}}{8}
    \end{align*}

    For a nonnegative $a$, the area when $b=\frac{a}{2}$ is greater than or equal to the area when $b=0$ and $b=a$, 0. Thus, by the Extreme Value Theorem, the maximum area attainable by a rectangle inscribed in an
    equilateral triangle of side $a$ is:

    \[
        \frac{a^2\sqrt{3}}{8}
    \]


    \begin{tbhtheorem}{Problem 5}
        Suppose you are at the beach, standing at the edge of the water with your dog. As you look westward into the ocean, you throw a tennis ball out into the water. The ball lands in the water 20 metres North and
        13 metres West of where you are standing. Your dog swims at 5 m/s but runs at 13 m/s. How far North does the dog run before jumping in the water to minimize his time to get to the ball?
    \end{tbhtheorem}

    Let $s$ be the distance the dog swims and $r$ be the distance the dog runs, both in metres. The distance $s$ can be determined using the Pythagorean Theorem to be

    \[
        s = \sqrt{r^2 - 40r+569},
    \]

    as demonstrated in the below figure.

    \begin{figure*}[hbt!]
        \centering
        \includegraphics[scale=0.15]{dog}
    \end{figure*}

    Let us define a new function, $T(r)$, as the time the dog takes to get to the ball. This time is given by the distance over the time:

    \begin{align*}
        T(r)   &= \frac{r}{\frac{dr}{dt}} + \frac{s}{\frac{ds}{dt}} \\
            &= \frac{r}{13} + \frac{\sqrt{r^2-40r+569}}{5}
    \end{align*}

    The function $T(r)$ is bounded on the closed interval $[0,20]$. By the Extreme Value Theorem, $T(r)$ must attain its minimum either at an endpoint of its domain or at a critical point. When $r=0$,

    \begin{align*}
        T(0)    &= \frac{0}{13} + \frac{\sqrt{0^2-40(0)+569}}{5} = \frac{\sqrt{569}}{5} \\
                &= 4.711
    \end{align*}

    When $r=20$,

    \begin{align*}
        T(20)   &= \frac{20}{13}+\frac{\sqrt{20^2-40(20)+569}}{5} = \frac{269}{65} \\
                &\approx 4.138
    \end{align*}

    \pagebreak
    \thispagestyle{6}

    To find the relative extrema of $T(r)$, we differentiate $T(r)$ with respect to $r$ and set the result = 0. Then,

    \begin{align*}
        \frac{dT}{dr}   &= \frac{1}{13} + \frac{1}{5}\cdot\frac{d}{dr}(r^2-40r+569)^{\frac{1}{2}} \\
                        &= \frac{1}{13} + \frac{1}{5}\left(\frac{1}{2}\left(r^2-40r+569\right)^{-\frac{1}{2}}(2r-40)\right) \\
        0               &= \frac{1}{13} + \frac{r-20}{5\sqrt{r^2-40r+569}} \\
        r               &= \frac{175}{12}
    \end{align*}

    When $r=\frac{175}{12}$,

    \begin{align*}
        T\left(\frac{175}{12}\right)   &= \frac{\frac{175}{12}}{13} + \frac{\sqrt{\left(\frac{175}{12}\right)^2-40\left(\frac{175}{12}\right)+569}}{5} \\
                                       &= \frac{1189}{360} \\
                                       &\approx 3.303 \text{ m}
    \end{align*}

    Because the smallest attainable time for the dog to travel to the ball out of the values given by $r=0,\frac{175}{12},20$ is 3.303, then the dog should run a distance of $\frac{175}{12}$ metres before swimming in
    order to minimize the time required to get to the ball.



\end{document}