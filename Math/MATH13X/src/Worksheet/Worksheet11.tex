% Preamble
\documentclass{article}


% Package Imports
\usepackage{../../../../mypackages}


% Macros
\usepackage{../../../../mymacros}


% Homework Details and Basic Document Settings
\pagestyle{fancy}
\lhead{\textbf{Eric Xia}}
\chead{MATH134 (Professor Ebru Bekyel): Worksheet 11}
\cfoot{\thepage}

\renewcommand\headrulewidth{0.4pt}
\renewcommand\footrulewidth{0.4pt}


% Title Page
\title{
    \vspace{2in}
    \textmd{\textbf{MATH134: Worksheet 11}}\\
    \normalsize\vspace{0.1in}\small{Due on December 11, 2020}\\
    \vspace{0.1in}\large{\textit{Professor Ebru Bekyel}}
    \vspace{3in}
}

\author{\textbf{Eric Xia}}
\date{}


% Problem Headers and Footers
\fancypagestyle{2}{\rhead{Problem 1A}}

%-------------------------------------------------------------------------------------------------------------------------------------------------------------------------------------------------------------------------
%-------------------------------------------------------------------------------------------------------------------------------------------------------------------------------------------------------------------------
%-------------------------------------------------------------------------------------------------------------------------------------------------------------------------------------------------------------------------
\begin{document}

    \maketitle
    \pagebreak

    \thispagestyle{2}


    \begin{tbhtheorem}{Problem 1}
        (a) Keeping in mind that the velocity is $v(t) = x'(t) = \frac{dx}{dt}$, acceleration is $a(t) = \frac{dv}{dt} = x''(t)$ and using the equation $x'' + {\omega}^2 x=0$ together with the chain rule show that

        \[
            \frac{dv}{dx} = -\frac{1}{v}\cdot {\omega}^2 x
        \]

        (b) Now use separation of variables together with the initial conditions $x(0) = x_0$ and $v(0) = 0$ to get an equation for $v$ in terms of $x$. (What is the sign of $v?$) \\
        (c) From $\frac{dx}{dt} = v$, it follows that $dt = \frac{dx}{v}$. Compute the integral

        \[
            \int_{x=x_0}^{x=0} dt
        \]

        (with $x$ being your varialbe of integration) to show that the time it takes the object to reach the origin is

        \[
            \frac{\pi}{2\omega}
        \]

        which you can see is independent of the starting position $x_0$!
    \end{tbhtheorem}

    \textbf{(a)} \\
    By definition,

    \begin{align*}
        x'              &= \frac{dx}{dt} \\
        \frac{1}{x'}    &= \frac{dt}{dx} \\
        \frac{1}{v}     &= \frac{dt}{dx}
    \end{align*}

    Since $x'' = \frac{dv}{dt}$ we can rewrite the given equation in Leibniz notation and plug in $\frac{dt}{dx} = \frac{1}{v}$:

    \begin{align*}
        \frac{dv}{dt}       + \omega^2 x        &= 0 \\
        \frac{dv}{dt} \cdot \frac{dt}{dx}       &= -\omega^2 x \cdot \frac{dt}{dx} \\
        \frac{dv}{dx}                           &= -\frac{1}{v} \omega^2 x
    \end{align*}

    \textbf{(b)}

    \begin{align*}
        \frac{dv}{dx}   &= -\frac{1}{v} \cdot \omega^2 x \\
        \int vdv        &= -\omega^2 \int x dx \\
        \frac{v^2}{2} + C_1     &= -\frac{\omega^2}{2}x^2 + C_2 \\
        \text{$C_1$ and $C_2$ are arbitary, let } C &= C_2 - C_1 \\
        \frac{v^2}{2}   &= -\frac{\omega^2}{2}x^2 + C \\
        v^2             &= -\omega^{2}x^2 + 2C
    \end{align*}

    Since $C$ is arbitrary, redefine $C$ from $2C$, substituting in the initial-value $(x_0,0)$. This initial value is since $x(t)$ and $v(t)$ are functions of $t$ which we do not consider in the new function $v(x)$.

    \begin{align*}
        v^2 &= -\omega^2 x^2 + C \\
        0   &= -\omega^2 x_0^2 + C \\
        C   &= \omega^2 x_0^2
    \end{align*}

    Plugging this value in for $C$,

    \begin{align*}
        v^2 &= -\omega^2 x^2 + \omega^2 x_0^2
    \end{align*}

    Since $v$ should be negative,

    \begin{align*}
        v(x)    &= -\sqrt{\omega^2 x_0^2 - \omega^2 x^2} \\
                &= -\omega\sqrt{x_0^2 - x^2}
    \end{align*}

    \pagebreak
    \thispagestyle{3}

    \textbf{(c)}

    \begin{align*}
        \int_{x=x_0}^{x=0}dt    &= \int_{x=x_0}^{x=0} \frac{dx}{v} \\
                                &= \int_{x_0}^{0} \frac{dx}{-\omega\sqrt{x_0^2 - x^2}} \\
                                &= -\frac{1}{\omega}\int_{x_0}^0 \frac{dx}{\sqrt{x_0^2-x^2}} \\
                                &= -\frac{1}{\omega}\int_{x_0}^0 \frac{dx}{x_0\sqrt{1-\left(\frac{x}{x_0}\right)^2}} \\
        u = \frac{x}{x_0}       &\implies du = \frac{dx}{x_0} \\
                                &= -\frac{x_0}{\omega x_0} \int_{x_0}^0 \frac{du}{\sqrt{1-u^2}} \\
                                &= -\frac{1}{\omega}\arcsin{u}\Big|_1^0 \\
                                &= \frac{1}{\omega} \cdot \frac{\pi}{2} \\
                                &= \frac{\pi}{2\omega}
    \end{align*}




    \begin{tbhtheorem}{Problem 2}
        Now we solve the initial value problem $x'' + {\omega}^2 x = 0$, $x(0) = x_0$, $x'(0) = 0$. \\
        (a) Start by thinking about, i.e. guess, two familiar functions $f(t)$ and $g(t)$ which are solutions of $x'' + {\omega}^2 x = 0$. You can think about the simpler case $\omega = 1$ first. Then,
        $C_1 f(t) + C_2 g(t)$ will be the general solution to $x'' + {\omega}^2 x = 0$. \\
        (b) Use your result above together with $x(0) = x_0$, $x'(0) = 0$ to show that the solution to the initial value problem is

        \[
            x(t) = x_0 \cos{\left(\omega t\right)}
        \]
    \end{tbhtheorem}

    \textbf{(a)}









    \pagebreak

    \begin{tbhtheorem}{Problem 3}
        Now we are moving onto two dimensions with a parametric curve $\left(x(\theta), y(\theta)\right)$. The object is sliding down, under (constant) gravitational force on this curve. The length of the curve from its
        lowest point is the arclength $s(t)$. If the arclength satisfies the initial value problem

        \[
            s'' + {\omega}^2 s = 0, s(0) = s_0, s'(0) = 0
        \]

        then the time it takes for the object to reach the bottom will be independent of $s_0$, where it starts in the curve, as established above. We want to find the parametric equations of a curve whose arclength
        satisfies this differential equation where the acceleration is the result of a constant gravitational force. \\

        A component of the gravitational force is normal to the curve and cancels out with the reaction force $F_N$ from the surface of its path. The tangential component of the gravitational force $F_T$ causes the
        acceleration $s''$. Let $\theta$ be the angle the tangent to the curve makes with the $x$-axis as shown (negative in the picture). \\

        (a) Compute $F_T$ in terms of the gravitational force $mg$ and $\theta$. \\
        (b) Use the fact that $s'' + {\omega}^{2}s = 0$ (and $F=ma$) to relate $\theta$ and $s$. \\
        (c) Differentiate your result above to get

        \[
            ds = \frac{g}{{\omega}^2}\cos{\theta}
        \]

        (d) Relate the slope of the curve $\frac{dy}{dx}$ to the angle $\theta$. \\
        (e) Use the definition of the arclength differential $ds$ to show

        \[
            ds = \frac{dx}{\cos{\theta}} \text{ and } ds = \frac{dy}{\sin{\theta}}
        \]

        (f) Now use your results in (c) and (e) above to express $dx$ and $dy$ in the form $F(\theta)d\theta$ ready to integrate with respect to $\theta$. \\
        (g) Integrate to get $x$ and $y$ with $\theta$ as a parameter to get

        \[
            x = \frac{g}{4\omega^2}(2\theta - \sin{2\theta}) + C_x \text{ and } y = -\frac{g}{4\omega^2}\cos{2\theta} + C_y
        \]

        where $C_x$ and $C_y$ are the integration constants.
    \end{tbhtheorem}

    \textbf{(a)}

    \[
        F_T = mg\sin{\theta}
    \]

    \textbf{(b)} \\


    \begin{figure*}[hbt!]
        \centering
        \includegraphics[scale=0.75]{Worksheet11_Problem3}
    \end{figure*}

    \begin{tbhtheorem}{Problem 4}
        Now, we want to see what the shape of this parametric curve is. You can graph it before you go on (choose random integration constants) and try to make a guess. \\
        (a) A cycloid is the path of a point on the circumference of a circle as the circle rolls on a surface. To get the equations for $\left(x(\theta), y(\theta)\right)$, you can use the picture below. \\
        Find equations of the coordinates of point $P$ which initially is at the bottom of the circle of radius $r$ when teh center of the circle is at $(0,r)$. The paremeter is $\theta$, which keeps track of how much
        the circle has rolled, at its starts at $\theta = 0$. \\
        (b) Now make the change of coordinates to your result in 3(g) including choices for the integration constants so the equations match. \\
        (c) Express the time it takes the object to fall in terms of $r$ and $g$. It does not depend on the initial starting position of the object!
    \end{tbhtheorem}

    \begin{figure*}[hbt!]
        \centering
        \includegraphics[scale=0.75]{Worksheet11_Problem4}
    \end{figure*}






\end{document}
