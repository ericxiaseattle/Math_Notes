% Preamble
\documentclass{article}


% Package Imports
\usepackage{../../../../mypackages}


% Macros
\usepackage{../../../../mymacros}


% Homework Details and Basic Document Settings
\pagestyle{fancy}
\lhead{\textbf{Eric Xia}}
\chead{MATH134 (Professor Ebru Bekyel): Worksheet 7}
\cfoot{\thepage}

\renewcommand\headrulewidth{0.4pt}
\renewcommand\footrulewidth{0.4pt}


% Title Page
\title{
    \vspace{2in}
    \textmd{\textbf{MATH134: Worksheet 7}}\\
    \normalsize\vspace{0.1in}\small{Due on November 13, 2020 at 11:59 PM}\\
    \vspace{0.1in}\large{\textit{Professor Ebru Bekyel}}
    \vspace{3in}
}

\author{\textbf{Eric Xia}}
\date{}


% Problem Headers and Footers
\fancypagestyle{2}{\rhead{Problem 1}\fancyfoot[L]{Problem 1 continued on next page\ldots}}
\fancypagestyle{3}{\rhead{Problem 1}\fancyfoot[L]{Problem 2 continued on next page\ldots}}
\fancypagestyle{4}{\rhead{Problem 2}}


%-------------------------------------------------------------------------------------------------------------------------------------------------------------------------------------------------------------------------
%-------------------------------------------------------------------------------------------------------------------------------------------------------------------------------------------------------------------------
%-------------------------------------------------------------------------------------------------------------------------------------------------------------------------------------------------------------------------
\begin{document}

    \maketitle
    \pagebreak

    \thispagestyle{2}

    \begin{tbhtheorem}{Problem 1}
        \textbf{(a)} Redo the sandbag problem from the lecture without the sandbag, only pulling up the cable, by slightly altering the integral we set up. Keep the same variable $y$ with ground at $y=0$ for comparison
        among groups and parts below. \\
        \textbf{(b)} Now, do the same problem changing the point of view: When the cable is at the top of the building, every vertical slice of length $\Delta y$ will have moved from its initial (hanging) position to the
        top of the building. (Imagine moving each link of the chain separately.) Find the work done with this approach. \\
        \textbf{(c)} Now, show that the work done in pulling the cable up is its weight times the distance its center of mass travels. \\
        \textbf{(d)} Finally, set up an integral to compute the work done in pulling the cable to the top of the buildilng if the density of the cable is given by $\lambda (y)$ instead of the fixed 0.2 kilograms per meter.
        Check part (c) still holds in this case.
    \end{tbhtheorem}

    \textbf{(a)}
    The bounds of our integral will be the $y$ at the top and bottom of the building. The integrand, $F(y)$, will be the weight of the cable at a height $y$ where $y=0$ is relative to the ground. Then

    \[
        W = \int_0^{32} F(y)dy
    \]

    and

    \begin{align*}
        W   &= \int_0^{32} 0.2g(32-y) \\
            &= 0.2g\int_0^{32} \\
            &= 0.2g\left[32y-\frac{y^2}{2}\right]\Big|_0^{32} \\
            &= 0.2g\left[32^2-\frac{32^2}{2}\right] \\
            &= 102.4g \\
            &= 1003.52 \text{ J}
    \end{align*}

    \textbf{(b)}
    \begin{align*}
        W   &= \int_0^{32} F(y)dy \\
            &= \int_0^{32} 0.2g(32-y) \\
            &= 1003.52 \text{ J}
    \end{align*}

    \textbf{(c)}
    The density of the cable is 0.2 kg/m and the cable is 32 m long, so the mass of the cable is $0.2*32=6.4$ kg. Thus the weight of the cable is

    \[
        F_g = 6.4g = 62.72 \text{ N}
    \]

    The change in the centre of mass of the cable is given by

    \[
        \Delta x_{\text{cm}} = 32-\frac{32}{2} = 16\text{ m}
    \]

    Thus, the work done is as follows.

    \begin{align*}
        W &= \vec{F}\cdot \Delta \vec{x} \\
          &= (62.72\text{ N})(16\text{ m})\\
          &= 1003.52\text{ J}
    \end{align*}

    \textbf{(d)}
    The weight of the cable, $F(y)$, is now

    \[
        F(y) = \lambda (y)g(32-y)
    \]

    and the work done is given by

    \begin{align*}
        W   &= \int_0^{32}\lambda(y)g(32-y)dy \\
            &= g\int_0^{32}\lambda(y)(32-y)dy
    \end{align*}

    \pagebreak
    \thispagestyle{3}

    Through Integration by Parts, setting $u=\lambda(y)$ and $\int dv=(32-y)dy$, we have

    \begin{align*}
        W   &= g\left(\lambda(y)\left(32y-\frac{y^2}{2}\right)\Big|_0^{32}-\int\left(32y-\frac{y^2}{2}\lambda '(y)dy\right)\right) \\
            &= g\left(512\lambda(y) - \int \left(32y-\frac{y^2}{2}\right)\lambda '(y)dy \right)
    \end{align*}

    When the density of the cable, $\lambda(y)$, is fixed at 0.2 kg/m,

    \begin{align*}
        W   &= g\left(512(0.2)-\int\left(32y-\frac{y^2}{2}\right)(0)dy\right) \\
            &= g(512(0.2)) \\
            &= 102.4g \\
            &= 1003.52 \text{ J}
    \end{align*}

    Note that this result is the same as that of part C.



    \begin{tbhtheorem}{Problem 2}
        \textbf{(a)} A flat math billboard is in the shape of (what else?) a parabola. Its top side is 6 meters wide and the billboard is 18 meters high, measured from the lowest to the highest point. It is mounted on a
        pole and the lowest point of the billboard is 15 meters above the ground. Before it was mounted on the pole, the billboard was originally lying flat on the ground. The billboard weighs 13 kilograms per square
        meter. Set up a definite integral for the work done in lifting this billboard up to where it now stands. Evaluate the integral and find the work done. \\
        \textbf{(b)} Show that the work done is the weight of the billboard times the distance its center of mass travels vertically.
    \end{tbhtheorem}
    
    \begin{figure*}[hbt!]
        \centering
        \includegraphics[]{2D}
    \end{figure*}

    \textbf{(a)}
    A parabola through the points (-3,18), (0,0), and (3,18) can be of the form $y=2x^2$. We can find the 2D area of the parabolic billboard to be

    \begin{align*}
        A   &= \int_{-3}^3 (18-2x^2)dx \\
            &= \left[18x-\frac{2}{3}x^3\right]\Big|_{-3}^3 \\
            &= 36 + 36 \\
            &= 72 \text{ m}^2
    \end{align*}

    The mass of the billboard is given by the density of it multiplied by the area such that

    \begin{align*}
        m   &= \frac{\rho}{A} \\
            &= \frac{13\text{ kg}}{\text{m}^2}\cdot 72 \text{ m}^2 \\
            &= 936 \text{ kg}
    \end{align*}

    Then the weight of the billboard is

    \[
        \text{Weight } = F_g = 936g = 9172.8\text{ N}
    \]

    The final centre of mass of the billboard is given by

    \begin{align*}
        y_{\text{cm, f}}    &= 15 + \frac{3h}{5} \\
                            &= 15 + \frac{3(18)}{5} \\
                            &= 25.8
    \end{align*}

    \pagebreak
    \thispagestyle{4}

    Thus the work done is as follows.

    \begin{align*}
        W   &= \int_0^{25.8} 9172.8dy \\
            &= \left[9172.8y\right]\Big|_0^{25.8} \\
            &= 9172.8(25.8) \\
            &= 236658.24 \text{ J}
    \end{align*}

    \textbf{(b)}
    Considering only the $y$-component, the initial centre of mass of the billboard is zero. We calculated the final centre of mass of the billboard in part A to be 25.8m, thus the displacement of the centre of mass of
    the billboard is 25.8. Multiplying this displacement by the weight, we see that

    \[
        25.8 \cdot 9172.8 = 236658.24 \text{ J}
    \]

    which is identical to the work calculated in part A.


\end{document}