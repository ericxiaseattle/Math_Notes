% Preamble
\documentclass{article}


% Package Imports
\usepackage{../../../../mypackages}


% Macros
\usepackage{../../../../mymacros}


% Homework Details and Basic Document Settings
\pagestyle{fancy}
\lhead{\textbf{Eric Xia}}
\chead{MATH134 (Professor Ebru Bekyel): Worksheet 6}
\cfoot{\thepage}

\renewcommand\headrulewidth{0.4pt}
\renewcommand\footrulewidth{0.4pt}


% Title Page
\title{
    \vspace{2in}
    \textmd{\textbf{MATH134: Worksheet 6}}\\
    \normalsize\vspace{0.1in}\small{Due on November 6, 2020 at 11:59 PM}\\
    \vspace{0.1in}\large{\textit{Professor Ebru Bekyel}}
    \vspace{3in}
}

\author{\textbf{Eric Xia}}
\date{}


% Problem Headers and Footers
\fancypagestyle{2}{\rhead{Problem 1}}


%-------------------------------------------------------------------------------------------------------------------------------------------------------------------------------------------------------------------------
%-------------------------------------------------------------------------------------------------------------------------------------------------------------------------------------------------------------------------
%-------------------------------------------------------------------------------------------------------------------------------------------------------------------------------------------------------------------------
\begin{document}

    \maketitle
    \pagebreak

    \thispagestyle{2}

    \color{blue} In this worksheet we work with the triangular region bounded by the lines $y=x$, $y=\frac{1}{\sqrt{3}}x$, and $x=3$ in the first quadrant. Last two questions are tricky! \color{black}

    \begin{figure*}[hbt!]
        \centering
        \includegraphics[]{volumes}
    \end{figure*}

    \begin{tbhtheorem}{Problem 1}
        Compute the area of the region. Set up integrals both with $dx$ and with $dy$. Which one was easier to do?
    \end{tbhtheorem}

    \begin{align*}
        y = x                   &\iff   x = y \\
        y = \frac{1}{\sqrt{3}}x &\iff   x = y\sqrt{3}
    \end{align*}

    The region is bounded by $[0,3]$ for $x$ because the region spans from the origin to the line $x=3$. The region is also bounded by $[0,3]$ for $y$ because the upper $y$-bound is given by $y=x=3$. Thus the area
    computed with respect to $x$ is given by:

    \begin{align*}
        A   &= \int^3_0 \left(x-\frac{1}{\sqrt{3}}x\right) \\
            &= \left[\frac{x^2}{2}-\frac{x^2}{2\sqrt{3}}\right] \Bigg|^3_0 \\
            &= \frac{9}{2} - \frac{9}{2\sqrt{3}} \\
            &= \frac{9-3\sqrt{3}}{2}
    \end{align*}

    The area computed with respect to $y$ is as follows:

    \begin{align*}
        A   &= \int_0^{\frac{3}{\sqrt{3}}} \left(y\sqrt{3}-y\right) + \int^3_{\frac{3}{\sqrt{3}}} (3-y)dy \\
            &= \left[\frac{y^2\sqrt{3}}{2}-\frac{y^2}{2}\right]\Big|_0^{\frac{3}{\sqrt{3}}} + \left[3y-\frac{y^2}{2}\right]\Big|^3_{\frac{3}\sqrt{3}} \\
            &= \left[\frac{y^2\sqrt{3}-y^2}{2}\right]\Bigg|_0^{\sqrt{3}} + \left[3y-\frac{y^2}{2}\right]\Bigg|^3_{\sqrt{3}} \\
            &= \frac{3\sqrt{3}-3}{2} + \left[\left(9-\frac{9}{2}\right)-\left(3\sqrt{3}-\frac{3}{2}\right)\right] \\
            &= \frac{3\sqrt{3}-3}{2} + \left[9-\frac{9}{2}-3\sqrt{3}+\frac{3}{2}\right] \\
            &= \frac{3\sqrt{3}-3}{2} + 6 - 3\sqrt{3} \\
            &= \frac{3\sqrt{3}-3}{2} + \frac{12}{2} - \frac{6\sqrt{3}}{12} \\
            &= \frac{9-3\sqrt{3}}{2}
    \end{align*}

    The area computed with respect to $x$ was much easier to do.

    \pagebreak

    \begin{tbhtheorem}{Problem 2}
        Find the volume of the solid obtained by rotating this region about the line $y=-1$.
    \end{tbhtheorem}

    The volume, $V$, is as follows:

    \begin{align*}
        V   &= \pi\int^3_0 \left((x+1)^2-\left(\frac{x}{\sqrt{3}}+1\right)^2\right)dx \\
            &= \pi\int^3_0 \left(x^2+2x+1-\left(\frac{x^2}{3}+\frac{2x}{\sqrt{3}}+1\right)\right)dx \\
            &= \pi\int^3_0 \left(x^2+2x-\frac{x^2}{3}-\frac{2x}{\sqrt{3}}\right) \\
            &= \pi\left[\frac{x^3}{3}+x^2-\frac{x^3}{9}-\frac{x^2}{\sqrt{3}}\right]\Bigg|^3_0 \\
            &= \pi\left(9 + 9 - 3 - \frac{9}{\sqrt{3}}\right) \\
            &= 15\pi - 3\pi\sqrt{3}
    \end{align*}


    \begin{tbhtheorem}{Problem 3}
        Find the volume of the solid obtained by rotating this region about the $y$-axis.
    \end{tbhtheorem}

    The volume $V$ is as follows.

    \begin{align*}
        V   &= 2\pi\int_0^3 x\left(x-\frac{x}{\sqrt{3}}\right) \\
            &= 2\pi\int_0^3 \left(x^2 - \frac{x^2}{\sqrt{3}}\right) \\
            &= 2\pi\left[\frac{x^3}{3}-\frac{x^3}{3\sqrt{3}}\right]\Big|_0^3 \\
            &= 2\pi\left(9 - \frac{9}{\sqrt{3}}) \\
            &= 2\pi\left(9 - 3\sqrt{3}) \\
            &= 18\pi - 6\sqrt{3}\pi
    \end{align*}


    \begin{tbhtheorem}{Problem 4}
        Now, switch your variable of integration and compute the volume of the solid obtained by rotating this region about the $y$-axis again.
    \end{tbhtheorem}

    The volume $V$ is given by:

    \begin{align*}
        V   &= \pi\int_0^{\frac{3}{\sqrt{3}}} \left((y\sqrt{3})^2+y^2\right)dy + \pi\int_{\sqrt{3}}^3 \left(3^2-y^2\right)dy \\
            &= \pi\int_0^{\frac{3}{\sqrt{3}}} (3y^2 - y^2)dy + \pi\int_{\sqrt{3}}^3 \left(9-y^2)dy \\
            &= \pi\int_0^{\frac{3}{\sqrt{3}}}2y^2 dy + \pi\int_{\sqrt{3}}^3 \left(9-y^2)dy \\
            &= \pi\left[\frac{2y^3}{3}\right]\Big|_0^{\frac{3}{\sqrt{3}}} + \pi\left[9y-\frac{y^3}{3}\right]\Big|_{\sqrt{3}}^3 \\
            &= \pi\left[\frac{2\left(\frac{3}{\sqrt{3}}\right)^3}{3}\right] + \pi\left[27-9-\left(9\sqrt{3}-\frac{(\sqrt{3})^3}{3}\right)\right] \\
            &= 2\pi\sqrt{3} + \pi (18-8\sqrt{3}) \\
            &= 2\pi\sqrt{3} + 18\pi - 8\pi\sqrt{3} \\
            &= 18\pi - 6\sqrt{3}\pi
    \end{align*}

    \pagebreak

    \begin{tbhtheorem}{Problem 5}
        Find the volume of the solid obtained by rotating this region about the ine $x=1$. Keep careful track of the overlap.
    \end{tbhtheorem}

    Let $V_1$ be the volume of the red region bounded by the lines $x=1$, $y-x$, and $y=\frac{x}{\sqrt{3}}$. Let $V_2$ be the volume of the blue region bounded by the lines $y=x$, $y=\frac{x}{\sqrt{3}}$, $x=1$, and $x=3$.
    Let $V_3$ be the purple overlapping region. Let the green point $P$ be the intersection of the line $y=\frac{x}{\sqrt{3}}$ and the \textit{reflection} of the line $y=x$ across $x=1$. The relationship between these
    variables is sketched below for clarity.
    
    \begin{figure*}[hbt!]
        \centering
        \includegraphics[scale=0.1]{volume}
    \end{figure*}

    Let us first solve for all the unknown variables that we will need to use, before finally computing the integrals. \\

    The line $y=x$ intersects the $x$-axis at the origin, which is 1 unit away from the axis of reflection, $x=1$. Thus, the reflection of $y=x$ will intersect the $x-$axis 1 unit in the opposite direction of $x=1$,
    i.e. the reflection of $y=x$ intersects the $x$-axis at (2,0). Then the $y-$intercept of the reflection will be 2 and hence the reflection of $y=x$ across $x=1$ is $y=-x+2$. \\

    To find the $x$-coordinate of $P$, we set up a system as follows.

    \[
        \begin{cases}
            y = -x+2 & (1)\\
            y = \frac{x}{\sqrt{3}} & (2)
        \end{cases}
    \]

    Substituting (2) into (1), we have

    \begin{align*}
        \frac{x}{\sqrt{3}}  &= -x+2 \\
        x                   &= -x\sqrt{3}+2\sqrt{3} \\
        x(1+\sqrt{3})       &= 2\sqrt{3} \\
        x                   &= 3-\sqrt{3}
    \end{align*}

    Using the shell method for $V_1$, our radius is $1-x$ and our height is $y_2-y_1=x-\frac{x}{\sqrt{3}}$. Thus, $V_1$ is given by

    \begin{align*}
        V_1 &= 2\pi\int_0^1 (1-x)\left(x-\frac{x}{\sqrt{3}}\right)dx \\
            &= 2\pi\int_0^1 \left(x-\frac{x}{\sqrt{3}}-x^2+\frac{x^2}{\sqrt{3}}\right) \\
            &= 2\pi\left[\frac{x^2}{2}-\frac{x^2}{2\sqrt{3}}-\frac{x^3}{3}+\frac{x^3}{3\sqrt{3}}\right]\Bigg|_0^1 \\
            &= 2\pi\left(\frac{1}{2}-\frac{1}{2\sqrt{3}}-\frac{1}{3}+\frac{1}{3\sqrt{3}}\right) \\
            &= 2\pi\left(\frac{1}{6}-\frac{\sqrt{3}}{18}\right) \\
            &= \frac{6\pi-\pi\sqrt{3}}{18}
    \end{align*}

    For $V_2$, our radius is $x-1$ and our height is still $x-\frac{x}{\sqrt{3}}$. So $V_2$ is given by

    \begin{align*}
        V_2 &= 2\pi\int_1^3 (x-1)\left(x-\frac{x}{\sqrt{3}}\right)dx \\
            &= 2\pi\int_1^3\left(x^2-\frac{x^2}{\sqrt{3}}-x+\frac{x}{\sqrt{3}}\right)dx \\
            &= 2\pi\left[\frac{x^3}{3}-\frac{x^3}{3\sqrt{3}}-\frac{x^2}{2}+\frac{x^2}{2\sqrt{3}}\right]\Bigg|_1^3 \\
            &= 2\pi\left[\left(9-\frac{9}{\sqrt{3}}-\frac{9}{2}+\frac{9}{2\sqrt{3}}\right) - \left(\frac{1}{3}-\frac{1}{3\sqrt{3}}-\frac{1}{2} + \frac{1}{2\sqrt{3}}\right)\right] \\
            &= 2\pi\left(\frac{42-14\sqrt{3}}{9}\right) \\
            &= \frac{84\pi-28\sqrt{3}\pi}{9}
    \end{align*}

    For $V_3$, the radius is $(x-1)$ and the height is $\left(x-\frac{x}{\sqrt{3}}\right)$, however the bounds are 1 and the $x$-coordinate of $P$, $3-\sqrt{3}$. $V_3$ is

    \begin{align*}
        V_3 &= 2\pi\int_1^{3-\sqrt{3}} (x-1)\left(x-\frac{x}{\sqrt{3}}\right) \\
            &= 2\pi\int_1^{3-\sqrt{3}} \left(x^2-\frac{x^2}{\sqrt{3}}-x+\frac{x}{\sqrt{3}}\right)dx \\
            &= 2\pi\left[\frac{x^3}{3}-\frac{x^3}{3\sqrt{3}}-\frac{x^2}{2}+\frac{x^2}{2\sqrt{3}}\right]\Bigg|_1^{3-\sqrt{3}} \\
            &= 2\pi\left[\left(\frac{(3-\sqrt{3})^3}{3}-\frac{(3-\sqrt{3})^3}{3\sqrt{3}}-\frac{(3-\sqrt{3})^2}{2}+\frac{(3-\sqrt{3})^2}{2\sqrt{3}}\right)-\left(\frac{1}{3}-\frac{1}{3\sqrt{3}}-\frac{1}{2}+\frac{1}{2\sqrt{3}}\right)\right] \\
            &= 2\pi\left(\frac{345-199\sqrt{3}}{18}\right) \\
            &= \frac{345\pi-199\sqrt{3}\pi}{9}
    \end{align*}

    Because $V_3$ overlaps with $V_1$ and $V_2$, it is already counted and therefore it follows that the total volume is given by
    \begin{align*}
        V   &= V_1 + V_2 - V_3 \\
            &= \frac{6\pi-\pi\sqrt{3}}{18} + \frac{84\pi-28\pi\sqrt{3}}{9} - \frac{345\pi-199\pi\sqrt{3}}{9} \\
            &= \frac{6\pi-\pi\sqrt{3}+168\pi-56\pi\sqrt{3}-690\pi+398\pi\sqrt{3}}{18} \\
            &= \frac{341\pi\sqrt{3}-516\pi}{18}
    \end{align*}

    \pagebreak


    \begin{tbhtheorem}{Problem 6}
        Finally, following Cindy's suggestion, we'll rotate this about the slant (red) line $y=x\sqrt{3}$. Here is the picture with the new axis of rotation: \\

        We'll do this by introducing a new variable $z$, which will keep track of how far we are from the origin on the red line and using washers. The black lines are all perpendicular to the red line. \\
        (a) What are the angles the three slant lines make with the positive $x$-axis? They will help with computations below. \\
        (b) Compute the distances $OA$ and $OB$. These will be our limits of integration. \\
        (c) Let $P_z$ be an arbitrary point on the axis of rotation which is $z$ units from the origin. I put $P_z$ in two places as examples. So $OP_z$ has length $z$. Compute the lengths $P_z Q_z$, $P_z R_z$ and
        $P_z S_Z$, all in terms of $z$. These will be related to the radii in the washer method. \\
        (d) Now, set up two integrals using washers and compute the volume of the solid obtained by rotating the region about the line $y=x\sqrt{3}$.
    \end{tbhtheorem}

    \begin{figure*}[hbt!]
        \centering
        \includegraphics[]{slant}
    \end{figure*}

    \textbf{(a)} \\
    The line $y=x$ makes an angle of $\frac{\pi}{4}$ with the positive $x$-axis because a 45-45-90 triangle can be drawn with the hypotenuse being the red line and the base being the $x$-axis on the interval $[0,3]$. \\

    The equation for the line $y=x\sqrt{3}$ can be rearranged such that it represents a tangent relationship like so:

    \begin{align*}
        y   &= x\sqrt{3} \\
        \frac{y}{x} &= \sqrt{3} \\
        \arctan(\sqrt{3})   &= \frac{\pi}{3}
    \end{align*}

    Thus, the angle between the line $y=x\sqrt{3}$ and the positive $x-$axis is $\frac{\pi}{3}$. Similarly, the equation for the line $y=\frac{x}{\sqrt{3}}$ can be rearranged such that it represents a tangent
    relationship:

    \begin{align*}
        y   &= \frac{x}{\sqrt{3}} \\
        \frac{y}{x} &= \frac{1}{\sqrt{3}} \\
        \arctan{\left(\frac{1}{\sqrt{3}}\right)}    &= \frac{\pi}{6}
    \end{align*}

    Thus, the angle between the line $y=\frac{x}{\sqrt{3}}$ and the positive $x$-axis is $\frac{\pi}{6}$. \\

    \textbf{(b)} \\
    The normal to the red line $y=x\sqrt{3}$ at A is known to pass through the intersection of $x=3$ and $y=\frac{x}{\sqrt{3}}$ which is $\left(3,\frac{3}{\sqrt{3}})=(3,\sqrt{3})$. A normal to $y=x\sqrt{3}$ will be of
    the form

    \[
        y = -\frac{1}{\sqrt{3}}x+b
    \]

    Plugging in the point $(3,\sqrt{3})$, we find that the normal to the red line $y=x\sqrt{3}$ that passes through $A$ is given by

    \begin{align*}
        \sqrt{3}    &= -\frac{3}{\sqrt{3}}+b \\
        b           &= 2\sqrt{3} \\
        y           &= -\frac{x}{\sqrt{3}}+2\sqrt{3}
    \end{align*}

    The point $A$ is the intersection of the normal above and the red line $y=x\sqrt{3}$. That is, $A$ is the solution to the system

    \[
        \begin{cases}
            y = -\frac{x}{\sqrt{3}}+2\sqrt{3} \\
            y = \frac{x}{\sqrt{3}}
        \end{cases}
    \]

    Solving the system, we see that $A$ is $\left(\frac{3}{2},2\sqrt{3}-\frac{\sqrt{3}}{2}\right)$. The distance OA is given by

    \begin{align*}
        OA  &= \sqrt{\left(\frac{3}{2}\right)^2 + \left(2\sqrt{3}-\frac{\sqrt{3}}{2}\right)^2} = 3
    \end{align*}

    To find the distance OB, we can start off similarly by considering that any normal to $y=x\sqrt{3}$ will have the form

    \[
        y = -\frac{x}{\sqrt{3}}+b
    \]

    and that the normal to $y=x\sqrt{3}$ at B passes through $(3,3)$. Plugging in $(3,3)$, we see that the normal at $B$ is

    \begin{align*}
        3   &= -\frac{3}{\sqrt{3}}+b \\
        b   &= 3 + \sqrt{3} \\
        y   &= -\frac{x}{\sqrt{3}}+3+\sqrt{3}
    \end{align*}

    The point $B$ is the intersection of the normal above and the red line $y=x\sqrt{3}$. That is, $A$ is the solution to the system

    \[
        \begin{cases}
            y = -\frac{x}{\sqrt{3}}+3+\sqrt{3} \\
            y = \frac{x}{\sqrt{3}}
        \end{cases}
    \]

    Solving the system, we see that $B$ is $\left(\frac{3\sqrt{3}+3}{4},\frac{9+5\sqrt{3}}{4}\right)$. The distance $OB$ is given by

    \[
        OB = \sqrt{\left(\frac{3\sqrt{3}+3}{4}\right)^2+\left(\frac{9+5\sqrt{3}}{4}\right)^2} = \sqrt{\frac{9\sqrt{3}+18}{2}}
    \]

    \textbf{(c)} \\
    The lengths of $P_z Q_z$ and $P_z R_z$ can be computed readily by setting up a trigonometric function relating the known side length $z$ and the angle the slant lines make with the positive $x$-axis. Some sketches
    of the right triangles used to form these trigonometric relationships are below for clarity. \\

    \begin{figure}[hbt!]
        \centering
        \begin{subfigure}[b]{.45\textwidth}
            \includegraphics[scale=0.06]{triangles}
        \end{subfigure}
        \begin{subfigure}[b]{.45\textwidth}
            \includegraphics[scale=0.12]{triangle3}
        \end{subfigure}
    \end{figure}

    The angle $\angle P_z O Q_z$ was found by subtracting the angle the line $y=x$ makes with the $x$-axis $\left(\frac{\pi}{4}\right)$ from the angle the line $y=x\sqrt{3}$ makes with the $x$-axis
    $\left(\frac{\pi}{3}\right)$. This gives us an angle of $\frac{\pi}{12}$. Similarly, the angle $\angle P_z O R_z$ was found by subtracting the angle the line $y=\frac{x}{\sqrt{3}}$ makes with the $x$-axis
    $\left(\frac{\pi}{6}\right)$ from the angle the line $y=x\sqrt{3}$ makes with the $x$-axis $\left(\frac{\pi}{3}\right)$. This gives us an angle of $\frac{\pi}{6}$. Using the definition of the tangent, we have that

    \begin{align*}
        \tan{\left(\frac{\pi}{12}\right)}   &= \frac{P_z Q_z}{z} \\
        P_z Q_z                             &= z\tan{\left(\frac{\pi}{12}\right)} \\
                                            &= z(2-\sqrt{3})
    \end{align*}

    and

    \begin{align*}
        \tan{\left(\frac{\pi}{6}\right)} &= \frac{P_z R_z}{z} \\
        P_z R_z                          &= z\tan{\left(\frac{\pi}{6}\right)} \\
                                         &= \frac{\sqrt{3}}{3}z
    \end{align*}

    To find the length of $P_z S_z$, I defined a new point, $C$, to represent the intersection of the lines $x=3$ and $y=x\sqrt{3}$. At point $C$, $y=3\sqrt{3}$. The distance $OC$ can be found to be 6 as shown:

    \begin{align*}
        OC  &= \sqrt{3^2 + (3\sqrt{3})^2} \\
            &= \sqrt{9+27} \\
            &= 6
    \end{align*}

    Because the length of $OP_z$ is $z$, $P_z C$ must have a length of $6-z$. Additionally, a right triangle is bounded by the lines $y=x\sqrt{3}$, $x=3$, and the $x$-axis. One of the non-right angles was found in part
    A to be $\frac{\pi}{3}$, hence the other non-right angle, coincidentally the angle $\angle OC S_z$, is $\frac{\pi}{6}$. Again by the definition of the tangent, it follows that

    \begin{align*}
        \tan{\left(\frac{\pi}{6}\right)}    &= \frac{6-z}{P_z S_z} \\
        P_z S_z                             &= \frac{6-z}{\tan{\left(\frac{\pi}{6}\right)}} \\
                                            &= \frac{6-z}{\frac{\sqrt{3}}{3}} \\
                                            &= 6\sqrt{3} -z\sqrt{3}
    \end{align*}

    \textbf{(d)} \\
    The volume, $V$, is as follows.

    \begin{align*}
        V   &= 
    \end{align*}



\end{document}