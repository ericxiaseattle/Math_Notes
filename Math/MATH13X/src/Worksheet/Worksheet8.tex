% Preamble
\documentclass{article}


% Package Imports
\usepackage{../../../../mypackages}


% Macros
\usepackage{../../../../mymacros}


% Homework Details and Basic Document Settings
\pagestyle{fancy}
\lhead{\textbf{Eric Xia}}
\chead{MATH134 (Professor Ebru Bekyel): Worksheet 8}
\cfoot{\thepage}

\renewcommand\headrulewidth{0.4pt}
\renewcommand\footrulewidth{0.4pt}


% Title Page
\title{
    \vspace{2in}
    \textmd{\textbf{MATH134: Worksheet 8}}\\
    \normalsize\vspace{0.1in}\small{Due on November 20, 2020}\\
    \vspace{0.1in}\large{\textit{Professor Ebru Bekyel}}
    \vspace{3in}
}

\author{\textbf{Eric Xia}}
\date{}


% Problem Headers and Footers
\fancypagestyle{2}{\rhead{Problem 1A}}
\fancypagestyle{3}{\rhead{Problem 2}\fancyfoot[L]{Problem 3A continued on next page\ldots}}
\fancypagestyle{4}{\rhead{Problem 3A}}
\fancypagestyle{5}{\rhead{Problem 3C}\fancyfoot[L]{Problem 3D continued on next page\ldots}}
\fancypagestyle{6}{\rhead{Problem 3D}}

%-------------------------------------------------------------------------------------------------------------------------------------------------------------------------------------------------------------------------
%-------------------------------------------------------------------------------------------------------------------------------------------------------------------------------------------------------------------------
%-------------------------------------------------------------------------------------------------------------------------------------------------------------------------------------------------------------------------
\begin{document}

    \maketitle
    \pagebreak

    \thispagestyle{2}



    \begin{tbhtheorem}{Problem 1A}
        Add the fractions on the left and set the numerators of the two sides equal:

        \[
            \frac{A}{x-1} + \frac{B}{x-2} = \frac{2x+3}{(x-1)(x-2)}
        \]

        The numerators are the same for every value of $x$. Pick two "nice" values of $x$ to find $A$ and $B$ quickly.
    \end{tbhtheorem}

    \begin{align*}
        2x+3    &= A(x-2) + B(x-1) \\
        x = 2   &\implies B = 7 \\
        x = 1   &\implies A = -5 \\
        \frac{2x+3}{(x-1)(x-2)} &= -\frac{5}{x-1} + \frac{7}{x-2}
    \end{align*}

    \begin{tbhtheorem}{Problem 1B}
        Evaluate

        \[
            \int \frac{3x+4}{x^2-3x-4}dx
        \]
    \end{tbhtheorem}

    The integrand can be decomposed into linear partial fractions like so:

    \begin{align*}
        \frac{3x+4}{x^2-3x-4}   &= \frac{3x+4}{(x-4)(x+1)} \\
                                &= \frac{A}{x-4} + \frac{B}{x+1} \\
        x = 4                   &\implies B = \frac{16}{5} \\
        x = -1                  &\implies A = -\frac{1}{5} \\
        \frac{3x+4}{x^2-3x-4}   &= -\frac{1}{5(x-4)} + \frac{16}{5(x+1)}
    \end{align*}

    Thus, the integral can be computed as follows.

    \begin{align*}
        \int \frac{3x+4}{x^2-3x-4}dx    &= -\frac{1}{5}\int \frac{1}{x-4} + \frac{16}{5}\int\frac{1}{x+1} \\
                                        &= -\frac{1}{5}\ln{|x+1|} + \frac{16}{5}\ln{|x-4|}+C
    \end{align*}

    \begin{tbhtheorem}{Problem 1C}
        When the power of the linear term is more than one, then you get more partial fractions, as many as the power of the term:

        \[
            \frac{2x^2+1}{(x+1)^3} = \frac{A}{x+1} + \frac{B}{(x+1)^2} + \frac{C}{(x+1)^3}
        \]

        Note that you run out of "nice" $x$ values before you get all the constants. Find $A$, $B$, $C$.
    \end{tbhtheorem}

    Multiplying each term in the given equality by $(x+1)^3$, we get

    \begin{align*}
        2x^2 + 1    &= A(x+1)^2 + B(x+1) + C \\
        x = -1      &\implies C = 3 \\
        2x^2 + 1    &= A(x+1)^2 + B(x+1) + 3 \\
        2x^2 + 1    &= Ax^2 + 2A + A + Bx + B + 3 \\
        2x^2 + 1    &= Ax^2 + x(2A + B) + (A+ B + 3)
    \end{align*}

    Matching each term of the expression $2x^2+1$ to the coefficients on the right-hand side of the equality, we form a system of equations to solve:

    \[
        \begin{bmatrix*}
            A + B + 3 &= 1 \\
            2A + B    &= 0 \\
            A         &= 2
        \end{bmatrix*}
    \]

    Solving this system, we find that $A=2$ and $B=-4$. Recall that $C=3$.

    \pagebreak
    \thispagestyle{3}


    \begin{tbhtheorem}{Problem 2}
        Partial fractions only work when the degree of the numerator is less than the degree of the denominator. Evaluate

        \[
            \int \frac{x^4 - 3x+1}{x^3-x}dx
        \]

        by first using long division and then using partial fractions on the remainder term.
    \end{tbhtheorem}

    The image below demonstrates performing long division on this fraction.
    
    \begin{figure*}[hbt!]
        \centering
        \includegraphics[scale=0.1]{longdiv}
    \end{figure*}

    Thus, the fraction can be rewritten and further decomposed like so:

    \begin{align*}
        \frac{x^4-3x+1}{x^3-x}  &= x + \frac{x^2-3x+1}{x^3-x} \\
                                &= x + \frac{x^2-3x+1}{x(x+1)(x-1)}
    \end{align*}

    Then

    \begin{align*}
        \frac{x^2-3x+1}{x(x+1)(x-1)}    &= \frac{A}{x} + \frac{B}{x+1} + \frac{C}{x-1} \\
        x^2 - 3x + 1                    &= A(x+1)(x-1) + Bx(x-1) + Cx(x+1) \\
        x = 0                           &\implies A = -1 \\
        x = -1                          &\implies B = \frac{5}{2} \\
        x = 1                           &\implies C = -\frac{1}{2} \\
        \frac{x^2-3x+1}{x(x+1)(x-1)}    &= -\frac{1}{x} + \frac{5}{2(x+1)}-\frac{1}{2(x-1)}
    \end{align*}

    Hence, the integral can be computed as follows.

    \begin{align*}
        \int \frac{x^4-3x+1}{x^3-x}dx   &= \int x dx - \int \frac{1}{x} + \frac{5}{2} \int \frac{1}{x+1}dx - \frac{1}{2}\int \frac{1}{x-1}dx \\
                                        &= \frac{x^2}{2} - \ln{|x|} + \frac{5}{2}\ln{|x+1|} - \frac{1}{2}\ln{|x-1|} + C
    \end{align*}

    \begin{tbhtheorem}{Problem 3A}
        Quadratic factors (which cannot be factored into linear terms) get linear terms in their numerators. Add the fractions on the right and set the numerators of the two sides equal:

        \[
            \frac{x^2-1}{(x^2+1)(x+3)} = \frac{Ax+B}{x^2+1} + \frac{C}{x+3}
        \]

        Again, the numerators will be the same for every value of $x$. Find $A$, $B$, and $C$.
    \end{tbhtheorem}

    Multiplying each term in the given equality by the denominator of the left-hand term,

    \begin{align*}
        x^2-1   &= (Ax+B)(x+3) + C(x^2+1) \\
        x = -3  &\implies C = \frac{4}{5} \\
        x^2 - 1 &= (Ax+B)(x+3) + \frac{4}{5}(x^2+1) \\
        x^2 - 1 &= Ax^2 + 3Ax + Bx + 3B + \frac{4}{5}x^2 + \frac{4}{5} \\
        x^2 - 1 &= x^2\left(A + \frac{4}{5}\right) + x\left(B+3A\right) + \left(3B + \frac{4}{5}\right)
    \end{align*}

    \pagebreak
    \thispagestyle{4}

    Matching each term in the left-hand side of the equality with its coefficient, we form a system of equations like so:

    \[
        \begin{bmatrix}
            3B + \frac{4}{5} = -1 \\
            3A + B = 0          \\
            3A + \frac{4}{5} = 1
        \end{bmatrix}
    \]

    Solving the above system, we find that $A = \frac{1}{5}$ and $B=-\frac{3}{5}$. Recall that $C=\frac{4}{5}$.

    \begin{tbhtheorem}{Problem 3B}
        Evaluate

        \[
            \int \frac{x+4}{x^3+2x^2+2x}dx
        \]

        by first using partial fractions.
    \end{tbhtheorem}

    Factoring the denominator of the integrand,

    \begin{align*}
        \frac{x+4}{x^3+2x^2+2x} &= \frac{x+4}{x\left(x^2+2x+2\right)}
    \end{align*}

    Then,

    \begin{align*}
        \frac{x+4}{x\left(x^2+2x+2\right)}  &= \frac{A}{x} + \frac{Bx+C}{x^2+2x+2}
    \end{align*}

    Multiplying each term in the above equality by the denominator of the left-hand side fraction,

    \begin{align*}
        x+4 &= A\left(x^2 + 2x+2\right) + x(Bx+C) \\
        x=0 &\implies A=2 \\
        x+4 &= 2\left(x^2+2x+2\right) + x(Bx+C) \\
        x+4 &= 2x^2 + 4x + 4 + Bx^2 + Cx \\
        x+4 &= x^2\left(B+2\right) + x\left(C+4\right) + 4 \\
    \end{align*}

    Matching each term on the left-hand side with its respective coefficient from the right-hand side of the above equality, we can form the system of equations below.

    \[
        \begin{bmatrix}
            4 + C = 1 \\
            2 + B = 0
        \end{bmatrix}
    \]

    Solving the system gives $B=-2$ and $C=-3$. Thus, the integral can be decomposed and computed like so:

    \begin{align*}
        \int \frac{x+4}{x^3+2x^2+2x}dx  &= \int \frac{2}{x} dx + \int \frac{-2x-3}{x^2+2x+2}dx \\
                                        &= 2\ln{x} + \int \frac{-2x-3}{x^2+2x+2}dx \\
                                        &= 2\ln{x} - \int \frac{2x}{x^2+2x+2} - \int \frac{3}{x^2+2x+2} \\
                                        &= 2\ln{x} - 2\int\frac{x}{x^2+2x+2} - 3\int\frac{1}{x^2+2x+2} \\
                                        &= 2\ln{x} - 2\int \frac{x}{\left(x+1\right)^2 + 1}dx - 3\int\frac{1}{\left(x+1\right)^2+1}dx \\
        u = x+1                         &\implies 2\ln{x} - 2\int \frac{u-1}{u^2 + 1}du - 3\int \frac{1}{u^2+1}du \\
                                        &= 2\ln{x} - 2\left[\int \frac{u}{u^2+1}du-\int\frac{1}{u^2+1}du\right]-3\int\frac{1}{u^2+1}du \\
                                        &= 2\ln{x} - 2\left[\frac{1}{2}\ln{\left|u^2+1\right|}-\arctan{(u)}\right]-3\arctan{(u)} + C \\
                                        &= 2\ln{x} -\ln{\left|u^2+1\right|} + 2\arctan{(u)}-3\arctan{(u)} + C \\
                                        &= 2\ln{x} -\ln{\left|u^2+1\right|} -\arctan{(u)} + C \\
                                        &= 2\ln{x} - \ln{\left|x^2+2x+2\right|}-\arctan{(x+1)} + C
    \end{align*}

    \pagebreak
    \thispagestyle{5}


    \begin{tbhtheorem}{Problem 3C}
        To integrate the above rational function, after doing partial fractions, you had to complete the square for the quadratic term. It may be better to complete the square \textbf{before} doing the partial fractions:

        \[
            \frac{x+4}{x^3+2x^2+2x} = \frac{A}{x} + \frac{B(x+1)+C}{\left(x+1\right)^2+1}
        \]
    \end{tbhtheorem}

    Multiplying each term in the above equality by the left-hand side's denominator, we have

    \begin{align*}
        x+4 &= A\left[\left(x+1\right)^2+1\right] + x\left[B(x+1)+C\right] \\
            &= A\left(x+1\right)^2 + A + x\left[B(x+1)+C\right] \\
            &= A\left(x+1\right)^2 + x\left[B(x+1)+C\right] + A \\
            &= A\left(x^2+2x+2\right) + x\left(Bx+C\right) \\
        x=0 &\implies A=2 \\
        x+4 &= 2\left(x^2+2x+2\right) + x\left(Bx+C\right) \\
        x+4 &= 2x^2 + 4x + 4 + Bx^2 + Cx \\
        x+4 &= x^2\left(B+2\right) + x\left(C+4\right) + 4
    \end{align*}

    Matching the terms on the left-hand side of the above equality with their respective coefficients represented by the right-hand side's terms, we can form a system like so:

    \[
        \begin{bmatrix}
            4 + C=1 \\
            2 + B = 0
        \end{bmatrix}
    \]

    Solving the above system, we find that $B=-2$ and $C=-3$. The rest of this problem is identical to that of Problem 3B so I will just copy and paste my response for that problem here. Thus, the integral can be
    decomposed and computed like so:

    \begin{align*}
        \int \frac{x+4}{x^3+2x^2+2x}dx  &= \int \frac{2}{x} dx + \int \frac{-2x-3}{x^2+2x+2}dx \\
        &= 2\ln{x} + \int \frac{-2x-3}{x^2+2x+2}dx \\
        &= 2\ln{x} - \int \frac{2x}{x^2+2x+2} - \int \frac{3}{x^2+2x+2} \\
        &= 2\ln{x} - 2\int\frac{x}{x^2+2x+2} - 3\int\frac{1}{x^2+2x+2} \\
        &= 2\ln{x} - 2\int \frac{x}{\left(x+1\right)^2 + 1}dx - 3\int\frac{1}{\left(x+1\right)^2+1}dx \\
        u = x+1                         &\implies 2\ln{x} - 2\int \frac{u-1}{u^2 + 1}du - 3\int \frac{1}{u^2+1}du \\
        &= 2\ln{x} - 2\left[\int \frac{u}{u^2+1}du-\int\frac{1}{u^2+1}du\right]-3\int\frac{1}{u^2+1}du \\
        &= 2\ln{x} - 2\left[\frac{1}{2}\ln{\left|u^2+1\right|}-\arctan{(u)}\right]-3\arctan{(u)} + C \\
        &= 2\ln{x} -\ln{\left|u^2+1\right|} + 2\arctan{(u)}-3\arctan{(u)} + C \\
        &= 2\ln{x} -\ln{\left|u^2+1\right|} -\arctan{(u)} + C \\
        &= 2\ln{x} - \ln{\left|x^2+2x+2\right|}-\arctan{(x+1)} + C
    \end{align*}


    \begin{tbhtheorem}{Problem 3D}
        When the power of the quadratic factor is more than one, you get as many partial fractions as the power of that term. Find $A,B,C$, and $D$ in

        \[
            \frac{x^3+x+1}{\left(x^2+4\right)^2} = \frac{Ax+B}{x^2+4} + \frac{Cx+D}{\left(x^2+4\right)^2}
        \]
    \end{tbhtheorem}

    Multiplying each term in the given equality by the left-hand side's denominator, we have

    \begin{align*}
        x^3 + x + 1 &= \left(Ax+B)\left(x^2+4\right) + Cx+D \\
        x^3 + x + 1 &= Ax^3 + 4Ax + Bx^2 + 4B + Cx + D \\
        x^3 + x + 1 &= Ax^3 + Bx^2 + x\left(4A + C\right) + \left(4B + D\right)
    \end{align*}

    \pagebreak
    \thispagestyle{6}

    Matching the terms on the left-hand side of the above equality with their respective coefficients represented by the right-hand side's terms, we can form a system like so:

    \[
        \begin{bmatrix}
            4B + D = 1 \\
            4A + C = 1 \\
            A = 1 \\
            B = 0
        \end{bmatrix}
    \]

    Solving the system, we fine that $A=1$, $B=0$, $C=-3$, and $D=1$.


    \begin{tbhtheorem}{Problem 3E}
        Write down the partial fractions expansion for

        \[
            \frac{5x^5 + 4x^4 + 3x^3 + 2x^2 + x + 1}{x^3(x+1)(2x+1)^2(x^2+1)(x^2+6x+13)^2}
        \]

        but do not evaluate the constants.
    \end{tbhtheorem}

    The partial fractions expansion is as follows.

    \[
        \frac{A}{x} + \frac{B}{x^2} + \frac{C}{x^3} + \frac{D}{x+1} + \frac{E}{2x+1} + \frac{F}{\left(2x+1\right)^2} + \frac{Gx+H}{x^2+1} + \frac{Ix+J}{x^2+6x+13} + \frac{Kx+L}{(x^2+6x+13)^2}
    \]



\end{document}
