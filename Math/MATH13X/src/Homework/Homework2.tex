% Preamble
\documentclass{article}


% Package Imports
\usepackage{../../../../mypackages}


% Macros
\usepackage{../../../../mymacros}


% Homework Details and Basic Document Settings
\pagestyle{fancy}
\lhead{\textbf{Eric Xia}}
\chead{MATH134 (Professor Ebru Bekyel): Week 1 Assignment}
\cfoot{\thepage}

\renewcommand\headrulewidth{0.4pt}
\renewcommand\footrulewidth{0.4pt}


% Title Page
\title{
    \vspace{2in}
    \textmd{\textbf{MATH134: Week 2 Assignment}}\\
    \normalsize\vspace{0.1in}\small{Due on October 12, 2020 at 5:45 PM}\\
    \vspace{0.1in}\large{\textit{Professor Ebru Bekyel}}
    \vspace{3in}
}

\author{\textbf{Eric Xia}}
\date{}


% Problem Headers and Footers
\fancypagestyle{page2}{\rhead{Section 2.2 Problem 38}}
% \fancypagestyle{page3}{\rhead{Section 1.2 Problem 74}}
% \fancypagestyle{page4}{\rhead{Section 1.3 Problem 58}\fancyfoot[L]{Section 1.4 Problem 52 continued on next page \ldots}}
% \fancypagestyle{page5}{\rhead{Section 1.4 Problem 52}}
% \fancypagestyle{page6}{\rhead{Section 1.8 Problem 6}\fancyfoot[L]{Section 1.8 Problem 6 continued on next page \ldots}}
% \fancypagestyle{page7}{\rhead{Section 1.8 Problem 6}}


%-------------------------------------------------------------------------------------------------------------------------------------------------------------------------------------------------------------------------
%-------------------------------------------------------------------------------------------------------------------------------------------------------------------------------------------------------------------------
%-------------------------------------------------------------------------------------------------------------------------------------------------------------------------------------------------------------------------
\begin{document}

    \maketitle
    \pagebreak


    \thispagestyle{page2}

    \begin{tbhtheorem}{Section 2.2 Problem 38}
        Give an $\epsilon,\delta$ proof for the statement
        \[
            \lim_{x\to 0} (2-5x) = 2
        \]
    \end{tbhtheorem}

    \begin{proof}
        Let $\epsilon$ be given (arbitrary) and $\delta=\frac{\epsilon}{5}$. Suppose that for every $\epsilon>0$ there exists $\delta >0$ such that $\forall x$,

        \begin{align*}
            0 < |x-0| < \delta           & \implies |(2-5x)-2|<\epsilon \\
            0 < |x| < \frac{\epsilon}{5} & \implies |-5x| < \epsilon
        \end{align*}

        Because the inequality above is true, therefore

        \[
            \lim_{x\to 0} (2-5x) = 2
        \]
    \end{proof}


    \begin{tbhtheorem}{Section 2.2 Problem 52}
        Give an $\epsilon,\delta$ proof for the statement
        \[
            \lim_{x\to 3} \sqrt{x+1} = 2
        \]
    \end{tbhtheorem}

    \begin{proof}

    \end{proof}

%    \begin{tbhtheorem}{Section 2.2 Problem 54}
%        Prove that, for the function
%        \begin{align*}
%            g(x) &=
%            \begin{cases}
%                x,  & x \text{ is rational} \\
%                0,  & x \text{ is irrational,}
%            \end{cases}
%            \\
%            \lim_{x\to 0} g(x) &= 0
%        \end{align*}
%    \end{tbhtheorem}
%
%    \begin{tbhtheorem}{Section 2.3 Problem 61E}
%        Calculate
%        \[
%            \lim_{h\to 0} \frac{f(x+h)-f(x)}{h}
%        \]
%        for the function
%        \[
%            f(x) = x^n,
%        \]
%        where $n$ is an arbitrary positive integer. Make a guess and confirm your guess by induction.
%    \end{tbhtheorem}
%
%    \begin{tbhtheorem}{Section 2.4 Problem 36}
%        Let
%        \[
%            f(x) =
%            \begin{cases}
%                A^2 x^2,        & x \leq 2 \\
%                (1-A) x,        & x > 2.
%            \end{cases}
%        \]
%        For what values of $A$ is $f$ continuous at 2?
%    \end{tbhtheorem}
%
%    \begin{tbhtheorem}{Section 2.4 Problem 49 (Important)}
%        Prove that $f$ is continuous at $c$ if and only if
%        \[
%            \lim_{h\to 0} f(c+h) = f(c).
%        \]
%    \end{tbhtheorem}
%
%    \begin{tbhtheorem}{Section 2.4 Problem 53}
%        Suppose the function $f$ has the property that there exists a number $B$ such that
%        \[
%            |f(x) - f(c) | \leq B|x-c|, \forall c-p \leq x \leq c+p
%        \]
%        Prove that $f$ is continuous at $c$.
%    \end{tbhtheorem}
%
%    \begin{tbhtheorem}{Section 2.5 Problem 36}
%        Evaluate the limit, taking $a$ and $b$ as nonzero constants:
%        \[
%            \lim_{x\to 0} \frac{\sin{(ax)}}{\sin{(bx)}}
%        \]
%    \end{tbhtheorem}
%
%    \begin{tbhtheorem}{Section 2.5 Problem 43}
%        Show that $lim_{x\to 0} x \sin{\left(\frac{1}{x}\right)}=0$. \\
%        HINT: Use the pinching theorem.
%    \end{tbhtheorem}
%
%    \begin{tbhtheorem}{Section 2.5 Problem 46}
%        Let $f$ be the Dirichlet function
%        \[
%            f(x) =
%            \begin{cases}
%                1,  & x \text{ is rational} \\
%                0,  & x \text{ is irrational.}
%            \end{cases}
%        \]
%        Show that $\lim_{x\to 0} x f(x) = 0$. \\
%        Note: This problem is basically the same as 2.2.38 but prof wants a distinct proof for this one.
%    \end{tbhtheorem}


\end{document}




