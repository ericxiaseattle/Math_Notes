% Preamble
\documentclass{article}


% Package Imports
\usepackage{../../../../mypackages}



% Macros
\usepackage{../../../../mymacros}


% Homework Details and Basic Document Settings
\pagestyle{fancy}
\lhead{\textbf{Eric Xia}}
\chead{MATH134 (Professor Ebru Bekyel): Homework 7}
\cfoot{\thepage}

\renewcommand\headrulewidth{0.4pt}
\renewcommand\footrulewidth{0.4pt}


% Title Page
\title{
    \vspace{2in}
    \textmd{\textbf{MATH134: Homework 7}}\\
    \normalsize\vspace{0.1in}\small{Due on November 16, 2020 at 5:45 PM}\\
    \vspace{0.1in}\large{\textit{Professor Ebru Bekyel}}
    \vspace{3in}
}

\author{\textbf{Eric Xia}}
\date{}


% Problem Headers and Footers
\fancypagestyle{2}{\rhead{Section 5.5.9 Problem 26}}

%-------------------------------------------------------------------------------------------------------------------------------------------------------------------------------------------------------------------------
%-------------------------------------------------------------------------------------------------------------------------------------------------------------------------------------------------------------------------
%-------------------------------------------------------------------------------------------------------------------------------------------------------------------------------------------------------------------------
\begin{document}

    \maketitle
    \pagebreak

    \thispagestyle{2}


    \begin{tbhtheorem}{Section 5.9 Problem 26}
        For a rod that extends from $x=a$ to $x=b$ and has mass density $\lambda = \lambda (x)$, the integral

        \[
            \int_a^b (x-c)\lambda (x)dx
        \]

        gives what is called the \textit{mass moment} of the rod about the point $x=c$. Show that the mass moment about the center of the mass is zero. (The center of mass can be defined as the point about which the mass
        moment is zero.)
    \end{tbhtheorem}

    A definition of the center of mass of a rod is given by the mass moment, $M$, divided by the mass of the rod, $m$ s.t.

    \begin{align*}
        x_{\text{cm}}   &= \frac{M}{m} \\
                        &= \frac{\int_a^b (x-c)\lambda(x)dx}{\int_a^b\lambda(x)dx} \\
                        &= 0
    \end{align*}

    The fraction defining the center of mass is only equal to zero when

    \[
        \int_a^b (x-c)\lambda(x)dx=0
    \]

    and thus the mass moment about the center of mass must be zero.




    \begin{tbhtheorem}{Section 5.9 Problem 35}
        Prove that two distinct continuous functions cannot have the same average on every interval.
    \end{tbhtheorem}

    Let $f(x)$ and $g(x)$ be two distinct continuous functions. For a contradiction, assume that the averages of $f(x)$ and $g(x)$ are equal on every interval $[a,b]$ s.t.

    \begin{align*}
        \frac{1}{b-a}\int_a^b f(x)dx    &= \frac{1}{b-a}\int_a^b g(x) \\
        \int_a^b f(x)dx                 &= \int_a^b g(x) \\
    \end{align*}

    Let $F(x)$ be defined as the antiderivative of $f(x)$ and $G(x)$ be defined as the antiderivative of $g(x)$. By the Fundamental Theorem of Calculus, it follows that

    \begin{align*}
        F(b) - F(a) &= G(b) - G(a) \\
        F(b) - G(b) &= F(a) - G(a)
    \end{align*}

    Let the constant $C$ represent $F(b)-G(b)$ such that

    \begin{align*}
        F(b) - G(b) &= C \\
        F(b)        &= G(b) + C
    \end{align*}

    As $b$ is an arbitrary value of $x$, this equality can be generalized to

    \[
        F(x) &= G(x) + C
    \]

    Differentiating,

    \begin{align*}
        \frac{d}{dx}F(x)    &= \frac{d}{dx} (G(x) + C) \\
        f(x)                &= g(x)
    \end{align*}

    This contradicts the assumption that $f(x)$ and $g(x)$ are distinct, hence, two distinct continuous functions cannot have the same average on every interval.

    \pagebreak

    \begin{tbhtheorem}{Section 6.4 Problem 30}
        Find the volume of the solid generated by revolving the entire triangular region of Exercise 29.\\
        (a) about the $x$-axis \\
        (b) about the $y$-axis
    \end{tbhtheorem}

    \begin{figure*}[hbt!]
        \centering
        \includegraphics[scale=0.075]{omega2}
    \end{figure*}

    \textbf{(a)} \\
    In the labeled diagram above, an equation for line $\overline{AB}$ can be written as

    \[
        y = \overline{AB} = \frac{h}{a}x
    \]

    The slope of line $\overline{BC}$ can be written as

    \[
        -\frac{h}{b-a}x
    \]

    When $x=b$, $y=0$ and so an equation for $\overline{BC}$ can be written as

    \begin{align*}
        y = \overline{BC} = -\frac{h}{b-a}(x+b)
    \end{align*}

    Thus, the volume of the solid generated by revolving the region about the $x$-axis is given by

    \begin{align*}
        V   &= \pi\left(\int_0^a \left(\frac{h}{a}x\right)^2 dx + \int_a^b \left(-\frac{h}{b-a}(x-b)\right)^2 dx\right) \\
            &= \pi\left(\frac{h^2}{a^2}\int_0^a x^2 dx + \frac{h^2}{(a-b)^2}\int_a^b(x-b)^2 dx) \\
            &= \pi\left(\frac{h^2}{a^2}\cdot\frac{a^3}{3} + \frac{h^2}{(a-b)^2}\left(-\frac{(a-b)^3}{3}\right)\right) \\
            &= \pi\frac{h^2b}{3}
    \end{align*}

    \textbf{(b)} \\
    Line $\overline{AB}$ can be rewritten in terms of $x$ like so:

    \[
        y = \overline{AB} = \frac{a}{h}y
    \]

    Similarly, line $\overline{BC}$ can be rewritten in terms of $x$ as follows.

    \[
        y = \overline{BC} = -\frac{b-a}{h}y-b
    \]

    Hence, the volume of the solid generated by revolving the region about the $y$-axis is given by

    \begin{align*}
        V   &= \pi\int_0^h\left(\left(-\frac{b-a}{h}y-b\right)^2-\left(\frac{a}{h}y\right)^2\right)dy \\
            &= \pi\int_0^h\left(\frac{b^2}{h^2}y^2 - \frac{2ba}{h^2} y^2 + \frac{2b(b-a)}{h}b+b^2\right)dy \\
            &= \pi\left(\frac{b^2}{h^2}\cdot\frac{y^3}{3} - \frac{2ba}{h^2}\cdot\frac{y^3}{3} + \frac{2b^2 y (b-a)}{h} + b^2 y\right)\Big|_0^h \\
            &= \pi\left(\frac{b^2 h}{3} - \frac{2bha}{3} + 2b^2 (b-a)+b^2 h\right) \\
            &= \pi\left(\frac{b^2 h - 2bha}{3} + 2b^2 (b-a) + b^2 h\right)
    \end{align*}

    \pagebreak


    \begin{tbhtheorem}{Section 6.4 Problems 1\, 2\, 3D}
        (Explain problems 1 and 2 briefly with a picture similar to the lecture and do problem 3). \\
        \textbf{Problem 1}. Show that $\bar{x}V=\int_a^b \pi x\left[f(x)\right]^2 dx$ where $V$ is the volume of $T$. \\

        HINT: Use the following principle: if a solid of volume $V$ consists of a finite number of pieces with volumes $V_1$, $V_2$, \dots, $V_n$ and the pieces have centroids $\bar{x}_1$, $\bar{x}_2$, \dots, $\bar{x}_n$,
        then $\bar{x}V=\bar{x}_1 V_1 + \bar{x}_2 V_2 + \dots + \bar{x}_n V_n$. \\

        Now revolve $\Omega$ around the $y$-axis and let $S$ be the resulting solid. By symmetry, the centroid $S$ lies on the $y$-axis and is determined solely by its $y$-coordinate $\bar{y}.$. \\

        \textbf{Problem 2}. Show that $\bar{y} V = \int_a^b \pi x\left[f(x)\right]^2 dx$, where $V$ is the volume of $S$. \\
        \textbf{Problem 3D}. Use the results in Problems 1 and 2 to locate the centroid of the solid generated by revolving the region below of the graph of $f(x)=\sqrt{x},x\in\ [0,1]$ \\
        (i) about the $x$-axis \\
        (ii) about the $y$-axis
    \end{tbhtheorem}

    \begin{tbhtheorem}{Section 6.5 Problem 40}
        A storage tank in the form a hemisphere topped by a cylinder is filled with oil that weights 60 pounds per cubic foot. The hemisphere has a 4-foot radius; the height of the cylinder is 8 feet. \\
        (a) How much work is required to pump the oil to the top of the tank? \\
        (b) How long would it take a $\frac{1}{2}-$horsepower motor to empty out the tank?
    \end{tbhtheorem}

    \textbf{(a)} \\
    The volume of the hemispherical bottom of the tank is

    \[
        V_{\text{hemi}} = \frac{4}{6}\pi r^3 \text{ ft}^3,
    \]

    where $r$ is the radius. Thus, the weight of the hemispherical part of the tank is given by

    \begin{align*}
        F_{g{\text{ hemi}}}   &= \frac{4}{6} \pi r^3 \text{ ft}^3 \cdot \frac{60 \text{ lbs}}{\text{ft}^3}  \\
                            &= 40\pi r^3 \text{ lbs.} \\
                            &= 558.980012 r^3 \text{ N}
    \end{align*}

    The work to pump oil to the top of the hemisphere is then

    \begin{align*}
        W_{\text{hemi}} &= 558.980012 \int_0^4 r^3 dr \\
                        &= 558.980012 \left[\frac{r^4}{4}\right]\Big|_0^4 \\
                        &= 35774.72077 \text{ J}
    \end{align*}

    The volume of the cylindrical part of the tank is

    \[
        V_{\text{cylinder}} = \pi r^2 h \text{ ft}^3 = 16\pi h \text{ ft}^3,
    \]

    where $h$ is the instantaneous height of the cylinder. Thus, the weight of the cylindrical part of the tank is given by

    \begin{align*}
        F_{g{\text{ cylinder}}} &= 16\pi h \text{ ft}^3 \cdot \frac{60 \text{ lbs}}{\text{ft}^3} \\
                                &= 960 \pi h \text{ lbs.} \\
                                &= 13415.520288 \text{ N}
    \end{align*}

    The work to pump oil from the bottom to the top of the cylinder is then

    \begin{align*}
        W_{\text{cylinder}} &= 13415.520288 \int_0^8 h dh \\
                            &= 13415.520288 \left[\frac{h^2}{2}\right]\Big|_0^8 \\
                            &= 429296.6492 \text{ J}
    \end{align*}

    Adding the works to pump oil through the hemisphere and the cylinder, it follows that the total work is

    \[
        W = 429296.6492 \text{ J } + 35774.72077 \text{ J } = 465071.37 \text{ J}
    \]

    \textbf{(b)} \\
    The time that it takes a $\frac{1}{2}-$horsepower engine to fill the tank is given by the power $P$ divided by the work necessary. $\frac{1}{2}$ horsepower is equal to 372.849936 W and so the time taken by the motor
    is

    \[
        t = \frac{465071.37 \text{ J}}{372.849936 \text{ W}} = 1247.34196 \text{s} = 20.789 \text{ min.}
    \]


    \begin{tbhtheorem}{Worksheet Problem 3}
        (a) Gasoline is stored in a cylindrical tank of radius 9 feet and length 23 feet lying under ground in a gas station. The tank is buried on its side with the highest part of the tank 7 feet below ground. The tank
        is initially half-full. Suppose that the filler cap of each car is 2 feet above the ground. Express the work done in pumping all the gasoline as an integral. The density of gasoline is 45 pounds per cubic foot. \\
        (b) Show that the work done is the weight of the gasoline times the distance its center of mass travels vertically. (Note that you only need the $y$-coordinate of its center of mass. What are the $x$ and $z$
        (axis not shown) coordinates of the center of mass of the gasoline?).
    \end{tbhtheorem}


    \begin{figure*}[hbt!]
        \centering
        \includegraphics[scale=0.75]{gas}
    \end{figure*}







    \begin{tbhtheorem}{Section 7.1 Problem 52}
        Set

        \[
            f(x) = \int_1^{2x} \sqrt{16+t^4} dt
        \]

        (a) Show that $f$ has an inverse. \\
        (b) Find $\left(f^{-1}\right)'(0)$.
    \end{tbhtheorem}

    \textbf{(a)} \\
    Because $f$ is defined as an integral and integrals are continuous by definition, $f$ is continuous on its domain. The expression

    \[
        \sqrt{16+t^4}
    \]

    is strictly increasing because $t^4$ is strictly increasing. It follows then that for all values of $x$, as $x$ increases, so does $f(x)$ and so $f(x)$ is both a strictly increasing function and one-to-one. Hence,
    $f(x)$ has an inverse.

    \textbf{(b)} \\
    Consider the equality

    \[
        \left(f^{-1}\right)'(x) = \frac{1}{f'\left(f^{-1}(x)\right)}
    \]

    Function $f$ is equal to zero when the bounds of the integrand in its definition are equal s.t.

    \[
        1 = 2x \implies x = \frac{1}{2}
    \]

    It follows then that

    \begin{align*}
        \frac{1}{f'\left(f^{-1}(0)\right)} = \frac{1}{f'\left(\frac{1}{2}\right)}
    \end{align*}

    and

    \begin{align*}
        \left(f^{-1}\right)'(0) &= \frac{1}{f'\left(\frac{1}{2}\right)}
    \end{align*}

    The derivative of $f$ at $\frac{1}{2}$ is as follows.

    \begin{align*}
        f'(x)   &= \frac{d}{dx}\int_1^{2x}\sqrt{16+t^4}dt \\
                &= 2\sqrt{16+(2x)^4} \\
                &= 2\sqrt{16+16x^4} \\
        f'\left(\frac{1}{2}\right)  &= 2\sqrt{16+16\left(\frac{1}{2}\right)^4} \\
                                    &= 2\sqrt{17}
    \end{align*}

    Hence,

    \begin{align*}
        \left(f^{-1}\right)'(0) &= \frac{1}{2\sqrt{17}} \\
                                &= \frac{\sqrt{17}}{34}
    \end{align*}





\end{document}