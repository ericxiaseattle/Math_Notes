% Preamble
\documentclass{article}


% Package Imports
\usepackage{../../../../mypackages}



% Macros
\usepackage{../../../../mymacros}


% Homework Details and Basic Document Settings
\pagestyle{fancy}
\lhead{\textbf{Eric Xia}}
\chead{MATH134 (Professor Ebru Bekyel): Homework 7}
\cfoot{\thepage}

\renewcommand\headrulewidth{0.4pt}
\renewcommand\footrulewidth{0.4pt}


% Title Page
\title{
    \vspace{2in}
    \textmd{\textbf{MATH134: Homework 8}}\\
    \normalsize\vspace{0.1in}\small{Due on November 23, 2020 at 5:45 PM}\\
    \vspace{0.1in}\large{\textit{Professor Ebru Bekyel}}
    \vspace{3in}
    }

\author{\textbf{Eric Xia}}
\date{}


% Problem Headers and Footers
\fancypagestyle{2}{\rhead{Section 7.3 Problem 66}}

%-------------------------------------------------------------------------------------------------------------------------------------------------------------------------------------------------------------------------
%-------------------------------------------------------------------------------------------------------------------------------------------------------------------------------------------------------------------------
%-------------------------------------------------------------------------------------------------------------------------------------------------------------------------------------------------------------------------
\begin{document}

    \maketitle
    \pagebreak

    \thispagestyle{2}


    \begin{tbhtheorem}{Section 7.3 Problem 66}
        Find a formula for the $n$th derivative.

        \[
            \frac{d^n}{dx^n}\left[\ln{(1-x)}\right]
        \]
    \end{tbhtheorem}

    Listing the first few derivatives of the given expression,

    \begin{align*}
        \frac{d}{dx}\left[\ln{(1-x)}\right]     &= \frac{1}{1-x} \\
        \frac{d^2}{dx^2}\left[\ln{(1-x)}\right] &= -\frac{1}{(1-x)^2} \\
        \frac{d^3}{dx^3}\left[\ln{(1-x)}\right] &= (-1)(-2)\frac{1}{(1-x)^3} \\
        \frac{d^4}{dx^4}\left[\ln{(1-x)}\right] &= (-1)(-2)(-3)\frac{1}{(1-x)^4} \\
                                                &= (-1)^3 \cdot\frac{3!}{(1-x)^4}
    \end{align*}

    Thus the $n$th derivative of the expression $\ln{(1-x)}$ can be generalized by the formula

    \[
        \frac{d^n}{dx^n}\left[\ln{(1-x)}\right] = (-1)^{n-1}\cdot\frac{(n-1)!}{(1-x)^n} \text{ } \forall n \geq 1
    \]

    \begin{tbhtheorem}{Section 7.4 Problem 72}
        Prove that for all $x>0$ and all positive integers $n$

        \[
            e^x > 1 + x + \frac{x^2}{2!} + \frac{x^3}{3!} + \dots + \frac{x^n}{n!}.
        \]

        Recall that

        \[
            n! = n(n-1)(n-2)\dots 3 \cdot 2 \cdot 1.
        \]

        HINT:

        \begin{align*}
            e^x &= 1 + \int_{0}^x e^t dt > 1 + \int_0^x dt = 1 + x \\
                &= 1 + \int_0^x e^t dt > 1 + \int_0^x (1+t)dt \\
                &= 1 + x + \frac{x^2}{2},
        \end{align*}

        and so on.
    \end{tbhtheorem}

    \begin{proof}
        Let $S$ be the set of positive integers $n$ for which $x>0$,

        \[
            e^x > 1 + x + \frac{x^2}{2!} + \frac{x^3}{3!} + \dots + \frac{x^n}{n!}.
        \]

        Then $1\in S$ since

        \[
            e^x > 1, \forall x > 0
        \]

        Assume that $k\in S$; that is, assume

        \[
            e^x > 1 + x + \frac{x^2}{2!} + \frac{x^3}{3!} + \dots + \frac{x^k}{k!}.
        \]

        Integrating both sides of this inequality, where the variable $x$ has been changed to $t$, it follows that

        \begin{align*}
            \int_{0}^x e^t dt   &> \int_0^x \left[1+t+\frac{t^2}{2!} + \frac{t^3}{3!} + \dots + \frac{t^k}{k!}\right]dt \\
            e^x - e^0           &> x + \frac{x^2}{2!} + \frac{x^3}{3!} + \dots + \frac{x^{k+1}}{(k+1)!} \text{ (by the inductive hypothesis)} \\
            e^x                 &> 1 + x + \frac{x^2}{2!} + \frac{x^3}{3!} + \dots + \frac{x^{k+1}}{(k+1)!}
        \end{align*}

        and so $k+1\in S$. Thus, by the principle of induction, it can be concluded that all positive integers are in $S$; that is,

        \[
            e^x > 1 + x + \frac{x^2}{2!} + \frac{x^3}{3!} + \dots + \frac{x^n}{n!}, \forall x > 0, n\in \mathbb{Z}^+
        \]
    \end{proof}

    \pagebreak
    \thispagestyle{3}


    \begin{tbhtheorem}{Section 7.4 Problem 73}
        Prove that, if $n$ is a positive integer, then

        \[
            e^x > x^n \text{ for all $x$ sufficiently large.}
        \]

        HINT: Exercise 72.
    \end{tbhtheorem}

    \begin{tcolorbox}[colback = red!10]
        \textbf{Lemma:}
        \begin{proof}
            Let $S$ be the set of positive integers $n$ for which $x>0$,

            \[
                e^x > 1 + x + \frac{x^2}{2!} + \frac{x^3}{3!} + \dots + \frac{x^n}{n!}.
            \]

            Then $1\in S$ since

            \[
                e^x > 1, \forall x > 0
            \]

            Assume that $k\in S$; that is, assume

            \[
                e^x > 1 + x + \frac{x^2}{2!} + \frac{x^3}{3!} + \dots + \frac{x^k}{k!}.
            \]

            Integrating both sides of this inequality, where the variable $x$ has been changed to $t$, it follows that

            \begin{align*}
                \int_{0}^x e^t dt   &> \int_0^x \left[1+t+\frac{t^2}{2!} + \frac{t^3}{3!} + \dots + \frac{t^k}{k!}\right]dt \\
                e^x - e^0           &> x + \frac{x^2}{2!} + \frac{x^3}{3!} + \dots + \frac{x^{k+1}}{(k+1)!} \text{ (by the inductive hypothesis)} \\
                e^x                 &> 1 + x + \frac{x^2}{2!} + \frac{x^3}{3!} + \dots + \frac{x^{k+1}}{(k+1)!}
            \end{align*}

            and so $k+1\in S$. Thus, by the principle of induction, it can be concluded that all positive integers are in $S$; that is,

            \[
                e^x > 1 + x + \frac{x^2}{2!} + \frac{x^3}{3!} + \dots + \frac{x^n}{n!}, \forall x > 0, n\in \mathbb{Z}^+
            \]
        \end{proof}
    \end{tcolorbox}

    \begin{proof}
        Let $S$ be the set of positive integers $n$ for which $x$ being sufficiently large,

        \[
            x^n \leq 1 + x + \frac{x^2}{2!} + \frac{x^3}{3!} + \dots + \frac{x^n}{n!}
        \]

        Then $1\in S$ because

        \[
            x\leq 1
        \]

        as $x$ is arbitrary and its value can be adjusted without restrictions. Assume $k\in S$ s.t.

        \[
            x^k \leq 1 + x + \frac{x^2}{2!} + \frac{x^3}{3!} + \dots + \frac{x^k}{k!}
        \]

        Integrating both sides of this inequality, where the variable $x$ has been changed to $t$, it follows that

        \begin{align*}
        \int_0^x t^k        &\leq \int_0^x \left[1+x+\frac{x^2}{2!} + \frac{x^3}{3!} + \dots + \frac{x^k}{k!}\right] \\
            x^k - x^0       &\leq x + \frac{x^2}{2!} + \frac{x^3}{3!} + \dots + \frac{x^{k+1}}{(k+1)!} \text{ (by the induction hypothesis)} \\
            x^k             &\leq 1 + x + \frac{x^2}{2!} + \frac{x^3}{3!} + \dots + \frac{x^{k+1}}{(k+1)!}
        \end{align*}

        Then $k+1\in S$ and hence all positive integers are in $S$ s.t.

        \[
            x^n \leq 1 + x + \frac{x^2}{2!} + \frac{x^3}{3!} + \dots + \frac{x^n}{n!}
        \]

        Since

        \[
            e^x > 1 + x + \frac{x^2}{2!} + \frac{x^3}{3!} + \dots + \frac{x^n}{n!}, \forall x > 0, n\in \mathbb{Z}^+,
        \]

        it follows that if $x$ is sufficiently large,

        \[
            e^x > x^n \text{ }\forall n\in \mathbb{Z}^+
        \]
    \end{proof}

    \pagebreak
    \thispagestyle{4}

    \begin{tbhtheorem}{Section 7.7 Problem 70}
        A person walking along a straight path at the rate of 6 feet per second is followed by a spotlight that is located 30 feet from the path. How fast is the spotlight turning at the instant the person is 50 feet
        past the point on the path that is closest to the spotlight?
    \end{tbhtheorem}

    Let the distance between the person and the point on the path closest to the spotlight be represented by $a$ and let the distance between the spotlight and the path be represented by $b$. Let $t$ represent time in
    seconds and let $\theta$ be the angle between the endpoints of $a$, relative to the spotlight. The relationship between these variables is depicted in the following diagram.
    
    \begin{figure*}[hbt!]
        \centering
        \includegraphics[scale=0.075]{spotlight}
    \end{figure*}

    The angle $\theta$ can be defined like so:

    \[
        \theta = \arctan{\left(\frac{a}{b}\right)}
    \]

    Differentiating both sides of this definition with respect to $t$, it follows that

    \begin{align*}
        \frac{d\theta}{dt}  &= \frac{1}{1+\left(\frac{a}{b}\right)^2}\cdot\frac{b\frac{da}{dt}-a\frac{db}{dt}}{b^2}
    \end{align*}

    Plugging in the known values,

    \begin{align*}
        \frac{d\theta}{dt}  &= \frac{1}{1+\left(\frac{50}{30}\right)^2}\cdot\frac{(30)(6)\cancel{-(50)(0)}}{30^2} \\
                            &= \frac{1}{1+\frac{25}{9}}\cdot\frac{180}{900} \\
                            &= \frac{9}{170} \text{ rad} \\
                            &= 3.031^{\circ}\text{/s}
    \end{align*}

    Hence, the spotlight is turning at the rate of $3.031^{\circ}$/s at the instant the person is 50 feet past the point on the path that is closest to the spotlight.


    \pagebreak
    \thispagestyle{5}

    \begin{tbhtheorem}{Section 8.2 Problem 78}
        You are familiar with the identity

        \[
            f(b) - f(a) = \int_a^b f'(x)dx.
        \]

        (a) Assume that $f$ has a continuous second derivative. Use integration by parts to derive the identity

        \[
            f(b) - f(a) = f'(a)(b-a)-\int_a^b f''(x)(x-b)dx.
        \]

        (b) Assume that $f$ has a continuous third derivative. Use the result in part (a) and integration by parts to derive the identity

        \[
            f(b) - f(a) = f'(a)(b-a) + \frac{f''(a)}{2}(b-a)^2 - \int_a^b \frac{f'''(x)}{2}(x-b)^2 dx.
        \]

        Going on in this manner, we are led to what are called Taylor series (Chapter 12).
    \end{tbhtheorem}

    (a) For the integral $\int f'(x)dx$, let $u=f'(x)\implies du=f''(x)dx$ and $\int dv = \int dx\implies v = x$. Because $b$ is an arbitrary chosen constant and constants are eliminated in differentiation, $v$ can be
    redefined as

    \[
        v:=x-b
    \]

    Following this, through Integration by Parts, the integral $\int f'(x)dx$ can be rewritten as

    \[
        \int f'(x)dx=(x-b)f'(x)-\int (x-b)f''(x)dx
    \]

    Hence for $x\in[a,b]$,

    \begin{align*}
        \int_a^b f'(x)dx    &= \left[(x-b)f'(x)\right]\Big|_a^b-\int_a^b (x-b)f''(x)dx \\
                            &= \left[(b-b)f'(x)-(a-b)f'(a)\right]-\int_a^b (x-b)f''(x)dx \\
                            &= f'(a)(b-a)-\int_a^b (x-b)f''(x)dx
    \end{align*}

    and by extension,

    \[
        f(b) - f(a) = f'(a)(b-a)-\int_a^b (x-b)f''(x)dx
    \]

    (b) For the integral $\int f''(x)(x-b)dx$, let $u=f''(x)\implies du=f'''(x)dx$ and $\int dv=\int (x-b)dx\implies v= \frac{(x-b)^2}{2}$. Following this, through Integration by Parts, the integral $\int f''(x)(x-b)dx$
    can be rewritten as

    \[
        \int f''(x)(x-b)dx = \frac{f''(x)}{2}(x-b)^2 - \int \frac{f'''(x)}{2}(x-b)^{2}dx
    \]

    Hence for $x\in[a,b]$,

    \begin{align*}
        \int_a^b f''(x)(x-b)dx  &= \left[\frac{f''(x)}{2}(x-b)^2\right]\Big|_a^b-\int_a^b \frac{f'''(x)}{2}(x-b)^2 dx \\
                                &= \left[\frac{f''(b)}{2}(b-b)^2-\frac{f''(a)}{2}(a-b)^2\right] - \int_a^b \frac{f'''(x)}{2}(x-b)^2 dx \\
                                &= -\frac{f''(a)}{2}(a-b)^2-\int_a^b \frac{f'''(x)}{2}(x-b)^2 dx
    \end{align*}

    Substituting the right side of the above equality for $\int f''(x)(x-b)dx$ in the derived identity from part A, it follows that

    \begin{align*}
        f(b) - f(a) &= f'(a)(b-a) - \left[-\frac{f''(a)}{2}(a-b)^2-\int_a^b \frac{f'''(x)}{2}(x-b)^2 dx\right] \\
                    &= f'(a)(b-a) + \frac{f''(a)}{2}(a-b)^2 + \int_a^b \frac{f'''(x)}{2}(x-b)^2 dx
    \end{align*}

    Notice that the term $(a-b)^2$ can be rewritten as $(b-a)^{2}$ because the order of the variables don't matter as the expression is squared. Hence,

    \[
        f(b) - f(a) = f'(a)(b-a) + \frac{f''(a)}{2}(b-a)^2 + \int_a^b \frac{f'''(x)}{2}(x-b)^2 dx
    \]

    \pagebreak
    \thispagestyle{6}


    \begin{tbhtheorem}{Section 8.3 Problem 53}
        (a) Use integration by parts to show that for $n>2$,

        \[
            \int \sin^n{x}dx = -\frac{1}{n}\sin^{n-1}{x} \cos{x} + \frac{n-1}{n}\int\sin^{n-2}{x}dx.
        \]

        (b) Then show that

        \[
            \int_0^{\frac{\pi}{2}}\sin^n{x}dx = \frac{n-1}{n} \int_0^{\frac{\pi}{2}}\sin^{n-2}{x}dx.
        \]

        (c) Verify the \textit{Wallis sine formulas:} \\
        for even $n\geq 2$,

        \[
            \int_0^{\frac{\pi}{2}}\sin^n{x}dx = \frac{(n-1)\dots5\cdot3\cdot1}{n\dots6\cdot4\cdot2}\cdot\frac{\pi}{2}.
        \]

        for odd $n\geq 3$,

        \[
            \int_{0}^{\frac{\pi}{2}} \sin^n{x}dx = \frac{(n-1)\dots4\cdot2}{n\dots5\cdot3}.
        \]
    \end{tbhtheorem}


    (a) Let us rewrite the integral as

    \[
        \int \sin^n{(x)}dx  = \int\sin^{n-1}(x)\sin{(x)}dx.
    \]

    Now setting up Integration by Parts, let

    \[
        u   = \sin^{n-1}{(x)} \implies du = (n-1)\sin^{n-2}{(x)}\cos{(x)}
    \]

    and

    \[
        \int dv = \sin{(x)}dx \implies v= -\cos{(x)}
    \]

    Then the integral can be rewritten as

    \begin{align*}
        \int \sin^n{(x)}dx  &=  \sin^{n-1}{(x)}\left(-\cos{(x)}\right) - (n-1)\int \sin^{n-2}{(x)}\left(\cos{(x)}\right)\left(-\cos{(x)}\right)dx \\
                            &= -\sin^{n-1}{(x)}\cos{(x)}-(n-1)\int \sin^{n-2}{(x)}\left(-\cos^{2}{(x)})dx \\
                            &= -\sin^{n-1}{(x)}\cos{(x)}-(n-1)\int \sin^{n-2}{(x)}\left(\sin^2{(x)}-1\right)dx \\
                            &= -\sin^{n-1}{(x)}\cos{(x)}-(n-1)\int \sin^n{(x)} + (n-1) \int \sin^{n-2}{(x)}dx \\
        n\int\sin^n{(x)}dx  &= -\sin^{n-1}{(x)}\cos{(x)} + (n-1) \int \sin^{n-2}{(x)}dx \\
        \int \sin^n{(x)}dx  &= -\frac{1}{n}\sin^{n-1}{(x)}\cos{(x)}+\frac{n-1}{n}\int\sin^{n-2}{(x)}dx
    \end{align*}

    Hence, we conclude that

    \[
        \int \sin^n{(x)}dx  &= -\frac{1}{n}\sin^{n-1}{(x)}\cos{(x)}+\frac{n-1}{n}\int\sin^{n-2}{(x)}dx, n > 2
    \]

    (b) \\
    Over the interval $x\in\left[0,\frac{\pi}{2}\right]$, the integral becomes

    \begin{align*}
        \int_0^{\frac{\pi}{2}}\sin^n{x}dx   &= \cancel{-\frac{1}{n}\sin^{n-1}{(x)}\cos{(x)}\Big|_0^{\frac{\pi}{2}}} + \frac{n-1}{n} \int_0^{\frac{\pi}{2}} \sin^{n-2}{(x)}dx \\
                                            &= \frac{n-1}{n} \int_0^{\frac{\pi}{2}} \sin^{n-2}{(x)}dx
    \end{align*}

    and thus our assertion. \\

    (c) \\
    Recall that

    \[
        \int_0^{\frac{\pi}{2}}\sin^n{x}dx = \frac{n-1}{n} \int_0^{\frac{\pi}{2}} \sin^{n-2}{(x)}dx
    \]

    Expanding the R.H.S., it follows that

    \begin{align*}
        \int_0^{\frac{\pi}{2}}\sin^n{x}dx = \frac{n-1}{n}\cdot\frac{n-3}{n-2}\cdot\frac{n-5}{n-4}\dots
    \end{align*}

    For $n\geq 2$ and of even values, the expansion of the integral above becomes

    \begin{align*}
        \int_0^{\frac{\pi}{2}}\sin^n{x}dx &= \frac{(n-1)\dots 5 \cdot 3 \cdot 1 }{n\dots 6 \cdot 4 \cdot 2}\cdot \int_0^{\frac{\pi}{2}} \sin^{0}{(x)}dx \\
                                          &= \int_0^{\frac{\pi}{2}}\sin^n{x}dx = \frac{(n-1)\dots5\cdot3\cdot1}{n\dots6\cdot4\cdot2}\cdot\frac{\pi}{2}
    \end{align*}

    Similarly, for $n\geq 3$ and of odd values, the expansion of the original integral is as follows.

    \begin{align*}
        \int_0^{\frac{\pi}{2}}\sin^n{x}dx   &= \frac{(n-1)\dots 4\cdot 2}{n\dots 5\cdot 3}\cdot\int_0^{\frac{\pi}{2}} \sin{(x)}dx \\
                                            &= \frac{(n-1)\dots4\cdot 2}{n\dots 5\cdot 3}
    \end{align*}

    \begin{tbhtheorem}{Section 8.3 Problem 54}
        Use Exercise 53 to show that

        \[
            \int_0^{\frac{\pi}{2}} \cos^n{x}dx = \int_0^{\frac{\pi}{2}}\sin^n{x}dx.
        \]
    \end{tbhtheorem}

    The R.H.S. can be expressed as

    \[
        \int_0^{\frac{\pi}{2}}\sin^n{x}dx = -\int^0_{\frac{\pi}{2}}\sin^n{\left(\frac{\pi}{2}-x\right)}dx
    \]

    By definition of the cosine function,

    \[
        \cos^n{(x)} = \sin^n{\left(\frac{\pi}{2}-x\right)}
    \]

    and so

    \begin{align*}
        -\int^0_{\frac{\pi}{2}}\sin^n{\left(\frac{\pi}{2}-x\right)}dx   &= -\int_{\frac{\pi}{2}}^0 \cos{(x)}dx \\
                                                                        &= \int_0^{\frac{\pi}{2}} \cos{(x)}dx
    \end{align*}

    Hence,

    \[
        \int_0^{\frac{\pi}{2}} \cos^n{x}dx = \int_0^{\frac{\pi}{2}}\sin^n{x}dx
    \]

    \pagebreak

    \begin{tbhtheorem}{Section 8.4 Problems 51-53}
        For Exercises 51-53, let $\Omega$ be the region under the curve $y=\sqrt{x^2-a^2}$ from $x=a$ to $x=\sqrt{2}a$. \\
        \textbf{51.} Sketch $\Omega$, find the area of $\Omega$, and locate the centroid. \\
        \textbf{52.} Find the volume of the solid generated by revolving $\Omega$ about the $x$-axis and determine the centroid of that solid. \\
        \textbf{53.} Find the volume of the solid generated by revolving $\Omega$ about the $y$-axis and determine the centroid of that solid.
    \end{tbhtheorem}

    \textbf{51}. The region $\Omega$ can be sketched like so:
    
    \begin{figure*}[hbt!]
        \centering
        \includegraphics[scale=0.1]{omega3}
    \end{figure*}

    Thus the area $A$ of $\Omega$ is given by

    \begin{align*}
        A   &= \int_a^{a\sqrt{2}} \sqrt{x^2-a^2}dx
    \end{align*}

    Through Trigonometric Substitution, a relationship between $x$ and $a$ may be formed like so:

    \begin{align*}
        x   &= a\sec{\theta} \\
        dx  &= a\sec{\theta}\tan{\theta}d\theta
    \end{align*}

    When $x=a\sqrt{2}$,

    \begin{align*}
        a\sqrt{2}   &= a\sec{\theta} \\
        \theta      &= \arcsec{\left(\sqrt{2}\right)} \\
                    &= \frac{\pi}{4}
    \end{align*}

    When $x=a$,

    \begin{align*}
        a       &= a\sec{\theta} \\
        \theta  &= \arcsec{(1)} \\
                &= 0
    \end{align*}

    Thus, the integral can be rewritten and computed with respect to $\theta$.

    \begin{align*}
        A   &= \int_0^{\frac{\pi}{4}} \sqrt{\left(a\sec{(\theta)}\right)^2-a^2}\left(a\sec{(\theta)}\tan{(\theta)}\right)d\theta \\
            &= \int_0^{\frac{\pi}{4}} \sqrt{a^2\left(\sec^2{(\theta)}-1\right)} \left(a\sec{(\theta)}\tan{(\theta)}\right)d\theta \\
            &= \int_0^{\frac{\pi}{4}} \sqrt{a^2}\sqrt{\tan^2{\theta}}  \left(a\sec{(\theta)}\tan{(\theta)}\right)d\theta \\
            &= a \int_0^{\frac{\pi}{4}} \tan{(\theta)} \left(a\sec{(\theta)}\tan{(\theta)}\right)d\theta \\
            &= a^2\int_0^{\frac{\pi}{4}} \tan^2{(\theta)} \sec{(\theta)}d\theta
    \end{align*}

    Now we will apply Integration by Parts, letting

    \[
        u = \tan{\theta} \implies du = \sec^2{(\theta)}d\theta
    \]

    and

    \[
        \int dv = \int \tan{(\theta)}\sec{(\theta)}d\theta \implies v = \sec{\theta}
    \]

    The integral can now be rewritten as follows.

    \begin{align*}
        A   &= a^2 \left[\tan{(\theta)}\sec{(\theta)}-\int \sec{(\theta)}\sec^2{(\theta)}d\theta\right]\Big|_0^{\frac{\pi}{4}} \\
            &= a^2 \left[\tan{(\theta)}\sec{(\theta)}\Big|_0^{\frac{\pi}{4}}-\int_0^{\frac{\pi}{4}}\sec^3{\theta}d\theta\right] \\
            &= a^2 \left[\sqrt{2}-\int_0^{\frac{\pi}{4}}\sec^3{(\theta)}d\theta\right]
    \end{align*}

    Now for the generalized (indefinite) integral

    \[
        \int\sec^3{(\theta)}d\theta,
    \]

    we can apply Integration by Parts again, this time letting

    \[
        u   = \sec{(\theta)} \implies du = \sec{(\theta)}\tan{(\theta)}d\theta
    \]

    and

    \[
        \int dv = \int \sec^2{(\theta)}d\theta \implies v = \tan{(\theta)}
    \]

    Hence,

    \begin{align*}
        \int\sec^3{(\theta)}d\theta     &= \sec{(\theta)}\tan{(\theta)} - \int \tan{(\theta)}\left(\sec{(\theta)\tan{(\theta)}}\right) d\theta \\
                                        &= \sec{(\theta)}\tan{(\theta)} - \int \sec{(\theta)}\tan^2{(\theta)}d\theta \\
                                        &= \sec{(\theta)}\tan{(\theta)} - \int \sec{(\theta)} \left(\sec^2{(\theta)}-1\right)d\theta \\
                                        &= \sec{(\theta)}\tan{(\theta)} - \left[\int \sec^3{(\theta)}d\theta - \int \sec{(\theta)}\right] d\theta \\
                                        &= \sec{(\theta)}\tan{(\theta)} - \int \sec^3{(\theta)} d\theta + \int \sec{(\theta)} d\theta \\
        2\int\sec^3{(\theta)}d\theta    &= \sec{(\theta)}\tan{(\theta)} + \int \sec{(\theta)}d\theta \\
                                        &= \sec{(\theta)}\tan{(\theta)} + \ln{\left|\sec{(\theta)}+\tan{(\theta)}\right|} \\
        \int\sec^3{(\theta)}d\theta     &= \frac{1}{2}\left[\sec{(\theta)}\tan{(\theta)} + \ln{\left|\sec{(\theta)}+\tan{(\theta)}\right|}\right]
    \end{align*}

    On the interval $\theta\in\left[0,\frac{\pi}{4}\right]$, the above equation may be rewritten as

    \begin{align*}
        \int_0^{\frac{\pi}{4}}\sec^3{(\theta)}d\theta   &= \frac{1}{2}\left[\sec{(\theta)}\tan{(\theta)}\Big|_0^{\frac{\pi}{4}}+\left(\ln{\left|\sec{\left(\frac{\pi}{4}\right)}
                                                                +\tan{\left(\frac{\pi}{4}\right)}\right|}+\ln{\left|\sec{(0)}+\tan{(0)}\right|}\right)\right] \\
                                                        &= \frac{1}{2}\left[\sqrt{2}+\ln{\left|\sqrt{2}+1\right|}\right]
    \end{align*}

    Substituting this expression for the integral $\int_0^{\frac{\pi}{4}}\sec^3{(\theta)}d\theta$ in the original equation for the area, it follows that

    \begin{align*}
        A   &= a^2 \left[\sqrt{2}-\int_0^{\frac{\pi}{4}}\sec^3{(\theta)}d\theta\right] \\
            &= a^2 \left[\sqrt{2}-\frac{\sqrt{2} + \ln{\left|\sqrt{2}+1\right|}}{2}\right]
    \end{align*}

    Notice that

    \[
        \sqrt{2} + 1 > 0
    \]

    and so the absolute value signs inside the natural logarithm can be removed. Additionally, the terms inside the brackets can be combined, making the area $A$ of the region $\Omega$

    \[
        A   &= a^2 \left[\frac{\sqrt{2}+\ln{\left(\sqrt{2}+1\right)}}{2}\right]
    \]

    \pagebreak

    The centroid of $\Omega$ is given by

    \[
        \left(\bar{x},\bar{y}\right),
    \]

    where

    \[
        \bar{x} = \frac{1}{A}\int_a^{a\sqrt{2}} x\sqrt{x^2-a^2}dx
    \]

    and

    \[
        \bar{y} = \frac{1}{2A} \int_a^{a\sqrt{2}} \sqrt{x^2-a^2}dx
    \]

    For the integral representing $\bar{x}$, let

    \[
        u = \sqrt{x^2-a^2} \implies du = \frac{dx}{2\sqrt{x^2-a^2}}
    \]

    and thus

    \begin{align*}
        \bar{x} &= \frac{1}{A}\int_a^{a\sqrt{2}} x\sqrt{x^2-a^2}dx \\
                &= \frac{1}{A}\int_0^a u^2 du \\
                &= \frac{1}{A}\left[\frac{u^3}{3}\right]\Big|_0^a \\
                &= \frac{1}{A}\cdot\frac{a^3}{3} \\
                &= \frac{a^3}{3A}
    \end{align*}

    Substituting in $A$, it follows that

    \begin{align*}
        \bar{x} &= \frac{a^3}{3a^2 \left(\frac{\sqrt{2}+\ln{\left(\sqrt{2}+1\right)}}{2}\right)} \\
                &= \frac{2a}{3\left(\sqrt{2}+\ln{(\sqrt{2}+1}\right)}
    \end{align*}

    For $\bar{y}$, notice that the integral is simply the value of $A$. Thus,

    \begin{align*}
        \bar{y} &= \frac{1}{2A} \int_a^{a\sqrt{2}} \sqrt{x^2-a^2}dx \\
                &= \frac{1}{2A}\cdot A \\
                &= \frac{1}{2}
    \end{align*}

    Hence, the centroid of $\Omega$ is given by

    \[
        (\bar{x}, \bar{y}) = \left(\frac{a}{3\left(\sqrt{2}-\frac{\sqrt{2} + \ln{\left(\sqrt{2}+1\right)}}{2}\right)}, \frac{1}{2}\right)
    \]

    \pagebreak
    \textbf{52.}

    \begin{figure*}[hbt!]
        \centering
        \includegraphics[scale=0.35]{omega4}
    \end{figure*}

    \pagebreak
    \textbf{53.}
    
    \begin{figure*}[hbt!]
        \centering
        \includegraphics[scale=0.35]{omega5}
    \end{figure*}


    

\end{document}