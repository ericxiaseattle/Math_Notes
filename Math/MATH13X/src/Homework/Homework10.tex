% Preamble
\documentclass{article}


% Package Imports
\usepackage{../../../../mypackages}



% Macros
\usepackage{../../../../mymacros}


% Homework Details and Basic Document Settings
\pagestyle{fancy}
\lhead{\textbf{Eric Xia}}
\chead{MATH134 (Professor Ebru Bekyel): Homework 10}
\cfoot{\thepage}

\renewcommand\headrulewidth{0.4pt}
\renewcommand\footrulewidth{0.4pt}


% Title Page
\title{
    \vspace{2in}
    \textmd{\textbf{MATH134: Homework 10}}\\
    \normalsize\vspace{0.1in}\small{Due on December 7, 2020 at 5:45 PM}\\
    \vspace{0.1in}\large{\textit{Professor Ebru Bekyel}}
    \vspace{3in}
}

\author{\textbf{Eric Xia}}
\date{}


% Problem Headers and Footers
\fancypagestyle{2}{\rhead{Section 9.1 Problem 30}}

%-------------------------------------------------------------------------------------------------------------------------------------------------------------------------------------------------------------------------
%-------------------------------------------------------------------------------------------------------------------------------------------------------------------------------------------------------------------------
%-------------------------------------------------------------------------------------------------------------------------------------------------------------------------------------------------------------------------
\begin{document}

    \maketitle
    \pagebreak

    \thispagestyle{2}


    \begin{tbhtheorem}{Section 9.1 Problem 30}
        Find the general solution of $y' + ry = 0$, $r$ constant. \\
        (a) Show that if $y$ is a solution and $y(a) = 0$ at some number $a\geq 0$, then $y(x) = 0$ for all $x$. (Thus a solution $y$ is either identically zero or never zero.) \\
        (b) Show that if $r < 0$, then all nonzero solutions are unbounded. \\
        (c) Show that if $r > 0$, then all solutions tend to 0 as $x\rightarrow \infty$. \\
        (d) What are the solutions if $r=0$?
    \end{tbhtheorem}

    \textbf{(a)}
    \begin{proof}
        Though the method of separable differential equations,

        \begin{align*}
            y' + ry &= 0 \\
            \frac{dy}{dx}       &= -ry \\
            \int \frac{dy}{y}   &= \int -r dx \\
            \ln{|y|}+C_1        &= -rx + C_2
        \end{align*}

        Since $C_1$ and $C_2$ are constants, let $C=C_2 - C_1$.

        \begin{align*}
            \ln{|y|}            &= -rx + C \\
            y                   &= Ce^{-rx} \\
            y(x)                &= Ce^{-rx}
        \end{align*}

        If $y(a)=0$ at some number $a\geq 0$, then

        \begin{align*}
            Ce^{-ra}    &= 0
        \end{align*}

        Since both $r$ and $a$ are constants, then the product $-ra$ must also be a constant. By definition, $e^{x}$ and by extension, $e^{-ra}$ cannot ever be zero and so it must be true that $C=0$. Thus,

        \begin{align*}
            y(x) &= 0e^{-rx} \\
                 &= 0
        \end{align*}

        and the solution to the given differential equation is equal to zero for all $x$.
    \end{proof}

    \textbf{(b)} \\
    Let $r < 0$. As $x$ approaches infinity,

    \begin{align*}
        \lim_{x\to\infty}y(x)   &= \lim_{x\to\infty} Ce^_{-rx} \\
                                &= Ce^{\infty}
    \end{align*}

    Recall that $C$ was redefined from $e^c$, thus $C$ cannot ever be zero. As the value of $y(x)$ never approaches zero, hence it is either unbounded in the negative or positive direction for all values of $x$ when
    $r < 0$.

    \textbf{(c)} \\

    Let $r > 0$. As $x$ approaches infinity,

    \begin{align*}
        \lim_{x\to\infty} Ce^{-rx} &= \lim_{x\to\infty} \frac{C}{e^{rx}} \\
                                    &= \frac{C}{e^{r\infty}} \\
                                    &= 0
    \end{align*}

    Thus, if $r > 0$, then all solutions of the original differential equation tend to 0 as $x$ approaches infinity. \\

    \textbf{(d)} \\

    Let $r = 0$ s.t. all solutions to the original differential equation are

    \begin{align*}
        y(x)    &= Ce^{-0x} \\
                &= C
    \end{align*}

    where $C$ is a constant.



    \pagebreak
    \thispagestyle{3}

    \begin{tbhtheorem}{Section 9.1 Problem 31}
        For the differential equation

        \[
            y' + p(x)y = 0
        \]

        with $p$ continuous on an interval $I$,

        (a) Show that if $y_1$ and $y_2$ are solutions, then $u=y_1 + y_2$ is also a solution. \\
        (b) Show that if $y$ is a solution and $c$ is a constant, then $u=Cy$ is also a solution.
    \end{tbhtheorem}

    \textbf{(a)}
    \begin{proof}
        Let $y_1$ and $y_2$ be solutions to the differential equation

        \[
            y' + p(x)y = 0
        \]

        where $p(x)$ is continuous over an interval $I$. Thus,

        \[
            y_1' + p(x)y_1 = 0
        \]

        and

        \[
            y_2' + p(x)y_2 = 0
        \]

        Summing both sides of these equations,

        \begin{align*}
            (y_1' + p(x)y_1) + (y_2' + p(x)y_2) &= 0 \\
            (y_1 + y_2)' + p(x)(y_1 + y_2)      &= 0
        \end{align*}

        Let $u=y_1 + y_2$ s.t.

        \[
            u' + p(x)u = 0
        \]

        As this equation has the same form as the original differential equation, thus $u$ must also be a solution.
    \end{proof}

    \textbf{(b)} \\
    \begin{proof}
        Let $y$ be a solution to the differential equation

        \[
            y' + p(x)y = 0
        \]

        for which $p(x)$ is continuous on an interval $I$. Thus, multiplying each term in the solution $y$ by a constant $C$,

        \begin{align*}
            Cy' + Cp(x)y = 0
        \end{align*}

        Let $u=Cy$ s.t.

        \[
            u' + p(x)u = 0
        \]

        As this equation has the same form as the original differential equation, thus $u=Cy$ must also be a solution.
    \end{proof}

    

    \begin{tbhtheorem}{Section 9.1 Problem 34}
        Show that if $y_1$ and $y_2$ are solutions of the differential equation

        \[
            y' + p(x)y = q(x)
        \]

        with $p$ and $q$ continuous on some interval $I$, then $y=y_1 - y_2$ is a solution of

        \[
            y' + p(x) y = 0
        \]
    \end{tbhtheorem}

    \begin{proof}
        Let $y_1$ and $y_2$ be solutions to the differential equation

        \[
            y' + p(x)y = q(x)
        \]

        for which $p$ and $q$ are continuous on some interval $I$. Thus,

        \[
            y_1' + p(x)y_1 = q(x)
        \]

        and

        \[
            y_2' + p(x)y_2 = q(x)
        \]

        Subtracting the equation for $y_2$ from the equation from $y_1$, it follows that

        \begin{align*}
            (y_1 - y_2)' + p(x)(y_1 - y_2)  = 0
        \end{align*}

        Let $u = y_1 - y_2$ s.t.

        \[
            u' + p(x)u = 0
        \]

        As this equation has the same form as the original differential equation, thus $u=y_1 - y_2$ must also be a solution.
    \end{proof}



    \begin{tbhtheorem}{Section 9.2 Problem 23}
        When an object of mass $m$ moves through air or a viscous medium, it is acted on by a frictional force that acts in the direction opposite to its motion. This frictional force depends on the velocity of the object
        and (within close approximation) is given by

        \[
            F(v) = -\alpha v - \beta v^2
        \]

        where $\alpha$ and $\beta$ are positive constants. \\
        (a) From Newton's second law, $F=ma$, we have

        \[
            m\frac{dv}{dt} = -\alpha v - \beta v^2
        \]

        Solve this differential equation to find $v=v(t)$. \\
        (b) Find $v$ if the object has initial velocity $v(0) = v_0$. \\
        (c) What happens to $v(t)$ as $t\rightarrow \infty$?
    \end{tbhtheorem}

    \textbf{(a)}
    \begin{align*}
        m \frac{dv}{dt}     &= -\alpha v - \beta v^2 \\
        \int \frac{dv}{-\alpha v - \beta v^2} &= \int \frac{dt}{m}\\
        \int \frac{dv}{-v(\beta v+\alpha)}    &= \int \frac{dt}{m}
    \end{align*}

    Through the expansion of partial fractions,

    \begin{align*}
        \frac{1}{-v(\beta v+\alpha)}  &= \frac{A}{-v} + \frac{B}{\beta v+\alpha} \\
        1                       &= A(\beta v+\alpha) + B(-v) \\
        v = 0                   &\implies A = \frac{1}{\alpha} \\
        v = -\frac{\alpha}{\beta}   &\implies B = \frac{\beta}{\alpha} \\
        \frac{1}{-v(\beta v + \alpha)}  &= \frac{\frac{1}{\alpha}}{-v} + \frac{\frac{\beta}{\alpha}}{\beta v + \alpha}
    \end{align*}

    Thus the L.H.S. integrand may be rewritten as the sum of two fractions:

    \begin{align*}
        -\frac{1}{\alpha}\int \frac{dv}{v} + \frac{\beta}{\alpha} \int \frac{dv}{\beta v + \alpha}  &= \int \frac{dt}{m} \\
        -\frac{1}{\alpha}\ln{|v|} + C_1 + \frac{1}{\alpha}\ln{|\beta v + \alpha|} + C_2             &= \frac{t}{m} + C_3
    \end{align*}

    As $C_1$, $C_2$, and $C_3$ are all arbitrary constants of integration, let $C = C_3 - C_2 - C_1$. Then,

    \begin{align*}
        \frac{1}{\alpha}\ln{|\beta v + \alpha|} - \frac{1}{\alpha}\ln{|v|}      &= \frac{t}{m} + C \\
        \frac{1}{\alpha}\ln{\left|\frac{\beta v + \alpha}{v}\right|}            &= \frac{t}{m} + C \\
        \ln{\left|\frac{\beta v + \alpha}{v}\right|}                            &= \frac{\alpha t}{m} + \alpha C
    \end{align*}

    Since $C$ is an arbitrary constant and $\alpha$ is a constant, let us redefine $C$ as the past value of $\alpha C$ and later, $e^C$:

    \begin{align*}
        \ln{\left|\frac{\beta v + \alpha}{v}\right|}                            &= \frac{\alpha t}{m} + C \\
        \frac{\beta v + \alpha}{v}                                 &= Ce^{\frac{\alpha t}{m}} \\
        \beta v + \alpha                                            &= Cve^{\frac{\alpha t}{m}} \\
        \beta v - Cve^{\frac{\alpha t}{m}}                          &= -\alpha \\
        v \left(\beta - Ce^{\frac{\alpha t}{m}}\right)              &= -\alpha \\
        v                                                               &= -\frac{\alpha}{\beta - Ce^{\frac{\alpha t}{m}}}
    \end{align*}

    Expressing $v$ as a function of $t$,

    \[
        v(t) = -\frac{\alpha}{\beta - Ce^{\frac{\alpha t}{m}}}
    \]

    \textbf{(b)}
    If $v(0) = v_0$, then

    \begin{align*}
        v_0     &= -\frac{\alpha}{\beta - C} \\
        \beta - C   &= -\frac{\alpha}{v_0} \\
        C           &= \frac{\alpha}{v_0} + \beta
    \end{align*}

    Substituting the R.H.S. for $C$ in the function $v(t)$, $v(t)$ may be rewritten as

    \begin{align*}
        v(t) &= -\frac{\alpha}{\beta - \left(\frac{\alpha}{v_0} + \beta\right)e^{\frac{\alpha t}{m}}} \\
             &= \frac{\alpha}{\left(\frac{\alpha}{v_0}+\beta\right)e^{\frac{\alpha t}{m}}-\beta}
    \end{align*}

    \textbf{(c)}
    \begin{align*}
        \lim_{t\to\infty} v(t)  &= \lim_{t\to\infty} \frac{\alpha}{\left(\frac{\alpha}{v_0}+\beta\right)e^{\frac{\alpha t}{m}}-\beta} \\
                                &= \frac{\alpha}{\infty} \\
                                &= 0
    \end{align*}

    Thus as $t$ approaches infinity, $v(t)$ converges to zero.

    \begin{tbhtheorem}{Section 9.2 Problem 24}
        A descending parachutist is acted on by two forces: a constant downward force $mg$ and the upward force of air resistance, which (within close approximation) is of the form $-\beta v^2$, where $\beta$ is a
        positive constant. (In this problem we are taking the downward direction as positive.) \\
        (a) Express $t$ in terms of the velocity $v$, the initial velocity $v_0$, and the constant $v_c = \sqrt{\frac{mg}{\beta}}$. \\
        (b) Express $v$ as a function of $t$. \\
        (c) Express the acceleration $a$ as a function of $t$. Verify that the acceleration never changes sign and in time tends to zero. \\
        (d) Show that in time $v$ tends to $v_c$. (This number $v_c$ is called the \textit{terminal velocity}).
    \end{tbhtheorem}


    \textbf{(a)} \\
    The net force will be sum of the forces of gravity and air resistance. Taking the downward direction as positive, through Newton's Second Law, it follows that

    \begin{align*}
        ma &= mg - \beta v^2 \\
        m \frac{dv}{dt} &= mg - \beta v^2
    \end{align*}

    Separating $v$ and $t$ to their respective sides,

    \begin{align*}
        \int \frac{dv}{mg-\beta v^2}    &= \int \frac{dt}{m}
    \end{align*}

    Through $u$-substitution with

    \[
        v = u \sqrt{\frac{mg}{\beta}} = uv_c
    \]

    the L.H.S. can be rewritten in terms of $u$:

    \begin{align*}
        \frac{1}{\sqrt{10}} \int \frac{du}{1-u^2}   &= \int \frac{dt}{m} \\
        \frac{1}{\sqrt{10}}\left(\frac{\ln{|u+1|}}{2}-\frac{\ln{|u-1|}}{2}\right) + C_1 &= \frac{t}{m} + C_2 \\
        \frac{1}{2\sqrt{10}}\ln{\left|\frac{u+1}{u-1}\right|}           + C_1 &= \frac{t}{m} + C_2
    \end{align*}

    Let $C = C_2 - C_1$. Substituting $u=\frac{v}{v_c}$ into the equation,

    \begin{align*}
        \frac{m}{2\sqrt{10}} \ln{\left|\frac{\frac{v}{v_c}+1}{\frac{v}{v_c}-1}\right|}  &= t + C
    \end{align*}

    When $t=0$,

    \begin{align*}
        C = \frac{m}{2\sqrt{10}}\ln{\left|\frac{\frac{v_0}{v_c} + 1}{\frac{v_0}{v_c} - 1}\right|}
    \end{align*}

    Thus,

    \[
        t = \frac{m}{2\sqrt{10}} \ln{\left|\frac{\frac{v}{v_c}+1}{\frac{v}{v_c}-1}\right|} - \frac{m}{2\sqrt{10}}\ln{\left|\frac{\frac{v_0}{v_c} + 1}{\frac{v_0}{v_c} - 1}\right|}
    \]

    \textbf{(b)} \\
    Continuing the work from part A,

    \begin{figure*}[hbt!]
        \centering
        \includegraphics[scale=0.1]{parachutist}
    \end{figure*}

    \textbf{(c)} \\
    The acceleration of the package is given by the derivative of $v$ with respect to $t$:

%    \begin{align*}
%        v &= \frac{1+e^{\frac{2t\sqrt{10}}{m}}\left(\frac{\frac{v_0}{v_c}+1}{\frac{v_0}{v_c}-1}\right)}{e^{\frac{2t\sqrt{10}}{m}}\left(\frac{\frac{v_0}{v_c}+1}{\frac{v_0}{v_c}-1}\right)-1} \\
%        \frac{dv}{dt}   &= \text{NO}
%    \end{align*}

    \textbf{(d)}
    \begin{figure*}[hbt!]
        \centering
        \includegraphics[scale=0.1]{parachutist2}
    \end{figure*}
    




    \begin{tbhtheorem}{Section 9.2 Problem 27}
        A rescue package of mass 100 kilograms is dropped form a plane flying at a height of 4000 meters. As the object falls, the air resistance is equal to twice its velocity. AFter 10 seconds, the package's
        parachute opens and the air resistance is now four times the square of its velocity. \\
        (a) What is the velocity of the package the instant the parachute opens? \\
        (b) What is the velocity of the package $t$ seconds after the parachute opens? \\
        (C) What is the terminal velocity of the package? \\

        HINT: There are two differential equations that govern the package's velocity and position: one for the free-fall period and one for the period after the parachute opens.
    \end{tbhtheorem}

    \textbf{(a)} \\
    The net force will be sum of the forces of gravity and air resistance. Taking the downward direction as positive, through Newton's Second Law, it follows that

    \begin{align*}
        ma  &= mg - 2v \\
        m\frac{dv}{dt}  &= mg - 2v
    \end{align*}

    Separating the two variables $v$ and $t$ on their respective sides,

    \begin{align*}
        \int \frac{dv}{mg-2v}   &= \int \frac{dt}{m} \\
        -\frac{1}{2}\ln{|mg-2v|} + C_1 &= \frac{t}{m} + C_2
    \end{align*}

    As $C_1$ and $C_2$ are arbitrary constants, we can define $C=C_2 - C_1$ and redefine $C$ each time a constant is multiplied to it. Thus,

    \begin{align*}
        -\frac{1}{2}\ln{|mg - 2v|} &= \frac{t}{m} + C \\
        \ln{|mg - 2v|}             &= -\frac{2t}{m} + C \\
        mg - 2v                    &= Ce^{-\frac{2t}{m}} \\
        v                          &= Ce^{-\frac{2t}{m}} + \frac{mg}{2}
    \end{align*}

    Expressing $v$ as a function of $t$, it follows that

    \[
        v(t) = Ce^{-\frac{2t}{m}} + \frac{mg}{2}
    \]

    Assuming the package has no initial velocity,

    \begin{align*}
        0   &= \frac{mg}{2} + C \\
        C   &= -\frac{mg}{2} \\
        v(t)    &= \frac{mg}{2}\left(1-e^{-\frac{2t}{m}}\right)
    \end{align*}

   Substituting 100 kilgorams, the weight of the package, for $m$,

    \[
        v(t) &= 50g\left(1-e^{-\frac{t}{50}}\right)
    \]

    The velocity of the package at the moment when the parachute opens is found by $v(10)$:

    \begin{align*}
        v(10)   &= 50g\left(1-e^{-\frac{1}{5}}\right) \\
                &= 88.822 \text{m/s}
    \end{align*}

    \textbf{(b)} \\
    The net force will be the sum of the forces of gravity and air resistance. Taking the downward direction as positive, through Newton's Second Law, it follows that

    \[
        m\frac{dv}{dt} = mg - 4v^2
    \]

    Separating $v$ and $t$ to their respective sides,

    \begin{align*}
        \int \frac{dv}{mg-4v^2} &= \int \frac{dt}{m}
    \end{align*}

    Let

    \[
        v = \frac{u\sqrt{mg}}{2}
    \]

    Rewriting the L.H.S. integral with respect to $u$,

    \begin{align*}
        \int \frac{du}{2\sqrt{mg}\left(1-u^2\right)}  &= \int \frac{dt}{m} \\
        \frac{1}{2\sqrt{mg}}\int\frac{du}{1-u^2}  + C_1   &= \int \frac{dt}{m} \\
        \frac{1}{2\sqrt{mg}}\left(\frac{\ln{|u+1|}}{2}-\frac{\ln{|u-1|}}{2}\right) + C_1 &= \int \frac{dt}{m} \\
        \frac{1}{4\sqrt{mg}}\ln{\left|\frac{u+1}{u-1}\right|}               + C_1        &= \frac{t}{m} + C_2
    \end{align*}

    Substituting $u$ for $\frac{2v}{\sqrt{mg}}$ and letting $C=C_2 - C_1$,

    \begin{align*}
        \ln{\left|\frac{\sqrt{mg}+2v}{\sqrt{mg}-2v}\right|} &= \sqrt{4t\sqrt{mg}}{m} + C \\
        \frac{\sqrt{mg}+2v}{\sqrt{mg}-2v}                   &= Ce^{\frac{4t\sqrt{mg}}{m}} \\
        v                                                   &= \frac{\sqrt{mg}}{2}\left(\frac{Ce^{\frac{4t\sqrt{mg}}{m}}-1}{1+Ce^{\frac{4t\sqrt{mg}}{m}}}\right)
    \end{align*}

    Expressing $v$ as a function of $t$,

    \[
        v(t) = \frac{\sqrt{mg}}{2}\left(\frac{Ce^{\frac{4t\sqrt{mg}}{m}}-1}{1+Ce^{\frac{4t\sqrt{mg}}{m}}}\right)
    \]

    Substituting $m=100$ and $g=9.8$ into the above equation,

    \[
        v(t) = 15.652\left(\frac{Ce^{1.25t}-1}{1+Ce^{1.25t}}\right)
    \]

    The velocity of the package is 88.822 when $t=0$. Thus,

    \begin{align*}
        88.822  &= 15.652\left(\frac{C-1}{1+C}\right) \\
        C       &= -1.428
    \end{align*}

    Substituting the R.H.S. for $C$ in the function $v(t)$, the velocity $t$ seconds after the parachute opens is given by

    \[
        v(t) = 15.652\left(\frac{-1.428e^{1.25t}-1}{1-1.428e^{1.25t}}\right)
    \]

    \textbf{(c)} \\
    By definition, terminal velocity is the velocity at which a free-falling object has a net force of zero, or when the force of gravity equals the force of air resistance. By Newton's second law, terminal velocity is
    given by

    \begin{align*}
        mg  &= 4v^2 \\
        v   &= \frac{\sqrt{mg}}{2}
    \end{align*}

    Substituting in $m=100$ and $g=9.8$ the terminal velocity is found to be 15.652 m/s.


    \begin{tbhtheorem}{Extra Problem}
        Do you take your coffee with cream? \\

        \textit{Newton's Law of Cooling} states that the rate of heat loss of a body is directly proportional to the difference in the temperatures between the body and its surroundings. Experimenting, I have found that
        12 ounces of $180^{\circ}$ F coffee in my favorite cup will take 20 minutes to cool to a drinking temperature of $100^{\circ}$ F in a $70^{\circ}$ F room. Assume that when I add cream to the coffee, the two
        liquids are mixed instantly, and the temperature of the mixture instantly becomes the weighed average of the temperature of the coffee and of the cream (weighted by the number of ounces of each fluid). Also,
        assume that the cooling constant of the liquid (the proportionality constant in the differential equation) does not change when I add the cream. I take my cofee with cream. I am going to add 2 ounces of cream
        at $40^{\circ}$ F to my coffee. In order to reach drinking temperature as quickly as possible, should I \\

        1. Add the cream immediately to my 12 ounces of $180^{\circ}$ F coffee and wait for it to cool down to $110^{\circ}$ F? \\
        2. Wait 5 minutes before adding the cream?
    \end{tbhtheorem}

    \pagebreak

    \begin{figure*}[hbt!]
        \centering
        \includegraphics[scale=0.4]{coffee1}
    \end{figure*}

    \begin{figure*}[hbt!]
        \centering
        \includegraphics[scale=0.4]{coffee2}
    \end{figure*}




\end{document}