% Preamble
\documentclass{article}


% Package Imports
\usepackage{../../../../mypackages}



% Macros
\usepackage{../../../../mymacros}


% Homework Details and Basic Document Settings
\pagestyle{fancy}
\lhead{\textbf{Eric Xia}}
\chead{MATH134 (Professor Ebru Bekyel): Homework 10}
\cfoot{\thepage}

\renewcommand\headrulewidth{0.4pt}
\renewcommand\footrulewidth{0.4pt}


% Title Page
\title{
    \vspace{2in}
    \textmd{\textbf{MATH134: Homework 11}}\\
    \normalsize\vspace{0.1in}\small{Due on December 14, 2020 at 5:45 PM}\\
    \vspace{0.1in}\large{\textit{Professor Ebru Bekyel}}
    \vspace{3in}
}

\author{\textbf{Eric Xia}}
\date{}


% Problem Headers and Footers
\fancypagestyle{2}{\rhead{Section 10.5 Project 10.5}}

%-------------------------------------------------------------------------------------------------------------------------------------------------------------------------------------------------------------------------
%-------------------------------------------------------------------------------------------------------------------------------------------------------------------------------------------------------------------------
%-------------------------------------------------------------------------------------------------------------------------------------------------------------------------------------------------------------------------
\begin{document}

    \maketitle
    \pagebreak

    \thispagestyle{2}

    In the early part of the seventeenth century Galileo Galilei observed the motion of stones projected from the tower of Pisa and
    concluded that their trajectory was parabolic. Using calculus,
    together with some simplifying assumptions, we obtain results
    that agree with Galileo’s observations. \\

    Consider a projectile fired at an angle $\theta$, $0 < \theta < \frac{\pi}{2}$, from a point $\left(x_0, y_0\right)$ with initial velocity $v_0$. (Figure A). The horizontal component of $v_0$ is $v_0\cos{\theta}$,
    and the vertical component is $v_0 \sin{\theta}$. (Figure B). Let $x=x(t),y=y(t)$ be parametric equations for the path of the projectile. 
    
    \begin{figure*}[hbt!]
        \centering
        \includegraphics[scale=0.75]{Homework11_Fig1}
    \end{figure*}

    We neglect air resistance and the curvature of the earth. Under these circumstances there is no horizontal acceleration and

    \[
        x''(t) = 0
    \]

    The only vertical acceleration is due to gravity; therefore,

    \[
        y''(t) = -g
    \]

    \begin{tbhtheorem}{Section 10.5 Project 10.5}
        \textbf{Problem 1.} Show that the path of the projectile (the trajectory) is given parametrically by the functions

        \[
            x(t) = \left(v_0 \cos{\theta}\right) t+ x_0, y(t) = -\frac{1}{2}gt^2 + \left(v_0 \sin{\theta}\right) t+ y_0
        \]

        \textbf{Problem 2.} Show that in rectangular coordinates the equation of the trajectory can be written

        \[
            y = -\frac{g}{2v_0^2}\sec^2{\theta}\left(x-x_0\right)^2+ \tan{\theta}\left(x-x_0\right) + y_0
        \]

        \textbf{Problem 3.} Measure distances in feet, time in seconds, and set $g=32 \text{ ft/sec}^2$. Take $\left(x_0,y_0\right)$ as the origin and the $x$-axis at ground level. Consider a projectile fired at an
        angle $\theta$ with initial velocity $v_0$. \\
        (a) Give parametric equations for the trajectory; give an equation in $x$ and $y$ for the trajectory. \\
        (b) Find the range of the projectile, which in this case is the $x$-coordinate of the point of impact. \\
        (c) How many seconds after firing does the impact take place? \\
        (d) Choose $\theta$ so as to maximize the range. \\
        (e) Choose $\theta$ so that the projectile lands at $x=b$. \\

        \textbf{Problem 4.} \\
        (a) Use a graphing utility to draw the path of the projectile fired at an angle of $30^{\circ}$ with initial velocity $v_0 = 1500 \text{ ft/sec}$. Determine the range of the projectile and the height reached. \\
        (b) Keeping $v_0 = 1500\text{ ft/sec}$, experiment with several values of $\theta$. Confirm that $\theta = \frac{\pi}{4}$ maximizes the range. What angle maximizes the height reached?
        \textbf{}
    \end{tbhtheorem}

    \textbf{Problem 1.} The position $p$ of a point particle is given by the basic kinematic equation

    \[
        p = p_0 + v_0 t + \frac{1}{2}at^2
    \]

    The projectile has initial position $\left(x_0,y_0\right)$. Since the horizontal component of $v_0$ is $v_0\cos{\theta}$, the vertical component of $v_0$ is $v_0\sin{\theta}$, the horizontal component of the
    acceleration is $x''(t)=0$ and the vertical component of the acceleration is $y''(t) = -g$, thus the path of the projectile can be written parametrically like so:

    \begin{align*}
        x(t) &= \left(v_0 \cos{\theta}\right) t + x_0 \\
        y(t) &= -\frac{1}{2}gt^2 + \left(v_0 \sin{\theta}\right)t + y_0
    \end{align*}

    \pagebreak
    \thispagestyle{3}

    \textbf{Problem 2.} Isolating $t$ from $x(t)$ and $y(t)$, we can form a system

    \begin{align*}
        t   &= \frac{x-x_0}{v_0 \cos{\theta}} \\
        t   &= \frac{v_0 \sin{\theta}+\sqrt{v_0^2 \sin^2{\theta} +2g(y_0 - y)}}{g}
    \end{align*}

    We will solve the above system for $y$ by setting the R.H.S. of the two sides equal:

    \begin{align*}
        \frac{x-x_0}{v_0 \cos{\theta}}  &= \frac{v_0 \sin{\theta}+\sqrt{v_0^2 \sin^2{\theta} +2g(y_0 - y)}}{g} \\
        \frac{g(x-x_0)}{v_0 \cos{\theta}} - v_0\sin{\theta} &= \sqrt{v_0^2\sin^2{\theta} + 2g(y_0 - y)} \\
        \frac{g^2\left(x-x_0\right)^2}{v_0^2\cos^2{\theta}} - \frac{2v_0\sin{\theta}g(x-x_0)}{v_0 \cos{\theta}} + v_0^2 \sin^2{\theta}  &= v_0^2 \sin^2{\theta} + 2g(y_0-y) \\
        \frac{g^2\left(x-x_0\right)^2}{v_0^2\cos^2{\theta}} - \frac{2g\sin{\theta}(x-x_0)}{\cos{\theta}}    &= 2g(y_0-y) \\
        \frac{g^2\left(x-x_0\right)^2}{2gv_0^2\cos^2{\theta}} - \tan{\theta}\left(x-x_0\right)  &= y_0 - y \\
        y   &= -\frac{g(x-x_0)^2}{2v_0^2\cos^2{\theta}} + \tan{\theta}(x-x_0) + y_0 \\
        y &= -\frac{g}{2v_0^2}\sec^2{\theta}\left(x-x_0\right)^2+ \tan{\theta}\left(x-x_0\right) + y_0
    \end{align*}

    and thus our assertion. \\

    \textbf{Problem 3.} \\
    (a) Recall from problem 1 that the trajectory of the projectile is given by

    \[
        x(t) = (v_0 \cos{\theta})t + x_0
    \]

    and

    \[
        y(t) = -\frac{1}{2}gt^2 + (v_0 \sin{\theta})t + y_0
    \]

    Substituting in the known information, it follows that

    \[
        x(t) = (v_0 \cos{\theta})t + x_0
    \]

    and

    \[
        y(t) = -16t^2 + (v_0 \sin{\theta})t + y_0
    \]

    (b) The range of the point of impact of the projectile can be found by setting $y=0$ and solving for $x$:

    \begin{align*}
        0   &= -\frac{g}{2v_0^2} \sec^2{\theta}(x-x_0)^2 + \tan{\theta}(x-x_0) + y_0 \\
        \frac{g}{2v_0^2}\left(x^2-2xx_0 + x_0^2)\sec^2{\theta} - \tan{\theta}(x-x_0)    &= y_0 \\
        \frac{g}{2v_0^2}x^2 - \frac{g}{v_0^2}xx_0 - x\tan{\theta}                       &= y_0 - x_0\tan{\theta} - \frac{g}{2v_0^2}x^2_0
    \end{align*}


    \begin{tbhtheorem}{Section 10.7 Problem 27}
        (a) Let $a > 0$. Find the length of the path traced out by

        \begin{align*}
            x(\theta)   &= 3a\cos{\theta} + a\cos{3\theta} \\
            y(\theta)   &= 3a\sin{\theta} - a\sin{3\theta}
        \end{align*}

        as $\theta$ ranges from 0 to $2\pi$. \\
        (b) Show that this path can also be parameterized by

        \begin{align*}
            x(\theta)   &= 4a\cos^3{\theta} \\
            y(\theta)   &= 4a\sin^3{\theta} \\
            0           &\leq \theta \leq 2 \pi
        \end{align*}
    \end{tbhtheorem}

    (a) The change in $x$ as $\theta$ ranges from 0 to $2\pi$ is given by

    \begin{align*}
        \delta x    &= 3a\cos{2\pi} + a\cos{6\pi} - 3a\cos{0} - a\cos{0} \\
                    &= 4a - 4a \\
                    &= 0
    \end{align*}

    Similarly, the change in $y$ on $\theta:[0,2\pi]$ is given by

    \begin{align*}
        \delta y    &= 3a\sin{2\pi} - a\sin{6\pi} - 3a\sin{0} + a\sin{0} \\
                    &= 0 + 0 \\
                    &= 0
    \end{align*}

    Thus, the length of the path traced out by the given parametric equations as $\theta$ ranges from $0$ to $2\pi$ is zero. \\

    (b) Expanding $\cos{3\theta}$ in $x(\theta)$, it follows that

    \begin{align*}
        x(\theta)   &= 3a\cos{\theta} + a\cos{2\theta + \theta}
    \end{align*}

    Similarly,

    \begin{align*}
        y(\theta)   &= 3a\sin{\theta} + a\sin{(2\theta+\theta)} \\
                    &= 3a\sin{\theta} + a\left(\sin{2\theta}\cos{\theta} + \cos{2\theta}\sin{\theta}\right) \\
                    &= 3a\sin{\theta} + a\left(\left(2\sin{\theta}\cos{\theta}\right)\cos{\theta} + \left(1-2\sin^2{\theta}\right)\sin{\theta}\right) \\
                    &= 3a\sin{\theta} + a\left(2\sin{\theta}\cos^2{\theta} + \sin{\theta} - 2\sin^3{\theta}\right) \\
                    &= 3a\sin{\theta} + a\left(2\sin{\theta} - 2\sin^3{\theta} + \sin{\theta} - 2\sin^3{\theta}\right) \\
                    &= 3a\sin{\theta} + a\left(3\sin{\theta} - 4\sin^3{\theta}\right) \\
                    &=
    \end{align*}

    \begin{tbhtheorem}{Section 10.7 Problem 43}
        Show that the curve $y=\cosh{x}$ has the property that for every interval $[a,b]$ the length of the curve from $x=a$ to $x=b$ equals the area under the curve from $x=a$ to $x=b$.
    \end{tbhtheorem}

    \begin{tbhtheorem}{Section 10.8 Problem 27}
        (\textit{An interesting property of the sphere}) Slice a sphere along two parallel planes a fixed distance apart. Show that the surface area of the band so obtained depends only on the distance between the
        planes, not on their location.
    \end{tbhtheorem}

    \begin{tbhtheorem}{Section 10.8 Project 10.8}
        \textbf{Problem 2.} \\
        (a) At the end of each arch, the cycloid comes to a cusp. Show that $x'$ and $y'$ are both 0 at the end of each arch. \\
        (b) Show that the area under an arch of the cycloid is three times the area of the rolling circle. \\
        (c) Find the length of an arch of the cycloid. \\

        \textbf{Problem 3.} \\
        (a) Locate the centroid of the region under the first arch of the cycloid. \\
        (b) Find the volume of the solid generated by revolving the region under an arch of the cycloid about the $x$-axis. \\
        (c) Find the volume of the solid generated by revolving the region under an arch of the cycloid about the $y$-axis.
    \end{tbhtheorem}



\end{document}