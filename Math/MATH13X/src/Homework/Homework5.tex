% Preamble
\documentclass{article}


% Package Imports
\usepackage{../../../../mypackages}



% Macros
\usepackage{../../../../mymacros}


% Homework Details and Basic Document Settings
\pagestyle{fancy}
\lhead{\textbf{Eric Xia}}
\chead{MATH134 (Professor Ebru Bekyel): Week 5 Assignment}
\cfoot{\thepage}

\renewcommand\headrulewidth{0.4pt}
\renewcommand\footrulewidth{0.4pt}


% Title Page
\title{
    \vspace{2in}
    \textmd{\textbf{MATH134: Homework 5}}\\
    \normalsize\vspace{0.1in}\small{Due on November 2, 2020 at 5:45 PM}\\
    \vspace{0.1in}\large{\textit{Professor Ebru Bekyel}}
    \vspace{3in}
}

\author{\textbf{Eric Xia}}
\date{}


% Problem Headers and Footers
\fancypagestyle{2}{\rhead{Question 1}\fancyfoot[L]{Section 5.2 Problem 12 continued on next page\ldots}}
\fancypagestyle{3}{\rhead{Section 5.2 Problem 12}}
\fancypagestyle{4}{\rhead{Section 5.2 Problem 27}}
\fancypagestyle{5}{\rhead{Section 5.3 Problem 32}\fancyfoot[L]{Section 5.3 Problem 21 continued on next page\ldots}}
\fancypagestyle{6}{\rhead{Section 5.3 Problem 21}\fancyfoot[L]{Section 5.3 Problem 36 continued on next page\ldots}}
\fancypagestyle{7}{\rhead{Section 5.3 Problem 36}}


%-------------------------------------------------------------------------------------------------------------------------------------------------------------------------------------------------------------------------
%-------------------------------------------------------------------------------------------------------------------------------------------------------------------------------------------------------------------------
%-------------------------------------------------------------------------------------------------------------------------------------------------------------------------------------------------------------------------
\begin{document}

    \maketitle
    \pagebreak


    \thispagestyle{2}

    \begin{tbhtheorem}{Question 1}
        Prove that a non-constant linear function is uniformly continuous on the real line. (This should be straightforward. Start with a continuity proof and make sure your delta does not depend on the point you
        choose.)
    \end{tbhtheorem}

    \begin{proof}
        A non-constant linear function can have the form

        \[
            f(x) = mx + b,
        \]

        where $m\not = 0$ and $b$ is a constant. For $f(x)$ to be uniformly continuous, for any given $\epsilon > 0$, the inequality

        \[
            |x-c| < \delta \implies |f(x) - f(c)| < \epsilon,
        \]

        where $c$ is a constant, must hold. Let $\delta = \frac{\epsilon}{|m|}$ and $\epsilon > 0$, such that

        \[
            \text{if } |x-c| < \delta = \frac{\epsilon}{|m|} \text{ then } |f(x) - f(c)| < \epsilon
        \]

        Substituting in $f(x)=mx+b$ and $f(c)=mc+b$, $|f(x) - f(c)| < \epsilon$ becomes

        \begin{align*}
            |mx + b - mc - b| &< \epsilon \\
            |m(x-c)|          &< \epsilon \\
            |m||x-c|          &< \epsilon
        \end{align*}

        As $|x-c| < \frac{\epsilon}{|m|}$, it follows that

        \begin{align*}
            |m|\cdot \frac{\epsilon}{|m|} &= \epsilon \\
            \epsilon                      &= \epsilon
        \end{align*}

        Hence, the non-constant linear function $f(x)$ is uniformly continuous on the real line.
    \end{proof}

    \begin{tbhtheorem}{Section 5.2 Problem 12}
        (a) Given that $P = \{x_0, x_1, \dots, x_n\}$ is an arbitrary partition of $[a,b]$, find $L_f (P)$ and $U_f(P)$ for $f(x)=x+3$. \\
        (b) Use your answers to part (a) to evaluate
        \[
            \int^b_a f(x)dx.
        \]
    \end{tbhtheorem}

    \textbf{(a)} \\
    On each subinterval $[x_{i-1}, x_i]$, the function $f(x)=x+3$ has a maximum of $x_i$ and a minimum of $x_{i-1}$. $L_f (P)$ is defined as the lower sum for $f(x)$, i.e. the sum of the areas of the rectangles
    in each interval using the lowest value of $f(x)$ as the height of the rectangle. $U_f (P)$ is defined as the upper sum for $f(x)$, i.e. the sum of the areas of the rectangles in each interval using the highest value
    of $f(x)$ as the height of the rectangle. It follows that

    \[
        L_f (P) = \sum^n_{i=1} \Delta x f(x_{n-1}) = \sum^n_{i=1} (x_n - x_{n-1})f(x_{n-1})
    \]

    and

    \[
        U_f (P) = \sum^n_{i=1} \Delta x f(x_n) = \sum^n_{i=1} (x_n - x_{n-1}) f(x_n)
    \]

    \textbf{(b)} \\
    As $x_{i-1} \leq x_i$ and the function $f(x)$ is strictly increasing, for every index $i$, the inequality

    \[
        2(x_{x-1}+3) \leq x_i + 3 + x_{x-1} + 3 \leq 2(x_i + 3)
    \]

    holds, and

    \[
        x_{i-1} + 3 \leq \frac{1}{2}(x_i + x_{i-1} + 6) \leq x_i + 3
    \]

    Multiplying by $\Delta x = x_i - x_{i-1}$, the middle term of the inequality becomes

    \[
        \frac{1}{2}(x_i - x_{i-1})(x_i + x_{i-1} + 6) = \frac{1}{2}(x^2_i + 6x_i - x^2_{i-1} - 6x_{i-1})
    \]

    It follows that

    \[
        \Delta x (x_{i-1}) \leq \frac{1}{2} (x^2_i + 6x_i - x^2_{i-1} - 6x_{i-1}) \leq \Delta x x_i
    \]

    \pagebreak
    \thispagestyle{3}

    The sum of the middle term collapses to:

    \begin{align*}
        \frac{1}{2} (x^2_1 + 6x_1 - x^2_0 - 6x_0 + x^2_2 + 6x_2 - x^2_1 - 6x_1 + \dots + x^2_n + 6x_n - x^2_{n-1} - 6x_{n-1})   &= \frac{1}{2}(-x^2_0 - 6x_0 + x^2_n + 6x_n) \\
                                                                                                                                &= \frac{1}{2}(-a^2 - 6a + b^2 + 6b)
    \end{align*}

    The sum of the terms on the left side of the inequality is $L_f (P)$ and the sum of the terms on the right side of the inequality is $U_f (P)$. Thus,

    \[
        L_f (P) \leq \frac{b^2-a^2 + 6b-6a}{2} \leq U_f (P)
    \]

    Since $P$ was chosen arbitrarily, we can conclude that this inequality holds for all partitions $P$ of $[a,b]$. It follows that

    \[
        \int^b_a f(x)dx = \frac{b^2-a^2+6b-6a}{2}.
    \]



    \begin{tbhtheorem}{Section 5.2 Problems 25-30}
        Assume that $f$ and $g$ are continuous, that $a<b$, and that $\int^b_a f(x)dx > \int^b_a g(x)dx$. Which of the statements necessarily holds for all partitions $P$ of $[a,b]$? Justify your answer. \\

        25. $L_g (P) < U_f (P)$. \\
        26. $L_g (P) < L_f (P)$. \\
        27. $L_g (P) < \int^b_a f(x)dx$. \\
        28. $U_g (P) < U_f (P)$. \\
        29. $U_f (P) > \int^b_a g(x)dx$. \\
        30. $U_g (P) < \int^b_a f(x)dx$.
    \end{tbhtheorem}

    \textbf{25.} \\
    Since $L_g (P)$ is the lower sum of $g(x)$ on an arbitrary interval $P$ and $U_g (P)$ is the upper sum of $g(x)$ on $P$, it follows that

    \[
        L_g (P) < \int^b_a g(x)dx
    \]

    and

    \[
        U_f (P) > \int^b_a f(x)dx
    \]

    Since $\int^b_a f(x)dx > \int^b_a g(x)dx$,

    \[
        L_g (P) < \int^b_a g(x)dx < \int^b_a f(x)dx < U_f (P)
    \]

    and the statement $L_g (P) < U_f (P)$ must always hold true. \\

    \textbf{26.} \\
    Consider when $f(x)=x$ and $g(x)=1$, $\Delta x = x_i - x_{i-1}$, and the arbitrary partition $P$ is defined as $P=[0,1]$, such that

    \[
        L_f (P) = \Delta x \cdot f(x_{i-1})
    \]

    and

    \[
        L_g (P) = \Delta x \cdot g(x_{i-1})
    \]

    At the index $i=1$,

    \[
        L_f (P) = 0(1-0) = 0
    \]

    and

    \[
        L_g (P) = 1(1-0) = 1
    \]

    Thus,

    \[
        L_f (P) < L_g (P)
    \]

    and the statement $L_g (P) < L_f (P)$ is not necessarily true for all partitions $P$ of $[a,b]$.

    \pagebreak
    \thispagestyle{4}

    \textbf{27.} \\
    By the definition of $L_g (P)$,

    \[
        L_g (P) < \int^b_a g(x)dx
    \]

    Because

    \[
        \int^b_a g(x)dx < \int^b_a f(x)dx,
    \]

    it follows that the inequality

    \[
        L_g (P) < \int^b_a f(x)dx
    \]

    is always true. \\

    \textbf{28.} \\
    Consider when $f(x)=x$ and $g(x)=1$, $\Delta x = x_i - x_{i-1}$, and the arbitrary partition $P$ is defined as $P=[0,1]$, such that

    \[
        U_f (P) = \Delta x \cdot f(x_i)
    \]

    and

    \[
        U_g (P) = \Delta x \cdot g(x_i)
    \]

    At the index $i=1$,

    \[
        U_f (P) = 1(1) = 1
    \]

    and

    \[
        U_g (P) = 1(1) = 1
    \]

    Thus,

    \[
        U_f (P) = U_g (P)
    \]

    and the statement $U_g (P) < U_f (P)$ is not necessarily true for all partitions $P$ of $[a,b]$. \\

    \textbf{29.} \\
    By the definition of $U_f (P)$,

    \[
        U_f (P)> \int^b_a f(x)dx
    \]

    Because

    \[
        \int^b_a g(x)dx < \int^b_a f(x)dx,
    \]

    it follows that the inequality

    \[
        U_f (P) > \int^b_a g(x)dx
    \]

    always holds. \\

    \textbf{30.} \\
    Consider when $f(x)=x$ and $g(x)=1$, $\Delta x = x_i - x_{i-1}$, and the arbitrary partition $P$ is defined as $P=[0,1]$, such that

    \[
        U_g (P) = \Delta x\cdot g(x_i) = 1(1) = 1
    \]

    Because the integral of a function is the area under the curve,

    \begin{align*}
        \int^b_a f(x)dx = \int^1_{0} xdx = \frac{1}{2}
    \end{align*}

    It follows that

    \[
        U_g (P) > \int^b_a f(x)dx
    \]

    and hence the statement $U_g (P) < \int^b_a f(x)dx$ is not necessarily true for all partitions $P$ of $[a,b]$.

    \pagebreak
    \thispagestyle{5}



    \begin{tbhtheorem}{Section 5.2 Problem 32}
        Let $P=\{x_0, x_1, x_2, \dots, x_{n-1}, x_n\}$ be a regular partition of the interval $[a,b]$. Show that if $f$ is continuous and decreasing on $[a,b]$, then
        \[
            U_f (P) - L_f (P) = [f(a) - f(b)] \Delta x
        \]
    \end{tbhtheorem}

    \begin{proof}
        As $P$ is a regular partition and $f$ is decreasing on $[a,b]$, we have that the expansion of the upper sum $U_f (P)$ is given by

        \[
            U_f (P) = f(x_1)\Delta x + f(x_2)\Delta x + \dots + f(x_{n-1}) \Delta x
        \]

        and the expansion of the lower sum $L_f (P)$ is given by

        \[
            L_f (P) = f(x_0)\Delta x + f(x_1) \Delta x + \dots + f(x_n)
        \]

        It follows then that

        \begin{align*}
            U_f (P) - L_f (P)   &= [f(x_0)\Delta x + f(x_1) \Delta x + \dots + f(x_{n-1})] - [f(x_1)\Delta x + f(x_2)\Delta x + \dots + f(x_n) \Delta x] \\
                                &= \Delta x([f(x_0) + f(x_1) + \dots + f(x_{n-1})] - [f(x_1) + f(x_2) + \dots + f(x_n)])
        \end{align*}

        This collapses to

        \[
            U_f (P) - L_f (P) = \Delta x [f(x_0) - f(x_n)]
        \]

        Substituting $a=x_0$ and $b=x_n$,

        \[
            U_f (P) - L_f (P) = [f(a)-f(b)]\Delta x
        \]
    \end{proof}




    \begin{tbhtheorem}{Section 5.3 Problem 21}
        Suppose that $f$ is differentiable with $f'(x)>0$ for all $x$, and suppose that $f(1)=0$. Set

        \[
            F(x) = \int^x_0 f(t)dt.
        \]

        Justify each statement. \\
        (a) $F$ is continuous. \\
        (b) $F$ is twice differentiable. \\
        (c) $x=1$ is a critical point of $F$. \\
        (d) $F$ takes on a local minimum at $x=1$. \\
        (e) $F(1)<0$. \\

        Make a rough sketch of the graph of $F$.
    \end{tbhtheorem}

    \textbf{(a)}
    Because $f$ is differentiable,

    \begin{align*}
        f(x)    &= \lim_{h\to 0} \frac{F(x+h)-F(x)}{h} \\
        f(x)    &= \frac{\lim_{h\to 0} (F(x+h)-F(x))}{\lim_{h\to 0} h} \\
        0       &= \lim_{h\to 0} F(x+h) -\lim_{h\to 0} F(x) \\
        F(x)    &= \lim_{h\to 0} F(x+h)
    \end{align*}

    Hence, by the definition of continuity, $F$ is continuous. \\


    \textbf{(b)}
    The first derivative of $F$ is given by

    \[
        F'(x) = \frac{d}{dx}\int^x_0 f(t)dt = f(x)
    \]

    Because $f$ is differentiable for all $x$, it follows that

    \[
        F''(x) = f'(x)
    \]

    must exist and hence, $F$ is twice differentiable. \\


    \textbf{(c)}
    When $x=1$,

    \[
        F'(1) = f(1) = 0
    \]

    The slope at a critical point is equal to zero, hence, $x=1$ is a critical point of $F$.

    \pagebreak
    \thispagestyle{6}

    \textbf{(d)}
    As $f'(x)>0 $ $\forall x$ and $f(1)=0$, it must be true that

    \[
        f(x)=
        \begin{cases}
            \text{negative},     & x<1 \\
            \text{positive},     & x>1
        \end{cases}
    \]

    Because

    \[
        F'(1) = f(1) = 0
    \]

    and $f$ is negative before $x=1$ and positive after $x=1$, it follows that $F$ takes a local minimum at $x=1$. \\

    \textbf{(e)} \\
    Because $f(x)<0$ when $x<1$ and an integral represents the area under a curve, thus

    \[
        \int^1_0 f(x)dx < 0
    \]

    \textbf{Rough Sketch of $\bm{F}$}: \\
    Because $F(1)<0$ and $x=1$ is a minimum of $F$, then a rough sketch of $F$ can be as follows
    \begin{figure*}[hbt!]
        \centering
        \includegraphics[scale=0.075]{sketch}
    \end{figure*}




    \begin{tbhtheorem}{Section 5.3 Problem 36}
        Let $F$ be everywhere continuous and set

        \[
            F(x) = \int^x_0 \left[t\int^t_1 f(u) du \right]dt.
        \]

        Find \\
        (a) $F'(x)$. \\
        (b) $F'(1)$. \\
        (c) $F''(x)$. \\
        (d) $F''(1)$.
    \end{tbhtheorem}

    \textbf{(a)} \\
    $F'(x)$ is given by

    \[
        F'(x) = \frac{d}{dx} \int^x_0 \left[t\int^t_1 f(u)du\right] dt
    \]

    By the Fundamental Theorem of Calculus,

    \[
        F'(x) = x\int^x_1 f(u)du
    \]

    \textbf{(b)} \\
    When $x=1$,

    \begin{align*}
        F'(1)   &= 1 \int^1_1 f(u)du \\
                &= 1(0) \\
                &= 0
    \end{align*}

    \textbf{(c)} \\
    Because

    \[
        F'(x) = x\int^x_1 f(u)du,
    \]

    it follows that

    \begin{align*}
        F''(x)  &= \frac{d}{dx} \left(x\int^x_1 f(u)du\right) \\
                &= \int^x_1 f(u)du + xf(x)
    \end{align*}

    \pagebreak
    \thispagestyle{7}

    \textbf{(d)} \\
    When $x=1$,

    \begin{align*}
        F''(1)  &= \int^1_1 f(u)du + (1)f(1) \\
                &= 0 + f(1) \\
                &= f(1)
    \end{align*}

\end{document}