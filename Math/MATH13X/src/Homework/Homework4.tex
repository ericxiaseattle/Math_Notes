% Preamble
\documentclass{article}


% Package Imports
\usepackage{../../../../mypackages}



% Macros
\usepackage{../../../../mymacros}


% Homework Details and Basic Document Settings
\pagestyle{fancy}
\lhead{\textbf{Eric Xia}}
\chead{MATH134 (Professor Ebru Bekyel): Week 3 Assignment}
\cfoot{\thepage}

\renewcommand\headrulewidth{0.4pt}
\renewcommand\footrulewidth{0.4pt}


% Title Page
\title{
    \vspace{2in}
    \textmd{\textbf{MATH134: Homework 4}}\\
    \normalsize\vspace{0.1in}\small{Due on October 26, 2020 at 5:45 PM}\\
    \vspace{0.1in}\large{\textit{Professor Ebru Bekyel}}
    \vspace{3in}
}

\author{\textbf{Eric Xia}}
\date{}


% Problem Headers and Footers
\fancypagestyle{2}{\rhead{Section 4.1 Problem 29}}



%-------------------------------------------------------------------------------------------------------------------------------------------------------------------------------------------------------------------------
%-------------------------------------------------------------------------------------------------------------------------------------------------------------------------------------------------------------------------
%-------------------------------------------------------------------------------------------------------------------------------------------------------------------------------------------------------------------------
\begin{document}

    \maketitle
    \pagebreak


    \thispagestyle{2}

    \begin{tbhtheorem}{Section 4.1 Problem 29}
        A number $c$ is called a \textit{fixed point} of $f$ if $f(c)=c$. Prove that if $f$ is differentiable on an interval $I$ and $f'(x)<1$ for all $x\in I$, then $f$ has at most one fixed point in $I$. \\
        HINT: Form $g(x)=f(x)-x$.
    \end{tbhtheorem}

    \begin{proof}
        Let $g(x)=f(x)-x$. Because $f'(x)$ is differentiable on the interval $I$, then

        \[
            g'(x) = f'(x) - 1
        \]

        Because $f'(x) < 1$ $\forall x\in I$,

        \[
            g'(x) = f'(x) - 1 < 0,\text{ }\forall x \in I.
        \]

        Then $g(x)=f(x)-x$ is strictly decreasing and because $x$ is linear, hence $f(x)$ must also be strictly decreasing for all $x$ in $I$. Then by the definition of a function, $f$ can only ever attain decreasing
        values and thus it has at most one fixed point in $I$.
    \end{proof}


    \begin{tbhtheorem}{Section 4.1 Problem 34}
        Show that the equation $x^n + ax + b = 0$ has at most three distinct real roots, where $n$ is an odd positive integer.
    \end{tbhtheorem}

    \begin{proof}
        Let $f(x)$ be a function and

        \[
            f(x) = 0 = x^n + ax + b.
        \]

        When $n=1$,

        \begin{align*}
            x+ax+b=0
        \end{align*}

        This is a linear equation and must have one solution, hence, there is exactly 1 critical point when $n=1$. 77

        For $n>1$, setting $f'(x)=0$ to find the critical points of $f(x)$ gives

        \begin{align*}
            f'(x)   &= nx^{n-1} + a = 0 \\
            nx^{n-1} &= -a
        \end{align*}

        When $a = 0$,

        \[
            x=0
        \]

        and $f(x)$ has only one critical point. When $a>0$,

        \[
            nx^{n-1} + a > 0
        \]

        because $n>1$ and so $f(x)$ has at most one critical point. When $a<0$,

        \begin{align*}
            nx^{n-1}    &= -a \\
            x           &= \pm (n+1)\sqrt{-\frac{1}{n}}
        \end{align*}

        and $f(x)$ has at most 2 critical points. \\

        Because $f'(x)$ exists for all $x$, $f(x)$ must be continuous for all $x$. $f(x)$ can have at most two critical points; this happens when $n>1$ and $a<0$, Hence, by Rolle's Theorem, $f(x)=x^n+ax+b=0$ can have at
        most three distinct real roots.
    \end{proof}


    \begin{tbhtheorem}{Section 4.2 Problem 59 (Do two proofs one using induction and one using calculus)}
        Let $n$ be an integer greater than 1. Show that $(1+x)^n>1+nx$ for all $x<0$.
    \end{tbhtheorem}

    \textbf{Induction Method:}
    \begin{proof}
        Let $S$ be the set of positive integers $n>1$ for which

        \[
            (1+x)^n > 1+nx, \text{ } \forall x < 0
        \]

        holds. Let $n=2$ such that

        \begin{align*}
            (1+x)^2      &> 1 + 2x \\
            x^2 + 2x + 1 &> 1 + 2x \\
            x^2          &> 0
        \end{align*}

        When $x<0$, $x\not = 0$ and $2\in S$. Assume $k\in S$; that is assume,

        \[
            (1+x)^k > 1 + kx, \text{ } \forall x < 0
        \]

        Consider

        \begin{align*}
            (1+x)^k + 1     &= (1+x)(1+x)^k
        \end{align*}

        By the induction hypothesis,

        \begin{align*}
            (1+x)(1+x)^k    &> (1+x)(1+kx) \\
                            &> kx^2 + kx + x + 1 \\
                            &> kx^2 + x(k+1) + 1
        \end{align*}

        Because $k>0$,

        \begin{align*}
            (1+x)(1+x)^k > kx^2 + x(k+1) + 1 > x(k+1) + 1
        \end{align*}

        and thus $k+1\in S$. Hence, by the principle of mathematical induction, we conclude that all positive integers greater than 1 are in $S$; that is, we can conclude that

        \[
            (1+x)^n > 1 + nx, \text{ } \forall x < 0
        \]
    \end{proof}

    \textbf{Calculus Method:}
    \begin{proof}
        Suppose $f(x) = (1+x)^n - 1 - nx$. Setting $f'(x)=0$, we have

        \begin{align*}
            f'(x) = 0 &= n(1+x)^{n-1}-n \\
                    n &= n(1+x)^{n-1} \\
                    1 &= (1+x)^{n-1}
        \end{align*}

        Because $n\geq 0$, then $x=0$. $f''(0)$ is given by:

        \begin{align*}
            f''(x)  &= n(n-1)(1+x)^{n-2} \\
            f''(0)  &= n(n-1)
        \end{align*}

        Because $n\geq 2$, $n(n-1)>0$ and $f(x)$ has a minimum at $x=0$. As $x<0$, it must be true that

        \begin{align*}
            f(x) &> f(0) \\
            f(x) &> 0 \\
            (1+x)^n - 1 - nx &> 0 \\
            (1+x)^n &> 1 + nx, \text{ } \forall x < 0
        \end{align*}
    \end{proof}

    \begin{tbhtheorem}{Section 4.3 Problem 42}
        Let $y=f(x)$ be differentiable and suppose that the graph of $f$ does not pass through the origin. The distance $D$ from the origin to a point $P(x, f(x))$ of the graph is given by

        \[
            D = \sqrt{x^2 + [f(x)]^2}.
        \]

        Show that if $D$ has a local extreme value at $c$, then the line through $(0,0)$ and $(c,f(c))$ is perpendicular to the line tangent to the graph of $f$ at $(c,f(c))$.
    \end{tbhtheorem}

    \begin{proof}
        The line tangent to the graph of $f$ at $(c,f(c))$ has the slope $f'(c)$. The slope of the line through (0,0) and $(c,f(c))$ is given by:

        \[
            \frac{f(c)-0}{c-0} = \frac{f(c)}{c}
        \]

        Because $D$ has an extremum at $c$, $D'(c)$ must be 0 or undefined. Thus, taking the derivative of $D'(x)$, we find that

        \begin{align*}
            D'(x)   &= \frac{1}{2\sqrt{x^2 + f(x)^2}} \cdot (2x+2f'(x)f(x)) \\
                    &= \frac{2x + 2f'(x)f(x)}{2\sqrt{x^2+f(x)^2}}
        \end{align*}

        \pagebreak

        As $f$ does not pass through (0,0),

        \[
            2\sqrt{x^2+f(x)^2} \not = 0
        \]

        for all values of $x$ in $f(x)$. Then $D'(x)$ cannot ever be undefined and it must be true that $D'(c)=0$. Hence,

        \begin{align*}
            \frac{2x+2f'(x)f(x)}{2\sqrt{x^2+f(x)^2}}    &= 0 \\
            2x + 2f'(x)f(x)                             &= 0 \\
            2f'(x)f(x)                                  &= -2x \\
            f'(x)                                       &= -\frac{x}{f(x)} \\
            f'(c)                                       &= -\frac{c}{f(c)}
        \end{align*}

        Because the slopes

        \[
            \frac{f(c)}{c}  \text{ and } -\frac{c}{f(c)}
        \]

        are perpendicular, then the line through (0,0) and $(c,f(c))$ must also be perpendicular to the line tangent to the graph of $f$ at $(c,f(c))$.
    \end{proof}


    \begin{tbhtheorem}{Section 4.4 Problem 39}
        Suppose that $f$ is continuous on $[a,b]$ and $f(a)=f(b)$. Show that $f$ has at least one critical point in $(a,b)$.
    \end{tbhtheorem}

    \begin{proof}
        If $f(x)$ is differentiable on $(a,b)$, then by the Mean Value Theorem, there must exist some $c$ such that

        \[
            f'(c) = \frac{f(b)-f(a)}{b-a} = 0.
        \]

        By the definition of a critical point, $(c,f(c))$ would be a critical point. \\

        If $f(x)$ is not differentiable on $(a,b)$, then the graph of $f(x)$ must contain some sharp corner or cusp on $(a,b)$. Because the slope at a sharp corner or cusp is undefined, then by the definition of a
        critical point, $f(x)$ has a critical point whenever its graph has a sharp corner or cusp. Thus, $f(x)$ must have some critical point on $(a,b)$.
    \end{proof}

    \begin{tbhtheorem}{Section 4.4 Problem 41}
        Give an example of a nonconstant function that takes on both its absolute maximum and absolute minimum on every interval.
    \end{tbhtheorem}

    The Dirichlet function, given below, attains both its maximum value (1) and minimum value (0) on any given interval, since for any two real numbers, there exists a rational and irrational number between them.

    \[
        f(x) =
        \begin{cases}
            1,  & x \text{ is rational} \\
            0,  & x \text{ is irrational}
        \end{cases}
    \]

    \begin{tbhtheorem}{Section 4.4 Problem 48}
        A piece of wire of length $L$ is to be cut into two pieces, one piece to form a square and the other piece to form an equilateral triangle. How should the wire be cut so as to \\
        (a) maximize the sum of the areas of the square and the triangle? \\
        (b) minimize the sum of the areas of the square and the triangle?
    \end{tbhtheorem}

    The perimeter of a square of side length $s$ is $4s$ and the perimeter of an equilateral triangle of side $a$ is $3a$. Thus,

    \begin{align*}
        L &= 4s + 3a \\
        a &= \frac{L-4s}{3}
    \end{align*}

    and the sum of the areas of the square and the triangle is given by:

    \[
        A = s^2 + \frac{\sqrt{3}}{4}a^2
    \]

    Setting $A$ as a function of $s$ and substituting

    \[
        a = \frac{L-4s}{3},
    \]

    \pagebreak

    we have

    \begin{align*}
        A(s)    &= s^2 + \frac{\sqrt{3}}{4}a^2 \\
                &= s^2 + \frac{\sqrt{3}}{4} \left(\frac{L-4s}{3}\right)^2 \\
                &= s^2 + \frac{\sqrt{3}}{36}(L-4s)^2
    \end{align*}

    Function $A(s)$ is bounded on the closed interval that has endpoints equal to the areas of when the entirety of the wire is made as a square or equilateral triangle, respectively. When all of the wire is made into
    a square,

    \begin{align*}
        L &= 4s \\
        s &= \frac{L}{4}
    \end{align*}

    and the area is

    \begin{align*}
        A\left(\frac{L}{4}\right)   &= \frac{L^2}{16}
    \end{align*}

    When all of the wire is made into an equilateral triangle,

    \begin{align*}
        L   &= 3a \\
        a   &= \frac{L}{3}
    \end{align*}

    and the area is

    \begin{align*}
        A\left(\frac{L}{3}\right)   &= \frac{L^2}{9}
    \end{align*}

    Thus, the area function, $A(s)$, is bounded on the closed interval:

    \[
        \left[\frac{L^2}{16}, \frac{L^2}{9}\right]
    \]

    Recall that

    \[
        A(s) = s^2 + \frac{\sqrt{3}}{36}(L-4s)^2
    \]

    Then

    \begin{align*}
        \frac{dA}{ds}   &= 2s + \frac{\sqrt{3}}{18}(L-4s)(-4) \\
                        &= 2s - \frac{4\sqrt{3}(L-4s)}{18} \\
                        &= 2s - \frac{2\sqrt{3}(L-4s)}{9}
    \end{align*}

    When $\frac{dA}{ds}=0$,

    \begin{align*}
        2s  - \frac{2\sqrt{3}(L-4s)}{9} &= 0 \\
        s                               &= \frac{\sqrt{3}(L-4s)}{9} \\
        9s                              &= \sqrt{3}(L-4s) \\
        9s                              &= L\sqrt{3} - 4s\sqrt{3} \\
        9s+4s\sqrt{3}                   &= L\sqrt{3} \\
        s(9+4\sqrt{3})                  &= L\sqrt{3} \\
        s                               &= \frac{L\sqrt{3}}{9+4\sqrt{3}}
    \end{align*}

    and

    \pagebreak

    \begin{align*}
        A\left(\frac{L\sqrt{3}}{9+4\sqrt{3}}\right) &= \left(\frac{L\sqrt{3}}{9+4\sqrt{3}}\right)^2 + \frac{\sqrt{3}}{36}\left(L-4\left(\frac{L\sqrt{3}}{9+4\sqrt{3}}\right)\right)^2
    \end{align*}

    Because $A(s)$ is bounded on a closed interval, then by the Extreme Value Theorem, there must be an absolute maximum and absolute minimum attained either at the endpoints of the interval or at a critical point.
    This means we must compare the values of the endpoints and the critical point we found to determine the absolute maximum and absolute minimum combined area that can be made by a wire of length $L$. Because of the
    atrociously complicated critical point, let us define the arbitrary length $L$ as $L=7$ units for the sake of computational simplicity. \\

    Plugging $L=7$ into the endpoints of the interval $A(s)$ is bounded on, the interval can be rewritten in decimal approximations to 3 digit places:

    \[
        [3.063, 5.444]
    \]

    The area attained at the critical point can be approximated as 1.332. Because $L$ is arbitrary, the relational areas given by the endpoints and the critical point when $L=7$ can be generalized to all values of $L$.
    1.332 is the smallest area in these three values, hence, the sum of the areas made by the wire is minimized when

    \[
        s = \frac{L\sqrt{3}}{9+4\sqrt{3}}
    \]

    and the wire is cut at this value of $s$. Similarly, 5.444 is the largest area in these three values and hence, the sum of the areas made by the wire is maximized when all of the wire is not cut and instead made into
    an equilateral triangle.


    \begin{tbhtheorem}{Section 4.6 Problem 48}
        Show that if a cubic polynomial $p(x)=x^3 + ax^2 + bx + c$ has a local maximum and a local minimum, then the midpoint of the line segment that connects the local high point to the local low point is a point of
        inflection.
    \end{tbhtheorem}

    \begin{proof}

    \end{proof}

    \begin{tbhtheorem}{Section 4.7 Problem 52}
        Sketch the graph of the function showing all vertical and oblique asymptotes.
        \[
            f(x) = \frac{1+x-3x^2}{x}
        \]
    \end{tbhtheorem}

    The graph of $f(x)$ is in red and the asymptotes, $y=-3x+1$ and $x=0$, are shown in blue.

    \begin{center}
        \begin{tikzpicture}
            \begin{axis}[
                axis lines = center,
                axis equal image,
                xmin = -10,
                xmax = 10,
                ymin = -10,
                ymax = 10
            ]
            \addplot[
                color = red,
                domain = 0:10,
                samples = 500
            ]
            {-3*x+1+(1/x)};
            \addplot[
                color = red,
                domain = -10:0,
                samples = 500
            ]
            {-3*x+1+(1/x)};
            \addplot[
                color = blue
            ]
            {-3*x+1};
            \draw[dashed, blue] (0,-10) -- (0,10);
            \end{axis}
        \end{tikzpicture}
    \end{center}
%
    \begin{tbhtheorem}{Section 4.8 Problem 26}
        Sketch the graph of the function
        \[
            f(x) = \frac{1}{x^3 - x}.
        \]
    \end{tbhtheorem}

    The graph of $f(x)$ is in red and the asymptotes $x=-1$, $x=0$, $x=1$, and $y=0$ are in blue.

    \begin{center}
        \begin{tikzpicture}
            \begin{axis}[
                axis lines = center,
                axis equal image,
                xmin = -5,
                xmax = 5,
                ymin = -5,
                ymax = 5,
            ]
            \addplot[
                samples = 500,
                color = red,
                domain = -5:-1
            ]
            {1/(x^3-x)};
            \addplot[
                samples = 500,
                color = red,
                domain = -1:-0.1
            ]
            {1/(x^3-x)};
            \addplot[
                samples = 500,
                color = red,
                domain = 0.1:1
            ]
            {1/(x^3-x)};
            \addplot[
                samples = 500,
                color = red,
                domain = 1:5
            ]
            {1/(x^3-x)};
            \draw[blue, dashed] (-1,-5) -- (-1,5);
            \draw[blue, dashed] (0,-5) -- (0,5);
            \draw[blue, dashed] (1,-5) -- (1,5);
            \draw[blue, dashed] (-5,0) -- (5,0);
            \end{axis}
        \end{tikzpicture}
    \end{center}

    \begin{tbhtheorem}{Section 4.9 Problem 48}
        To estimate the height of a bridge, a man drops a stone into the water below. How high is the bridge \\
        (a) if the stone hits the water 3 seconds later? \\
        (b) if the man hears the splash 3 seconds later? \\
        (Use 1080 ft/s as the speed of sound).
    \end{tbhtheorem}

    (a) \\
    The gravitational acceleration of an object is about 32 ft/$\text{s}^2$. One of the foundational equations of kinematics is

    \begin{align*}
        y = y_0 + v_0 t + \frac{1}{2}at^2,
    \end{align*}

    where $y$ is the object's position on a vertical axis where the positive direction has higher position, $v$ is the object's velocity, $a$ is the acceleration, and $t$ is time in seconds.
    Because the rock does not have any velocity to begin with,

    \begin{align*}
        0   &= y_0 - 16t^2 \\
        y_0 &= 16(3)^2 \\
            &= 144
    \end{align*}

    Hence, the bridge is 144 feet high. \\

    (b) \\
    Let $t_w$ be the time the stone takes to hit the water and let $t_m$ be the time it takes for the sound from this interaction to reach the man. Since the man hears the splash after three seconds,

    \[
        t_w + t_m = 3
    \]

    Substituting $t_w$ for $t$ in the equation from part (a),

    \[
        y_0 = 16(t_w)^2
    \]

    Additionally, position is given by velocity multiplied by time. Thus,

    \[
        y_0 = 1080t_m
    \]

    Now we solve the system below for $y_0$.

    \begin{align*}
        \begin{cases}
            t_w + t_m = 3 \\
            y_0 = 16 t_w^2 \\
            y_0 = 1080t_m
        \end{cases}
    \end{align*}

    \begin{align*}
        16 t_w^2        &= 1080t_m \\
        16 t_w^2        &= 1080(3-t_w) \\
        16t_w^2 +1080t_w -3240 &= 0 \\
        t_w   &= \frac{-1080\pm\sqrt{1080^2-4(16)(-3240)}}{2(16)} \\
              &\approx -70.377, 2.877
    \end{align*}

    Because time cannot be negative,

    \begin{align*}
        y_0 &= 16\left(\frac{-1080+\sqrt{1080^2-4(16)(-3240)}}{2(16)}\right)^2 \\
            &\approx 132.466
    \end{align*}

    Thus, the bridge is approximately 132.466 feet high.


    \begin{tbhtheorem}{Section 4.10 Problem 26}
        Water flows from a faucet into a hemispherical basin 14 inches in diameter at the rate of 2 $\text{in}^3\text{/s}$. How fast does the water rise \\
        (a) when the water is exactly halfway to the top? \\
        (b) just as it runs over? \\

        (The volume of a spherical segment is given by $\pi rh^2 - \frac{1}{3}\pi h^3$ where $r$ is the radius of the sphere and $h$ is the depth of the segment.)
    \end{tbhtheorem}

    (a) \\
    The volume of the basin, $V$, is given by

    \[
        V = \pi rh^2 - \frac{\pi}{3} h^2
    \]

    Because the value of $r$ does not change, i.e. $r$ is a constant,

    \begin{align*}
        \frac{dV}{dt}   &= 2\pi r h \frac{dh}{dt} - \pi h^2 \frac{dh}{dt}
    \end{align*}

    When the water is exactly halfway to the top, $r = 7$ and $h=3.5$ and $\frac{dV}{dt}=2$ such that

    \begin{align*}
       2\pi (7)(3.5) \frac{dh}{dt} - \pi (3.5)^2 \frac{dh}{dt}  &= 2 \\
        \frac{dh}{dt} \left(2\pi (7)(3.5) - \pi (3.5)^2) = 2 \\
        \frac{dh}{dt}   &= \frac{2}{49\pi - 12.25\pi} \\
                        &\approx 0.017
    \end{align*}

    So the water rises at approximately 0.017 in/s when the water is exactly halway to the top.

    (b) \\
    Just as the water runs over, $r=7$ and $h=7$ such that

    \begin{align*}
        2\pi (7)(7) \frac{dh}{dt} - \pi (7)^2 \frac{dh}{dt} &= 2 \\
        49\pi \frac{dh}{dt} &= 2 \\
        \frac{dh}{dt}   &= \frac{2}{49\pi} \\
            &\approx 0.013
    \end{align*}

    So the water rises at approximately 0.013 in/s just as the basin runs over.


    \begin{tbhtheorem}{Section 4.10 Problem 40}
        An airplane is flying at constant speed and altitude on a line that will take it directly over a radar station on the ground. At the instant the plane is 12 miles from the station, it is noted that the plane's
        angle of elevation is $\text{30}^{\circ}$ and is increasing at the rate of $0.5^{\circ}$ per second. Give the speed of the plane in miles per hour.
    \end{tbhtheorem}

    Let $a$ be the altitude of the plane, $b$ be the horizontal path the plane takes as it nears the radio station, $c$ be the distance from the plane to the radio station, and $\theta$ be the angle of elevation of the
    plane. The relationship between these variables is shown in the figure below.
    
    \begin{figure*}[hbt!]
        \centering
        \includegraphics[scale=0.05]{plane}
    \end{figure*}

    \pagebreak

    When $c=12$,

    \begin{align*}
        \frac{12}{\sin{\left(\frac{\pi}{2}\right)}} &= \frac{a}{\sin{\left(\frac{\pi}{3}\right)}} \\
        12\sin{\left(\frac{\pi}{3}\right)}          &= a\sin{\left(\frac{\pi}{2}\right)} \\
        a                                           &= \frac{12\sin{\left(\frac{\pi}{3}\right)}}{\sin{\left(\frac{\pi}{2}\right)}} \\
                                                    &= 12 \cdot \frac{\sqrt{3}}{2} \\
                                                    &= 6\sqrt{3}
    \end{align*}

    The relationship between $a,b$, and $\theta$ is given by

    \begin{align*}
        \tan{\theta} &= \frac{b}{a} \\
        b            &= a \tan{\theta}
    \end{align*}

    Because $a$ represents altitude which is unchanging, i.e. $a$ is constant. 0.5 degrees is $\frac{\pi}{360}$ radians and so at the moment when the angle of elevation, $\theta$, is $\frac{\pi}{6}$, Then,

    \begin{align*}
        \frac{db}{dt}   &= a \sec^2{\theta}\frac{d\theta}{dt} \\
                        &= \frac{a}{\cos^2{\left(\frac{\pi}{6}\right)}}\cdot \left(\frac{\pi}{360}\right) \\
                        &= \frac{4a}{3}\cdot \left(\frac{\pi}{360}\right) \\
                        &= \frac{\pi a}{270}
    \end{align*}

    Plugging $a=6\sqrt{3}$ into this,

    \begin{align*}
        \frac{db}{dt}   &= \frac{6\pi\sqrt{3}}{270} \\
                        &= \frac{\pi\sqrt{3}}{45} \text{ mi/s}
    \end{align*}

    Converting this to mph, we have

    \begin{align*}
        \frac{\pi\sqrt{3}\text{ mi}}{45\text{ s}}   \cdot \frac{60^2 \text{ s}}{\text{hr}} = 80\pi\sqrt{3} \text{ mph}
    \end{align*}

    Thus, the speed of the plane is $80\pi\sqrt{3}$ mph when the plane is 12 miles from the station.

    \begin{tbhtheorem}{Section 4.11 Problem 20}
        View the earth as a sphere of radius 4000 miles. The volume of ice that covers the north and south poles is estimated to be 8 million cubic miles. Suppose that this ice melts and the water produced distributes
        itself uniformly over the surface of the earth. Estimate the depth of this water.
    \end{tbhtheorem}

    Assuming the earth is a perfect sphere, its volume is given by

    \[
        V = \frac{4}{3}\pi r^3
    \]

    The differentials $dV$ and $dr$ can then be related by the equation below.

    \[
        dV = 4\pi r^2 dr
    \]

    Because $dV = 8 \cdot 10^6$ and $r=4\cdot 10^3$,

    \begin{align*}
        dr  &= \frac{8\cdot 10^6}{4\pi (4\cdot 10^3)^2} \\
            &= \frac{1}{8\pi} \\
        &\approx 0.393
    \end{align*}

    Hence, the depth of the water would increase by approximately 0.393 miles.




\end{document}




