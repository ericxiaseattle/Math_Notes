\section{Differential Equations}

        \textit{Example:}

        \begin{align*}
                \frac{dy}{dx}                                   &= (y^2 + y)(4x+2) \\
                \int \frac{dy}{y^2+y}                           &= \int 4x + 2 dx \\
                \int \frac{1}{y}dy - \int \frac{1}{y+1}dy       &= 2x^2 + 2x + C_2 \\
                \ln{|y|} - \ln{|y+1|}+C_1                       &= 2x^2 + 2x + C_2 \\
                \text{Set } C                                   &= C_2 - C_1 \\
                \ln{\left|\frac{y}{y+1}\right|}                 &= 2x^2 + 2x + C \\
                \left|\frac{y}{y+1}\right|                      &= e^{2x^2+2x+C} \\
                \left|\frac{y}{y+1}\right|                      &= e^C e^{2x^2+2x} \\
                \frac{y}{y+1}                                   &= Ce^{2x^2+2x} \\
                y                                               &= Ce^{2x^2+2x}\cdot y + Ce^{2x^2+2x} \\
                \left(1-Ce^{2x^2+2x}\right)y                    &= Ce^{2x^2+2x} \\
                y                                               &= \frac{Ce^{2x^2+2x}}{1-Ce^{2x^2+2x}} \\
                y                                               &= \frac{1}{Ce^{-2x^2-x^2}-1}
        \end{align*}

        \textit{Example 2: Mixing Problem} \\
        A tank initially contains 1000 litres of pure water. Brine that contains 0.07 kilograms of salt per liter of water enters the tank at a rate of 5 litres per minute. In addition, brine that contains 0.04 kilograms
        of salt per liter of water enters the tank at a rate of 10 litres per minute. The solution is kept thoroughly mixed and drains from the tank at a rate of 15 litres per minute. Find the amount of salt in the tank
        as a function of time. Also, find the concentration of salt in the tank "eventually". \\

        Let $s(t)$ be the amount of salt in kg at a given time $t$ in minutes. We have $\frac{ds}{dt}=\text{in}-\text{out}$, thus

        \begin{align*}
                \frac{ds}{dt}   &= (0.07)5 + (0.04)10 - s\frac{15}{1000} \\
                                &= 0.75 - 0.015s \\
                                &= -0.015(s-50) \\
                \frac{ds}{s-50} &= -0.015 dt \\
                \int \frac{ds}{s-50}    &= \int -0.015 dt \\
                \ln{|s-50|}             &= -0.015t + C \\
                |s-50|                  &= e^{-0.015t+C} \\
                |s-50|                  &= Ce^{-0.015t} \\
                s-50                    &= Ce^{-0.015t} \\
                s                       &= 50 + Ce^{-0.015t}
        \end{align*}

        \textit{Example 3: Newton's Law of Cooling} \\
        Newton's Law of Cooling states that the rate of cooling of an object is proportional to the temperature difference between the object and its surroundings. The cooling constant, which is the
        proportionality constant in the differential equation, remains constant throughout this question. Experimenting, you found that 12 ounces of $180^{\circ}$ F coffee in your favorite cup will take 20 minutes to
        cool to a drinking temperature of $115^{\circ}$ F in a $70^{\circ}$ F room. Set up a differential equation for the temperature of the coffee, solve it, and find the cooling constant. \\

        Let $P_s=70$ be the temperature of the surrounding environment which stays constant. Let $P$ be the temperature of the object at minute $t$. We know that $\frac{dP}{dt}$ is proportional to the difference in
        temperature between the system and the environment, hence

        \begin{align*}
                \frac{dP}{dt}   &~ (P-P_s) \\
                                &= -k(P-P_s) \\
                \int \frac{dP}{P-P_s}   &= \int -k dt
        \end{align*}

        We will need to use information in the problem to figure out $k$ and $C$.

