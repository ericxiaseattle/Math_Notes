\section{Vector Functions}


    \subsection{Vector Functions and Space Curves}      % 13.1

        \textbf{Vector function:} a function whose domain is a set of real numbers and whose range is a set of vectors \\

        Let $r(t)$ be a vector function whose values are three-dimensional vectors. If $f(t)$, $g(t)$, and $h(t)$ are the components of the vector $r(t)$, then $f$, $g$, and $h$ are real-valued functions called the
        \textbf{component functions} of \textbf{r} and we can write

        \[
            \mathbf{r}(t) = \langle f(t), g(t), h(t)\rangle = f(t)\mathbf{i} + g(t)\mathbf{j} + h(t) \mathbf{k}
        \]

        The limit of a vector function $r(t) = \langle f(t), g(t), h(t) \rangle$ is defined by taking the limits of its component functions like so:

        \[
            \lim_{t\to a} r(t) = \langle \lim_{t\to a} f(t), \lim_{t\to a}g(t), \lim_{t\to a}h(t)\rangle
        \]

        iff the limits of the component functions exist. \\

        Alternatively, $\lim_{t\to a} r(t) = L$ iff for every $\epsilon > 0$ there is a number $\delta > 0$ s.t. if

        \[
            0 < |t - a| < \delta
        \]

        then

        \[
            |r(t) - b| < \epsilon
        \]

        A vector function \mathbf{r} is \textbf{continuous at $\mathbf{a}$} if

        \[
            \lim_{t\to a} r(t) = r(a)
        \]

        Suppose that $f$, $g$, and $h$ are continuous real-valued functions on an interval $I$. Then the set $C$ of all points $(x,y,z)$ in space, where

        \[
            x = f(t), y = g(t), z = h(t)
        \]

        and $t$ varies throughout the interval $I$, is called a \textbf{space curve}. Considering the vector function $r(t)$, $r(t)$ is the position vector of the point $P(f(t), g(t), h(t))$ on $C$. \\

        Plane curves can also be described by vectors. Consider the curve given by the parametric equations $x = t^2 -2t$ and $y = t+1$ which could be described by

        \[
            r(t) = \langle t^2 - 2t, t + 1\rangle = (t^2 - 2t)\mathbf{i} + (t+1) \mathbf{j}
        \]

        where $\mathbf{i} = \langle 1, 0\rangle$ and $\mathbf{j}= \langle 0, 1\rangle$. \\

        \textit{\blue{Example:}} Sketch the curve whose vector equation is

        \[
            r(t) = \cos{t}\mathbf{i} + \sin{t}\mathbf{j} + t\mathbf{k}
        \]

        The parametric equations for this curve are

        \[
            x = \cos{t}, y = \sin{t}, z =t
        \]

        Because $x^2 + y^2 = \cos^2{t} + \sin^2{t} = 1$ for all values of $t$, the curve must lie on the circular cylinder $x^2 + y^2 = 1$. The point $(x,y,z)$ lies directly above the point $(x,y,0)$, which moves
        counterclockwise around the circle $x^2 + y^2 = 1$ in the $xy$-plane. The projection of the curve onto the $xy$-plane has vector equation $r(t) = \langle \cos{t}, \sin{t}, 0\rangle$. Since $z=t$, the curve
        spirals upwards around the cylinder as $t$ increases. The curve shown below is called a \textbf{helix}.

        \begin{figure*}[hbt!]
            \centering
            \includegraphics[]{Resources/13.1_Helix}
        \end{figure*}

        \textit{\blue{Example 2:}} Find a vector equation and parametric equations for the line segment that joins the point $P(1,3,-2)$ to the point $Q(2,-1,3)$. \\

        Recall that a vector equation for the line segment that joins the tip of the vector $r_0$ to the tip of the vector $r_1$ is:

        \[
            r(t) = (1-t) r_0 + tr_1, 0 \leq t \leq 1
        \]

        Taking $r_0 = \langle 1,3,-2\rangle$ and $r_1 = \langle 2, -1, 3\rangle$ to obtain a vector equation of the line segment from $P$ to $Q$, it follows that

        \[
            r(t) = (1-t)\langle 1,3,-2\rangle + t\langle 2, -1, 3\rangle, 0 \leq t\leq 1
        \]

        or

        \[
            r(t) = \langle 1 + t, 3-4t, -2 + 5t\rangle, 0 \leq t\leq 1
        \]

        and the corresponding parametric equations are

        \[
            x = 1+t, y = 3-4t, z = -2 + 5t, 0 \leq t \leq 1
        \]

        \textit{\blue{Example 3:}} Find a vector function that represents the curve of intersection of the cylinder $x^2 + y^2 = 1$ and the plane $y + z = 2$. \\

        The below figures show the intersection of the plane and the cylinder, and the curve of intersection $C$, an ellipse, respectively.

        \begin{figure*}[hbt!]
            \centering
            \caption*{Intersection of Plane and Cylinder}
            \includegraphics[scale = 0.6]{Resources/13.1_Intersection_Plane_Cylinder}
        \end{figure*}

        \begin{figure*}[hbt!]
            \centering
            \caption*{Intersection $C$, an Ellipse}
            \includegraphics[scale = 0.6]{Resources/13.1_Intersection_C}
        \end{figure*}

        The projection of $C$ onto the $xy$-plane is the circle $x^2 + y^2 = 1, z = 0$. Thus we can write

        \[
            x = \cos{t}, y = \sin{t}, 0 \leq t\leq 2\pi
        \]

        From the equation of the plane, we have

        \[
            z = 2- y = 2-\sin{t}
        \]

        and we can write parametric equations for $C$ as

        \[
            x = \cos{t}, y = \sin{t},  z = 2-\sin{t}, 0 \leq t \leq 2\pi
        \]

        The corresponding vector equation is

        \[
            r(t) = \cos{t}\mathbf{i} + \sin{t}\mathbf{j} + (2-\sin{t})\mathbf{k}, 0 \leq t\leq 2\pi
        \]

        This equation is known as a \textit{parameterization} of the curve $C$. \\

        Space curves are much more difficult to draw by hand than plane curves and for an accurate representation we must use technology. Below are some examples of computer-generated space curves.

        \begin{figure*}[hbt!]
            \centering
            \caption*{A \textbf{Toroidal Spiral}}
            \includegraphics[scale = 0.6]{Resources/13.1_Toroidal_Spiral}
        \end{figure*}

        \begin{figure*}[hbt!]
            \centering
            \caption*{A \textbf{Trefoil Knot}}
            \includegraphics[scale = 0.6]{Resources/13.1_Trefoil_Knot}
        \end{figure*}

    \subsection{Derivatives and Integrals of Vector Functions}      % 13.2

        \begin{theorem}{Definition of Derivative of Vector Function}
            The derivative $\mathbf{r'}$ of a vector function $\mathbf{r}$ is given by

            \[
                \frac{d\mathbf{r}}{dt} = \mathbf{r}'(t) = \lim_{h\to 0} \frac{\mathbf{r}(t+h) - \mathbf{r}(t)}{h}
            \]

            if the limit exists.
        \end{theorem}

        If the points $P$ and $Q$ have position vectors $r(t)$ and $r(t+h)$, then $\vec{PQ}$ represents the vector $r(t+h) - r(t)$, which can be regarded as a secant vector. If $h > 0$, then the scalar multiple
        $\left(\frac{1}{h}\right)\left(r(t+h)-r(t)\right)$ has the same direction as $r(t+h)-r(t)$. As $h$ converges to 0, the scalar multiple $\left(\frac{1}{h}\right)\left(r(t+h)-r(t)\right)$ has the same direction
        as $r(t+h)-r(t)$. As $h\to 0$, this vector approaches a vector that lies on the tangent line. For this reason, the vector $r'(t)$ is called the \textbf{tangent vector} to the curve defined by \textbf{r} at the
        point $P$, provided that $\mathbf{r}'(t)$ exists and $\mathbf{r}'(t)\not = 0$. The \textbf{tangent line} to $C$ at $P$ is defined to be the line through $P$ parallel to the tangent vector $\mathbf{r}'(t)$. \\

        \textbf{Unit tangent vector:}

        \[
            \mathbf{T}(t) = \frac{\mathbf{r}'(t)}{|\mathbf{r}'(t)|}
        \]

        When $0 < h < 1$, multiplying the secant vector by $\frac{1}{h}$ "stretches" the vector, as shown below.

        \begin{figure*}[hbt!]
            \centering
            \includegraphics[scale = 0.75]{Resources/13.2_Stretching}
        \end{figure*}

        Geometrically, the derivative tells us that as $h \to 0$, the quotient of the secant vector $r(t+h) - r(t)$ and $h$ approaches the tangent vector $r(t)$. \\

        \textbf{Differentiation Rules for Vector Functions:}
        \begin{center}
            \begin{tabular}{|c|c|}
                \hline
                $\frac{d}{dt}\left[u(t)+v(t)\right]=u'(t) + v'(t)$  & Sum/Difference \\
                \hline
                $\frac{d}{dt}[cu(t)] = cu'(t)$  & Constant Multiple \\
                \hline
                $\frac{d}{dt}[f(t)u(t)]=f'(t)u(t)+f(t)u'(t)$    & Product \\
                \hline
                $\frac{d}{dt}[u(t)\cdot v(t)] = u'(t)\cdot v(t) + u(t) \cdot v'(t)$ & Product, Dot \\
                \hline
                $\frac{d}{dt}[u(t)\times v(t)] = u'(t) \times v(t) + u(t) \times v'(t)$ & Product, Cross \\
                \hline
                $\frac{d}{dt}[u(f(t))] = f'(t) u'(f(t))$    & Chain Rule \\
                \hline
            \end{tabular}
        \end{center}

        \textit{\blue{Example:}} Show that if $|r(t)| = c$ where $c$ is a constant, then $r'(t)$ is orthogonal to $r(t)$ for all $t$. \\

        Since

        \[
            r(t)\cdot r(t) = |r(t)|^2 = c^2
        \]

        and $c^2$ is a constant, it follows that

        \[
            0 = \frac{d}{dt}[r(t) \cdot r(t)] = r'(t) \cdot r(t) + r(t) \cdot r'(t) = 2r'(t) \cdot r(t)
        \]

        Hence, $r'(t) \cdot r(t) = 0$ which means that $r'(t)$ is orthogonal to $r(t)$. Geometrically, this result tells us that if a curve lies on a sphere with center the origin, then the tangent vector $r'(t)$
        is always perpendicular to the position vector $r(t)$.

        \begin{figure*}[hbt!]
            \centering
            \includegraphics[scale = 0.75]{Resources/13.2_Orthogonal}
        \end{figure*}

        Let $r(t)= \langle f(t), g(t),h(t)\rangle $ be a continuous vector function. Then

        \begin{align*}
            \int_a^b r(t)dt &= \lim_{n\to \infty} \sum^n_{i=1}r(t_i)\Delta t \\
                            &= \lim_{n\to\infty} \left[\left(\sum^n_{i=1} f(t_i)\Delta t\right)\mathbf{i} + \left(\sum^n_{i=1}g(t_i)\Delta_t\right)\mathbf{j} + \left(\sum^n_{i=1}h(t_i)\Delta t\right)\mathbf{k}\right] \\
                            &= \left(\int_a^b f(t)dt\right)\mathbf{i} + \left(\int_a^b g(t)dt\right)\mathbf{j} + \left(\int_a^b h(t)dt\right) \mathbf{k}
        \end{align*

        \begin{theorem}{FTC for Continuous Vector Functions}
            Let $R$ be an antiderivative of $r$, that is, $R'(t) = r(t)$. Then

            \[
                \int_a^b r(t)dt = R(t)\Big|_a^b = R(b) - R(a)
            \]
        \end{theorem}

    \subsection{Arc Length and Curvature}       % 13.3

        \begin{theorem}{Arc Length}
            For a space curve with vector equation $r(t) = \langle f(t), g(t), h(t) \rangle, a \leq t \leq b$, or equivalently, the parametric equations $x=f(t), y=g(t), z=h(t)$, where $f', g', h'$ are continuous.
            If the curve is traversed exactly once as $t$ increases from $a$ to $b$, then it can be shown that its length is

            \[
                L   = \int_a^b \sqrt{[f'(t)]^2 + [g'(t)]^2 + [h'(t)]^2}dt
            \]

            or

            \[
                L   = \int_a^b \sqrt{\left(\frac{dx}{dt}\right)^2 + \left(\frac{dy}{dt}\right)^2 + \left(\frac{dz}{dt}\right)^2}dt
            \]

            or

            \[
                L = \int_a^b |r'(t)| dt
            \]
        \end{theorem}

        Geometrically, the length of a space curve is the limit of lengths of inscribed polygons:

        \begin{figure*}[hbt!]
            \centering
            \includegraphics[scale = 0.75]{Resources/13.3_Space_Curve}
        \end{figure*}

        \textit{\blue{Example:}} Find the length of the arc of the circular helix with vector equation $r(t) = \cos{t}\mathbf{i} + \sin{t}\mathbf{j} + t\mathbf{k}$ from the point $(1,0,0)$ to the point $(1,0,2\pi)$. \\

        Since $r'(t) = -\sin{t}\mathbf{i} + \cos{t}\mathbf{j} + \mathbf{k}$, we have

        \[
            |r'(t)| = \sqrt{(-\sin{t})^2 + \cos^2{t} + 1} = \sqrt{2}
        \]

        The arc from $(1,0,0)$ to $(1,0,2\pi)$ is described by the parameter interval $0\leq t \leq 2\pi$, and so it follows that

        \[
            L = \int_0^{2\pi} |r'(t)| dt = \int_0^{2\pi} \sqrt{2}dt = 2\sqrt{2}\pi
        \]

        \begin{theorem}{Arc Length Function}
            Suppose that $C$ is a curve given by a vector function

            \[
                r(t) = f(t) \mathbf{i} + g(t) \mathbf{j} + h(t) \mathbf{k}, a\leq t\leq b
            \]

            where $r'$ is continuous and $C$ is traversed exactly once as $t$ increases from $a$ to $b$. The \textbf{arc length function} $s$ is given by

            \[
                s(t) = \int_a^t |r'(u)| du = \int_a^t \sqrt{\left(\frac{dx}{du}\right)^2 + \left(\frac{dy}{du}\right)^2 + \left(\frac{dz}{du}\right)^2}du
            \]
        \end{theorem}

        \begin{corollemma}{Arc Length Function: Corollary}
            Differentiating both sides of the above equation, it follows that

            \[
                \frac{ds}{dt} = |r'(t)|
            \]
        \end{corollemma}

        This corollary is useful when \textit{parameterizing a curve with respect to arc length}. If a curve $r(t)$ is already given in terms of a paramter and $s(t)$ is the arc length function foun, then we may be able
        to solve for $t$ as a function of $s:t = t(s)$. Thus, the curve can be reparamterized in terms of $s$ by substituting for $t:r = r(t(s))$. If $s=3$ for instance, $r(t(3))$ is the position vector of the point
        3 units of length along the curve from its starting point. \\

        \textit{\blue{Example 2:}} Reparametrize the helix $r(t) = \cos{t}\mathbf{i} + \sin{t}\mathbf{j} + \mathbf{k}$ with respect to arc length measured from $(1,0,0)$ in the direction of increasing $t$. \\

        The initial point $(1,0,0)$ corresponds to the parameter value $t=0$. From the first example we have that

        \[
            \frac{ds}{dt} = |r'(t)|
        \]

        and so

        \[
            s = s(t) = \int_0^t |r'(u)|du = \int_0^t \sqrt{2}du = \sqrt{2}t
        \]

        Therefore $t= \frac{s}{\sqrt{2}}$ and the required reparametrization is obtained by substituting for $t$:

        \[
            r(t(s)) = \cos{\left(\frac{s}{\sqrt{2}}\right)}\mathbf{i} + \sin{\left(\frac{s}{\sqrt{2}}\right)\mathbf{j} + \left(\frac{s}{\sqrt{2}}\right)\mathbf{k}}
        \]

        A parameterization $\mathbf{r}(t)$ is called \textbf{smooth} on an interval $I$ if $\mathbf{r}'$ is continuous and $\mathbf{r}'(t)\not = 0$ on $I$. A curve is called \textbf{smooth} if it has a smooth
        parametrization. A smooth curve hass no sharp corners or cusps; when the tangent vector turns, it does so continuously.

        \begin{theorem}{Unit Tangent Vector}
            For a smooth curve $C$ defined by the vector function $r$, the unit tangent vector is

            \[
                \mathbf{T}(t) = \frac{\mathbf{r}'(t)}{|\mathbf{r}'(t)|}
            \]

            This vector indicates the direction of the curve. $\mathbf{T}(t)$ changes direction slowly when the curve is relatively straight, but it changes direction more quickly when $C$ twists or turns more sharply.
        \end{theorem}

        The curvature of $C$ at a given point is a measure of how quickly the curve changes direction at that point. Because the unit tangent vector has constant length, only changes in direction contribute to the rate
        of change of \textbf{T}.

        \begin{theorem}{Curvature}
            The \textbf{curvature} of a curve is

            \[
                \kappa = \left|\frac{d\mathbf{T}}{ds}\right|
            \]

            where \textbf{T} is the unit tangent vector.
        \end{theorem}

        The curvature is easier to compute if it is expressed in terms of the paramter $t$ instead of $s$, so we use the Chain Rule to write

        \[
            \frac{d\mathbf{T}}{dt} = \frac{d\mathbf{T}}{ds}\frac{ds}{dt}
        \]

        and

        \[
            \kappa = \left|\frac{d\mathbf{T}}{ds}\right| = \left|\frac{\frac{d\mathbf{T}}{dt}}{\frac{ds}{dt}}\right|
        \]

        But $\frac{ds}{dt}=|r'(t)|$, so

        \begin{corollemma}{Curvature: Corollary}
            \[
                \kappa (t) = \left|\frac{\mathbf{T}'(t)}{\mathbf{r}'(t)}\right|
            \]
        \end{corollemma}

        \textit{\blue{Example 3:}} Show that the curvature of a circle of radius $a$ is $\frac{1}{a}$. \\

        Taking the circle to have center the origin, a parametrization is

        \[
            \mathbf{r}(t) = a\cos{t}\mathbf{i} + a\sin{t}\mathbf{j}
        \]

        Therefore

        \[
            \mathbf{r}'(t) = -a\sin{t}\mathbf{i} + a\cos{t}\mathbf{j}
        \]

        and

        \[
            |r'(t)| = a
        \]

        so

        \[
            T(t) = \frac{r'(t)}{|r'(t)|} = -\sin{t}\mathbf{i} + \cos{t}\mathbf{j}
        \]

        and

        \[
            T'(t) = -\cos{t}\mathbf{i} - \sin{t}\mathbf{j}
        \]

        This gives $|T'(t)| = 1$ and it follows that

        \[
            \kappa (t) = \frac{|T'(t)|}{|r'(t)|} = \frac{1}{a}
        \]

        \begin{theorem}{Curvature, Alternative}
            Although the other curvature formula can be used in all cases to compute the curvature, the formula below is often more convenient to apply. For the vector function \textbf{r}, the curvature is given by

            \[
                \kappa (t) = \frac{|r'(t) \times r''(t)|}{|r'(t)|^3}
            \]
        \end{theorem}

        \begin{proof}
            Since $\mathbf{T = \frac{r'}{|r'|}}$ and $\mathbf{|r'|} = \frac{ds}{dt}$, we have

            \[
                r' = |r'|T = \frac{ds}{dt}T
            \]

            so the Product Rule gives

            \[
                r'' = \frac{d^2 s}{dt^2}T + \frac{ds}{dt}T'
            \]

            Using the fact that $T \times T = 0$, we have

            \[
                r' \times r'' = \left(\frac{ds}{dt}\right)^2 (T \times T')
            \]

            Now $|T(t)|=1$ for all $t$, so \textbf{T} and \textbf{T'} are orthogonal. Hence,

            \[
                |r' \times r''| = \left(\frac{ds}{dt}\right)^2 |T \times T'| = \left(\frac{ds}{dt}\right)^2 |T||T'| = \left(\frac{ds}{dt}\right)^2|T'|
            \]

            Thus

            \[
                |T'| = \frac{|r' \times r''|}{\left(\frac{ds}{dt}\right)^2} = \frac{|r' \times r''|}{|r'|^2}
            \]

            and

            \[
                \kappa = \frac{|T'|}{|r'|} = \frac{|r' \times r''|}{|r'|^3}
            \]
        \end{proof}

        \textit{\blue{Example 4:}} Find the curvature of the twisted cubic $r(t) = \langle t, t^2, t^3\rangle$ at a general point and at $(0,0,0)$. \\

        We first compute the required ingredients:

        \begin{align*}
            r'(t)   &= \langle 1, 2t, 3t^2\rangle \\
            r''(t)  &= \langle 0, 2, 6t\rangle \\
            r'(t) \times r''(t) &=
            \begin{vmatrix}
                \mathbf{i}  & \mathbf{j}    & \mathbf{k} \\
                1           & 2t            & 3t^2 \\
                0           & 2             & 6t
            \end{vmatrix}
            = 6t^2 \mathbf{i} - 6t\mathbf{j} + 2\mathbf{k} \\
            |r'(t) \times r''(t)|   &= \sqrt{36t^4 + 36t^2 + 4} = 2\sqrt{9t^4 + 9t^2 + 1}
        \end{align*}

        Thus

        \[
            \kappa (t) = \frac{|r'(t) \times r''(t)|}{|r'(t)|^3} = \frac{2\sqrt{1+9t^2 + 9t^4}}{\left(1+4t^2+9t^4\right)^{\frac{3}{2}}}
        \]

        At the origin, where $t=0$, the curvature is $\kappa(0) = 2$.

        \begin{corollemma}{Curvature, Special Case}
            For the special case of a plane curve with equation $y=f(x)$, we choose $x$ as the parameter and write $r(x) = x\mathbf{i} + f(x)\mathbf{j}$. Then $r'(x) = \mathbf{i} + f'(x)|mathbf{j}$ and
            $\mathbf{r}''(x)=f''(x)\mathbf{j}$. Since $\mathbf{i\times j = k}$ and $\mathbf{j\times j} = 0$, it follows that $r'(x) \times r''(x) = f''(x)k$. We also have $|r'(x)| = \sqrt{1+[f'(x)]^2}$ and thus

            \[
                \kappa (x) = \frac{|f''(x)|}{\left[1+(f'(x))^2\right]^{|frac{3}{2}}}
            \]
        \end{corollemma}

        \textit{\blue{Example 5:}} Find the curvature of the parabola $y=x^2$ at the points (0,0), (1,1), and (2,4). \\

        Since $y' = 2x$ and $y'' = 2$, we have

        \[
            \kappa (x) = \frac{|y''|}{\left[1+(y')^2\right]^{\frac{3}{2}}} = \frac{2}{\left(1+4x^2\right)^{\frac{3}{2}}}
        \]

        The curvature at (0,0) is $\kappa (0) = 2$. At (1,1) it is $\kappa (1) = \frac{2}{5^{\frac{3}{2}}}\approx 0.18$. At (2,4) it is $\kappa (2) = \frac{2}{17^{\frac{3}{2}}}\approx 0.03$. \\

        At a given point on a smooth space curve $r(t)$, there are many vectors that are orthogonal to the unit tangent vector $\mathbf{T}(t)$. We single out one by observing that, becasue $|\mathbf{T}(t)| = 1\forall t$,
        we have $\mathbf{T}(t)\cdot \mathbf{T}'(t) = 0$, so $\mathbf{T}'(t)$ is orthogonal to $\mathbf{T}(t)$. Noe that in general $\mathbf{T}'(t)$ is itself not a unit vector. But at any point where $\kappa \not = 0$
        we can define the \textbf{principle unit normal vector} $\mathbf{N}(t)$ (\textbf{unit normal}) as

        \[
            \mathbf{N}(t) = \frac{\mathbf{T}'(t)}{|\mathbf{T}'(t)|}
        \]

        Geometrically, the unit normal vector indicates the direction in which the curve is turing at each point. The vector $\mathbf{B}(t) = \mathbf{T}(t) \times \mathbf{N}(t)$ is called the \textbf{binormal vector} and
        it is perpendicular to both $\mathbf{T}$ and $\mathbf{N}$ and is also a unit vector.

        \begin{figure*}[hbt!]
            \centering
            \includegraphics[scale = 0.75]{Resources/13.3_TNB}
        \end{figure*}

        \textit{\blue{Example 6:}} Find the unit normal and binormal vectors for the circular helix

        \[
            \mathbf{r}(t) = \cos{t}\mathbf{i} + \sin{t}\mathbf{j} + t\mathbf{k}
        \]

        We first compute the ingredients needed for the unit normal vector:

        \begin{align*}
            r'(t)   &= -\sin{t}i + \cos{t}j + k \\
            |r'(t)| &= \sqrt{2} \\
            T(t)    &= \frac{r'(t)}{|r'(t)|} = \frac{1}{\sqrt{2}} (-\sin{t}i + \cos{t}j + k) \\
            T'(t)   &= \frac{1}{\sqrt{2}}\left(-\cos{t}i - \sin{t}j) \\
            |T'(t)| &= \frac{1}{\sqrt{2}} \\
            N(t)    &= \frac{T'(t)}{|T'(t)|} = -\cos{t}i - \sin{t}j = \langle -\cos{t}, -\sin{t}, 0\rangle
        \end{align*}

        This shows that the normal vector at any point on the helix is horizontal and points toward the $z$-axis. The binormal vector is

        \begin{align*}
            B(t)    &= T(t) \times N(t) \\
                    &= \frac{1}{\sqrt{2}}
                        \begin{bmatrix}
                            \mathbf{i}  & \mathbf{j}    & \mathbf{k} \\
                            -\sin{t}    & \cos{t}       & 1 \\
                            -\cos{t}    & -\sin{t}      & 0
                        \end{bmatrix} \\
                    &= \frac{1}{\sqrt{2}}\langle \sin{t}, -\cos{t}, 1\rangle
        \end{align*}

        \textbf{TNB frame:} refers to the set of orthogonal vectors $T, N, B$, which are the tangent, normal, and binormal vectors.

        \begin{figure*}[hbt!]
            \centering
            \includegraphics[scale = 0.75]{Resources/13.3_TNB_Frame}
        \end{figure*}

        \textbf{Normal plane:} determined by normal and binormal vectors N and B at a point $P$ on a curve $C$ \\
        $\bullet$ consists of all lines orthogonal to the tangent vector $\mathbf{T}$ \\

        \textbf{Osculating plane:} determined by vectors T and N at a point $P$ on a curve $C$ \\
        $\bullet$ comes from Latin \textit{osculum} "to kiss" because it is the plane that comes closest to containing the part of the curve near $P$. \\

        \textbf{Oscalating circle (circle of curvature)}: the circle that lies in the osculating plane of $C$ at $P$, has the same tangnet as $C$ as $P$, lies on the concave side of $C$ (toward which N points), and has
        radius $\rho = \frac{1}{\kappa}$ (the reciprocal of the curvature) \\
        $\bullet$ the circle that best describes how $C$ behaves near $P$ \\
        $\bullet$ share the same tangnet, normal, and curvature at $P$.

    \subsection{Motion in Space: Velocity and Acceleration}     % 13.4

        \begin{theorem}{Velocity vector}
            The average velocity over a time interval of length $h$ and its limit is the velocity vector $v(t)$ at time $t$:

            \[
                v(t) = \lim_{h\to 0} \frac{r(t+h) - r(t)}{h} = r'(t)
            \]
        \end{theorem}

        \begin{theorem}{Speed}
            The speed of a particle at time $t$ is the magnitude of the velocity vector:

            \[
                |v(t)| = |r'(t)| = \frac{ds}{dt}
            \]
        \end{theorem}

        \begin{theorem}{Acceleration}
            \[
                a(t) = v'(t) = r''(t)
            \]
        \end{theorem}

        \textit{\blue{Example:}} A moving particle starts at an intiial position $r(0) = \langle 1, 0, 0 \rangle$ with initial velocity $v(0) = i - j + k$. Its acceleration is $a(t) = 4ti + 6tj + k$. Find its velocity
        and position at time $t$. \\

        Since $a(t) = v'(t)$, we have

        \begin{align*}
            v(t)    &= \int a(t)dt \\
                    &= \int (4t\mathbf{i} + 6t\mathbf{j} + \mathbf{k}) dt \\
                    &= 2t^2 \mathbf{i} + 3t^2 \mathbf{j} + t\mathbf{k} + C
        \end{align*}

        Because $v(0) = i-j+k$, $C=i-j+k$ and

        \begin{align*}
            v(t)    &= 2t^2 \mathbf{i} + 3t^2 \mathbf{j} + t\mathbf{k} + \mathbf{i-j+k} \\
                    &= (2t^2 + 1)\mathbf{i} + (3t^2 -1)\mathbf{j} + (t+1)\mathbf{k}
        \end{align*}

        Since $v(t) = r'(t)$, we have

        \begin{align*}
            r(t)    &= \int v(t) dt \\
                    &= \int \left[(2t^2 + 1)\mathbf{i} + (3t^2 -1)\mathbf{j} + (t+1)\mathbf{k}\right] dt \\
                    &= \left(\frac{2}{3}t^3 +t\right)\mathbf{i} + (t^3 -t)\mathbf{j} + \left(\frac{1}{2}t^2 + t\right) \mathbf{k} + \mathbf{D}
        \end{align*}

        Putting $t=0$, we find that $\mathbf{D=r}(0)=\mathbf{i}$, so the position at time $t$ is given by

        \begin{align*}
            \mathbf{r}(t) = \left(\frac{2}{3}t^3 + t + 1\right) \mathbf{i} + (t^3 - t)\mathbf{j} + \left(\frac{1}{2}t^2 + t\right) \mathbf{k}
        \end{align*}

        In general,

        \[
            v(t) = v(t_0) + \int_{t_0}^t a(u) du
        \]

        and

        \[
            r(t) = r(t_0) + \int_{t_0}^t v(u) du
        \]

        \textit{\blue{Example 2:}} An object with mass $m$ that moves in a circular path with constant angular speed $\omega$ has position vector $r(t) = a\cos{\omega t}\mathbf{i}+a\sin{\omega t}\mathbf{j}$. Find the
        force acting on the object and show that it is directed toward the origin. \\

        We have

        \begin{align*}
            v(t)    &= r'(t) = -a\omega \sin{\omega t}\mathbf{i} + a\omega \cos{\omega t} \mathbf{j} \\
            a(t)    &= v'(t) = -a\omega^2 \cos{\omega t}\mathbf{i} - a\omega^2 \sin{\omega t}\mathbf{j}
        \end{align*}

        Thus Newton's Second Law gives the force as

        \[
            F(t) = ma(t) = -m\omega^2 (a\cos{\omega t}\mathbf{i} + a\sin{\omega t}\mathbf{j}) = -m\omega^2 \mathbf{r}(t)
        \]

        The object moving with position $P$ has angular speed $\omega = \frac{d\theta}{dt}$. \\

        \textit{\blue{Example 3:}} A projectile is fired with angle of elevation $\alpha$ and initial velocity $v_0$. Assuming that air resistance is negligible and the only external force is due to gravity, find the
        position function $r(t)$ of the projectile. What value of $\alpha$ maximizes the range (the horizontal distance traveled)?

        \begin{figure*}[hbt!]
            \centering
            \includegraphics[scale = 0.75]{Resources/13.4_Projectile}
        \end{figure*}

        We set up the axes so that the projectile starts at the origin. Since the force due to gravity acts downward, we have

        \[
            F = m\mathbf{a} = -mg\mathbf{j}
        \]

        where $g=|a|\approx 9.8 \text{ m/s}^2$. Thus

        \[
            \mathbf{a} = -g\mathbf{j}
        \]

        Since $\mathbf{v}'(t) = \mathbf{a}$, we have

        \[
            v(t) = -gt\mathbf{j} + C
        \]

        where $C= v(0) = v_0$. Therefore

        \[
            r'(t) = v(t) = -gt\mathbf{j} + v_0
        \]

        Integrating again, we obtain

        \[
            r(t) = -\frac{1}{2}gt^2 \mathbf{j} + t\mathbf{v_0} + D
        \]

        But $D = r(0) = 0$, so the position vector of the projectile is given by

        \begin{theorem}{Position Vector of Projectile}
            r(t) = -\frac{1}{2}gt^2 \mathbf{j} + t\mathbf{v_0}
        \end{theorem}

        If we write $|v_0| = v_0$, then

        \[
            v_0 = v_0 \cos{\alpha} \mathbf{i} + v_0 \sin{\alpha} \mathbf{j}
        \]

        and it follows that

        \[
            r(t) = (v_0 \cos{\alpha})t\mathbf{i} + \left[(v_0 \sin{\alpha})t - \frac{1}{2}gt^2\right]\mathbf{j}
        \]

        The parametric equations of the trajectory are therefore

        \begin{theorem}{Projectile Trajectory, Parametric Equations}
            \[
                x = (v_0 \cos{\alpha}) t
            \]

            and

            \[
                y = (v_0 \sin{\alpha})t - \frac{1}{2}gt^{2}
            \]
        \end{theorem}

        The horizontal distance $d$ is the value of $x$ when $y=0$. Setting $y=0$, we obtain $t=0$ or $t=\frac{2v_0 \sin{\alpha}}{g}$. This second value of $t$ then gives

        \[
            d = x = (v_0 \cos{\alpha}) \frac{2v_0 \sin{\alpha}}{g} = \frac{v_0^2 (2\sin{\alpha}{\cos{\alpha}})}{g} = \frac{v^2_0 \sin{2\alpha}}{g}
        \]

        Clearly $d$ has its maximum value when $\sin{2\alpha} = 1$, that is, $\alpha = 45^{\circ}$. \\

        \textit{\blue{Example 4:}} A projectile is fired with muzzle speed 150 m/s and angle of elevation $45^{\circ}$ from a position 10 m above ground level. Where does the projectile hit the ground, and with what
        speed? \\

        If we place the origin at ground level, then the initial position of the projectil is (0, 10) and so we need to adjust the parametric equation for $y$ by adding 10. With $v_0 = 150$ m/s, $\alpha = 45^{\circ}$,
        and $g = 9.8 \text{ m/s}^2$, we have

        \begin{align*}
            x   &= 150\cos{45^{\circ}}t = 75\sqrt{2}t \\
            y   &= 10 + 150\sin{45^{\circ}}t - \frac{1}{2}(9.8)t^2 = 10 + 75\sqrt{2}t - 4.9t^2
        \end{align*}

        Impact occurs when $y=0$, that is, $4.9t^2 - 75\sqrt{2}t-10 = 0$. Then

        \[
            t = \frac{75\sqrt{2} + \sqrt{11250+196}}{9.8} \approx 21.74
        \]

        Then $x=75\sqrt{2}(21.74)\approx 2306$, so the projectile hits the ground about 2306 m away. The velocity of the projectile is

        \[
            v(t) = r'(t) = 75\sqrt{2}\mathbf{i} + (75\sqrt{2} - 9.8t) \mathbf{j}
        \]

        so its speed at impact is

        \[
            |v(21.74)| = \sqrt{(75\sqrt{2})^2 + (75\sqrt{2} - 9.8\cdot 21.74)^2} \approx 151 \text{ m/s}
        \]

        We know that

        \[
            T(t) = \frac{r'(t)}{|r'(t)|} = \frac{v(t)}{|v(t)|} = \frac{\mathbf{v}}{v}
        \]

        and so

        \[
            \mathbf{v} = v\mathbf{T}
        \]

        Differentiating both sides with repsect to $t$,

        \[
            \mathbf{a} = \mathbf{v'} = v'\mathbf{T} + v\mathbf{T'}
        \]

        Using the expression for curvature, we have

        \[
            \kappa = \frac{|\mathbf{T'}|}{|\mathbf{r'}|} = \frac{|T'|}{v}\implies |T'| = \kappa v
        \]

        The unit normal vector was defined as $N = \frac{T'}{|T'|}$, so we have

        \[
            T' = |T'|N = \kappa v \mathbf{N}
        \]

        and our equation becomes

        \[
            \mathbf{a} = v'\mathbf{T} + \kappa v^2 \mathbf{N}
        \]

        \begin{theorem}{Tangential and Normal Components of Acceleration}
            Letting $a_T$ and $a_N$ be the tangential and normal components of acceleration, respectively, we have

            \[
                a_T = v' = \frac{v\cdot a}{v} = \frac{r'(t)\cdot r''(t)}{|r'(t)|}
            \]

            and

            \[
                a_N = \kappa v^2 = \frac{|r'(t) \times r''(t)|}{|r'(t)|^3} |r'(t)|^2 = \frac{|r'(t) \times r''(t)|}{|r'(t)|}
            \]
        \end{theorem}

