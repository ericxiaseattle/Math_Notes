\section{Partial Derivatives}

    \subsection{Functions of Several Variables}     % 14.1

        A \textbf{function $\mathbf{f}$ of two variables} is a rule that assigns to each ordered pair of real numbers $(x,y)$ in a set $D$ a unique real number denoted by $f(x,y)$. The set $D$ is the \textbf{domain} of
        $f$ and its \textbf{range} is hte set of values that $f$ takes on, that is, $\{f(x,y)|(x,y) \in D\}$. \\

        We often write $z=f(x,y)$ to make the explicit value taken on by $f$ at the general point $(x,y)$. The variables $x$ and $y$ are \textbf{independent variables} and $z$ is the dependent variable. A function of
        two variables is just a function whose domain is a subset of $\mathbb{R^2}$ and whose range is a subset of $\mathbb{R}$. \\

        \textit{\blue{Example 1:}} Consider the following function.

        \[
            f(x,y) = \frac{\sqrt{x+y+1}}{x-1}
        \]

        The expression for $f$ makes sense if the denominator is not 0 and the quantity under the square root sign is nonnegative. So the domain of $f$ is 

        \[
            D = \{(x,y)|x+y+1\geq 0, x\not = 1\}
        \]

        The inequality $x+y+1\geq 0$, or $y\geq -x-1$, describes the points that lie on or above the line $y=-x-1$, while $x\not = 1$ means that the points on the line $x=1$ must be excluded from the domain. Using this
        information, we can sketch the domain of $f(x,y)$:

        \begin{figure*}[hbt!]
            \centering
            \includegraphics[scale = 0.75]{Resources/14.1_Domain_Sketch}
        \end{figure*}

        \textit{\blue{Example 2:}} Now consider the function $f(x,y) = x\ln{(y^2 - x)}$. \\

        The expression $\ln{y^2 - x}$ is only defined when $y^2 - x > 0$, that is, $x < y^2$. Hence, the domain of $f$ is

        \[
            D = \{(x,y)|x < y^2\}
        \]

        That is the set of points to the left of the parabola $x=y^2$:

        \begin{figure*}[hbt!]
            \centering
            \includegraphics[scale = 0.75]{Resources/14.1_Domain_Sketch2}
        \end{figure*}

        The \textbf{Cobb-Douglas Production Function:} \\

        Let $P$ be the total production (the monetary value of all goods produced in a year), let $L$ be the amount of labor (the total number of person-hours worked in a year), and let $K$ be the amount of capital
        invested (the monetary value of all machinery equipment and buildings). Then the production can be modeled with the function

        \[
            P(L,K) = b L^{\alpha} K^{1-\alpha}
        \]

        Using federal economic data and the method of least squares, they found that

        \[
            P(L,K) = 1.01 L^{0.75} K^{0.25}
        \]

        The domain of this function is $\{(L,K)|L\geq 0, K\geq 0\}$ because $L$ and $K$ represent labor and capital and are therefore never negative. \\

        \textit{\blue{Example 3:}} Consider the function $g(x,y)=\sqrt{9-x^2-y^2}$. The domain of $g$ is

        \[
            D = \{(x,y) | 9-x^2 - y^2 \geq 0\} = \{(x,y) | x^2 + y^2 \leq 9 \}
        \]

        which is the disk with center $(0,0)$ and radius 3. The range of $g$ is

        \[
            \{z| z = \sqrt{9-x^2 - y^2}, (x,y) \in D\}
        \]

        Since $z$ is a positive square root, $z\geq 0$. Also, because $9-x^2 - y^2 \leq 9$, we have

        \[
            \sqrt{9-x^2 - y^2 } \leq 3
        \]

        So the range isa

        \[
            \{z| 0 \leq z \leq 3\} = [0,3]
        \]

        If $f$ is a function of two variables with domain $D$, then the \textbf{graph} of $f$ is the set of all points $(x,y,z)$ in $\mathbb{R^3}$ such that $z=f(x,y)$ and $(x,y)$ is in $D$. Just as the graph of a
        function $f$ of one variable is a curve $C$ with equation $y=f(x)$, so the graph of a function $f$ of two variables is a surface $S$ with equation $z=f(x,y)$. We can visualize the graph $S$ of $f$ as
        lying directly above or below its domain $D$ in the $xy$-plane.

        \begin{figure*}[hbt!]
            \centering
            \includegraphics[scale = 0.75]{Resources/14.1_Surface}
        \end{figure*}

        \textit{\blue{Example 4:}} Sketch the graph of the function $f(x,y)=6-3x-2y$ \\

        The graph of $f$ has the equation $z=6-3x-2y$, or $3x+2y+z=6$, which represents a plane. To graph the plane we first find the intercepts, which are $(2,0,0)$, $(0,3,0)$, and $(0,0,6)$. This helps us sketch
        the portion of the graph that lies in the first octant:

        \begin{figure*}[hbt!]
            \centering
            \includegraphics[scale = 0.75]{Resources/14.1_Linear_Function}
        \end{figure*}

        The function is a special case of the function

        \[
            f(x,y) = ax + by + c
        \]

        which is called a \textbf{linear function}. The graph of such a function has the equation

        \[
            z = ax + by + c
        \]

        or

        \[
            ax + by - z + c =0
        \]

        so it is plane.

        \textit{\blue{Example 5:}} Sketch the graph of $g(x,y)=\sqrt{9-x^2 - y^2}$. \\

        The graph has equation $z=\sqrt{9-x^2-y^2}$. We square both sides of this equation to obtain $z^2 = 9-x^2 - y^2$, or $x^2 + y^2 + z^2 = 9$, which recognize as an equation of the sphere with center the origin 
        and radius 3. But, since $z\geq 0$, the graph of $g$ is just the top half of this sphere.

        \begin{figure*}[hbt!]
            \centering
            \includegraphics[scale = 0.75]{Resources/14.1_Graph}
        \end{figure*}

        \textit{\blue{Example 6:}} Find the domain and range and sketch the graph of $h(x,y) = 4x^2 + y^2$. \\

        Notice that $h(x,y)$ is defined for all possible ordered pairs of real numbers $(x,y)$, so the domain is $\mathbb{R^2}$, the entire $xy$-plane. The range of $h$ is the set $[0,\infty)$ of all nonnegative real
        numbers. [Notice that $x^2 \geq 0$ and $y^2 \geq 0$, so $h(x,y) \geq 0$ for all $x$ and $y$.] The graph of $h$ has the equation $z=4x^2 + y^2$, which is the elliptic paraboloid sketched below. Horizontal traces
        are ellipses and vertical traces are parabolas.

        \begin{figure*}[hbt!]
            \centering
            \includegraphics[scale = 0.75]{Resources/14.1_Graph2}
        \end{figure*}

        The \textbf{level curves (contour curves)} of a function $f$ of two variables are the curves with equations $f(x,y)=k$, where $k$ is a constant (in the range of $f$) \\
        $\bullet$ is the set of all points in the domain of $f$ at which $f$ takes on a given value $k$ \\
        $\bullet$ shows where the graph of $f$ has height $k$ \\

        Notice in the figure below that the level curves $f(x,y) = k$ are just the traces of the graph of $f$ in the horizontal plane $z=k$ projected down to the $xy$-plane.

        \begin{figure*}[hbt!]
            \centering
            \includegraphics[scale = 0.75]{Resources/14.1_Contour_Traces}
        \end{figure*}

        \textit{\blue{Example 7:}} \\

        Sketch some level curves of the function $h(x,y) = 4x^2 + y^2 + 1$. \\

        The level curves are

        \[
            4x^2 + y^2 + 1 = k
        \]

        or

        \[
            \frac{x^2}{\frac{1}{4}(k-1)} + \frac{y^2}{k-1} = 1
        \]

        which for $k > 1$, describes a family of ellipses with semiaxes $\frac{1}{2}\sqrt{k-1}$ and $\sqrt{k-1}$. The figure below on the left shows a contour map of $h$ drawn by a computer. The figure below on the right
        shows these level curves lifted up to the graph of $h$ (an elliptic paraboloid) where they become horizontal traces.

        \begin{figure*}[hbt!]
            \centering
            \includegraphics[scale = 0.75]{Resources/14.1_Lifting_Level_Curves}
        \end{figure*}

        A \textbf{function of three variables}, $f$, is a rule that assigns to each ordered triple $(x,y,z)$ in a domain $D\subset \mathb{R}^3$ a unique real number denoted by $f(x,y,z)$. For instance, the temperature
        $T$ at a point on the surface of the earth depends on the longitude $x$ and latitude $y$ of the point and on the time $t$, so we could write $T=f(x,y,t)$. \\

        \textit{\blue{Example 8:}} Find the domain of $f$ if

        \[
            f(x,y,z) = \ln{(z-y)} + xy\sin{z}
        \]

        The expression for $f(x,y,z)$ is defined as long as $z-y>0$, so the domain of $f$ is

        \[
            D = \{(x,y,z) \in \mathbb{R}^3 | z > y\}
        \]

        This is a \textbf{half-space} consisting of all points that lie above the plane $z=y$.

        \textit{\blue{Example 9:}} Find the level surfaces of the function

        \[
            f(x,y,z) = x^{2} + y^2 + z^2
        \]

        The level surfaces are $x^2 + y^2 + z^2 = k$, where $k\geq 0$. These form a family of concentric spheres with radius $\sqrt{k}$. Thus, as $(x,y,z)$ varies over any sphere with center $O$, the value of
        $f(x,y,z)$ remains fixed.

        \begin{figure*}[hbt!]
            \centering
            \includegraphics[scale = 0.75]{Resources/14.1_Surfaces}
        \end{figure*}

        A \textbf{function of $\mathbf{n}$ variables} is a rule that assigns a number $z=f(x_1, x_2, \dots, x_n)$ to an $n$-tuple $(x_1, x_2, \dots, x_n)$ of real numbers. We denote by $\mathbb{R}^n$ the set of all such
        $n$-tuples.

        \[
            C = f(x_1, x_2, \dots, x_n) = c_1 x_1 + c_2 x_2 + \dots + c_n x_n
        \]

        The function $f$ is a real-valued function whose domain is a subset of $\mathbb{R}^n$. Sometimes we use vector notation to write such functions more compactly. If $\mathbf{x} = \langle x_1, x_2, \dots, x_n\rangle$,
        we often write $\mathbf{f(x)}$ in place of $\mathbf{f(x_1, x_2, \dots, x_n)}$. With this notation we can rewrite $f$ as

        \[
            \mathbf{f(x) = c\dot x}
        \]

        where $\mathbf{c} = \langle c_1, c_2, \dots, c_n\rangle$ and $\mathbf{c\cdot x}$ denotes the dot product of the vectors \textbf{c} and \textbf{x} in $V_n$. \\

        Thus there are three ways of looking at a function $f$ defined on a subset of $\mathbb{R}^n$:

        1. As a function of $n$ real variables $x_1, x_2, \dots, x_n$. \\
        2. As a function of a single point variable $(x_1, x_2, \dots, x_n)$ \\
        3. As a function of a single vector variable $\mathbf{x} = \langle x_1, x_2, \dots, x_n\rangle$.

    \subsection{Limits and Continuity}  % 14.2

        Let $f$ be a function of two variables whose domain $D$ includes points arbitrarily close to $(a,b)$. Then we say that the \textbf{limit of} $f(x,y)$ as $(x,y)$ approaches $(a,b)$ is $L$ and we write

        \[
            \lim_{(x,y)\to (a,b)} f(x,y) = L
        \]

        if for every number $\epsilon > 0$ there is a corresponding number $\delta > 0$ such that if

        \[
            (x,y) \in D
        \]

        and

        \[
            0 < \sqrt{(x-a)^2 + (y-b)^2} < \delta
        \]

        then

        \[
            |f(x,y) - L| < \epsilon
        \]

        \begin{theorem}{Theorem}
            If $f(x,y)\to L_1$ as $(x,y)\to (a,b)$ along a path $C_1$ and $f(x,y) \to L_2$ as $(x,y)\to (a,b)$ along a path $C_2$, where $L_1 \not = L_2$, then $\lim_{(x,y)\to (a,b)}f(x,y)$ does not exist.
        \end{theorem}

        \textit{\blue{Example:}} Show that the limit below does not exist.

        \[
            \lim_{(x,y)\to (0,)} \frac{x^2 - y^2}{x^2 + y^2}
        \]

        Let $f(x,y) = \frac{x^2 - y^2}{x^2 + y^2}$. First approach $(0,0)$ along the $x$-axis. Then $y=0$ gives $f(x,0) = \frac{x^2}{x^2} = 1$ for all $x\not = 0$, so

        \[
            f(x,y)\to 1 \text{ as } (x,y)\to (0,0) \text{ along the } x\text{-axis}
        \]

        We now approach along the $y$-axis by putting $x=0$. Then $f(0,y)=\frac{-y^2}{y^2}=-1$ for all $y\not = 0$, so

        \[
            f(x,y)\to -1\text{ as } (x,y)\to (0,0) \text{ along the }y\text{-axis}
        \]

        \textit{\blue{Example 2:}} Find the limit below if it exists.

        \[
            \lim_{(x,y)\to (0,0)} \frac{3x^2 y}{x^2 + y^2}
        \]

        We can show that the limit along any line through the origin is 0. This doesn't prove that the given limit is 0, but the limits along the parabolas $y=x^2$ and $x=y^2$ also turn out to be 0, so we begin to suspect
        that the limit does exist and is equal to 0. \\

        Let $\epsilon > 0$. We want to find $\delta > 0$ s.t.

        \[
            \text{if } 0 < \sqrt{x^2 + y^2} < \delta \text{ then } \left|\frac{3x^2 y}{x^2 + y^2} - 0\right| < \epsilon
        \]

        that is, if $0 < \sqrt{x^2 + y^2} < \delta$, then $\frac{3x^2 |y|}{x^2 + y^2} < \epsilon$. But $x^2 \leq x^2 + y^2$ since $y^2 \geq 0$, so $\frac{x^2}{x^2 + y^2} \leq 1$ and therefore

        \[
            \frac{3x^2 |y|}{x^2 + y^2}\leq 3|y| = 3\sqrt{y^2} \leq 3 \sqrt{x^2 + y^2}
        \]

        Thus if we choose $\delta = \frac{\epsilon}{3}$ and let $0 < \sqrt{x^2 + y^2} < \delta$, then

        \[
            \left|\frac{3x^2 y}{x^2 + y^2} - 0\right| \leq 3 \sqrt{x^2 + y^2}  < 3\delta = 3\left(\frac{\epsilon}{3}\right) = \epsilon
        \]

        Hence,

        \[
            \lim_{(x,y) \to (0,0)} \frac{3x^2 y}{x^2 + y^2} = 0
        \]

        A function $f$ of two variables is \textbf{continuous} at $(a,b)$ if

        \[
            \lim_{(x,y) \to (a,b)} f(x,y) = f(a,b)
        \]

        $f$ is \textbf{continuous on} $D$ if $f$ is continuous at every point $(a,b)$ in $D$. \\

        A \textbf{polynomial function of two variables} is a sum of terms of the form $cx^m y^n$ where $c$ is a constant and $m$ and $n$ are nonnegative integers. A \textbf{rational function} is a ratio of polynomials.
        By examination the functions $f(x,y) = x,g(x,y) = y,$ and $h(x,y)=c$ are continuous. Since any polynomial can be built up out of the simple functions $f,g,$ and $h$ by multiplication and addition, it follows that
        \textit{all polynomials are continuous on} $\mathbb{R}^2$.

        \begin{theorem}{Theorem}
            If $f$ is defined on a subset $D$ of $\mathbb{R}^n$, then $\lim_{x\to a} f(x) = L$ means that for every number $\epsilon > 0$ there is a corresponding number $\delta > 0$ s.t.

            \[
                \text{if } x\in D \text{ and } 0 < |x-a| < \delta \text{ then } |f(x) - L| < \epsilon
            \]
        \end{theorem}

    \subsection{Partial Derivatives}    % 14.3

        The \textbf{partial derivative} of $f$ with respect to $x$ at $(a,b)$ and denote it by $f_x (a,b)$. Thus

        \[
            f_x (a,b) = g'(a)
        \]

        where

        \[
            g(x) = f(x,b)
        \]

        By the definition of the derivative

        \[
            g'(a) = \lim_{h\to 0} \frac{g(a+h) - g(a)}{h}
        \]

        and so

        \[
            f_x (a,b) = \lim_{h\to 0} \frac{f(a+h,b) - f(a,b)}{h}
        \]

        By this, if $f$ is a function of two variables, its \textbf{partial derivatives} are the functions $f_x$ and $f_y$ defined by

        \begin{align*}
            f_x (x,y)   &= \lim_{h\to 0} \frac{f(x+h, y) - f(x,y)}{h} \\
            f_y (x,y)   &= \lim_{h\to 0} \frac{f(x,y+h) - f(x,y)}{h}
        \end{align*}

        To find partial derivatives of $z=f(x,y)$, \\
        1. To find $f_x$, regard $y$ as a constant and differentiate $f(x,y)$ with respect to $x$. \\
        2. To find $f_y$, regard $x$ as a constant and differentiate $f(x,y)$ with respect to $y$. \\

        The partial derivatives of $f$ at $(a,b)$ are the slopes of the tangents to $C_1$ and $C_2$:

        \begin{figure*}[hbt!]
            \centering
            \includegraphics[scale = 0.75]{Resources/14.3_Slopes}
        \end{figure*}

        \textit{\blue{Example:}} If $f(x,y) = 4 - x^2 - 2y^2$, find $f_x (1,1)$ and $f_y (1,1)$ and interpret these numbers as slopes. \\

        We have

        \begin{align*}
            f_x (x,y) &= -2x \\
            f_y (x,y) &= -4y \\
            f_x (1,1) &= -2 \\
            f_y (1,1) &= -4
        \end{align*}

        The graph of $f$ is the paraboloid $z=4-x^2 - 2y^2$ and the vertical plane $y=1$ intersects it in hte parabola $z=2-x^2, y=1$. The slope of the tangent line to this parabola at the point $(1,1,1)$ is the parabola
        $z=3-2y^2,x=1$, and the slope of the tangent line at $(1,1,1)$ is $f_y (1,1) = -4$.

        \begin{figure}[hbt!]
            \centering
            \begin{subfigure}[b]{.45\textwidth}
                \includegraphics[scale = 0.75]{Resources/14.3_Paraboloid}
            \end{subfigure}
            \begin{subfigure}[b]{.45\textwidth}
                \includegraphics[scale = 0.75]{Resources/14.3_Paraboloid2}
            \end{subfigure}
        \end{figure}

        \textit{\blue{Example 2:}} Find $\frac{\partial z}{\partial x}$ and $\frac{\partial z}{\partial y}$ if $z$ is defined implicitly as a function of $x$ and $y$ by the equation

        \[
            x^3 + y^3 + z^3 + 6xyz = 1
        \]

        To find $\frac{\partial z}{\partial x}$, we differentiate implicitly with respect to $x$, being careful to treat $y$ as a constant:

        \[
            3x^2 + 3z^2 \frac{\partial z}{\partial x} + 6yz + 6xy \frac{\partial z}{\partial x} = 0
        \]

        Solving this equation for $\frac{\partial z}{\partial x}$, we have

        \[
            \frac{\partial z}{\partial x} = -\frac{x^2 + 2yz}{z^2 + 2xy}
        \]

        Similarly, implicit differentiation with respect to $y$ gives

        \[
            \frac{\partial z}{\partial x} = -\frac{y^2 + 2xz}{z^2 + 2xy}
        \]

        If $f$ is a function of two variables, then its partial derivatives $f_x$ and $f_y$ are also functions of two variables, so we can consider their partial derivatives $(f_x)_x, (f_x)y,$ and $(f_y)_y$, which are
        called the \textbf{second partial derivatives} of $f$. If $z=f(x,y)$, we use the following notaiton:

        \begin{align*}
            (f_x)_x &= f_{xx} = f_{11} = \frac{\partial}{\partial x} \left(\frac{\partial f}{\partial x}\right) = \frac{\partial^2 f}{\partial x^2} = \frac{\partial^2 z}{\partial x^2} \\
            (f_x)_y &= f_{xy} = f_{12} = \frac{\partial}{\partial y}\left(\frac{\partial f}{\partial x}\right) = \frac{\partial^2 f}{\partial y \partial x} = \frac{\partial^2 z}{\partial y \partial x} \\
            (f_y)_z &= f_{yx} = f_{21} = \frac{\partial}{\partial x}\left(\frac{\partial f}{\partial y}\right) = \frac{partial^2 f}{\partial x \partial y} = \frac{\partial^2 z}{\partial x \partial y} \\
            (f_y)_y &= f_{yy} = f_{22} = \frac{\partial}{\partial y}\left(\frac{\partial f}{\partial y}\right) = \frac{|partial^2 f}{\partial y^2} = \frac{\partial^2 z}{\partial y^2}
        \end{align*}

        Thus the notation $f_{xy}$ or $\frac{\partial^2 f}{\partial y \partial x}$ means that we first differentiate with respect to $x$ and then with respect to $y$, whereas in computing $f_{yx}$ the order is reversed.

        \begin{theorem}{Clairaut's Theorem}
            Suppose $f$ is defined on a disk $D$ that contains the point $(a,b)$. If the functions $f_{xy}$ and $f_{yz}$ are both continuous on $D$, then

            \[
                f_{xy} (z,b) = f_{yx} (a,b)
            \]
        \end{theorem}

        Partial derivatives of order 3 or higher can be defined. For instance,

        \[
            f_{xyy} = (f_{xy})_y = \frac{\partial}{\partial y}\left(\frac{\partial^2 f}{\partial y \partial x}\right) = \frac{\partial^3 f}{\partial y^2 \partial x}
        \]

        Using Clairaut's Theorem, it can be shown that $f_{xyy} = f_{yxy} = f_{yyx}$ if these functions are continuous. \\

        \textit{\blue{Example 3:}} Calculate $f_{xxyz}$ if $f(x,y,z) = \sin{3x+yz}$.

        \begin{align*}
            f_x &= 3\cos{(3x+yz)} \\
            f_{xx}  &= -9\sin{(3x+yz)} \\
            f_{xxy} &= -9z\cos{(3x+yz)} \\
            f_{xxyz}&= -9\cos{(3x+yz)} + 9yz\sin{(3x+yz)}
        \end{align*}

        \begin{axiom}{Laplace's Equation}
            \[
                \frac{\partial^2 u}{\partial x^2} + \frac{\partial^2 u}{\partial y^2} = 0
            \]

            Solutions of this equation are called \textbf{harmonic functions} and play a role in problems of heat conduction, fluid flow, and electric potential.
        \end{axiom}

        \textit{\blue{Example 4:}} Show that the function $u(x,y) = e^x \sin{y}$ is a solution of Laplace's equation. \\

        We first compute the needed second-order partial derivatives:

        \begin{align*}
            u_x &= e^x \sin{y}, & u_y = e^x \cos{y} \\
            u_{xx} &= e^x \sin{y}, & u_{yy} = -e^x \sin{y}
        \end{align*}

        So

        \[
            u_{xx} + u_{yy} = e^x \sin{y} - e^x \sin{y} = 0
        \]

        Therefore $u$ satisfies Laplace's equation. \\

        \begin{axiom}{Wave Equation}
            The \textbf{wave equation}

            \[
                \frac{\partial^2 u}{\partial t^2} = a^2 \frac{\partial^2 u}{\partial x^2}
            \]

            describes the motion of a waveform, which could be an ocean wave, a sound wave, a light wave, or a wave traveling along a vibrating string.
        \end{axiom}

        \textit{\blue{Example 5:}} Verify that the function $u(x,t) = \sin{(x-at)}$ satisfies the wave equation.

        \begin{align*}
            u_x &= \cos{(x-at)} & u_t &= -a\cos{(x-at)} \\
            u_{xx}  &= -\sin{(x-at)} & u_{tt} = -a^2 \sin{(x-at)} = a^2 u_{xx}
        \end{align*}

        So $u$ satisfies the wave equation. \\

        If the production function is denoted by $P=P(L,K)$, then the partial derivative $\frac{\partial P}{\partial L}$ is the rate at which produces changes with respect to the amount of labor. The marginal production
        with respect to labor or the \textbf{marginal productivity of labor}. Likewise, the partial derivative $\frac{\partial P}{\partial K}$ is the rate of change of production with respect to capital and is called the
        \textbf{marginal productivity of capital}. In these terms, the assumptions made by Cobb and Douglas can be stated as follows: \\

        1. If either labor or capital vanishes then so will production \\
        2. The marginal productivity of labor is proportional to the amount of production per unit of labor \\
        3. The marginal productivity of capital is proportional to the amount of production per unit of capital \\

        Because the production per unit of labor is $\frac{P}{L}$, assumption 2 says that

        \[
            \frac{\partial P}{\partial L} = \alpha \frac{P}{L}
        \]

        for some constant $\alpha$. If we keep $K$ constant ($K = K_0$), then this partial differential equation becomes an ODE:

        \[
            \frac{dP}{dL} = \alpha \frac{P}{L}
        \]

        If we solve this separable DE then

        \[
            P(L,K_0) = C_1 (K_0) L^{\alpha}
        \]
)
        Similarly,

        \[
            P(L_0,K) = C_2 (L_0) K^{\beta}
        \]

        Comparing the latter two equations, we have

        \[
            P(L,K) = b L^{\alpha} K^{\beta}
        \]

        where $b$ is a constant that is independent of both $L$ and $K$.


    \subsection{Tangent Planes and Linear Approximations}   % 14.4

        Suppose a surface $S$ has equation $z=f(x,y)$ where $f$ has continuous first partial derivatives, and let $P(x_0, y_0, z_0)$ be a point on $S$. As in the preceding section, let $C_1$ and $C_2$ be the curves
        obtained by intersecting the vertical planes $y=y_0$ and $x=x_0$ with the surface $S$. Then the point $P$ lies on both $C_1$ and $C_2$. Let $T_1$ and $T_2$ be the tangent lines to the curves $C_1$ and $C_2$ at
        the point $P$. Then the \textbf{tangent plane} to the surface $S$ at the point $P$ is defined to be the plane that contains both tangent lines $T_1$ and $T_2$.

        \begin{figure*}[hbt!]
            \centering
            \includegraphics[scale = 0.75]{Resources/14.4_Tangent_Plane}
        \end{figure*}

        Suppose $f$ has continuous partial derivatives. An equation of the tangent plane to the surface $z=f(x,y)$ at the point $P(x_0, y_0, z_0)$ is

        \[
            z - z_0 = f_x (x_0, y_0) (x-x_0) + f_y (x_0, y_0) (y-y_0)
        \]

        Note the similarity between the equation of a tangent plane and the equation of a tangent line:

        \[
            y - y_0 = f'(x_0) (x-x_0)
        \]

        \textit{\blue{Example:}} Find the tangent plane to the elliptic paraboloid $z=2x^2 + y^2$ at the point $(1,1,3)$. \\

        Let $f(x,y) = 2x^2 + y^2$. Then

        \begin{align*}
            f_x (x,y)   &= 4x & f_y (x,y) &= 2y \\
            f_x (1,1)   &= 4  & f_y (1,1) &= 2
        \end{align*}

        Then the tangent plane at $(1,1,3)$ is given by

        \[
            z - 3 = 4(x-1) + 2(y-1)
        \]

        or

        \[
            z = 4x + 2y - 3
        \]

        The linear function whose graph is this tangent plane, namely

        \[
            L(x,y) = f(a,b) + f_x (a,b)(x-a) + f_y (a,b) (y-b)
        \]

        is called the \textbf{linearization} of $f$ at $(a,b)$ and the approximation

        \[
            f(x,y) \approx f(a,b) + f_x (a,b) (x-a) + f_y (a,b) (y-b)
        \]

        is called the \textbf{linear approximation} or the \textbf{tangent plane approximation} of $f$ at $(a,b)$. \\

        \begin{axiom}{Differentiable}
            If $z=f(x,y)$, then $f$ is \textbf{differentiable} at $(a,b)$ if $\Delta z$ can be expressed in the form

            \[
                \Delta z = f_x (a,b) \Delta x + f_y (a,b) \Delta y + \epsilon_1 \Delta x + \epsilon_2 \Delta y
            \]

            where $\epsilon_1$ and $\epsilon_2 \to 0$ as $(\Delta x, \Delta y)\to (0,0)$.
        \end{axiom}

        \begin{theorem}{Theorem}
            If the partial derivatives $f_x$ and $f_y$ exist near $(a,b)$ and are continuous at $(a,b)$, then $f$ is differentiable at $(a,b)$.
        \end{theorem}

        \textit{\blue{Example 2:}} Show that $f(x,y) = xe^{xy}$ is differentiable at $(1,0)$ and find its linearization there. Then use it to approximate $f(1.1, -0.1)$.

        \begin{align*}
            f_x(x,y)    &= e^{xy} + xye^{xy}    & f_y (x,y) &= x^2 e^{xy} \\
            f_x(1,0)    &= 1                    & f_y (1,0) &= 1
        \end{align*}

        Both $f_x$ and $f_y$ are continuous functions, so $f$ is differentiable. The linearization is then

        \begin{align*}
            L(x,y)  &= f(1,0) + f_x (1,0)(x-1) + f_y (1,0)(y-0) \\
                    &= 1 + 1(x-1) + 1\cdot y \\
                    &= x + y
        \end{align*}

        The corresponding linear approximation is

        \[
            xe^{xy} \approx x + y
        \]

        so

        \[
            f(1.1, -0.1) \approx 1.1 - 0.1 = 1
        \]

        \textit{\blue{Example 3:}} Let the heat index (perceived temperature) $I$ as a function of the actual temperature $T$ and the relative humidity $H$ and gave the following table of values from the National Weather
        Service. Find a linear approximation for the heat index $I = f(T,H)$ when $T$ is near $96^{\circ}$ F and $H$ is near 70\%. Use it to estimate the heat index when the temperature is $97^{\circ}$ F and the relative
        humidity is 72\%. 

        \begin{figure*}[hbt!]
            \centering
            \includegraphics[scale = 0.75]{Resources/14.4_Heat_Index}
        \end{figure*}

        We read from the table that $f(96, 70) = 125$. In section 14.3 we used the tabular values to estimate that $f_T (96, 70) \approx 3.75$ and $f_H (96, 70) \approx 0.9$. So the linear approximation is

        \begin{align*}
            f(T,H)  &\approx f(96,70) + f_T (96,70)(T-96) + f_H (96,70) (H-70) \\
                    &\approx 125 + 3.75(T-96) + 0.9(H-70)
        \end{align*}

        In particular,

        \[
            f(97,72) \approx 125 + 3.75(1) + 0.9(2) = 130.55
        \]

        Therefore, when $T=97^{\circ}$F and $H=72\%$, the heat index is

        \[
            I   \approx 131^{\circ}\text{F}
        \]

        Recall that for a one variable function $y=f(x)$, the differential $dx$ is an independent variable and the differential of $y$ is given by

        \[
            dy = f'(x) dx
        \]

        For a differentiable function of two variables $z=f(x,y)$, the \textbf{differentials} $dx$ and $dy$ are independent variables and the differential $dz$, also called the \textbf{total differential}, is defined by

        \[
            dz = f_x (x,y) dx + f_y (x,y) dy = \frac{\partial z}{\partial x}dx + \frac{\partial z}{\partial y}dy
        \]

        If we take $dx=\Delta x = x-a$ and $dy = \Delta y = y-b$ then the differential of $z$ is

        \[
            dz = f_x (a,b) (x-a) + f_y (a,b) (y-b)
        \]

        So in differential notation, the linear approximation is given by

        \[
            f(x,y) \approx f(a,b) + dz
        \]

        \textit{\blue{Example 4:}} If $z=f(x,y) = x^2 + 3xy - y^2$, find the differential $dz$. If $x$ changes from 2 to 2.05 and $y$ changes from 3 to 2.96, compare the values of $\Delta z$ and $dz$. \\

        We have

        \begin{align*}
            dz  &= \frac{\partial z}{\partial x}dx + \frac{\partial z}{\partial y} dy \\
                &= (2x + 3y) dx \\
                &= (3x - 2y) dy
        \end{align*}

        Putting $x=2, dx = \Delta x = 0.05, y = 3$, and $dy = \Delta y = -0.04$, we get

        \[
            dz = [2(2) + 3(3)] 0.05 + [3(2) - 2(3)](-0.04) = 0.65
        \]

        The increment of $z$ is

        \begin{align*}
            \Delta z    &= f(2.05, 2.96) - f(2,3) \\
                        &= \left[(2.05)^2 + 3(2.05)(2.96) - (2.96)^2\right] - \left[2^2 + 3(2)(3) - 3^2 \right]
        \end{align*}

        Note that $\Delta z \approx dz$ but $dz$ is easier to compute.

        \textit{\blue{Example 5:}} The base radius and height of a right circular cone are measured as 10 cm and 25 cm, respectively, with a possible error in measurement of as much as 0.1 cm in each. Use differentials
        to estimate the maximum error in the calculated volume of the cone. \\

        The volume $V$ of a cone with base radius $r$ and height $h$ is $V = \frac{\pi r^2 h}{3}$. So the differential of $V$ is

        \[
            dV = \frac{\partial V}{\partial r}dr + \frac{\partial V}{\partial h} dh = \frac{2\pi rh}{3}dr + \frac{\pi r^2}{3}dh
        \]

        Since each error is at most 0.1 cm, we have $|\Delta r|\leq 0.1, |\Delta h| \leq 0.1$. To estimate the largest error in the volume we take the largest error in the measurement of $r$ and of $h$. Therefore we take
        $dr=0.1$ and $dh = 0.1$ along with $r=10,h=25$. This gives

        \[
            dV = \frac{500\pi}{3}(0.1) + \frac{100\pi}{3}(0.1) = 20\pi
        \]

        Thus the maximum error in the calculated volume is about $20\pi \text{ cm}^3\approx 63 \text{ cm}^3$. \\

        For a function of three variables, the \textbf{linear approximation} is

        \[
            f(x,y,z) \approx f(a,b,c) + f_x(a,b,c)(x-a) + f_y(a,b,c)(y-b) + f_z)a,b,c)(z-c)
        \]

        and the linearization $L(x,y,z)$ is the right side of this expression. If $w=f(x,y,z)$ then the \textbf{increment} of $w$ is

        \[
            \Delta w = f(x + \Delta x, y + \Delta y, z + \Delta z) - f(x,y,z)
        \]

        The differential $dw$ is defined like so:

        \[
            dw = \frac{\partial w}{\partial x}dx + \frac{\partial w}{\partial y}dy + \frac{\partial w}{\partial z}dz
        \]

        \textit{\blue{Example 6:}} The dimensions of a rectangular box are measured to be 75 cm, 60 cm, and 40 cm, and each measurement is correct to within 0.2 cm. Use differentials to estimate the largest possible
        error when the volume of the box is calculated from these measurements. \\

        If the dimensions of the box are $x,y,$ and $z$, its volume is $V=xyz$ and so

        \begin{align*}
            dV = \frac{\partial V}{\partial x}dx + \frac{\partial V}{\partial y}dy + \frac{\partial V}{\partial z}dz = yz dx + xz dy + xy dz
        \end{align*}

        We are given that $|\Delta x| \leq 0.2, |\Delta y| \leq 0.2,$ and $|\Delta z| \leq 0.2$. To estimate the largest error in the volume, we therefore use $dx = 0.2$, $dy = 0.2$, and $dz = 0.2$ together with
        $x=75, y = 60,$ and $z=40$:

        \[
            \Delta V \approx dV = (60)(40)(0.2) + (75)(40)(0.2) + (75)(60)(0.2) = 1980
        \]

        Thus an error of only 0.2 cm in measuring each dimension could lead to an error of approximately 1980 $\text{cm}^3$ in the calculated volume! This may seem like a large error, but it's only about 1\% of the
        volume of the box.

    \subsection{The Chain Rule}     % 14.5


    \subsection{Directional Derivatives and the Gradient Vector}    % 14.6
    \subsection{Maximum and Minimum Values}     % 14.7

        A function of two variables has a \textbf{local maximum} at $(a,b)$ if $f(x,y) \leq f(a,b)$ when $(x,y)$ is near $(a,b)$. [This means that $f(x,y) \leq f(a,b)$ for all points $(x,y)$ in some disk with center
        $(a,b)$.] The number $f(a,b)$ is called a \textbf{local maximum value}. If $f(x,y) \geq f(a,b)$ when $(x,y)$ is near $(a,b)$, then $f$ has a \textbf{local minimum} at $(a,b)$ and $f(a,b)$ is a
        \textbf{local minimum value}. If the inequalities hold for \textit{all} points $(x,y)$ in the domain of $f$, then $f$ has an \textbf{absolute maximum} (or \textbf{absolute minimum}) at $(a,b)$.

        \begin{theorem}{Theorem}
            If $f$ has a local maximum or minimum at $(a,b)$ and the first-order partial derivatives of $f$ exist there, then $f_x (a,b) = 0$ and $f_y (a,b) = 0$.
        \end{theorem}

        A point $(a,b)$ is called a \textbf{critical point} of $f$ if $f_x (a,b) = 0$ and $f_y (a,b) = 0$, or if one of these partial derivatives does not exist. \\

        \textit{\blue{Example 1:}} Find the extreme values of $f(x,y) = y^2 - x^2$. \\

        Since $f_x = -2x$ and $f_y = 2y$, the only critical point is $(0,0)$. Notice that for points on the $x$-axis we have $y=0$, so $f(x,y) = -x^2 < 0$ if $x\not = 0$. However, for points on the $y$-axis we have
        $y=0$, so $f(x,y) = y^2 > 0$ if $y\not = 0$. Thus every disk with center $(0,0)$ contains points where $f$ takes positive values as well as points where $f$ takes negative values. Therefore $f(0,0) = 0$ can't
        be an extreme value for $f$, so $f$ has no extreme value.

        \textit{\blue{Example 2:}} Let $f(x,y) = x^2 + y^2 - 2x - 6y + 14$. Then

        \begin{align*}
            f_x (x,y)   &= 2x -2 \\
            f_y (x,y)   &= 2y - 6
        \end{align*}

        These partial derivatives are equal to 0 when $x=1$ and $y=3$, so the only critical point is $(1,3)$. By completing the square, we find that

        \[
            f(x,y) = 4 + (x-1)^2 + (y-3)^2
        \]

        Since $(x-1)^2 \geq 0$ and $(y-3)^2 \geq 0$, we have $f(x,y) \geq 4$ for all values of $x$ and $y$. Therefore $f(1,3) = 4$ is a local minimum, and in fact it is the absolute minimum of $f$. This can be confirmed
        geometrically from the graph of $f$, which is the elliptic paraboloid with vertex $(1,3,4)$.

        \begin{axiom}{Second Derivative Test}
            Suppose the second partial derivatives of $f$ are continuous on a disk with center $(a,b)$, and suppose that $f_x (a,b) = 0$ and $f_y (a,b) = 0$ [that is, $(a,b)$ is a critical point of $f$]. Let

            \[
                D = D(a,b) = f_{xx} (a,b) f_{yy} (a,b) - [f_{xy} (a,b)]^2
            \]

            (a) If $D > 0$ and $f_{xx} (a,b) > 0$, then $f(a,b)$ is a local minimum. \\
            (b) If $D > 0$ and $f_{xx} (a,b) < 0$, then $f(a,b)$ is a local maximum. \\
            (c) If $D < 0$, then $f(a,b)$ is not a local maximum or minimum.
        \end{axiom}

        In case (c) the point $(a,b)$ is called a \textbf{saddle point} of $f$ and the graph of $f$ crosses its tangent plane at $(a,b)$. \\

        If $D = 0$ the test gives no information and $f$ could have a local maximum or local minimum at $(a,b)$, or $(a,b)$ could be a saddle point of $f$. \\

        To remember the formula for $D$, it's helpful to write it as a determinant:

        \[
            D =
            \begin{vmatrix}
                f_{xx}  & f_{xy} \\
                f{yx}   & f_{yy}
            \end{vmatrix}
            = f_{xx} f_{yy} - \left(f_{xy}\right)^2
        \]

        \textit{\blue{Example 3:}} Find the local maximum and minimum values and saddle points of $f(x,y) = x^4 + y^4 - 4xy + 1$. \\

        We first locate the critical points:

        \[
            f_x = 4x^3 - 4y, f_y = 4y^3 - 4x
        \]

        Setting these partial derivatives equal to 0, we obtain the equations

        \[
            x^3 - y = 0, y^3 - x = 0
        \]

        Solving these equations by substitution, we get

        \begin{align*}
            0   &= x^9 - x \\
                &= x(x^8 - 1) \\
                &= x(x^4 - 1)(x^4 + 1) \\
                &= x(x^2 - 1)(x^2 + 1)(x^4 + 1)
        \end{align*}

        so there are three real roots: $x = 0, 1, -1$. The three critical points are $(0,0), (1,1), (-1,-1)$. Next we calculate the second partial derivatives and $D(x,y):$

        \[
            f_{xx} = 12x^2, f_{xy} = -4, f_{yy} = 12y^2
        \]

        and

        \[
            D(x,y) = f_{xx} f_{yy} - \left(f_{xy}\right)^2 = 144x^2 y^2 - 16
        \]

        Since $D(0,0) = -16 < 0$, it follows that the origin is a saddle point; that is, $f$ has no local maximum or minimum at $(0,0)$. \\
        Since $D(1,1) = 128 > 0$ and $f_{xx} (1,1) = 12 > 0$, we see from case (a) of the test that $f(1,1) = -1$ is a local minimum. \\
        Similarly, $(-1, -1) = 128 > 0$ and $f_{xx} (-1, -1) = 12 > 0$, so $f(-1, -1) = -1$ is also a local minimum. \\

        The graph of $f$ is shown below:

        \begin{figure*}[hbt!]
            \centering
            \includegraphics[scale = 0.75]{Resources/14.7_Example_Graph}
        \end{figure*}

        A contour map of the function $f$ is shown below. Note how the level curves near $(1,1)$ and $(-1,-1)$ are oval in shape, thus indicating that was we move away from $(1,1)$ or $(-1,-1)$ in any direction the 
        values of $f$ are increasing. The level curves near $(0,0)$ resemble hyperbolas. They reveal that as we move away from the origin (where the value of $f$ is 1, the values of $f$ decrease in some directions but
        increase in other directions. Thus the contour map suggests the presence of the minima and saddle point that we found above.

        \begin{figure*}[hbt!]
            \centering
            \includegraphics[scale = 0.75]{Resources/14.7_Example_Contour}
        \end{figure*}

        \textit{\blue{Example 4:}} Find the shortest distance from the point $(1,0,-2)$ to the plane $x+2y + z = 4$. \\

        The distance from any point $(x,y,z)$ to the point $(1,0,-2)$ is

        \[
            d = \sqrt{(x-1)^2 + y^2 + (z+2)^2}
        \]

        but if $(x,y,z)$ lies on the plane $x+2y + z = 4$, then $z=4 - x - 2y$ and so we have

        \[
            d = \sqrt{(x-1)^2 + y^2 + (6-x-2y)^2}
        \]

        We can minimize $d$ by minimizing the simpler expression

        \[
            d^2 = f(x,y) = (x-1)^2 + y^2 + (6-x-2y)^2
        \]

        By solving the equations

        \begin{align*}
            f_x &= 2(x-1) - 2(6 - x - 2y) = 4x + 4y - 14 = 0 \\
            f_y &= 2y - 4(6 - x - 2y) = 4x + 10y - 24 = 0
        \end{align*}

        we find that the only critical point is $\left(\frac{11}{6}, \frac{5}{3}\right)$. Since $f_{xx} = 4, f_{xy} = 4,$ and $f_{yy} = 10$, we have

        \[
            D(x,y) = f_{xx} f_{yy} - \left(f_{xy}\right)^2 = 24 > 0
        \]

        and $f_{xx} > 0$, so by the Second Derivatives Test $f$ has a local minimum at $\left(\frac{11}{6}, \frac{5}{3}\right)$. Intuitively we can see that this local minimum is actually an absolute minimum because
        there must be a point on the given plane that is closest to $(1,0,-2)$. If $x = \frac{11}{6}$ and $y=\frac{5}{3}$, then

        \begin{align*}
            d   &= \sqrt{(x-1)^2 + y^2 + (6 - x - 2y)^2} \\
                &= \sqrt{\left(\frac{5}{6}\right)^2 + \left(\frac{5}{3}\right)^2 + \left(\frac{5}{6}\right)^2} \\
                &= \frac{5}{6}\sqrt{6}
        \end{align*}

        The shortest distance from $(1,0,-2)$ to the plane $x + 2y + z = 4$ is $\frac{5}{6}\sqrt{6}$. \\

        \textit{\blue{Example 6:}} A rectangular box without a lid is to be made from 12 $\text{m}^2$ of cardboard. Find the maximum volume of such a box. \\

        Let the length, width, and height of the box (in meters) be $x,y,z$. Then the volume of the box is

        \[
            V = xyz
        \]

        We can express $V$ as a function of just two variables $x$ and $y$ by using the fact that the area of the four sides and the bottom of the box is

        \[
            2xz + 2yz + xy = 12
        \]

        Solving this equation for $z$, we get $z = \frac{12 - xy}{2(x+y)}$, so the expression for $V$ becomes

        \[
            V = xy \frac{12-xy}{2(x+ y)} = \frac{12xy - x^2 y^2}{2(x+y)}
        \]

        We compute the partial derivatives:

        \[
            \frac{\partial V}{\partial x} = \frac{y^2 (12 - 2xy - x^2)}{2(x+y)^2}, \frac{\partial V}{\partial y} = \frac{x^2 (12 - 2xy - y^2)}{2(x+y)^2}
        \]

        If $V$ is a maximum, then $\frac{\partial V}{\partial x} = \frac{\partial V}{\partial y} = 0,$ but $x= 0$ or $y= 0$ gives $V= 0$, so we must solve the equations

        \begin{align*}
            12 - 2xy - x^2 &= 0 \\
            12 - 2xy - y^2 &= 0
        \end{align*}

        These eimply that $x^2 = y^2$ and so $x = y$. (Note that $x$ and $y$ must both be positive in this problem.) If we put $x=y$ in either equation we get $12 - 3x^2 = 0$, which gives $x=2, y=2, $ and
        $z = \frac{12 - 2\cdot 2}{2(2+2)} = 1$. \\

        We could use the Second Derivatives Test to show that this gives a local maximum of $V$, or we could simply argue from the physical nature of htis problem that there must be an absolute maximum volume, which has
        to occur at a critical point of $V$, so it must occur when $x=2, y=2, z=1$. Then $v=2\cdot 2 \cdot 1 = 4$, so the maximum volume of the box is 4 $\text{m^3}$. \\

        \textbf{Closed set} in $\mathbb{R}^2$: one that contains all its boundary point \\
        \textbf{Bounded set} in $\mathbb{R}^2$: one that is contained within some boundary

        \begin{theorem}{Extreme Value Theorem}
            If $f$ is continuous on a closed, bounded set $D$ in $\mathbb{R}^2$, then $f$ attains an absolute maximum value $f(x_1, y_1)$ and an absolute minimum value $f(x_2, y_2)$ at some points $(x_1, y_1)$ and
            $(x_2, y_2)$ in $D$.
        \end{theorem}

        \textbf{Steps for finding the absolute maximum and minimum values of a continuous function $f$ on a closed, bounded set $D$:} \\
        1. Find the values of $f$ at the critical points of $f$ in $D$. \\
        2. Find the extreme values of $f$ on the boundary of $D$. \\
        3. The largest of the values from steps 1 and 2 is the absolute maximum value; the smallest of these values is the absolute minimum value. \\

        \textit{\blue{Example 7:}} Find the absolute maximum and minimum values of the function $f(x,y) = x^2 - 2xy + 2y$ on the rectangle $D = \{(x,y) | 0\leq x \leq 3, 0 \leq y \leq 2\}$. \\

        Since $f$ is a polynomial it is continuous on the closed, bounded rectangle $D$, so there must be both an absolute maximum and an absolute minimum. We first find the critical points which occur when

        \[
            f_x = 2x - 2y = 0, f_y = -2x + 2 = 0
        \]

        so the only critical point is $(1,1)$ and the value of $f$ there is $f(1,1) = 1$. Looking at the values of $f$ on the boundary of $D$ consists of the four line segments $L_1$, $L_2$, $L_3$, $L_4$, shown on the
        figure below. On $L_1$ we have $y=0$ and

        \[
            f(x,0) = x^2, 0 \leq x \leq 3
        \]

        \begin{figure*}[hbt!]
            \centering
            \includegraphics[scale = 0.75]{Resources/14.7_Example_Graph}
        \end{figure*}

        This is an increasing function of $x$ so its minimum value is $f(0,0) = 0$ and its maximum value is $f(3,0) = 9$. On $L_2$ we have $x=3$ and

        \[
            f(3, y) = 9 - 4y, 0 \leq y \leq 2
        \]

        This is a decreasing function of $y$ so its maximum value is $(3,0) = 9$ and its minimum value is $f(3,2) = 1$. On $L_3$ we have $y=2$ and

        \[
            f(x,2) = x^2 - 4x + 4, 0 \leq x \leq 3
        \]

        Notice that the minimum value of this function is $f(2,2) = 0$ and the maximum value is $f(0,2) = 4$. Finally on $L_4$ we have $x=0$ and

        \[
            f(0,y) = 2y, 0 \leq y \leq 2
        \]

        with maximum value $f(0,2) = 4$ and minimum value $f(0,0) = 0$. Thus on the boundary, the minimum value of $f$ is 0 and the maximum is 9. \\

        Comparing these values with the value $f(1,1) = 1$ at the critical point, we can conclude that the absolute maximum value of $f$ on $D$ is $f(3,0) = 9$ and the absolute minimum value is $f(0,0) = f(2,2) = 0$.

    \subsection{Lagrange Multipliers}   % 14.8