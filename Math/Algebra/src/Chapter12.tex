\section{The Cauchy-Schwartz Inequality}

    The Cauchy-Schwartz Inequality, or the Cauchy-Bunyakovsky-Schwartz Inequality, states
    that for all sequences of real numbers $a_i$ and $b_i$, we have \\

    \begin{equation*}
        (\sum^n_{i=1}a_i^2)
        (\sum^n_{i=1}b_i^2)
        \geq
        (\sum^n_{i=1}a_ib_i)^2
    \end{equation*}

    \noindent Equality holds if and only if $a_i=kb_i$ for some non-zero constant
    $k\in\mathbb{R}$. \\

    \noindent \color{blue} \textit{Example: If $x^2+y^2+z^2=1$, what is the maximum
    value of $x+2y+3z$?} \color{black} \\

    \noindent We have $(x+2y+3z)^2\leq(1^2+2^2+3^2)(x^2+y^2+z^2)=14$. Hence, $x+2y+3z\leq\sqrt{14}$
    with equality holding when $\frac{x}{1}=\frac{y}{2}=\frac{z}{3}$. Together with
    $x^2+y^2+z^2=1$, we get \\

    \begin{equation*}
        x=\frac{1}{\sqrt{14}},
        y=\frac{2}{\sqrt{14}},
        z=\frac{3}{\sqrt{14}}
    \end{equation*}