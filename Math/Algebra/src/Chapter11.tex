\section{Sequences and Series}

    \subsection{Introduction to Sequences and Series}
        A \textbf{sequence} is an ordered list of numbers, whereas a \textbf{series} is the
        sum of the terms of a sequence. We can represent a series with $\{a_n\}^\infty_{n=1}$,
        where the sequence starts with index $n=1$ and runs to infinity. The other notation,
        summation notation, is written $\sum^{10}_{n=1}a_n$, where the sequence runs from $n=1$
        to infinity. \\

        \noindent For example, the expansion of
        $\{a_n\}_{n=1}^{n=10}$ is $a_n=n^2=1,4,,9,16,25,36,49,64,81,100$. \\

        \noindent Conventionally, the following symbols are used:

        \begin{center}
            \begin{tabular} {|c|c|}
                \hline
                $a$
                & first term in sequence                                                    \\
                \hline
                $n$
                & number of terms in sequence                                               \\
                \hline
                $S_n$
                & sum of first $n$ terms in sequence                                        \\
                \hline
                $d$
                & Common difference between any two consecutive terms, arithmetic sequences \\
                \hline
                $r$
                & Common ratio between two consecutive terms, geometric sequences           \\
                \hline
            \end{tabular}
        \end{center}



    \subsection{Arithmetic Progressions} \\
        \textbf{Arithmetic Progressions} are sequences containing numbers which differ from each
        other by a common difference, $d$. \\

        \noindent \textbf{Formula for Arithmetic Sequences} \\

        \begin{equation*}
            a_n=a_1+d(n-1)
        \end{equation*}

        \noindent $a_n$ is the $n^{th}$ term, $a_1$ is the first term, $n$ is the index \\

        \noindent  The sum of the first $n$ terms is given by one of the three formulas. \\

        \begin{align*}
            S_n=\frac{n}{2}[2a+d(n-1)] \\
            S_n=\frac{n}{2}[a+a_n] \\
            S_n=n\cdot (middle term)
        \end{align*}

        \noindent Find the sum of the first 50 odd positive integers. \\

        \begin{equation*}
            S_n=\frac{n}{2}(2a+d(n-1))
            \implies
            S_{50}=25\cdot (2+49\cdot 2)=2500
        \end{equation*}



    \subsection{Geometric Sequences and Series}
        \textbf{Geometric Sequences} include terms that are multiplied by a ratio iteratively.
        They are given by the following formula. \\

        \begin{equation*}
            a_n=a\cdot r^{n-1}
        \end{equation*}

        \noindent $a_n$ is the $n^{th}$ term, $a$ is the first term, $r$ is the common ratio. \\

        \noindent The sum of a geometric sequence is given by \\

        \begin{equation*}
            S_n = \begin{cases}
                      a\cdot (\frac{r^n-1}{r-1}), & r\not=1 \\
                      a\cdot n, & r=1
            \end{cases}
        \end{equation*}

        \noindent The sum to infinity of a geometric sequence, where $|r|<1$, is given by \\

        \begin{equation*}
            S_\infty=\frac{a}{1-r}
        \end{equation*}

        \noindent \color{blue} \textit{Example: After striking the floor, a tennis ball bounces
        to $\frac{2}{3}$ of the height from which it last fell. What is the total vertical
        distance it travels before it comes to rest when it is dropped from a vertical height of
        100$m$?} \color{black} \\

        \noindent If $h$ is the height in meters, $e$ is a number such that $0<e<1$, and $S$ is
        the total vertical distance covered before coming to rest, then \\

        \begin{align*}
            S &= h+2(eh)+2(e^2h)+2(e^3h)+2(e^4)h + \dots \\
            &= h+2eh(1+e+e^2+e^3+\dots) \\
            &= h+2eh \cdots \frac{1}{1-e}, \because e<1 \\
            &= h\left(\frac{1+e}{1-e}\right)
        \end{align*}

        \noindent Since it is given that $h=100$ and $e=\frac{2}{3}$, \\

        \begin{equation*}
            S=100\left(\frac{1+\frac{2}{3}}{1-\frac{2}{3}}\right)=500 \text{ meters}
        \end{equation*}



    \subsection{Binomial Expansion}
        The \textbf{Binomial Theorem} allows us to expand binomials such that \\

        \begin{equation*}
            (a+b)^n
            = \sum^n_{k=0}\binom{n}{k}a^{n-k}b^k
        \end{equation*}

        \noindent where $\binom{n}{k}$, pronounced "n choose k" because it describes how many
        ways to choose $k$ elements from a set of $n$, is given by the formula \\

        \begin{equation*}
            \binom{n}{k}=\frac{n!}{k!(n-k)!}
        \end{equation*}

        \noindent In the Binomial Theorem, $\binom{n}{k}$ determines the coefficients of the
        expanded binomial. Coefficients of Binomials follow Pascal's Triangle and they match up
        like so: \\

        \begin{figure} [hbt!]
            \centering
            \includegraphics[scale = 0.5] {Resources/Unit11Sequences/pascal.PNG}
        \end{figure}

        \noindent \color{blue} \textit{Example 1: Expand $(y+5)^4$} \color{black} \\

        \begin{align*}
            (y+5)^4 &= \binom{4}{0}y^45^0
            + \binom{4}{1}y^35^1
            + \binom{4}{2}y^25^2
            + \binom{4}{3}y^15^3
            + \binom{4}{4}y^05^4 \\
            &= y^4+20y^3+150y^2+500y+625
        \end{align*}

        \noindent \color{blue} \textit{Example 2: What is the coefficient of $x^3 in (2x+4)^8?$}
        \color{black} \\
        The term containing $x^3$ is \\

        \begin{align}
            \binom{8}{5}(2x)^34^5 &= 56(2x^3)(4^5) \\
            &= 458752x^3
        \end{align}

        \noindent Hence, the coefficient is 458752.