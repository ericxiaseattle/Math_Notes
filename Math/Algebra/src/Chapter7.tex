\section{Radical Functions}

    Radical Functions are functions that contain a fractional exponent, that is, a variable
    under a radicand. When graphing radical functions, it is important to keep the domain in
    mind. For the parent function $f(x)=\sqrt{x}$, the domain is $x\geq 0$. This results from
    the fact that real solutions will not come from the square root of negative numbers. \\

    \noindent For the square root function $f(x)=a\sqrt{x}$, we notice that a value
    $|a|>0\implies$ a vertical stretch and $0<|a|<1\implies$ a vertical shrink. If $a<0$ then
    the graph will be vertically reflected across the $x$-axis. \\

    \noindent Radical functions are graphed by plotting many points while keeping in mind the
    domain and connecting them with a line.