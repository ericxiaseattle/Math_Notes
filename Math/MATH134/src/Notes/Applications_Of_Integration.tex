\section{Applications of Integration}

    \subsection{General Steps for Applications of Integration Problems}

    1. Chop up the quantity $Q = \sum \Delta Q$ \\
    2. Estimate $\Delta Q$ \\
    3. Sum $dQ$ by integrating

    \color{blue} \textit{Example: A rod 6 metres long is placed on the $x$-axis from $x=0$ to $x=6$. The mass density function is $\lambda (x) = \frac{12}{\sqrt{x+1}}$. \\
    $\bullet$ What is the total mass? \\
    $\bullet$ Where is the centre of mass? \\
    $\bullet$ What is the average density?}
    \color{black} \\

    fill in later

    \subsection{Centre of Mass}

    For a rod with varying density,

    \[
        \text{Centre of mass } = \frac{\text{moment}}{\text{total mass}},
    \]

    where

    \[
        \text{moment } = \int_a^b x\cdot \lambda (x)
    \]

    Additionally,

    \[
        \text{Average density } = \frac{\text{mass}}{\text{interval of integration}}
    \]



\subsection{Centroid of a Region and Pappus's Theorem on Volumes (6.4)}

        \begin{figure*}[hbt!]
            \centering
            \includegraphics[scale=0.5]{pappus}
        \end{figure*}

        \color{blue} \textit{Example: Find the centroid of a quarter disk of radius $r$ (put it in the first quadrant for computations)} \color{black}

    \subsection{Work (6.5)}

        If force $F$ is constant, when you move an object a distance $x$, the work done is

        \[
            W = F \cdot x
        \]

        So if I am to use my three steps for integration applications, how do I "chop up" work to get $\Delta W$?

        \[
            \Delta Work = F(x) \Delta x
        \]

        It follows that

        \[
            W   = \int_a^b F(x)dx
        \]

        \color{blue} \textit{Example: It requires 6N to keep a spring stretched 0.2m beyond its natural length. Find the work done to stretch it from its original length of 0.7m to length of 1m.} \color{black}

        We have

        \[
            6 = k(0.2) \implies k = 30
        \]

        Then the work needed is

        \[
            W = \int_0^{0.3=1-0.7} 30xdx
        \]

        \color{blue} \textit{Example 2: A sandbag is pulled from ground level to the top of a 32m building by a cable that is 0.2 kg/m. Initially the sandbag has a mass of 52kg but it leaks at a constant rate so when it
        reaches the roof its mass is 36kg. Find the work done.} \color{black}

        We have

        \[
            W = \int_0^{32} F(y) dy,
        \]

        where $F(y)$ is the cable's weight and $y$ is the sand. The cable's weight is

        \[
            (32-y)0.2g
        \]

        and the rate of change is

        \[
            \frac{36-52}{32}=-0.5
        \]

        So our work is

        \[
            W = \int_0^{32} \left[(32-y)0.2g+(-0.5y+52)g\right]dy
        \]

    \subsection{One-to-one Functions}

        \textbf{Theorem:} if a function $f$ is one-to-one, then there is a unique function, dentoed $f^{-1}$, defined on the range of $f$, such that $f\left(f^{-1}(x)\right) = x$ for all $x$ in the range of $f$ and
        $f^{-1}\left(f(x)\right)=x$ for all $x$ in the domain of $f$.


    \subsection{The Natural Logarithm and Euler's Constant}

        For $x>0$, the natural logarithm function is defined by

        \[
            \ln{x} = \int_1^x \frac{1}{t}dt
        \]

        Because $\ln{x}$ is increasing and differentiable on its domain, it follows that when $x>1$,

        \[
            \int_1^x \frac{1}{t}dt > 0
        \]

        and $\ln{x}>0$. Similarly, when $x<1$,

        \[
            \int_1^x \frac{1}{t}dt < 0
        \]

        and $\ln{x}>0$. \\

        \blue{\textbf{Proofs of some Properties of the Natural Logarithm:}} \\
        If $a,b>0$ and $r$ is a rational number, then \\
        i. $\ln{1} = 0$ \\
        ii. $\ln{(ab)} = \ln{a} + \ln{b}$ \\
        iii. $\ln{\left(\frac{a}{b}\right)} = \ln{a} - \ln{b}$

        \textbf{i.} By definition,

        \[
            \ln{1} = \int_1^1 \frac{1}{t}dt = 0.
        \]

        \textbf{ii.} We have

        \[
            \ln{\left(ab\right)} = \int_1^{ab} \frac{1}{t}dt = \int_1^a \frac{1}{t}dt + \int_a^{ab} \frac{1}{t}dt.
        \]

        Using $u$-substitution on the last integral in this expression, let $u=\frac{t}{a}\implies du = \frac{dt}{a}$. When $t=a,$ $u=1$, and when $t=ab$, $u=b$. It follows then that

        \[
            \ln{(ab)} = \int_1^a \frac{1}{t}dt + \int_a^{ab} \frac{1}{t}dt = \int_1^a \frac{1}{t}dt + \int_1^{ab} \frac{a}{t}\cdot\frac{1}{a}dt = \int_1^a \frac{1}{t}dt + \int_1^b \frac{1}{u}du = \ln{a} + \ln{b}
        \]

        \textbf{iii.} Note that

        \[
            \frac{d}{dx}\ln{\left(x^r\right)} = \frac{rx^{r-1}}{x^r} = \frac{r}{x}
        \]

        Furthermore,

        \[
            \frac{d}{dx}\left(r\ln{x}\right) = \frac{r}{x}
        \]

        Since the derivatives of these two functions are the same, then by the FTC, they must differ by a constant s.t.

        \[
            \ln{\left(x^r\right)} = r\ln{x} + C
        \]

        for some constant $C$. Setting $x=1$, it follows that

        \begin{align*}
            \ln{\left(1^r\right)}   &= r\ln{(1)} + C \\
            0                       &= r(0) + C \\
            C                       &= 0
        \end{align*}

        Hence, $\ln{\left(x^r\right)} = r\ln{x}$. \\

        There is a function, $f(z)$ which is the inverse of $\ln{x}$ with domain $\mathbb{R}$ and range $\mathbb{R}^{-1}$. $\ln{x}$ is an increasing function on its domain, so it is injective and hence, there is an
        unique function $f(x)$ on the domain $\left(-\infty, \infty\right)$ and range $\left(0,\infty\right)$. This function has the following values:

        \[
            f(1) = e \text{ because } \ln{e} = 1 \\
            f(0) = 1 \text{ because } \ln{1} = 0
        \]

        when $b\in \alpha$, $f(b) = e^b$ because $\ln{e^b} = b\ln{e} = b$. \\

        \color{\textbf{Properties of $f(x) = e^x$}}: \\
        i. $e^x$ is increasing because it is the inverse of an increasing function \\
        ii. $f^{-1}\left(f(x)\right) = x$, any invertible function

    \subsection{Exponential and Logarithmic Functions}

        Since

        \[
            x^{\frac{p}{q}} = \left(x^p\right)^{\frac{1}{q}},
        \]

        we have

        \begin{align*}
            f(x)    &= x^r, r\in \mathbb{R} \\
                    &= e^{r\ln{x}},x > 0 \\
                    &= e^{\frac{p}{q}\ln{x}} \\
                    &= e^{\ln{\left(x^{\frac{p}{q}}\right)}}
        \end{align*}

        \textbf{Defining $a^x$ and $\log_a {x}$}:

        \begin{align*}
            a^x &= e^{x\ln{a}} \\
            \frac{d}{dx}a^x &= \frac{d}{dx}e^{x\ln{a}} \\
                            &= e^{x\ln{a}}\cdot\ln{a} \\
                            &= (\ln{a})a^x
        \end{align*}

        Additionally,

        \[
            \log_a {x} = \frac{\ln{x}}{\ln{a}}
        \]





    \subsection{Error in Linear Approximations}

        From HW question:

        \[
            f(x) = f(b) + f'(b)(x-b) + \int_b^x f''(t)(x-t)dt
        \]

        Note that $f(x)$ is the actual function value, whereas $f(b)+f'(b)(x-b)$ is the linear approximated value. Thus, the error is the actual value minus the approximated value s.t.

        \[
            f(x) - L(x) = \int_b^x f''(t)(x-t)dt
        \]

        We want to bound the error. If $b < x$,

        \[
            b < t < x \implies x - t > 0
        \]

        and

        \[
            \int_b^x f''(t)(x-t)dt \leq \int_b^x |f''(t)| (x-t)dt
        \]

        Let $M$ be a (positive) number such that

        \[
            |f''(t)| \leq M
        \]

        when $t$ is between $b$ and $x$. Then,

        \begin{align*}
            \int_b^x f''(t)(x-t)dt  &\leq \int_b^x M\cdot (x-t) dt \\
                                    &= -\frac{M(x-t)^2}{2} \Big|_b^x \\
            u = x-t, du = -dt       \\
                                    &= \frac{M(x-b)^2}{2}
        \end{align*}

        It follows that

        \[
            \left|\int_b^x f''(t)(x-t)dt\right| \leq \frac{M|x-b|^2}{2}
        \]

        This is known as Taylor's Inequality, where $M$ is an upper bound for $|f''(t)|$ when $t$ is between $x$ and $b$.


    \subsection{Numerical Integration}

        







