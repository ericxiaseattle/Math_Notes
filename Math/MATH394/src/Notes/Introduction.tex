\section{Introduction}

    \subsection{Permutations and Combinations}

        \textbf{Cardinality:} the number of elements in a given set $A$, denoted by #$A$

        \begin{axiom}{Property}
            Let $A$ and $B$ be finite sets and $k$ a positive integer. Assume that there is a function $f$ from $A$ onto $B$ so that each element of $B$ is the image of exactly $k$ elements of $A$. (Such a function is
            called $k$\text{-to-one}.) Then $#A = k\cdot #B$.
        \end{axiom}

        Below is a graphical illustration of this property. Set $A$ has 12 elements represented by bullets, set $B$ has 4 elements represented by squares, and set $C$ has 4 elements represented by triangles. Each element
        of $B$ is matched up with exactly three elements of $A$ and each element of $C$ is matched up with one element of $B$. Thus $#A = 3\cdot #B$. The elements of $B$ and $C$ are matched up one-to-one, so $#B=#C$. \\

        \textit{\blue{Example:}} To dress up for school in the morning, Joyce chooses from 3 dresses (red, yellow, or green), 3 blouses (also red, yellow, or green), and 2 pairs of shoes. She refuses to wear a dress and
        a blouse of matching colors. How many different outfits can she choose from? \\

        We can imagine that Joyce goes through the choices one by one: first the dress, then the blouse, and finally the shoes. There are three choices for the dress. Once the dress is chosen, there are two choices for
        the blouse, and finally two choices for the shoes. By the multiplication principle, she has $3\cdot 2 \cdot 2 = 12$ different outfits. The construction of the $n$-tuples in the general multiplication rule may be
        visualized by a \textbf{decision tree}, with arrows pointing to the available choices of the subsequent step. The dresses are the triangles and the blouses the squares, labeled by the colors. The shoes are the
        circles. The possible outfits resulting from the three choices can thus be read off from the directed paths through the tree along the arrows.

        \begin{figure*}[hbt!]
            \centering
            \includegraphics[scale = 0.75]{Assets/Appendices_Decision_Tree}
        \end{figure*}

        \begin{corollemma}{Corollary of General Multiplication Product}
            Let $A_1, A_2, \dots, A_n$ be finite sets. Then

            \[
                #(A_1 \times A_2 \times \dots \times A_n) = (#A_1)\cdot (#A_2) \dots (#A_n) = \Pi_{i=1}^n (#A_i)
            \]

            where the capital pi notation is shorthand for a product
        \end{corollemma}

        \textit{\blue{Example 2:}} In a certain country license plates have three letters followed by three digits. How many different license plates can we construct if the country's alphabet contains 26 letters? \\

        Each license plate can be described as an element of hte set $A\times A \times A \times B \times B \times B$ where $A$ is the set of letters and $B$ is the set of digits. Since $#A=26$ and $#B=10$, the answer is
        $26^3 \cdot 10^3=17,576,000$. \\

        \textit{\blue{Example 3:}} We flip a coin three times and then roll a die twice. We record the resulting sequence of outcomes in order. How many different sequences are there? \\

        The outcome of a coin flip is an element of the set $C=\{H,T\}$ and a die roll gives an element of the set $D=\{1,2,3,4,5,6\}$. Each possible sequence is an element of the set $C\times C\times C \times D \times D$,
        which has $2^3 \cdot 6^2 = 288$ elements. \\

        \textit{\blue{Example 4:}} How many distinct subsets does a set of size $n$ have? \\

        Each subset can be encoded by an $n$-tuple with entries 0 or 1, where the $i$th entry is 1 if the $i$th element of the set is in the subset (and 0 if it is not). Thus the number of subsets is the same as the
        number of elements in the Cartesian product

        \[
            \{0,1\} \times \cdot \times \{0,1\} = \{0,1\}^n,
        \]

        which is $2^n$. Note that the empty set $\emptyset$ and set itself are included in the count. These correspond to the $n$-tuples $(0,0,\dots,0)$ and $(1,1,\dots,1)$, respectively.

        \begin{axiom}{Permutations}
            Consider all $k$-tuples $(a_1, \dots, a_k)$ that can be constructed from a set $A$ of size $n(n\geq k)$ without repetition. So each $a_i \in A$ and $a_i \not = a_j$ if $i\not = j$. The total number of these
            $k$-tuples is

            \[
                (n)_k = n\cdot (n-1) \dots (n-k+1) = \frac{n!}{(n-k)!}
            \]

            In particular, with $k=n$, each $n$-tuple is a \textbf{permutation} of the set $A$. So the total number of orderings of a set of $n$ elements is $n! = n\cdot (n-1)\dots 2 \cdot 1$.
        \end{axiom}

        \begin{proof}
            Construct these $k$-tuples sequentially. Choose the first entry out of the full set $A$ with $n$ alternatives. If we have the first $j$ entries chosen then the next one can be any one of the remaining
            $n-j$ elements of $A$. Thus the total number of $k$-tuples is the product $n\cdot (n-1) \dots (n-k+1).$
        \end{proof}

        $(n)_k$ is the \textbf{descending factorial}. \\

        \textit{\blue{Example 5:}} I have a 5-day vacation from Monday to Friday in Santa Barbara. I have to specifically assign one day to exploring the town, one day to hiking the mountains, and one day for the beach.
        I can take up at most one activity per day. In how many ways can I schedule these activities? \\

        The underlying set has 5 elements and we choose a 3-tuple. For example, (Mon, Fri, Tue) means Monday in town, Friday for hiking, and Tuesday on the beach. The number of choices is $5\cdot 4 \cdot 3 = 60$. \\

        \begin{theorem}{Binomial Coefficient}
            Let $n$ and $k$ be nonnegative integers with $0\leq k \leq n$. The number of distinct subsets of size $k$ that a set of size $n$ has is given by the \textbf{binomial coefficient}

            \[
                \binom{n}{k} = \frac{n!}{k!(n-k)!}
            \]
        \end{theorem}

        Conventionally, $0! = 1$ gives $\binom{n}{0} = 1$ which makes sense because the empty set $\emptyset$ is the unique subset with zero elements. The binomial coefficient $\binom{n}{k}$ is defined as 0 for integers
        $k < 0$ and $k > n$. \\

        \textit{\blue{Example 6:}} In a class there are 12 boys and 14 girls. How many different teams of 7 pupils with 3 boys and 4 girls can be created? \\

        Imagine that first we choose the 3 boys and then the 4 girls. We can choose the 3 boys in $\binom{12}{3}$ different ways and the 4 girls in $\binom{14}{4}$ different ways. Thus we can form the team in
        $\binom{12}{3}\cdot \binom{14}{4}$ different ways. \\

        \textit{\blue{Example 7:}} Let $A=\{1,2,3,4,5,6\}$. \\
        (a) How many different two-element subsets of $A$ are there that have one element from $\{1,2,3,4\}$ and one element from $\{5,6\}$? \\

        The exhaustive list shows that the answer is 8:

        \[
            \{1,5\}, \{1,6\}, \{2,5\},\{2,6\},\{3,5\},\{3,6\},\{4,5\},\{4,6\}
        \]

        We get the correct answer by first picking one element from $\{1,2,3,4\}$ (four choices) and then one element from $\{5,6\}$ (two choices) and multiplying: $4\cdot 2 = 8$. Another way to arrive at the same answer
        is to take

        \[
            \binom{6}{2} - \binom{4}{2} - \binom{2}{2} = 15 - 6 - 1 = 8
        \]

        \begin{theorem}{Multinomial Coefficient}
            Let $n$ and $r$ be positive integers and $k_1, \dots, k_r$ nonnegative integers such that $k_1 + \dots + k_r = n$. The number of ways of assigning labels $1, 2, \dots, r$ to $n$ items so that, for each
            $i=1,2,\dots,r$ exactly $k_i$ items receive label $i$, is the \textbf{multinomial coefficient}

            \[
                \binom{n}{k_1, k_2, \dots, k_r} = \frac{n!}{k_1 ! k_2 ! \dots k_r !}
            \]
        \end{theorem}

        \textit{\blue{Example 8:}} 120 students signed up for a class. The class is divided into 4 sections numbered 1, 2, 3, and 4, which will have 25, 30, 31, 34 students, respectively. How many ways are there to
        divide up the students among the four sections? \\

        The answer is

        \[
            \binom{120}{25,30,31,34} = \frac{120!}{25! \cdot 30! \cdot 31! \cdot 34!}
        \]

        \textit{\blue{Example 9:}} If you flip a fair coin 4 times what is the probability that you will get exactly 2 tails? \\

        We are trying to find the combination of 2 heads from 4 coins:

        \begin{align*}
            C(4,2) = \frac{4!}{2! (4-2)!} = 6
        \end{align*}

        The total number of all combinations of 4 flips is:

        \[
            C = 2^4 = 16
        \]

        Thus the probability is

        \[
            P(\text{Exactly 2H}) = \frac{6}{16}
        \]

        \textit{\blue{Example 10:}} There are 9 students in a class: 5 boys and 4 girls. If the teacher picks a group of 4 at random,w hat is the probability that everyone in the group is a boy? \\

        The number of ways to pick a group of 4 students out of 9 is

        \[
            \frac{9!}{4! (9-4)!}
        \]

        and the number of ways to pick a group of 4 students out of 5 is

        \[
            \frac{5!}{4!(5-4)!}
        \]

        Thus the probability is

        \[
            \frac{\frac{5!}{4!(5-4)!}}{\frac{9!}{4!(9-4)!}} = \frac{5}{126}
        \]

    \subsection{Binomial Coefficients}

        A finite sum may be represented by

        \[
            \sum_{k=1}^n a_k = a_1 + a_2 + \dots + a_n
        \]

        A product may be represented similarly like so:

        \[
            \Pi_{k=1}^n a_k = a_1 \cdot a_2 \cdot a_3 \dots a_n
        \]

        Exponentials and logarithms can be used to convert between sums and products:

        \[
            \Pi_{k=1}^n a_k = \exp{\left(\ln{\Pi_{k=1}^n} a_k\right)} = e^{\sum^n_{k=1}\ln{a_k}}, a_k > 0
        \]

        By convention, an empty sum has value zero while an empty product has value 1. This is because 0 is the additive identity and 1 is the multiplicative property. The $0! = 1$ is an instance of this convention.

        \begin{axiom}{Summation Identities}
            Let $n$ be a positive integer. Then

            \begin{align*}
                1 + 2 + \dots + n   &= \frac{n(n+1)}{2} \\
                1^2 + 2^2 + \dots + n^2 &= \frac{n(n+1)(2n+1)}{6} \\
                1^3 + 2^3 + \dots + n^3 &= \frac{n^2 (n+1)^2}{4}
            \end{align*}
        \end{axiom}

        This continues to hold for arbtirary upper and lower summation limits:

        \[
            \sum^b_{k=a} k = a + a(a+1) + (a+2) + \dots + b = (b-a+1) \frac{a+b}{2}
        \]

        \begin{theorem}{Binomial theorem}
            Let $n$ be a positive integer. Then for any $x,y$,

            \[
                (x+y)^n = \sum^n_{k=0} \binom{n}{k} x^k y^{n-k}
            \]
        \end{theorem}

        \begin{theorem}{Multinomial theorem}
            Let $n$ and $r$ be positive integers. Then for any $x_1, \dots, x_r$,

            \begin{align*}
                (x_1 + x_2 + \dots + x_r)^n &= \sum_{\substack{k_1 \geq 0,k_2 \geq 0,\dots, k_r \geq 0 \\ k_1 + k_2 + \dots + k_r = n}} \binom{n}{k_1,k_2,\dots,k_r} x_1^{k_1}x_2^{k_2}\dots x_r^{k_r}
            \end{align*}

            The sum runs over $r$-tuples $(k_1, k_2, \dots, k_r)$ of nonnegative integers that add up to $n$.
        \end{theorem}