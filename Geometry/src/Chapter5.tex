\section{Transformations}
    \color{purple} \textbf{The 4 Main Geometrical Transformations:} \color{black} \\

    \begin{figure} [hbt!]
        \centering
        \begin{subfigure}[b]{.45\linewidth}
            \includegraphics[scale=0.5]{Resources/Unit5Transformations/rotation.PNG}
            \caption*{Rotation}
        \end{subfigure}
        \begin{subfigure}[b]{.45\linewidth}
            \includegraphics[scale=0.5]{Resources/Unit5Transformations/reflection.PNG}
            \caption*{Reflection}
        \end{subfigure}
        \begin{subfigure}[b]{.45\linewidth}
            \includegraphics[scale=0.5]{Resources/Unit5Transformations/translation.PNG}
            \caption*{Translation}
        \end{subfigure}
        \begin{subfigure}[b]{.45\linewidth}
            \includegraphics[scale=0.5]{Resources/Unit5Transformations/homothety.PNG}
            \caption*{Homothety}
        \end{subfigure}
    \end{figure}

    \noindent When homothety is used to transform a figure, the figure and its result are
    similar. If a figure is transformed by any method besides homothety, the figure and its
    result are congruent.