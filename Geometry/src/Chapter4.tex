\section{Circles}

    \subsection{Circle Basics}
        A \textbf{circle} is a 2D figure composed of all the points that are a given distance
        (\textbf{radius}) away from a centre. The \textbf{diameter} of a circle is the distance
        from one side of the circle to the other, passing through the centre of the circle.
        It is double the length of the radius. The \textbf{circumference} of a circle is the
        distance around the circle. \\

        \noindent When $d$ is the diameter of a circle, the circumference of said circle, $C$,
        is given by \\

        \begin{equation*}
            C = \pi d
        \end{equation*}

        \noindent The area of any circle is given by \\

        \begin{equation*}
            A=\pi r^2
        \end{equation*}

        \noindent A circle has about 80\% of the area of an equal-width square. \\

        \noindent A \textbf{tangent} is a line that intersects the edge of a circle only once
        and is perpendicular at the intersection point. A line that intersects a circle twice
        is called a \textbf{secant}. A \textbf{chord} is a line segment travelling between any
        two points on a circle's circumference. A \textbf{diameter} is also a chord that passes
        through the centre of the circle. An \textbf{arc} is any section of the circumference.
        A \textbf{sector} is any part of a circle enclosed by two radii and their intercepted
        arc. A \textbf{segment} is any slice on a circle made by a chord. A quarter of a circle
        is a \textbf{quadrant} and half a circle is a \textbf{semicircle}. \\

        \begin{figure} [hbt!]
            \centering
            \includegraphics[scale = 0.8] {Resources/Unit4Circles/lines.PNG}
            \includegraphics[scale = 0.8] {Resources/Unit4Circles/slices.PNG}
        \end{figure}

        \noindent The arc of a circle, $s$, where $\theta$ is the intercepted angle between the
        two radii composing the arc \textit{measured in RADIANS}, is given by \\

        \begin{equation*}
            s=r\theta
        \end{equation*}

        \noindent The area of a sector, where $\theta$ is the intercepted angle between the two
        radii composing the sector \textit{measured in RADIANS}, is given by \\

        \begin{equation*}
            A=\frac{1}{2} r^2 \theta
        \end{equation*}

        \noindent \color{purple} \textbf{The 7 Common Circle Theorems:} \color{black} \\

        \noindent \color{purple} \textbf{1. Angle at the Centre Theorem} \color{black} \\

        \begin{figure} [hbt!]
            \centering
            \includegraphics[scale=0.6]{Resources/Unit4Circles/circle1.PNG}
        \end{figure}

        \noindent The angle formed at the centre of a circle by lines originating from any two
        points on the circle's circumference is double the angle formed on the circumference of
        the circle by lines originating from the same points. \\

        \pagebreak
        \noindent \color{purple} \textbf{2. Angle in a Semicircle Theorem} \color{black} \\
        \begin{figure} [hbt!]
            \centering
            \includegraphics[scale=0.75]{Resources/Unit4Circles/circle2.PNG}
        \end{figure}

        \noindent An angle formed by constructing lines from the ends of the diameter of a circle
        to its circumference form a right angle. \\

        \noindent \color{purple} \textbf{3. Angles in the Same Segment} \color{black} \\

        \begin{figure} [hbt!]
            \centering
            \includegraphics[scale=0.7]{Resources/Unit4Circles/circle3.PNG}
        \end{figure}

        \noindent Angles formed from two points on the circumference are equal to other angles
        in the same arc formed from said points. \\

        \noindent \color{purple} \textbf{4. Cyclic Quadrilaterals} \color{black} \\

        \begin{figure} [hbt!]
            \centering
            \includegraphics[scale=0.6]{Resources/Unit4Circles/circle4.PNG}
        \end{figure}

        \noindent The opposite angles of a cyclic quadrilateral are equal. \\

        \noindent \color{purple}\textbf{5. Radius to a Tangent} \color{black} \\

        \begin{figure} [hbt!]
            \centering
            \includegraphics[scale=0.75]{Resources/Unit4Circles/circle5.PNG}
        \end{figure}

        \noindent The radius and tangent that intercept any point on a circle's circumference
        form a right angle. \\

        \pagebreak
        \noindent \color{purple} \textbf{6. Tangents from a Point to a Circle} \color{black} \\

        \begin{figure} [hbt!]
            \centering
            \includegraphics[scale=0.75]{Resources/Unit4Circles/circle6.PNG}
        \end{figure}

        \noindent When two tangents are constructed on the same circle, the distance between
        their point of intersection with the circle and point of intersection with each other
        is the same. Furthermore, when a line from the intersection point of the two tangents
        to the centre of the circle is constructed, the angles formed between said line and
        either tangent are equal. \\

        \noindent \color{purple} \textbf{7. Alternate Segment Theorem} \color{black} \\

        \begin{figure} [hbt!]
            \centering
            \includegraphics[scale=0.75]{Resources/Unit4Circles/circle7.PNG}
        \end{figure}

        \noindent The alternate angles inside two segments constructed within a circle are equal.



    \subsection{Power of a Point Theorem}
        The \color{purple} \textbf{Power of a Point Theorem} \color{black} is a relationship
        holding between the lengths of the line segments formed when two lines intersect a circle
        and each other. There are three possibilities for which this theorem holds: \\

        \noindent 1. The two lines are chords of the circle and intersect inside the circle.
        In this case, we have $AE\cdot CE = BE\cdot DE$. \\

        \begin{figure} [hbt!]
            \centering
            \includegraphics[scale=0.75]{Resources/Unit4Circles/power1.PNG}
        \end{figure}

        \noindent 2. One line is a tangent and the other is a secant. In this case, we have
        $AB^2=BC\cdot BD$. \\

        \begin{figure} [hbt!]
            \centering
            \includegraphics[scale=0.75]{Resources/Unit4Circles/power2.PNG}
        \end{figure}

        \pagebreak
        \noindent 3. Both lines are secants and intersect outside the circle. In this case,
        $CB\cdot CA = CD\cdot CE$. \\

        \begin{figure} [hbt!]
            \centering
            \includegraphics[scale=0.75]{Resources/Unit4Circles/power3.PNG}
        \end{figure}