\section{Solids}

    \subsection{Basics of Solids}
        A \textbf{solid} is a 3D figure. A \textbf{face} of a solid is a flat surface. A
        \textbf{edge} of a solid is where 2 or more faces meet. A \textbf{vertex} is where 2 or
        more straight edges meet. \\



    \subsection{Polyhedra}
        \noindent A \textbf{polyhedron} is a solid composed only of flat faces and each face is
        a polygon. Common polyhedra include the platonic solids, prisms, and pyramids. \\

        \noindent The \textbf{Platonic Solids} are solids where each face is the same regular
        polygon and the same number of polygons meet at each vertex. In $\mathbb{R}^3$ there are
        5 Platonic Solids, of which each was theorized by Plato to represent the five elements:
        earth, air, fire, water, and the universe. \\

        \begin{figure} [hbt!]
            \centering
            \begin{subfigure}[b]{.45\linewidth}
                \includegraphics[scale=0.25]{Resources/Unit6Solids/tetrahedron.png}
                \caption*{Tetrahedron}
            \end{subfigure}
            \begin{subfigure}[b]{.45\linewidth}
                \includegraphics[scale=0.5]{Resources/Unit6Solids/cube.PNG}
                \caption*{Cube}
            \end{subfigure}
            \begin{subfigure}[b]{.45\linewidth}
                \includegraphics[scale=1]{Resources/Unit6Solids/octahedron.PNG}
                \caption*{Octahedron}
            \end{subfigure}
            \begin{subfigure}[b]{.45\linewidth}
                \includegraphics[scale=0.25]{Resources/Unit6Solids/dodecahedron.PNG}
                \caption*{Dodecahedron}
            \end{subfigure}
            \begin{subfigure}[b]{.45\linewidth}
                \includegraphics[scale=0.4]{Resources/Unit6Solids/icosahedron.PNG}
                \caption*{Icosahedron}
            \end{subfigure}
        \end{figure}

        \noindent A \textbf{prism} is a polyhedron with identical ends and the same cross-section
        along its length. Let $B$ and $h$ represent the base and height of a prism. Then  \\

        \noindent \color{purple} \textbf{Volume of a Prism:} \color{black}
        \begin{equation*}
            V = Bh
        \end{equation*}

        \noindent A \textbf{pyramid} is any polyhedron where there is a polygon base and all
        other faces are triangles. The other faces all connect the base to the \textbf{apex}. \\

        \noindent \color{purple} \textbf{Volume of a Pyramid:} \color{black} \\
        \begin{equation*}
            V = \frac{Bh}{3}
        \end{equation*}



    \subsection{Non-Polyhedrons}
        \textbf{Nonpolyhedrons} have curved surfaces. Common non-polyhedrons include spheres,
        cylinders, cones, and tori. \\

        \noindent \textbf{Spheres} are 3D figures where all points on the surface are the same
        distance from a center. Spheres are perfectly symmetrical, have no edges nor vertices,
        and only have one surface. Of all the solids, spheres have the greatest volume to surface
        area ratio. A \textbf{hemisphere} is half of a sphere. \\

        \pagebreak
        \noindent \color{purple} \textbf{Surface Area of a Sphere:} \color{black} \\

        \begin{equation*}
            A = 4\pi r^2
        \end{equation*}

        \noindent \color{purple} \textbf{Volume of a Sphere:} \color{black} \\

        \begin{equation*}
            \frac{4}{3}\pi r^3
        \end{equation*}

        \noindent A \textbf{cylinder} is a non-polyhedron with straight parallel sides and an
        ellipse cross-section. A \textbf{Right Cylinder} is a cylinder with its two bases
        directly aligned on top of each other. An \textbf{Oblique Cylinder} is any cylinder
        that is not right.

        \begin{figure} [hbt!]
            \centering
            \includegraphics[scale=0.5]{Resources/Unit6Solids/cylinders.png}
            \caption*{Right and Oblique Cylinders}
        \end{figure}

        \noindent \color{purple} \textbf{Surface Area of a Cylinder}: \color{black}

        \begin{equation*}
            A=2\pi rh+2\pi r^2
        \end{equation*}

        \noindent \color{purple} \textbf{Volume of a Cylinder:} \color{black}

        \begin{equation*}
            V=\pi r^2 h
        \end{equation*}

        \noindent A \textbf{cone} is a non-polyhedron that tapers smoothly from a flat base
        to an \textbf{apex}. It can be formed by rotating a triangle $360^\circ$ in along
        the dependent axis. When the apex of a cone is aligned with the centre of the base,
        it is a \textbf{right cone}. Otherwise, the cone is \textbf{oblique}. \\

        \begin{figure} [hbt!]
            \centering
            \includegraphics[scale=0.5]{Resources/Unit6Solids/cones.jpg}
            \caption*{Right and Oblique Cylinders}
        \end{figure}

        \noindent \color{purple} \textbf{Surface Area of a Right Cone}: \color{black} \\

        \begin{equation*}
            A=\pi r(r+\sqrt{h^2+r^2})
        \end{equation*}

        \noindent \color{purple} \textbf{Volume of a Cone}: \color{black} \\

        \begin{equation*}
            V = \frac{\pi r^3h}{3}
        \end{equation*}

        \noindent The \textbf{torus} is a solid of revolution generated by revolving a
        circle in $\mathbb{R}^3$ about an axis coplanar with the circle. Its shape resembles
        that of a donut. For tori, $r$ is the smaller radius of the circle cross-section and
        $R$ is the larger radius around which the circle is swept, such that \\

        \begin{figure} [hbt!]
            \centering
            \includegraphics[scale=0.5]{Resources/Unit6Solids/torus.png}
        \end{figure}

        \noindent \color{purple} \textbf{Surface Area of a Torus}: \color{black} \\

        \begin{equation*}
            A = 4 {\pi}^2 Rr
        \end{equation*}

        \noindent \color{purple} \textbf{Volume of a Torus}: \color{black} \\

        \begin{equation*}
            V = 2 {\pi}^2 Rr^2
        \end{equation*}