\section{Rational Functions}

    A rational function is any function that can be expressed as the ratio of two polynomial
    functions, where the denominator is not equal to 0. The domain of a rational function
    $f(x)=\frac{P(x)}{Q(x)}$ is the set of all points for which $Q(x)\not=0$.
    \textbf{Singularities} are the $x$-values at which rational functions are undefined, for
    which $Q(x)\not=0$. \\

    \noindent \textbf{Oblique Asymptotes} are asymptotes that are neither perpendicular nor
    parallel, rather, they are inclined. Vertical asymptotes occur at all singularities for
    rational functions, and a rational function can have at most one horizontal/oblique asymptote. \\

    \noindent If a function $f(x)=\frac{P(x)}{Q(x)}$ has a highest degree of $n$ in the numerator
    and $m$ in the denominator then: \\

    % Table with asymptote types
    \begin{center}
        \begin{tabular}{|c|c|}
            \hline
            $n>m$ & No Horizontal Asymptote (Although if $n=m+1$ then there is an Oblique Asymptote)            \\
            \hline
            $n<m$ & $x$-axis is a Horizontal Asymptote                                                          \\
            \hline
            $n=m$ & Horizontal Asymptote exists at $y=\frac{\text{Coefficient of }        n}{\text{Coefficient of } m}$ \\
            \hline
        \end{tabular}
    \end{center}

    \noindent \color{blue} \textit{Example 1: Find any horizontal or oblique asymptotes for
    $f(x)=\frac{2x^2+x+1}{x^2+16}$} \color{black} \\
    Because $n=m$, there will be one horizontal asymptote and no oblique asymptote, given by
    $y=\frac{2}{1}=2$. \\

    \noindent \color{purple} \textbf{Steps for Graphing Rational Functions:} \color{black} \\
    1. Find the intercepts \\
    2. Find the vertical asymptotes if they exist by setting the denominator equal to zero and solving \\
    3. Find the horizontal or oblique asymptote if it exists \\
    4. Sketch at least one point in each region divided by the vertical asymptotes.
    Add more points for more accuracy. \\
    5. Sketch the graph \\

    \noindent \color{blue} \textit{Example 1: Sketch the graph of $f(x)=\frac{3x+6}{x-1}$}
    \color{black} \\
    Starting with the intercepts, the $y$-intercept is \\

    \begin{equation*}
        f(0)=\frac{6}{-1}=-6\implies (0, -6)
    \end{equation*}

    \noindent and the $x$-intercepts will be \\

    \begin{equation*}
        3x+6=0, x=-2\implies (-2,0)
    \end{equation*}

    \noindent Now let's find the asymptotes, starting with the vertical asypmtote. \\

    \begin{equation*}
        x-1=0\implies x=1
    \end{equation*}

    \noindent Since $n=m$, there will be a horizontal asymptote a \\

    \begin{equation*}
        y=\frac{3}{1}=3
    \end{equation*}

    \noindent After plugging some $x$-values into the function, we can find the general shape of the graph.
    Now we sketch the graph with its asymptotes. \\

    % Graph of f(x) with asymptotes identified
    \begin{center}
        \begin{tikzpicture}
            \begin{axis}[
                axis lines = center,
                xmin = -20,
                xmax = 20,
                ymin = -20,
                ymax = 20,
            ]
            % f(x)
            \addplot [unbounded coords=jump,
                domain=-20:-2,
                samples=41,
                color=red,
            ]
            {(3*x+6)/(x-1)};
            \addlegendentry{$f(x)$}
            % Asymptote 1
            \addplot [unbounded coords=jump,
                domain=-2:1,
                samples=16,
                color=red,
            ]
            {(3*x+6)/(x-1)};
            % Asymptote 2
            \addplot [unbounded coords=jump,
                domain=1:20,
                samples=46,
                color=red,
            ]
            {(3*x+6)/(x-1)};
            %Vertical Asymptote
            \draw[dashed] (1,\pgfkeysvalueof{/pgfplots/ymin}) --
            (1,\pgfkeysvalueof{/pgfplots/ymax})
            (\pgfkeysvalueof{/pgfplots/xmin},3) --
            (\pgfkeysvalueof{/pgfplots/xmax},3);
            \end{axis}
        \end{tikzpicture}
    \end{center}