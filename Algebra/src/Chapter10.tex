\section{Conics}

    \subsection{Introduction to Conics}
        From Analytic Geometry, \textbf{Conic Sections} are the intersection of a plane and
        2 opposite-facing solid cones from different angles. \\

        \begin{figure} [hbt!]
            \centering
            \includegraphics [scale=0.6] {Resources/Unit10Conics/conics.PNG}
        \end{figure}

        \noindent These curves can be defined using a straight line (\textbf{directrix}) and a
        point (\textbf{focus}). The distance from the focus to a point on the curve and the
        distance perpendicularly from the directrix to that point will always be the same ratio.
        For ellipses, this ratio is less than 1. For parabolas, the ratio is 1, hence the
        distances are equal. For hyperbolas, this ratio is greater than 1. This ratio is called
        \textbf{eccentricity}, which graphically shows us how "un-circular" the curve is. The
        larger the eccentricity, the less curved it is. Circles have an eccentricity of 0. \\

        \begin{figure} [hbt!]
            \centering
            \includegraphics [scale=0.5] {Resources/Unit10Conics/ecc.PNG}
        \end{figure}

        \noindent The \textbf{latus rectum} runs parallel to the directrix and passes through the focus. \\
        \noindent \textbf{Length of the Latus Rectum:} \\

        \begin{center}
            \begin{tabular}{|c|c|}
                \hline
                \textbf{Parabolas} & 4x focal length  \\
                \hline
                \textbf{Circles}   & the diameter     \\
                \hline
                \textbf{Ellipses}  & $\frac{2b^2}{a}$ \\
                \hline
            \end{tabular}
        \end{center}

        \noindent Above, $a$ and $b$ are the \textbf{major} and \textbf{minor axes}, respectively.

        \begin{figure} [hbt!]
            \centering
            \includegraphics[scale = 0.6] {Resources/Unit10Conics/latrect.PNG}
            \includegraphics[scale = 0.6] {Resources/Unit10Conics/lactrect2.PNG}
        \end{figure}

        \noindent The \textbf{General Equation} that covers all conic equations is given by \\

        \begin{equation*}
            Ax^2+Bxy+Cy^2+Dx+Ey+F=0
        \end{equation*}

        \noindent where $A,B,C,D,E,F$ are all constants. \\

        \noindent \color{purple} \textbf{Process to Determine Conic Type from General Form:} \color{black} \\
        1. Are both variables squared? \\
        If no, it's a parabola. If yes, go to next step. \\
        2. Do the squared terms have opposite signs? \\
        If yes, it's a hyperbola. If no, go to next step. \\
        3. Are the squared terms multiplied by the same number? \\
        If yes, it's a circle. If no, it's an ellipse.



    \pagebreak
    \subsection{Ellipses}
        For the equations below, $a$ is the length of half of the major axis, $b$ is the length of
        half of the minor axis, and $A,C,D,E,F$ are constants, respectively. \\

        \begin{center}
            \begin{tabular}{|c|c|}
                \hline
                \textbf{General Form}
                & $Ax^2+Cy^2+Dx+Ey+F=0$                       \\
                \hline
                \textbf{Standard Form, Horizontal Major Axis}
                & $\frac{(x-h)^2}{a^2}+\frac{(y-k)^2}{b^2}=1$ \\
                \hline
                \textbf{Standard Form, Vertical Major Axis}
                & $\frac{(x-h)^2}{b^2}+\frac{(y-k)^2}{a^2}=1$ \\
                \hline
            \end{tabular}
        \end{center}

        \noindent Center: $(h,k)$ \\
        Length of Major Axis: $2a$ \\
        Length of Minor Axis: $2b$ \\
        Distance between center and foci, represented by $c$, is $c^2=a^2-b^2,a>b>0$ \\

        \noindent The major axis runs between the vertices, whereas the minor axis runs between
        the co-vertices. \\

        \begin{figure} [hbt!]
            \centering
            \includegraphics [scale=0.6] {Resources/Unit10Conics/ellipse1.jpg}
        \end{figure}

        \noindent \color{purple} \textbf{Graphing Ellipses:} \color{black} \\
        Determine the major axis, vertices, co-vertices, and foci. \\
        \textbf{Equation is in form $\frac{(x-h)^2}{a^2}+\frac{(y-k)^2}{b^2}=1, a>b$}

        \begin{center}
            \begin{tabular} {|c|c|}
                \hline
                Center                         & $(h,k)$              \\
                \hline
                Major Axis                     & parallel to $x$-axis \\
                \hline
                Coordinates of the Vertices    & $(h\pm a, k)$        \\
                \hline
                Coordinates of the Co-vertices & $(h,k \pm b)$        \\
                \hline
                Coordinates of the Foci        & $(h\pm c, k)$        \\
                \hline
            \end{tabular}
        \end{center}

        \noindent \textbf{Equation is in form $\frac{(x-h)^2}{b^2}+\frac{(y-k)^2}{a^2}=1, a>b$} \\

        \begin{center}
            \begin{tabular} {|c|c|}
                \hline
                Center                         & $(h,k)$              \\
                \hline
                Major Axis                     & parallel to $y$-axis \\
                \hline
                Coordinates of the Vertices    & $(h, k\pm a)$        \\
                \hline
                Coordinates of the Co-vertices & $(h\pm b, k)$        \\
                \hline
                Coordinates of the Foci        & $(h, k\pm c)$        \\
                \hline
            \end{tabular}
        \end{center}

        \noindent \color{blue} \textit{Example 1: Graph $\frac{(x+2)^2}{4}+\frac{(y-5)^2}{9}=1$}
        \color{black} \\
        \noindent Because $a$ is the always the bigger number, $a^2=9$ and $b^2=4$. Because
        $9>4$ the major axis is parallel to the $y$-axis. The center $(h,k)$ is then $(-2,5)$. The
        vertices are $(h,k\pm a)=(-2,2),(-2,8)$. The co-vertices are $(h\pm b,k)=(-4,5),(0,5)$.
        Since $c^2=a^2-b^2=9-4=5\implies c=\pm\sqrt{5}$, the foci are $(h,k\pm c)=(-2,5-\sqrt{5}),
        (-2,5\pm 5)$. \\

        \begin{figure} [hbt!]
            \centering
            \includegraphics [scale=0.5] {Resources/Unit10Conics/ellipse2.PNG}
        \end{figure}

        \noindent \color{blue} \textit{Example 2: Graph the ellipse given by
        $4x^2+9y^2-40x+36y+100=0$} \color{black}  \\

        \begin{align*}
            (4x^2-40x)+(9y^2+36y) &= -100 \\
            4(x^2-10x) + 9(y^2+4y) &= -100 \\
            4(x^2-10x+25)+9(y^2+4y+4) &= -100 + 100 +36 \\
            4(x-5)^2+9(y+2)^2 &= 36 \\
            \frac{(x-5)^2}{9}+\frac{(y+2)^2}{4}=1
        \end{align*}

        \noindent Because $9>4$. the major axis is parallel to the $x$-axis. Since $a^2=9$ and
        $b^2=4$, $c^2=9-4\implies c=\pm \sqrt{5}$. The center is $(5, -2)$. The vertices are
        $(2,-2),(8,-2)$. The co-vertices are $(5,-4),(5,0)$. The foci are $(5-\sqrt{5},-2),
        (5+\sqrt{5},-2)$. \\

        \begin{figure} [hbt!]
            \centering
            \includegraphics [scale=0.5] {Resources/Unit10Conics/ellipse3.PNG}
        \end{figure}



    \subsection{Circles}
        A \textbf{circle} is the set of all points on a plane that are a fixed distance from the
        center. A circle is not a function because it fails the vertical line test. A circle is a
        special type of ellipse with equations below. \\

        \begin{center}
            \begin{tabular} {|c|c|}
                \hline
                \textbf{General Form}
                & $x^2+y^2+Cx+Dy+E=0$   \\
                \hline
                \textbf{Standard Form}
                & $(x-h)^2+(y-k)^2=r^2$ \\
                \hline
            \end{tabular}
        \end{center}

        \noindent Above, the center is given by $(h,k)$ and the radius as $r$.

        \noindent \color{purple} \textbf{Graphing Circles:} \color{black} \\
        Determine the center $(h,k)$ and radius $r$ and graph it.



    \subsection{Parabolas}
        We have already graphed and worked with parabolas in the past, so here are the general
        forms of parabolas when used in the context of conics. \\

        \begin{center}
            \begin{tabular} {|c|c|c|}
                \hline
                \textbf{Parabola, Horizontal Axis}
                & $(y-k)^2=4p(x-h),p\not=0$
                & Vertex is $(h,k)$ \\
                & & Focus is $(h+p,k)$            \\
                & & Directrix is the line $x=h-p$ \\
                & & Axis is the line $y=k$        \\
                \hline
                \textbf{Parabola, Vertical Axis}
                & $(x-h)^2=4p(y-k),p\not=0$
                & Vertex is $(h,k)$ \\
                & & Focus is $(h,k+p)$            \\
                & & Directrix is the line $y=k-p$ \\
                & & Axis is the line $x=h$        \\
                \hline
            \end{tabular}
        \end{center}



    \subsection{Hyperbolas}
        Hyperbolas look like this: \\
        \begin{figure} [hbt!]
            \centering
            \includegraphics [scale=0.3] {Resources/Unit10Conics/hyperbola.png}
        \end{figure}

        \begin{center}
            \begin{tabular} {|c|c|}
                \hline
                \textbf{General Form}
                & $Ax^2-Cy^2+Dx+Ey+F=0$                       \\
                \hline
                \textbf{Hyperbola, Horizontal Transverse Axis}
                & $\frac{(x-h)^2}{a^2}-\frac{(y-k)^2}{b^2}=1$ \\
                \hline
                \textbf{Hyperbola, Vertical Transverse Axis}
                & $\frac{(y-k)^2}{a^2}-\frac{(x-h)^2}{b^2}=1$ \\
                \hline
            \end{tabular}
        \end{center}

        \noindent Center: $(h,k)$ \\
        Distance between vertices: $2a$ \\
        Distance between foci: $2c$ \\
        $c^2=a^2+b^2$ \\

        \noindent The eccentricity, $e$, has the formula \\
        $e=\frac{\sqrt{a^2+b^2}}{a}$  \\

        \noindent The reciprocal function $f(x)=\frac{1}{x}$ is a hyperbola.