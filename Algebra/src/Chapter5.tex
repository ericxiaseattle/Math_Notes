\section{More Functions}

    \subsection{Composite Functions}
        Function composition refers to when one function is applied to the results of another.
        The formal notation of "the result of function $f$ sent through function $g$ is written
        $(g\circ f)(x)$ which is also the same as saying $g(f(x))$. \\

        \noindent \color{blue} \textit{Example: Given $f(x)=3x+2$ and $g(x)=x+5$,
        find $(f\circ g)(x)$}. \color{black} \\

        \begin{align*}
            (f\circ g)(x) &= f(x+5) \\
            &= 3(x+5)+2 \\
            &= 3x+17
        \end{align*}



    \subsection{Inverse Functions}
        An \textbf{inverse function} is a function that reverses another function. If a function
        $f(x)=y$ then its inverse, denoted by the superscript -1, is given by $f^{-1}(y)=x$.
        The inverse function is also commonly wrote as $g(x)$. \\

        \noindent An \textbf{invertible function} is a function that has an inverse. A function
        is invertible if and only if the function is \textbf{one-to-one}, or for each $y$-value,
        there must be only one value of $x$ such that $Y\rightarrow X$. Such a function is also
        described as \textbf{bijective}. A function is bijective only if and only if it passes the
        horizontal line test. Inverse functions are symmetrical across the line $y=x$. \\

        \noindent \color{blue} \textit{Example: Given $f(x)=\sqrt{x-3}$, find $g^{-1}(x)$ for
        $x\geq0$}. \color{black} \\

        \begin{align*}
            y=f(x) &= \sqrt{x-3} \\
            x &= \sqrt{y-3} \\
            x^2 &= y-3 \\
            x^2+3 &= y \\
            g^{-1}(x) &= x^2+3
        \end{align*}