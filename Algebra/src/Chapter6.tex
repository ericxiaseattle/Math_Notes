\section{Exponents and Logarithms}

    \subsection{Review of Exponents and Logarithms}
        \color{purple} \textbf{Exponent Laws:} \color{black} \\
        1. $a^ma^n=a^{m+n}$ \\
        2. $(a^m)^n=a^{mn}$ \\
        3. $(ab)^m=a^mb^m$ \\
        4. $\frac{a^m}{a^n}=a^{m-n},a\not=0$ \\
        5. $(\frac{a}{b})^m=\frac{a^m}{b^m},b\not=0$ \\
        6. $a^-m=\frac{1}{a^m},a\not=0$ \\
        7. $a^{\frac{1}{n}}=\sqrt[n]{a}$ \\
        8. $a^0=1,a\not=0$ \\
        9. $a^{\frac{m}{n}}=\sqrt[n]{a^m}=(\sqrt[n]{a})^m$ \\

        \noindent Logarithms are the inverse operation of exponents. In other words, \\
        $y=\log_ax\iff x=a^y, a>0$ \\
        where $a$ is the base. Conventionally, if a base is unspecified then it is 10 such that
        $\log x = \log _{10} x$. \\

        \noindent The natural logarithm, $\ln{x}$, is a logarithm with base $e$. \\

        \noindent \color{purple} \textbf{Logarithm Properties:} \color{black} \\
        1. $\log _a xy=\log _a x+\log _a y$ \\
        2. $\log _a \frac{x}{y}=\log _a x - \log _a y$ \\
        3. $\log _a x^y=y\cdot\log _a x$ \\
        4. $\log _a a^x=x$ \\
        5. $a^{\log _a x} = x$ \\
        6. $\log _a \frac{1}{x}=-\log _a {x}$ \\

        \noindent \color{purple} \textbf{Common Logarithms:} \color{black} \\
        1. $\ln{e}=1$ \\
        2. $\log _a {1}=0, a>0$ \\
        3. $\log _a {0} = $ UND \\
        4. $\log _a {a} = 1, a>0$



    \subsection{Graphing Exponential and Logarithmic Functions}
        The function $y=e^x$ is the inverse of $y=\ln{x}$. From the graph, we can see that the
        two functions are symmetric across the line $y=x$. There is a horizontal asymptote at
        $y=0$ in the graph of $e^x$ and a vertical asymptote at $x=0$ in the graph of $y=\ln{x}$.

        \begin{center}
            \begin{tikzpicture}
                \begin{axis}[
                    axis lines = center,
                    xmin = -10,
                    xmax = 10,
                    ymin = -10,
                    ymax = 10
                ]
                %y=ln(x)
                \addplot [
                    domain=0:10,
                    samples=100,
                    color=red
                ]
                {ln(x)};
                \addlegendentry{$y=\ln{x}$}
                %y=e^x
                \addplot [
                    domain=-5:10,
                    samples=100,
                    color=blue
                ]
                {e^x};
                \addlegendentry{$y=e^x$}
                %y=x
                \addplot[
                    domain = -10:10,
                    samples=100,
                    color=black,
                    style=dashed
                ]
                {x};
                \end{axis}
            \end{tikzpicture}
        \end{center}