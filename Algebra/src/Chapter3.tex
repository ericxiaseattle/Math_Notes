\section{Polynomials}

    \subsection{Long Division}
        \color{purple} \textbf{Dividend = Divisor $\cdot$ Quotient + Remainder} \color{black} \\
        \noindent $f(x)=d(x)\cdot q(x)+r(x)$ \\
        \noindent Note that the degree of $r(x)$ is always less than that of $d(x)$. \\

        \noindent \color{blue} \textit{Example 1: Divide $2x^2-5x-1$ by $x-3$} \color{black} \\

        \begin{figure} [hbt!]
            \centering
            \includegraphics[scale = 0.8] {Resources/Unit3Polynomials/longdiv.PNG}
        \end{figure}

        \begin{equation*}
            \therefore 2x^2-5x-1 = (x-3)(2x+1)+2
        \end{equation*}

        \noindent where the remainder is 2. \\

        \noindent \color{purple} \textbf{The Remainder Theorem:} \color{black} \\
        \noindent When a polynomial $f(x)$ is divided by $x-c$ the remainder is $f(c)$. \\

        \noindent \color{blue} \textit{Example 2: Find the remainder when $2x^2-5x-1$ is
        divided by $x-5$} \color{black} \\

        \begin{align*}
            f(5) &= 2(5)^2-5(5)-1 \\
            &= 50-25-1 \\
            &= 24
        \end{align*}

        \noindent Thus, the remainder is 24. \\
        \noindent \color{purple} \textbf{The Factor Theorem:} \color{black} \\
        When $f(c)=0$, $x=c$ is a factor of $f(x)$. The converse of this is also true.



    \subsection{Synthetic Division}
        Synthetic division is a shortcut for long division. It only works when the divisor is
        of the first degree. The remainder produced by synthetic division is also the the value
        of the function at the remainder, $f(r)$ when $r$ is the remainder. \\

        \noindent \color{blue} \textit{Example: Divide $x^3-2x^2+3x-4$ by $x-2$} \color{black} \\

        \noindent 1. Draw a table like so and write $c$ on the left side of the table, for a
        divisor in the form $x-c$. In this case, $c=2$. \\

        \begin{figure} [hbt!]
            \centering
            \includegraphics [scale = 0.5] {Resources/Unit3Polynomials/synthdiv1.png}
        \end{figure}

        \noindent 2. Add the coefficients of the polynomial's terms under the line as is. In
        this case, the coefficients corresponding to $x^3, x^2, x^1,$ and $x^0$ are $1,-2,3,$
        and $-4$, respectively. \\

        \begin{figure} [hbt!]
            \centering
            \includegraphics [scale = 0.5] {Resources/Unit3Polynomials/synthdiv2.png}
        \end{figure}

        \noindent 3. Drop the first coefficient, in this case 1, below the newly made bar. Then
        write the product of $c$ and the first coefficient below the second coefficient. Now
        write the sum of the second coefficient and this product below the bar. \\

        \begin{figure} [hbt!]
            \centering
            \includegraphics [scale = 0.5] {Resources/Unit3Polynomials/synthdiv3.png}
        \end{figure}

        \noindent 4. Find the product of $c$ and the calculated sum, in this case the second
        green number. Find the sum of this product and the third coefficient. Repeat this
        process for the rest of the coefficients. The last calculated sum is the remainder,
        in this case the purple number.

        \begin{figure} [hbt!]
            \centering
            \includegraphics [scale = 0.5] {Resources/Unit3Polynomials/synthdiv4.png}
            \includegraphics [scale = 0.5] {Resources/Unit3Polynomials/synthdiv5.png}
        \end{figure}

        \noindent 5. The three green sums are the coefficients of the quotient. The quotient
        should be of degree 2, since its largest term $x^3$ divided by the divisor's largest
        term, $x$, is $x^2$. Thus, $x^2$ corresponds to the first green number, and so on.
        Here, $c$ refers to the constant term of the polynomial.

        \begin{figure} [hbt!]
            \centering
            \includegraphics [scale = 0.5] {Resources/Unit3Polynomials/synthdiv6.png}
        \end{figure}

        \noindent Thus,

        \begin{align*}
            \frac{x^3-2x^2+3x-4}{x-2} &= 1x^2+0x+3+\frac{2}{x-2} \\
            &= x^2+3+\frac{2}{x-2}
        \end{align*}



    \subsection{Vieta's Formulas}
        Vieta’s Formulas relate the coefficients of polynomials to the sums and products of their
        roots. Whereas Vieta’s Formulas seem trivial in quadratic applications, they become
        extremely useful in complex polynomials with many roots or roots that are hard to derive.
        Vieta’s Formulas can be viewed as a shortcut for finding solutions of a polynomial quickly
        simply through the sums and products of the roots. \\

        \noindent Consider the polynomial $x^2+2x-15=(x-3)(x+5) \implies x=-5,3$. Using Vieta's
        Formulas, we can find the sum of the roots $3+(-5)=-2$ and $3\cdot(-5)=-15$ directly,
        without having to find each root directly. \\

        \noindent By the Remainder Factor Theorem, a polynomial $f(x)$ has roots $r_1$ and $r_2$
        in the form $f(x)=A(x-r_1)(x-r_2)=Ax^2-A(r_1+r_2)x+Ar_1 r_2$ for some constant $A$.
        Comparing coefficients with $f(x) = ax^2+bx+c$, we can conclude that $a=A, b=-A(r_1+r_2)$,
        and $c=Ar_1 r_2$. Hence, we get: \\

        \noindent \color{purple} \textbf{Vieta's Formulas for Quadratics:} \color{black} \\
        \noindent Given $f(x) = ax^2+bx+c$ if the equation $f(x)=0$ has roots $r_1$ and $r_2$, then

        \begin{equation*}
            r_1 + r_2 = -\frac{b}{a}, r_1 r_2 = \frac{c}{a}
        \end{equation*}

        \noindent Alternatively,

        \begin{equation*}
            b=-(p+q),c=pq
        \end{equation*}

        \noindent It is important to note that the precondition for using Vieta's Formula is that
        the polynomial $f(x)$ must be put in a \textbf{monic} form, that is, the leading coefficient
        $a$ must be 1. \\

        \noindent \color{blue} \textit{Example 1: If $\alpha$ and $\Beta$ are the roots of the
        quadratic $x^2-4x+9=0$, what are the values of} \\
        \noindent 1. $\alpha+\Beta$ \\
        \noindent 2. $\alpha\Beta$ \\
        \noindent 3. $\alpha^2\Beta^2$ \color{black} \\

        \noindent 1.
        \begin{equation*}
            \alpha + \Beta = -\frac{b}{a} = -\frac{-4}{1} = 4
        \end{equation*}

        \noindent 2.
        \begin{equation*}
            \alpha\Beta = \frac{c}{a} = \frac{9}{1} = 9
        \end{equation*}

        \noindent 3.
        \begin{align*}
            \alpha^2+\Beta^2 &= (\alpha + \Beta)^2 - 2\alpha\Beta \\
            &= 4^2 - 2(9) \\
            &= -2
        \end{align*}

        \noindent For this question, the roots were $2\pm i\sqrt{5}$. vieta's Formulas offer a
        simpler approach to compute these expressions without potentially making calculation
        mistakes. \\

        \noindent \color{blue} \textit{Example 2: What are the roots of the quadratic $x^2-5x+6$?} \color{black} \\
        \noindent If $p$ and $q$ are the roots of the quadratic, then $p+q=5$ and $pq=6$. Solving
        this system, it's easy to see that the roots are 2 and 3. \\

        \noindent \color{blue} \textit{Example 3: Find a quadratic with roots 2 and 5.} \color{black} \\
        $b=-(2+5)=-7, c=(2)(5)=10$ \\
        Hence, the desired quadratic is $x^2-7x+10$.

        \pagebreak
        \noindent \color{blue} \textit{Example 4: Find a quadratic with roots $3+2i$ and $3-2i$.} \color{black} \\
        \noindent $b=-[(3+2i)+(3-2i)] = -6,c=(3+2i)(3-2i)=13$. \\
        \noindent The desired quadratic is $x^2-6x+13$.

        \noindent We can also generalize these formulas to higher-degree polynomials: \\
        \noindent \color{purple} \textbf{Vieta's Formula:} \color{black} \\
        Let $P(x)=a_n x^n + a_{n-1}x^{n-1}+\dots+a_0$ be a polynomial with complex coefficients
        and degree $n$, having complex roots $r_n, r_{n-1},\dots,r_1$. Then for any integer
        $0\leq k \leq n$, \\

        \begin{equation*}
            \sum\limits_{1\leq i_1 < i_2<\dots<i_k\leq n}   r_{i_1}r_{i_2}\dots r_{i_k}
            = (-1)^k \frac{a_{n-k}}{a_n}
        \end{equation*}

        \noindent Vieta's Formulas for quadratic polynomials written in summation notation is \\
        \begin{equation*}
            \sum\limits_{i=1}^n r_i     = -\frac{a_{n-1}}{a_n}, r_1r_2\dots r_n=(-1)^n\frac{a_0}{a_n}
        \end{equation*}

        \noindent \color{blue} \textit{Example 5: Suppose $k$ is a number such that the cubic
        polynomial $P(x)=-2x^3+48x^2+k$ has three integer roots that are all prime numbers. How
        many possible distinct values are there for $k$?} \color{black} \\

        \noindent Let $p,q,$ and $r$ denote the three integer roots of $P(x)$. Then by Vieta's
        Formula, we have $pq+qr+pr=0$ but since $p,q,r$ are prime, each of them are strictly
        greater than 1 and hence no such $P(x)$ exists.