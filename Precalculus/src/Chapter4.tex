\section{Matrices}

    \subsection{Fundamentals of Matrices}
        A \textbf{matrix} is an array of numbers. The \textbf{Identity Matrix} is a special matrix
        equivalent to the number 1 and of the form \\

        \noindent $I$ =

        \begin{bmatrix}
            1 & 0 & 0 \\
            0 & 1 & 0 \\
            0 & 0 & 1
        \end{bmatrix}. \\

        \noindent It has the same number of rows as columns and has 1s on the main diagonal and 0s
        everywhere elsewhere. When we multiply by $I$ the original is unchanged. \\

        \noindent The \textbf{determinant} is a number that can be calculated from a
        \textbf{square matrix} (same number of rows as columns) that gives us information regarding
        systems of linear equations and matrix inverses. The symbol for determinants are two pipes on
        either side of a matrix, much like the absolute value symbol. For example, $|A|$ is the
        determinant of $A$.

        \noindent \textbf{Addition/Subtraction:} add the values in the corresponding positions \\
        \noindent
        \begin{bmatrix}
            a & b \\
            c & d
        \end{bmatrix}
        +
        \begin{bmatrix}
            e & f \\
            g & h
        \end{bmatrix}
        =
        \begin{bmatrix}
            a+e & b+f \\
            c+g & d+h
        \end{bmatrix} \\

        \noindent \textbf{Negation:} Distribute the opposite sign \\

        \noindent -
        \begin{bmatrix}
            a & b \\
            c & d
        \end{bmatrix}
        =
        \begin{bmatrix}
            -a & -b \\
            -c & -d
        \end{bmatrix} \\

        \noindent \textbf{Constant Multiple:} Distribute the constant \\

        \noindent k
        \begin{bmatrix}
            a & b \\
            c & d
        \end{bmatrix}
        =
        \begin{bmatrix}
            ka & kb \\
            kc & kd
        \end{bmatrix} \\

        \noindent \textbf{Matrix Multiplication:} Take the dot product of the rows and columns
        of the matrices being multiplied together. When matrices are multiplied together,
        the number of columns in the $1^{st}$ matrix must equal the number of rows in the $2^{nd}$
        matrix. The resulting matrix will always have the same number of rows as the $1^{st}$ matrix
        and the same number of columns as the $2^{nd}$ matrix. \\

        \noindent \textit{Example:} \\

        \noindent
        \begin{bmatrix}
            1 & 2 & 3 \\
            4 & 5 & 6
        \end{bmatrix}
        $\times$
        \begin{bmatrix}
            7 & 8 \\
            9 & 10 \\
            11 & 12
        \end{bmatrix}
        =
        \begin{bmatrix}
            58 & 64 \\
            139 & 154
        \end{bmatrix} \\

        \noindent $1^{st}$ row and $1^{st}$ column: \\
        \begin{align*}
            (1,2,3) \bullet (7,9,11) &= 1\cdot 7 + 2\cdot 9 + 3\cdot 11 \\
            &= 58
        \end{align*}

        \noindent $1^{st}$ row and $2^{nd}$ column: \\
        \begin{align*}
            (1,2,3) \bullet (8,10,12) &= 1\cdot 8 +2\cdot 10+3\cdot 12 \\
            &= 64
        \end{align*}

        \noindent $2^{nd}$ row and $1^{st}$ column: \\
        \begin{align*}
            (4,5,6) \bullet (7,9,11) &= 4\cdot 7+5\cdot 9+6\cdot 11 \\
            &= 139
        \end{align*}

        \noindent $2^{nd}$ row and $2^{nd}$ column: \\
        \begin{align*}
            (4,5,6) \bullet (8,10,12) &= 4\cdot 8+5\cdot 10+6 \cdot 12 \\
            &= 154
        \end{align*}

    \subsection{Solving Higher-Order Systems Using Augmented Matrices}
    \subsection{The Gauss-Jordan Method}