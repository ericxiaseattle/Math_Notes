\section{Trigonometry}

    \subsection{Trigonometric Functions}
        For any angles $A$,$B$, and $C$, the following definitions hold true. \\

        \begin{center}
            \begin{tabular}{ccc}
                $\sin A = \frac{a}{c}$
                & $\cos A = \frac{b}{c}$
                & $\tan A = \frac{a}{b}$ \\
                $\csc A = \frac{c}{a}$
                & $\sec A = \frac{c}{b}$
                & $\cot A = \frac{b}{a}$\\\\
                $\sin B = \frac{b}{c}$
                & $\cos B = \frac{a}{c}$
                & $\tan B = \frac{b}{a}$
            \end{tabular}
        \end{center}

        \begin{figure} [h]
            \centering
            \includegraphics [scale=0.4] {Resources/Unit2Trig/Trig.fig1.png}
        \end{figure}

        \noindent From the figure, it is easy to tell that $\sin A$ and $\csc A$, $\cos A$ and
        $\sec A$, and $\tan A$ and $\cot A$ are reciprocal functions. Hence, it is usually
        easier to just work with $\sin A$, $\cos A$, and $\tan A$ Additionally,

        \begin{center}
            \begin{tabular}{ccc}
                $\frac{\sin A}{\cos A}=\tan A$
                & and &
                $\frac{\cos A}{\sin A}=\cot A$
            \end{tabular}
        \end{center}

        \noindent Manipulating the trigonometric definitions, we get

        \begin{center}
            \begin{tabular}{cccc}
                $a=c\sin A$
                & $a=c\cos B$
                & $a = b\tan A$ & \\
                %%%
                $b=c\sin B$
                & $b=c\cos A$
                & $b=a\tan B$ & \\
                %%%
                $c=a\csc A$
                & $c=a\sec B$
                & $c=b\csc B$
                & $c=b\sec A$
            \end{tabular}
        \end{center}

        \noindent For $\triangle ABC$, we have $a^2+b^2=c^2$. It follows that

        \begin{equation*}
            (\sin ^2 A) + (\cos ^2 A) = \frac{a^2}{c^2} + \frac{b^2}{c^2}=1
        \end{equation*}

        \noindent Simplified,this is known as the \color{purple} \textbf{Pythagorean Identity}
        \color{black}. Dividing both sides of the equation by $\sin ^2 A$ and $\cos ^2 A$
        respectively, we get

        \begin{align*}
            sin^2 A + cos^2 A = 1 \\
            1 + cot^2 A = csc^2 A \\
            tan^2 A + 1 = sec^2 A
        \end{align*}



    \subsection{Sum and Difference}
        \begin{align*}
            \sin (\alpha \pm \beta) = \sin \alpha \cos \beta \pm \cos \alpha \sin \beta \\
            \cos (\alpha \pm \beta) = \cos \alpha \cos \beta \mp \sin \alpha \sin \beta
        \end{align*}

        \noindent By the definition of the tangent function, we have

        \begin{align*}
            tan(\alpha \pm \beta) &= \frac{sin(\alpha \pm \beta)}{cos(\alpha \pm \beta)}
            = \frac{\sin \alpha \cos \beta \pm \cos \alpha \sin \beta}{\cos \alpha \cos \beta \mp \sin \alpha \sin \beta} \\
            &= \frac{\frac{\sin\alpha}{\cos\alpha}\pm\frac{\sin\beta}{\cos\beta}}{1\mp\frac{\sin\alpha\sin\beta}{\cos\alpha\cos\beta}}
            = \frac{\tan\alpha \pm \tan\beta}{1 \mp \tan\alpha\tan\beta}
        \end{align*}



    \subsection{Multiple Angle Formulas}

        By defining $\alpha = \beta$ in the sum formulas, we get

        \begin{align*}
            \sin (2\alpha) = 2 \sin\alpha \cos\alpha \\
            \cos (2\alpha) = \cos ^2 \alpha - \sin ^2 \alpha \\
            \tan (2\alpha) = \frac{2\tan\alpha}{1-\tan ^2\alpha}
        \end{align*}

        \noindent We can derive the half-angle and triple angle formulas similarly using the
        same method by manipulating the sum formulas.



    \subsection{The Unit Circle}
        By rotating a right triangle around and reflecting across a unit circle, we can clearly
        see the periodic properties of the basic trig functions, $\sin\theta$, $\cos\theta$,
        and $\tan\theta$.

        \begin{center}
            \begin{tabular}{cc}
                $\sin(\theta \pm \frac{\pi}{2})=\pm\cos\theta$
                & $\cos(\theta+\frac{\pi}{2})=\mp\sin\theta$ \\
                $\sin(-\theta)=-\sin\theta$
                & $\cos(-\theta)=\cos\theta$ \\
                $\sin(\pi-\theta)=\sin\theta$
                & $\cos(\pi-\theta)=-\cos\theta$
            \end{tabular}
        \end{center}

        \noindent Through computing the $\sin x$ and $\cos x$ of the 30-60-90 and 45-45-90
        triangles, we find the coordinates of chosen points on the unit circle to be given in
        the figure below.

        \begin{figure}[h]
            \centering
            \includegraphics[scale = 0.2] {Resources/Unit2Trig/unit_circle.png}
            \caption*{Figure 2}
        \end{figure}



    \subsection{Sum and Product}
        \noindent The \color{purple} \textbf{product to sum} \color{black} identities can be
        realized through expanding the sum and difference functions on\textbf{} the right side
        of the identity.

        \begin{align*}
            \cos(\alpha)\cos(\beta)=\frac{\cos(\alpha+\beta)+\cos(\alpha-\beta)}{2} \\
            \sin(\alpha)\sin(\beta)=\frac{\cos(\alpha-\beta)-\cos(\alpha+\beta)}{2} \\
            \sin(\alpha)\cos(\beta)=\frac{\sin(\alpha+\beta)+\sin(\alpha-\beta)}{2}
        \end{align*}

        \noindent The \color{purple} \textbf{sum to product identities} \color{black} can be
        proved through expanding the sums within the product side of the identity. We can also
        infer the \color{purple} \textbf{difference to product} \color{black} identities by the
        same method.

        \begin{align*}
            \sin x + \sin y = 2\sin(\frac{x+y}{2})\cos(\frac{x-y}{2}) \\
            \cos x + \cos y = 2\cos(\frac{x+y}{2})\cos(\frac{x-y}{2}) \\
            \sin x - \sin y = 2 \cos (\frac{x+y}{2}) \sin(\frac{x-y}{2}) \\
            \cos x - \cos y = 2 \sin (\frac{x+y}{2}) \sin(\frac{y-x}{2}) \\
            \tan x \pm \tan y = \tan(x\pm y)(1\mp \tan x \tan y)
        \end{align*}



    \pagebreak
    \subsection{Graphing Trig Functions}
        All of the six fundamental trigonometric functions are \textbf{odd} functions
        (symmetric across origin) except $\cos x$ and $\sec x$, \textbf{even} functions
        (symmetric across $y$-axis). Recall that for odd functions $f(-x)=-f(x)$ and for even
        functions $f(-x)=f(x)$. A function $f(x)$ is \textbf{sinusoidal} if it can be written in
        the form $f(x)=a\sin[b(x+c)]+d$ for real constants $a, b, c,$ and $d$. Since
        $\cos x = sin (x+\frac{\pi}{2})$, $f(x)=\cos x$ is sinusoidal. \\

        \begin{center}
            \begin{tikzpicture}
                \begin{axis}[width=4in,axis equal image,
                    xmin = -8,
                    xmax = 8,
                    ymin = -6,
                    ymax = 6,
                    axis lines = center,
                    no marks,
                    xticklabels={-2$\pi$,-1.5$\pi$,...$\pi$,2$\pi$},
                    xtick={-6.2832,-4.7124,...,6.2832},
                    xlabel=$x$,
                    ylabel=$y$
                ]
                %sin
                \addplot[
                    domain = -8:8,
                    samples=200,
                    color=red
                ]
                {sin(deg(x))};
                \addlegendentry{$f(x)=\sin{x}$}
                %cos
                \addplot[
                    domain = -8:8,
                    samples=200,
                    color=blue
                ]
                {cos(deg(x))};
                \addlegendentry{$f(x)=\cos{x}$}
                %tan part 1
                \addplot[
                    unbounded coords=jump,
                    domain = -7.85:-4.72,
                    samples=200,
                    color=purple
                ]
                {tan(deg(x))};
                \addlegendentry{$f(x)=\tan{x}$}
                %tan part 2
                \addplot[
                    unbounded coords=jump,
                    domain = -4.70:-1.58,
                    samples=200,
                    color=purple
                ]
                {tan(deg(x))};
                %tan part 3
                \addplot[
                    unbounded coords=jump,
                    domain = -1.56:1.56,
                    samples=200,
                    color=purple
                ]
                {tan(deg(x))};
                %tan part 4
                \addplot[
                    unbounded coords=jump,
                    domain = 1.58:4.68,
                    samples=200,
                    color=purple
                ]
                {tan(deg(x))};
                %tan part 5
                \addplot[
                    unbounded coords=jump,
                    domain = 4.73:7.81,
                    samples=200,
                    color=purple
                ]
                {tan(deg(x))};
                % TODO: Fix dysfunctional asymptote
                % \draw[dashed,thick,green] (-4.70,4) -- (-4.71,-4);
                \end{axis}
            \end{tikzpicture}
        \end{center}

        \noindent The graph of $f(x)=\tan x$ has vertical asymptotes at $x=\frac{\pi}{2}\pm\pi$. \\

        \begin{center}
            \begin{tikzpicture}
                \begin{axis}[width=4in,axis equal image,
                    xmin = -8,
                    xmax = 8,
                    ymin = -6,
                    ymax = 6,
                    axis lines = center,
                    no marks,
                    xticklabels={-2$\pi$,-1.5$\pi$,...$\pi$,2$\pi$},
                    xtick={-6.2832,-4.7124,...,6.2832},
                    xlabel=$x$,
                    ylabel=$y$
                ]
                %csc part 1
                \addplot[
                    unbounded coords=jump,
                    domain = -7.84:-6.36,
                    samples=200,
                    color=red
                ]
                {cosec(deg(x))};
                \addlegendentry{$f(x)=\csc{x}$}
                %sec part 1
                \addplot[
                    unbounded coords=jump,
                    domain = -7.85:-4.84,
                    samples=200,
                    color=blue
                ]
                {sec(deg(x))};
                \addlegendentry{$f(x)=\sec{x}$}
                %cot part 1
                \addplot[
                    unbounded coords=jump,
                    domain = -7.85:-6.3,
                    samples=200,
                    color=purple
                ]
                {cot(deg(x))};
                \addlegendentry{$f(x)=\cot{x}$}
                %csc part 2
                \addplot[
                    unbounded coords=jump,
                    domain = -6.15:-3.25,
                    samples=200,
                    color=red
                ]
                {cosec(deg(x))};
                %csc part 3
                \addplot[
                    unbounded coords=jump,
                    domain = -3.02:-0.11,
                    samples=200,
                    color=red
                ]
                {cosec(deg(x))};
                %csc part 4
                \addplot[
                    unbounded coords=jump,
                    domain = 0.10:3.02,
                    samples=200,
                    color=red
                ]
                {cosec(deg(x))};
                %csc part 5
                \addplot[
                    unbounded coords=jump,
                    domain = 3.26:6.16,
                    samples=200,
                    color=red
                ]
                {cosec(deg(x))};
                %csc part 6
                \addplot[
                    unbounded coords=jump,
                    domain = 6.4:7.85,
                    samples=200,
                    color=red
                ]
                {cosec(deg(x))};
                %sec part 2
                \addplot[
                    unbounded coords=jump,
                    domain = -4.59:-1.68,
                    samples=200,
                    color=blue
                ]
                {sec(deg(x))};
                %sec part 3
                \addplot[
                    unbounded coords=jump,
                    domain = -1.45:1.45,
                    samples=200,
                    color=blue
                ]
                {sec(deg(x))};
                %sec part 4
                \addplot[
                    unbounded coords=jump,
                    domain = 1.67:4.60,
                    samples=200,
                    color=blue
                ]
                {sec(deg(x))};
                %sec part 5
                \addplot[
                    unbounded coords=jump,
                    domain = 4.81:7.85,
                    samples=200,
                    color=blue
                ]
                {sec(deg(x))};
                %cot part 2
                \addplot[
                    unbounded coords=jump,
                    domain = -6.17:-3.26,
                    samples=200,
                    color=purple
                ]
                {cot(deg(x))};
                %cot part 3
                \addplot[
                    unbounded coords=jump,
                    domain = -3.0:-0.1,
                    samples=200,
                    color=purple
                ]
                {cot(deg(x))};
                %cot part 4
                \addplot[
                    unbounded coords=jump,
                    domain = 0.1:3.03,
                    samples=200,
                    color=purple
                ]
                {cot(deg(x))};
                %cot part 5
                \addplot[
                    unbounded coords=jump,
                    domain = 3.26:6.16,
                    samples=200,
                    color=purple
                ]
                {cot(deg(x))};
                %cot part 6
                \addplot[
                    unbounded coords=jump,
                    domain = 6.36:7.85,
                    samples=200,
                    color=purple
                ]
                {cot(deg(x))};
                \end{axis}
            \end{tikzpicture}
        \end{center}

        \noindent The above three graphs of the functions all have vertical asymptotes at
        sections where the function is undefined and $x$ is a multiple of $\frac{\pi}{2}$.