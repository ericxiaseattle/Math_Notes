% Preamble
\documentclass{article}

% Packages
\usepackage[legalpaper, portrati, margin=0.9in]{geometry}
\usepackage{amsmath}
\usepackage{bm}
\usepackage{amssymb}
\usepackage{gensymb}
\usepackage{mathtools}
\usepackage{xcolor}
\usepackage{caption}
\usepackage{subcaption}
\usepackage{pgfplots}
\pgfplotsset{compat=1.17}
\usepackage{tikz}
\usepackage{tkz-euclide}


% File info
\title{Precalculus Notes}
\author{Eric Xia}
\date{Last Updated 10 June 2020}

% Document
\begin{document}

    \maketitle
    \tableofcontents
    \pagebreak

%%%-------------------------------------------NEW SECTION--------------------------------------%%%

    \section{Introductory Set Theory}

        \subsection{Introduction to Sets}
            Forget what you think math is. Forget even what numbers are. Instead of numbers, think
            about math in terms of "things". What is a set? Simply put, a \textbf{set} is a
            \textit{collection} of "things" with a shared defining characteristic. This is
            exactly like the array data structure in Computer Science. For example, a set of
            clothing may include shirts, pants, hats, jackets, socks, etc. and a set of colors
            may include red, blue, green, purple, brown, etc. \\

            \noindent In their purest form, sets are pretty useless. However, they are the
            foundation of mathematics and can be seen in the many branches of mathematics,
            including graph theory, algebra, real analysis, complex analysis, number theory,
            and so on. Set theory is important because it is all about using logic to connect
            numbers, models, axioms, and more. Without set theory, mathematics would not have
            meaning, and may as well be a bunch of scribbles.\\

            \noindent The \textbf{universal set} is one that contains everything relevant to the
            focus. For example, in number theory the universal set is all of the integers.
            In Calculus the universal set is generally the real numbers and in complex analysis
            the universal set is the complex numbers.\\

            \noindent Sets are composed of many \textbf{elements}, separated by commas and enclosed
            by curly braces. For example, a set of clothes may be defined as \{shirts, pants, socks,
            jackets,\dots\}. We add the \textbf{ellipsis} (dots) after all the definite elements to
            indicate that the set can keep going on forever, as there are many more types of clothing
            that we don't need to bother defining.

        \subsection{Order and Equality}
            For sets, the arrangement of the elements does not matter. For example, the set
            $\{1,2,3,4\}$ is the same set as $\{4,1,3,2\}$. In set theory, \textbf{order} is not the
            arrangement of the elements, rather it is \textit{the size of the set}. The preferred
            term for this is \textbf{cardinality}, the \textit{number of elements a set has}. \\

            \noindent In general, mathematicians use capital letters to represent sets and lowercase
            letters to represent elements in that set. For example, in the set $A=\{a,\dots\}$, $a$
            is an element of the set $A$. If an element $a$ is in the set $A$, then we can write this
            with the symbol $\in$. If $a$ is not in $A$ then we denote this with $\notin$. For
            example, if $A=\{1,2,3\}$ then it is valid to say that $1\in A$ and $5\notin A$. \\

            \noindent Two sets are equal if they have the same elements. We use the equals sign to
            show equality. For example, if $A$ is the set defined by the first four positive whole
            numbers and $B=\{4,2,1,3\}$, then $A=B$. Remember that the arrangement of the elements
            in a set do not matter.


        \pagebreak
        \subsection{Notation}
            Some of these symbols and their corresponding topics will be covered in later sections.
            It's just convenient to put everything on set notation in one section.

            \begin{center}
                \begin{tabular}{|c|c|c|}
                    \hline
                    \textbf{Symbol} & \textbf{Meaning} & \textbf{Example} \\\hline
                    $\{ \}$ & \textbf{Set} & $\{1,2,3,4\}$ \\ \hline
                    $A \cup B$ & \textbf{Union}: in A or B or both & $C\cup D=\{1,2,3,4,5\}$ \\ \hline
                    $A\cap B$ & \textbf{Intersection}: in both A and B & $C\cap D=\{3,4\}$ \\ \hline
                    $A\subseteq B$ & \textbf{Subset}: A has some or all elements of B & $\{3,4,5\}\subseteq D$ \\ \hline
                    $A\subset B$ & \textbf{Proper Subset}: A has some elements of B & $\{3,5\}\subset D$ \\ \hline
                    $A\not\subset B$ & \textbf{Not Subset}: A is not a subset of B & $\{ 1,6 \} \not\subset C$ \\ \hline
                    $A\supseteq B$ & \textbf{Superset}: A has the same elements of B or more & $\{1,2,3\}\supseteq \{1,2,3\}$ \\ \hline
                    $A \supset B$ & \textbf{Proper Superset}: A has all of B's elements and more & $\{1,2,3,4\} \supset \{1,2,3\}$ \\ \hline
                    $A \not\supset B$ & \textbf{Not Superset}: A is not a superset of B & $\{1,2,6\} \not\supset \{1,9\}$ \\ \hline
                    %%%
                    $A^c$ & \textbf{Complement}: elements not in A & $D^c=\{1,2,6,7\}$  \\
                    & & when $\mathbb{U}=\{1,2,3,4,5,6,7\}$ \\ \hline
                    $A-B$ & \textbf{Difference}: in A but not in B & $\{1,2,3,4\}-\{3,4\}=\{1,2\}$ \\ \hline
                    $a \in A$ & \textbf{Element}: $a$ is in $A$ & $3 \in \{1,2,3,4\}$ \\ \hline
                    $b \notin A$ & \textbf{Not Element}: $b$ is not in $A$ & $6 \notin \{1,2,3,4\}$ \\ \hline
                    $\emptyset$ & \textbf{Empty Set}: $\{\}$ & $\{1,2\} \cap\{3,4\}=\emptyset$ \\ \hline
                    $\mathbb{U}$ & \textbf{Universal Set}: set of all possible values in the area of interest &   \\ \hline
                    $P(A)$ & \textbf{Power Set}: all subsets of A & $P(\{1,2\})=\{\{\},\{1\},\{2\},\{1,2\}\}$ \\ \hline
                    %%%
                    $A=B$ & \textbf{Equality}: both A and B have the same elements & $\{3,4,5\}=\{5,4,3\}$ \\ \hline
                    $A\times B$ & \textbf{Cartesian Product}: set of ordered pairs from A and B & $\{1,2\}\times\{3,4\}$ \\
                    & & $=\{(1,3),(1,4),(2,3),(2,4)\}$  \\ \hline
                    $|A|$ & \textbf{Cardinality}: the number of elements in $A$ & $|\{3,4\}|=2$ \\ \hline
                    %%%
                    $|,:$ & \textbf{Such That} & $\{n|n>0\}=\{1,2,3,\dots\}$ \\ \hline
                    $\forall$ & \textbf{For All} & $\forall x>1,x^2.x$ \\ \hline
                    $\exists$ & \textbf{There Exists} & $\exists$  $x|x^2>x$ \\ \hline
                    $\therefore$ & \textbf{Therefore} & $a=b \therefore b =a$ \\ \hline
                    $\because$ & \textbf{Because} & \\ \hline
                    %%%
                    $\mathbb{N}$ & \textbf{Natural Numbers} & $\{1,2,3,\dots\}$ or $\{0,1,2,3,\dots\}$ \\ \hline
                    $\mathbb{Z}$ & \textbf{Integers} & $\{\dots,-3,-2,-1,0,1,2,3,\dots\}$ \\ \hline
                    $\mathbb{Q}$ & \textbf{Rational Numbers} & \\ \hline
                    $\mathbb{A}$ & \textbf{Algebraic Numbers} & \\ \hline
                    $\mathbb{R}$ & \textbf{Real Numbers} & \\ \hline
                    $\mathbb{I}$ & \textbf{Imaginary Numbers} & $3i$ \\ \hline
                    $\mathbb{C}$ & \textbf{Complex Numbers} & $2+5i$ \\ \hline
                \end{tabular}
            \end{center}


        \subsection{Fundamental Laws of Set Algebra}
            \begin{center}
                \begin{tabular}{|c|c|}
                    \hline
                    $A\cup B=B \cap A$ & \textbf{Commutative} Property \\
                    $A\cap B = B\cap A$ & \\ \hline
                    $(A\cup B)\cup C = A\cup (B\cup C)$ & \textbf{Associative} Property \\
                    $(A\cap B)\cap C = A\cap (B\cap C)$ & \\ \hline
                    $A\cup (B \cap C) = (A\cup B) \cap (A\cup C)$ & \textbf{Distributive} Property \\
                    $A\cap (B \cup C) = (A\cap B) \cup (A\cap C)$ & \\ \hline
                \end{tabular}

                \begin{tabular}{|c|c|}
                    \hline
                    $A \cup \emptyset = A$ & \textbf{Identity} \\
                    $A \cap \mathbb{U}=A$ & \\ \hline
                    $A \cup A^C = \mathbb{U}$ & \textbf{Complement} \\
                    $A \cap A^C = \emptyset$ & \\ \hline
                \end{tabular}
            \end{center}\\

            \noindent \color{purple} \textbf{The Principle of Duality} \color{black} states that
            for any true statement about sets, the \textit{dual }statement obtained by interchanging
            $\cup$ and $\cap$, $\cup$ and $\emptyset$, and reversing inclusions is also true. A
            statement is \textbf{self-dual} if it is equal to its dual. \\

            \begin{center}
                \begin{tabular}{|c|c|}
                    \hline
                    $A\cup A=A$ & \textbf{Idempotent} Laws \\
                    $A \cap A = A$ & \\ \hline
                    $A\cup \mathbb{U}=\mathbb{U}$ & \textbf{Domination} Laws\\
                    $A\cap \emptyset = \emptyset$ & \\ \hline
                    $A\cup (A \cap B) = A$  & \textbf{Absorption} Laws \\
                    $A\cap (A \cup B) = A$ & \\ \hline
                \end{tabular}
            \end{center}

        \pagebreak
        \subsection{Complements}
            The \textbf{complement} of a set $A$ refers to the elements not in $A$. In the figure
            below, if $A$ is the area colored red in the left image, then the complement of $A$,
            denoted $A^c$ is everything else, as shown in the right image.

            \begin{figure}[hbt!]
                \centering
                \includegraphics[scale=0.4]{Resources/Unit1SetTheory/complement1.PNG}
                \caption*{Figure 1}
            \end{figure}

            \noindent Below are some laws about complements.

            \begin{center}
                \begin{tabular}{|c|c|}
                    \hline
                    $(A\cup B)^C = A^C \cap B^C$ & \textbf{De Morgan's Laws} \\
                    $(A\cap B)^C = A^C \cup B^C$ & \\ \hline
                    $(A^C)^C=A$ & \textbf{Involution} Law, \textit{also known as "double complement law"} \\ \hline
                    $\emptyset^C=\mathbb{U}$ & \\ \hline
                    $\mathbb{U}^C=\emptyset$ & \\ \hline
                \end{tabular}
            \end{center}\\

            \noindent If $A\cup B=\mathbb{U}$ and $A\cap B=\emptyset$, then $B=A^C$.

            \subsection{Inclusion}

            \noindent \textbf{Subsets} are parts of a set. For example, in the set $\{1,2,3,4,5\}$,
            one subset is $\{1,2,3\}$. Two others are $\{3,4\}$ and $\{1\}$. However, $\{1,6\}$ is
            not a subset since the element 6 is not in the parent set. We use \textbf{$A \subseteq B$}
            to denote that A is a subset of B.

            \begin{figure} [hbt!]
                \centering
                \includegraphics[scale=0.6]{Resources/Unit1SetTheory/subsets.PNG}
                \caption*{Figure 2}
            \end{figure}

            \noindent \color{blue} \textit{Example 1: Let $A$ be all the multiples of 4 and let
            $B$ be all the multiples of 2.} Is $A$ a subset of $B$? Is $B$ a subset of $A$?
            \color{black} \\

            $A=\{\dots,-8,-4,0,4,8,\dots\}$\\
            $B=\{\dots,-8,-6,-4,-2,0,2,4,6,8,\dots\}$\\
            By pairing elements from $A$ and $B$, we can see that every element of $A$ is also an
            element of $B$, but not every element of $B$ is an element of $A$.\\

            \begin{figure} [hbt!]
                \centering
                \includegraphics[scale=0.6]{Resources/Unit1SetTheory/subset2.PNG}
                \caption*{Figure 3}
            \end{figure}

            \noindent $\therefore A \subseteq B, B\not\subseteq A$ \\
            \noindent $A$ is a \textbf{proper subset} of $B$ if and only if every element in $A$ is
            also in $B$ and there exists \textit{at least one element} in $B$ that is \textit{not}
            in $A$. We use $\{1,2,3\}\subset\{1,2,3,4\}$ to denote that the first set is a proper
            subset of the second set since the element $4$ is not in the first set. In another
            example, $\{1,2,3\}\subseteq\{1,2,3\}$ but $\{1,2,3\}\not\subset\{1,2,3\}$.


        \subsection{Null Sets}
            An \textbf{empty} set, or \textbf{null} set, is one with \textit{no elements}.
            Represented by $\emptyset$, an example of this is "even numbers that are also odd".
            Obviously, no number like this exists, so the set is null. Furthermore,
            \textit{every empty set is a subset}. Intuitively, since we cannot find any elements
            in the empty set that are not in set $A$, then all elements in the empty set are in $A$.


    \pagebreak
%%%-------------------------------------------NEW SECTION--------------------------------------%%%

    \section{Trigonometry}

        \subsection{Trigonometric Functions}
            For any angles $A$,$B$, and $C$, the following definitions hold true. \\

            \begin{center}
                \begin{tabular}{ccc}
                    $\sin A = \frac{a}{c}$
                    & $\cos A = \frac{b}{c}$
                    & $\tan A = \frac{a}{b}$ \\
                    $\csc A = \frac{c}{a}$
                    & $\sec A = \frac{c}{b}$
                    & $\cot A = \frac{b}{a}$\\\\
                    $\sin B = \frac{b}{c}$
                    & $\cos B = \frac{a}{c}$
                    & $\tan B = \frac{b}{a}$
                \end{tabular}
            \end{center}

            \begin{figure} [h]
                \centering
                \includegraphics [scale=0.4] {Resources/Unit2Trig/Trig.fig1.png}
            \end{figure}

            \noindent From the figure, it is easy to tell that $\sin A$ and $\csc A$, $\cos A$ and
            $\sec A$, and $\tan A$ and $\cot A$ are reciprocal functions. Hence, it is usually
            easier to just work with $\sin A$, $\cos A$, and $\tan A$ Additionally,

            \begin{center}
                \begin{tabular}{ccc}
                    $\frac{\sin A}{\cos A}=\tan A$
                    & and &
                    $\frac{\cos A}{\sin A}=\cot A$
                \end{tabular}
            \end{center}

            \noindent Manipulating the trigonometric definitions, we get

            \begin{center}
                \begin{tabular}{cccc}
                    $a=c\sin A$
                    & $a=c\cos B$
                    & $a = b\tan A$ & \\
                    %%%
                    $b=c\sin B$
                    & $b=c\cos A$
                    & $b=a\tan B$ & \\
                    %%%
                    $c=a\csc A$
                    & $c=a\sec B$
                    & $c=b\csc B$
                    & $c=b\sec A$
                \end{tabular}
            \end{center}

            \noindent For $\triangle ABC$, we have $a^2+b^2=c^2$. It follows that

            \begin{equation*}
                (\sin ^2 A) + (\cos ^2 A) = \frac{a^2}{c^2} + \frac{b^2}{c^2}=1
            \end{equation*}

            \noindent Simplified,this is known as the \color{purple} \textbf{Pythagorean Identity}
            \color{black}. Dividing both sides of the equation by $\sin ^2 A$ and $\cos ^2 A$
            respectively, we get

            \begin{align*}
                sin^2 A + cos^2 A = 1 \\
                1 + cot^2 A = csc^2 A \\
                tan^2 A + 1 = sec^2 A
            \end{align*}


        \subsection{Sum and Difference}
            \begin{align*}
                \sin (\alpha \pm \beta) = \sin \alpha \cos \beta \pm \cos \alpha \sin \beta \\
                \cos (\alpha \pm \beta) = \cos \alpha \cos \beta \mp \sin \alpha \sin \beta
            \end{align*}

            \noindent By the definition of the tangent function, we have

            \begin{align*}
                tan(\alpha \pm \beta) &= \frac{sin(\alpha \pm \beta)}{cos(\alpha \pm \beta)}
                = \frac{\sin \alpha \cos \beta \pm \cos \alpha \sin \beta}{\cos \alpha \cos \beta \mp \sin \alpha \sin \beta} \\
                &= \frac{\frac{\sin\alpha}{\cos\alpha}\pm\frac{\sin\beta}{\cos\beta}}{1\mp\frac{\sin\alpha\sin\beta}{\cos\alpha\cos\beta}}
                = \frac{\tan\alpha \pm \tan\beta}{1 \mp \tan\alpha\tan\beta}
            \end{align*}

        \subsection{Multiple Angle Formulas}

            By defining $\alpha = \beta$ in the sum formulas, we get

            \begin{align*}
                \sin (2\alpha) = 2 \sin\alpha \cos\alpha \\
                \cos (2\alpha) = \cos ^2 \alpha - \sin ^2 \alpha \\
                \tan (2\alpha) = \frac{2\tan\alpha}{1-\tan ^2\alpha}
            \end{align*}

            \noindent We can derive the half-angle and triple angle formulas similarly using the
            same method by manipulating the sum formulas.

        \subsection{The Unit Circle}
            By rotating a right triangle around and reflecting across a unit circle, we can clearly
            see the periodic properties of the basic trig functions, $\sin\theta$, $\cos\theta$,
            and $\tan\theta$.

            \begin{center}
                \begin{tabular}{cc}
                    $\sin(\theta \pm \frac{\pi}{2})=\pm\cos\theta$
                    & $\cos(\theta+\frac{\pi}{2})=\mp\sin\theta$ \\
                    $\sin(-\theta)=-\sin\theta$
                    & $\cos(-\theta)=\cos\theta$ \\
                    $\sin(\pi-\theta)=\sin\theta$
                    & $\cos(\pi-\theta)=-\cos\theta$
                \end{tabular}
            \end{center}

            \noindent Through computing the $\sin x$ and $\cos x$ of the 30-60-90 and 45-45-90
            triangles, we find the coordinates of chosen points on the unit circle to be given in
            the figure below.

            \begin{figure}[h]
                \centering
                \includegraphics[scale = 0.2] {Resources/Unit2Trig/unit_circle.png}
                \caption*{Figure 2}
            \end{figure}

        \subsection{Sum and Product}
            \noindent The \color{purple} \textbf{product to sum} \color{black} identities can be
            realized through expanding the sum and difference functions on\textbf{} the right side
            of the identity.

            \begin{align*}
                \cos(\alpha)\cos(\beta)=\frac{\cos(\alpha+\beta)+\cos(\alpha-\beta)}{2} \\
                \sin(\alpha)\sin(\beta)=\frac{\cos(\alpha-\beta)-\cos(\alpha+\beta)}{2} \\
                \sin(\alpha)\cos(\beta)=\frac{\sin(\alpha+\beta)+\sin(\alpha-\beta)}{2}
            \end{align*}

            \noindent The \color{purple} \textbf{sum to product identities} \color{black} can be
            proved through expanding the sums within the product side of the identity. We can also
            infer the \color{purple} \textbf{difference to product} \color{black} identities by the
            same method.

            \begin{align*}
                \sin x + \sin y = 2\sin(\frac{x+y}{2})\cos(\frac{x-y}{2}) \\
                \cos x + \cos y = 2\cos(\frac{x+y}{2})\cos(\frac{x-y}{2}) \\
                \sin x - \sin y = 2 \cos (\frac{x+y}{2}) \sin(\frac{x-y}{2}) \\
                \cos x - \cos y = 2 \sin (\frac{x+y}{2}) \sin(\frac{y-x}{2}) \\
                \tan x \pm \tan y = \tan(x\pm y)(1\mp \tan x \tan y)
            \end{align*}


        \pagebreak
        \subsection{Graphing Trig Functions}
            All of the six fundamental trigonometric functions are \textbf{odd} functions
            (symmetric across origin) except $\cos x$ and $\sec x$, \textbf{even} functions
            (symmetric across $y$-axis). Recall that for odd functions $f(-x)=-f(x)$ and for even
            functions $f(-x)=f(x)$. A function $f(x)$ is \textbf{sinusoidal} if it can be written in
            the form $f(x)=a\sin[b(x+c)]+d$ for real constants $a, b, c,$ and $d$. Since
            $\cos x = sin (x+\frac{\pi}{2})$, $f(x)=\cos x$ is sinusoidal. \\

            \begin{center}
                \begin{tikzpicture}
                    \begin{axis}[width=4in,axis equal image,
                        xmin = -8,
                        xmax = 8,
                        ymin = -6,
                        ymax = 6,
                        axis lines = center,
                        no marks,
                        xticklabels={-2$\pi$,-1.5$\pi$,...$\pi$,2$\pi$},
                        xtick={-6.2832,-4.7124,...,6.2832},
                        xlabel=$x$,
                        ylabel=$y$
                    ]
                    %sin
                    \addplot[
                        domain = -8:8,
                        samples=200,
                        color=red
                    ]
                    {sin(deg(x))};
                    \addlegendentry{$f(x)=\sin{x}$}
                    %cos
                    \addplot[
                        domain = -8:8,
                        samples=200,
                        color=blue
                    ]
                    {cos(deg(x))};
                    \addlegendentry{$f(x)=\cos{x}$}
                    %tan part 1
                    \addplot[
                        unbounded coords=jump,
                        domain = -7.85:-4.72,
                        samples=200,
                        color=purple
                    ]
                    {tan(deg(x))};
                    \addlegendentry{$f(x)=\tan{x}$}
                    %tan part 2
                    \addplot[
                        unbounded coords=jump,
                        domain = -4.70:-1.58,
                        samples=200,
                        color=purple
                    ]
                    {tan(deg(x))};
                    %tan part 3
                    \addplot[
                        unbounded coords=jump,
                        domain = -1.56:1.56,
                        samples=200,
                        color=purple
                    ]
                    {tan(deg(x))};
                    %tan part 4
                    \addplot[
                        unbounded coords=jump,
                        domain = 1.58:4.68,
                        samples=200,
                        color=purple
                    ]
                    {tan(deg(x))};
                    %tan part 5
                    \addplot[
                        unbounded coords=jump,
                        domain = 4.73:7.81,
                        samples=200,
                        color=purple
                    ]
                    {tan(deg(x))};
                    %Asymptote not working
                    %\draw[dashed,thick,green] (-4.70,4) -- (-4.71,-4);
                    \end{axis}
                \end{tikzpicture}
            \end{center}

            \noindent The graph of $f(x)=\tan x$ has vertical asymptotes at $x=\frac{\pi}{2}\pm\pi$. \\

            \begin{center}
                \begin{tikzpicture}
                    \begin{axis}[width=4in,axis equal image,
                        xmin = -8,
                        xmax = 8,
                        ymin = -6,
                        ymax = 6,
                        axis lines = center,
                        no marks,
                        xticklabels={-2$\pi$,-1.5$\pi$,...$\pi$,2$\pi$},
                        xtick={-6.2832,-4.7124,...,6.2832},
                        xlabel=$x$,
                        ylabel=$y$
                    ]
                    %csc part 1
                    \addplot[
                        unbounded coords=jump,
                        domain = -7.84:-6.36,
                        samples=200,
                        color=red
                    ]
                    {cosec(deg(x))};
                    \addlegendentry{$f(x)=\csc{x}$}
                    %sec part 1
                    \addplot[
                        unbounded coords=jump,
                        domain = -7.85:-4.84,
                        samples=200,
                        color=blue
                    ]
                    {sec(deg(x))};
                    \addlegendentry{$f(x)=\sec{x}$}
                    %cot part 1
                    \addplot[
                        unbounded coords=jump,
                        domain = -7.85:-6.3,
                        samples=200,
                        color=purple
                    ]
                    {cot(deg(x))};
                    \addlegendentry{$f(x)=\cot{x}$}
                    %csc part 2
                    \addplot[
                        unbounded coords=jump,
                        domain = -6.15:-3.25,
                        samples=200,
                        color=red
                    ]
                    {cosec(deg(x))};
                    %csc part 3
                    \addplot[
                        unbounded coords=jump,
                        domain = -3.02:-0.11,
                        samples=200,
                        color=red
                    ]
                    {cosec(deg(x))};
                    %csc part 4
                    \addplot[
                        unbounded coords=jump,
                        domain = 0.10:3.02,
                        samples=200,
                        color=red
                    ]
                    {cosec(deg(x))};
                    %csc part 5
                    \addplot[
                        unbounded coords=jump,
                        domain = 3.26:6.16,
                        samples=200,
                        color=red
                    ]
                    {cosec(deg(x))};
                    %csc part 6
                    \addplot[
                        unbounded coords=jump,
                        domain = 6.4:7.85,
                        samples=200,
                        color=red
                    ]
                    {cosec(deg(x))};
                    %sec part 2
                    \addplot[
                        unbounded coords=jump,
                        domain = -4.59:-1.68,
                        samples=200,
                        color=blue
                    ]
                    {sec(deg(x))};
                    %sec part 3
                    \addplot[
                        unbounded coords=jump,
                        domain = -1.45:1.45,
                        samples=200,
                        color=blue
                    ]
                    {sec(deg(x))};
                    %sec part 4
                    \addplot[
                        unbounded coords=jump,
                        domain = 1.67:4.60,
                        samples=200,
                        color=blue
                    ]
                    {sec(deg(x))};
                    %sec part 5
                    \addplot[
                        unbounded coords=jump,
                        domain = 4.81:7.85,
                        samples=200,
                        color=blue
                    ]
                    {sec(deg(x))};
                    %cot part 2
                    \addplot[
                        unbounded coords=jump,
                        domain = -6.17:-3.26,
                        samples=200,
                        color=purple
                    ]
                    {cot(deg(x))};
                    %cot part 3
                    \addplot[
                        unbounded coords=jump,
                        domain = -3.0:-0.1,
                        samples=200,
                        color=purple
                    ]
                    {cot(deg(x))};
                    %cot part 4
                    \addplot[
                        unbounded coords=jump,
                        domain = 0.1:3.03,
                        samples=200,
                        color=purple
                    ]
                    {cot(deg(x))};
                    %cot part 5
                    \addplot[
                        unbounded coords=jump,
                        domain = 3.26:6.16,
                        samples=200,
                        color=purple
                    ]
                    {cot(deg(x))};
                    %cot part 6
                    \addplot[
                        unbounded coords=jump,
                        domain = 6.36:7.85,
                        samples=200,
                        color=purple
                    ]
                    {cot(deg(x))};
                    \end{axis}
                \end{tikzpicture}
            \end{center}

            \noindent The above three graphs of the functions all have vertical asymptotes at
            sections where the function is undefined and $x$ is a multiple of $\frac{\pi}{2}$.

    \pagebreak
%%%-------------------------------------------NEW SECTION--------------------------------------%%%

    \section{Vectors and Other Coordinate Systems}

        \subsection{3D Coordinate Systems}
            In the 2D Rectangular (Cartesian) Coordinate System, we have the $x$-axis and the
            $y$-axis. Points are of the form $(x,y)$. In the 3D Rectangular Coordinate System,
            we add the $z$-axis and points are of the form $(x,y,z)$. In the 2D system, we divided
            space into quadrants. Similarly, in the 3D system, we divide space into octants, numbered
            from I to VIII.

            \begin{figure} [hbt!]
                \centering
                \includegraphics[scale=0.3]{Resources/Unit3Vectors/Octants.png}
                \caption*{Figure 1}
            \end{figure}

            \noindent We refer to the set of all ordered triples of real numbers $(x,y,z)$ as
            $\mathbb{R}^3$, where
            $\mathbb{R}^3\implies\mathbb{R}*\mathbb{R}*\mathbb{R}=\{(x,y,z)|x,y,z\in\mathbb{R}\}$.
            Collectively, the $xy, xz,$ and $yz$-planes are known as the coordinate planes, where
            the $xy$-plane has $z=0$. Likewise, the $xz$ and $yz$ planes have $y=0$ and $x = 0$,
            respectively. \\

            \noindent Let's say we wanted to find the distance between two points given by
            $P_1(x_1,y_1,z_1)$ and $P_2(x_2,y_2,z_2)$. We first construct a rectangular box as
            in Figure 2, where $P_1$ and $P_2$ are opposite vertices and the faces of the box are
            parallel to the coordinate planes. If $A(x_2,y_1,z_1)$ and $B(x_2,y_2,z_1)$ are the
            vertices of the box indicated in the figure, then

            \begin{figure}[h]
                \centering
                \includegraphics[scale=0.1]{Resources/Unit3Vectors/Fig2.jpg}
                \caption*{Figure 2}
            \end{figure}

            \begin{center}
                \begin{tabular}{ccc}
                    $|P_1A|=|x_2-x_1|$
                    &  $|AB|=|y_2-y_1|$
                    &  $|BP_2|=|z_2-z_1|$
                \end{tabular}
            \end{center}

            \noindent Because triangles $P_1BP_2$ and $P_1AB$ are both right-angled, two applications
            of the Pythagorean Theorem give

            \begin{align*}
                |P_1P_2|^2=|P_1B|^2+|BP_2|^2 \\
                |P_1B|^2=|P_1A|^2+|AB|^2
            \end{align*}

            \noindent Combining these equations, we get

            \begin{align*}
                |P_1P2|^2 &= |P_1B|^2+|BP_2|^2\\
                &= |x_2-x_1|^2+|y_2-y_1|^2+|z_2-z_1|^2\\
                &= (x_2-x_1)^2+(y_2-y_1)^2+(z^2-z^1)^2
            \end{align*}

            \noindent \color{purple} \textbf{Distance Formula in 3D} \color{black} The distance
            $|P_1P_2|$ between the points $P_1(x_1,y_1,z_1)$ and $P_2(x_2,y_2,z_2)$ is given by

            \begin{equation*}
                |P_1, P_2|=\sqrt{(x_2-x_1)^2+(y_2-y_1)^2+(z_2-z_1)^2}
            \end{equation*}

            \noindent Using this formula, we can find the \color{purple} \textbf{Standard Form of
            a $\mathbb{R}^3$ Sphere}\color{black}, centered at $C(h,k,l)$ and radius $r$:

            \begin{equation*}
                (x-h)^2+(y-k)^2+(z-l)^2=r^2
            \end{equation*}

            \noindent \color{blue} \textit{Example 1 Show that $x^2+y^2+z^2+4x-6y+2z+6=0$ is the
            equation of a sphere, and find its center and radius.}\color{black}\\

            \noindent We rewrite the given equation by completion of squares:
            \begin{align*}
                (x^2+4x+4)+ &= (y^2-6y+9)+(z^2+2z+1)=-6+4+9+1\\
                &= (x+2)^2+(y-3)^2+(z+1)^2=8
            \end{align*}

            \noindent Comparing this equation with the standard form, we find the center to be
            $(-2,3,-1)$ and the radius to be $\sqrt{8}=2\sqrt{2}.$

            \noindent \color{blue} \textit{Example 2 What region in $\mathbb{R}^3$ is represented
            by the following inequalities?}

            \begin{center}
                \begin{tabular}{cc}
                    $1\leqx^2+y^2+z^2\leq4$
                    & $z\leq0$
                \end{tabular}
            \end{center}
            \color{black}

            \noindent The inequalities can be rewritten as

            \begin{equation*}
                1\leq\sqrt{x^2+y^2+z^2}\leq2
            \end{equation*}

            \noindent so they represent the points $(x,y,z)$ whose distance from the origin is at least 1 and
            at most 2. But we are also given that $z\leq0$, so the points lie on or below the
            $xy$-plane. Thus, the given inequalities represent the region that lies between or on
            the spheres $x^2+y^2+z^2=1$ and $x^2+y^2+z^2=4$ and beneath or on the $xy$-plane.
            This region is drawn in Figure 3.

            \begin{figure}[h]
                \centering
                \includegraphics[scale=0.1]{Resources/Unit3Vectors/Fig3.jpg}
                \caption*{Figure 3}
            \end{figure}


        \subsection{Introduction to Vectors}
            A \textbf{vector} is a quantity specifying both \emph{magnitude} and \emph{direction},
            often represented by an arrow. The length of the arrow corresponds with the magnitude
            of the vector. We denote vectors with the boldface letter "\textbf{v}" or with
            "\overrightarrow{$v$}". \\

            \noindent\textbf{Displacement vectors}, of the form $v=\overrightarrow{AB}$ or
            $u=\overrightarrow{CD}$, represent the magnitude and direction travelled from the
            \textbf{initial point}, A (the \emph{tail}), to the \textbf{terminal point}, B
            (the \emph{tip}). If the vectors $v$ and $u$ have the same magnitude and direction,
            even if they are in different positions, then we can say that $u$ and $v$ are
            \textbf{equivalent} (or \emph{equal}) and we write \textbf{$u=v$}. The
            \textbf{zero vector} (also known as the \emph{null vector}) is denoted by a point with a
            0 next to it. It has a magnitude of 0 since all its components are 0 and it is the only
            vector with no specific direction. The \textbf{position vector}
            (also known as \emph{location vector} or \emph{radius vector}) always starts at the
            origin, \emph{O}.\\

            \noindent To represent the movement of a particle between the points $A(x_1,y_1,z_1)$
            and $B(x_2,y_2,z_2)$, where the \textbf{components} are separated by commas, we use the
            following two notations. It is extremely important to remember to not mix up
            \overrightarrow{v} and \overrightarrow{w}
            $when describing the magnitude and distance travelled between $A$ and $B$ or $B$ and $ A.

            \begin{align*}
                \overrightarrow{AB} = \overrightarrow{v} = \langle x_2-x_1, y_2-y_1, z_2-z_1\rangle\\
                \overrightarrow{BA} = \overrightarrow{w} = \langle x_1-x_2, y_1-y_2, z_1-z_2\rangle
            \end{align*}

            \noindent \color{blue} \textit{Example 1: Give the vector from $(a)$ $(1,-3,-5)$ to
            $(2,-7,0)$ and $(b)$ the position vector for $(-90,4)$.} \color{black} \\
            $(a)$ $\langle2-1,-7-(-3),0-(-5)\rangle$\\
            $=\langle1,-4,5\rangle$\\
            $(b)$ There isn't much to this problem besides acknowledging that the position vector
            is just the components of the vector. The answer is $\langle-90,4\rangle$. \\

            \noindent The \textbf{magnitude} of the vector
            $\overrightarrow{v}=\langle x_1,y_1,z_1\rangle$ is given by

            \begin{align*}
                ||\overrightarrow{v}||=\sqrt{x_1^2+y_1^2+z_1^2}
            \end{align*}

            \noindent Hence, if $||\overrightarrow{v}||=0$ then $\overrightarrow{v}=\overrightarrow{0}.$

            \noindent\color{blue} \textit{Example 2: Determine the magnitude of
            $(a)$ $\overrightarrow{u}=\langle\frac{1}{\sqrt{5}},-\frac{2}{\sqrt{5}}\rangle$ and
            $(b)$ $\overrighatarrow{w}=\langle0,0,0\rangle.$} \color{black} \\
            (a) $||\overrightarrow{u}||=\sqrt{\frac{1}{5}+\fraq{4}{5}}=\sqrt{1}=1$\\
            (b) $||\overrightarrow{w}||=\sqrt{0+0}=0$\\

            \noindent A \textbf{unit vector} is a vector with a magnitude of 1 and is represented
            by a circumflex over a variable. (Example: $\hat{j}$ has a magnitude of 1) In
            $\mathbb{R}^3$ there are three \textbf{standard basis vectors}, given by

            \begin{center}
                \begin{tabular}{ccc}
                    $i=\langle1,0,0\rangle$
                    & $j=\langle0,1,0\rangle$
                    & $k=\langle0,0,1\rangle$
                \end{tabular}
            \end{center}

            \noindent Note that standard basis vectors are also unit vectors.\\

            \noindent \color{blue} Example 3 \color{black} The vector
            $\overrightarrow{v}=\langle6,-4,0\rangle$ starts at the point $P=(-2,5,-1)$. At what
            point does the vector end?\\

            \noindent \emph{Solution} Recall that the components of a vector are always the coordinates of the
            terminal point minus the coordinates of the starting point. So, if the ending point of the
            vector is given by $Q=(x_2,y_2,z_2)$ then we know that the vector $\overrightarrow{v}$ can
            be written as

            \begin{equation*}
                \overrightarrow{v}=\overrightarrow{PQ}=\langle x_2+2,y_2-5,z_2+1\rangle
            \end{equation*}

            \noindent We are given the components of $\overrightarrow{v}$ so we can set the components
            of the vector above to the given components. Doing so gives

            \begin{equation*}
                \langle x_2+2,y_2-5,z_2+1\rangle=\langle6,-4,0\rangle
            \end{equation*}

            \noindent If two vectors are equal then their components must also be equal. Hence,

            \begin{center}
                \begin{align*}
                    x_2+2=6\implies x_2=4\\
                    y_2-5=-4\implies y_2=1\\
                    z_2+1=0\implies z_2=-1
                \end{align*}
            \end{center}

            \noindent The endpoint of the vector is then $Q(4,1,-1)$.


        \subsection{Vector Arithmetic}
            \textbf{Sum/Difference Rule} The sum and difference of two vectors $\overrightarrow{a}$
            and $\overrightarrow{b}$ are given by

            \begin{equation}
                \overrightarrow{a}\pm\overrightarrow{b}=\langle x_1\pm x_2, y_1 \pm y_2, z_1 \pm z_2\rangle
            \end{equation}

            \noindent The addition and subtraction of two vectors can easily be visualized with the
            \textbf{Triangle Law} (Fig 4) and the \textbf{Parallelogram Law} (Fig 5).

            \begin{figure}[h]
                \centering
                \begin{minipage}{.5\textwidth}
                    \centering
                    \includegraphics[scale=0.3]{Resources/Unit3Vectors/Triangle.PNG}
                    \captionOf{Figure 4}
                \end{minipage}%
                \begin{minipage}{.5\textwidth}
                    \centering
                    \includegraphics[scale = 0.3] {Parallelogram.PNG}
                    \captionOf{Figure 5}
                \end{minipage}
            \end{figure}

            \noindent \textbf{Scalar Multiplication} If $c$ is a scalar and $\overrightarrow{v}$ is
            a vector, then the \textbf{scalar multiple} $c\overrightarrow{v}$ is the vector whose
            magnitude is $|c|$ times the magnitude of $\overrightarrow{v}$ and whose direction is
            the same as $v$ if $c>0$ and is opposite to $\overrightarrow{v}$ if $c<0$. If $c=0$ or
            $\overrightarrow{v}=0$, then $c\overrightarrow{v}=0$.

            \begin{equation}
                c\overrightarrow{v}=\langle cx_1, cy_1, cz_1 \rangle
            \end{equation}

            \noindent Two nonzero vectors are \textbf{parallel} if they are scalar multiples of one
            another. In particular, the vector $-\overrightarrow{v}=(-1)\overrightarrow{v}$ has the
            same magnitude as $\overrightarrow{v}$ but faces the opposite direction. We refer to
            this as the \textbf{negative} of $\overrightarrow{v}$. Suppose that $\overrightarrow{v}$
            and $\overrightarrow{u}$ are parallel vectors. Then there must be a number $c$ such that
            $\overrightarrow{a}=c\overrightarrow{b}$.

            \noindent \color{blue} Example 1 \color{black} Determine if
            $\overrightarrow{a} = \langle 2,-4,1\rangle, \overrightarrow{b}=\langle -6,12,-3\rangle$ \\
            \emph{Solution} The vectors are parallel since $\overrightarrow{b}=-3\overrightarrow{a}$. \\

            \noindent \color{blue} Example 2 \color{black} Find a unit vector that faces the same
            direction as $\overrightarrow{w}=\langle-5,2,1\rangle.$\\
            \emph{Solution} We first need to determine the magnitude of $\overrightarrow{w}$.\\
            $||\overrightarrow{w}||=\sqrt{25+4+1}=\sqrt{30}$.
            Then the unit vector, $\overrightarrow{u}$ is given by\\
            $\overrightarrow{u}=\frac{1}{||\overrightarrow{w}||}\overrightarrow{w}
            =\frac{1}{\sqrt{30}}\langle-5,2,1\rangle=\langle-\frac{5}{\sqrt{30}},
            \frac{2}{\sqrt{30}},\frac{1}{\sqrt{30}}\rangle$.\\
            We can check that $\overrightarrow{u}$ is an unit vector by finding its magnitude.\\
            $||\overrightarrow{u}||=\sqrt{\frac{25}{30}+\frac{4}{30}+\frac{1}{30}}=\sqrt{\frac{30}{30}}=1$\\
            $\overrightarrow{u}$ also faces the same direction as $\overrightarrow{w}$ since it is
            only a scalar multiple of $\overrightarrow{w}$ and $c > 0$.\\

            \noindent This example helps us establish the generality that, \emph{given a vector}
            $\overrightarrow{w}$, $\overrightarrow{u}=\frac{\overrightarrow{w}}{||\overrightarrow{w}||}$
            \emph{will be a unit vector that faces the same direction as} $\overrightarrow{w}$.\\

            \noindent Revising standard basis vectors, using scalar multiplication we can write
            $\overrightarrow{v}=\langle x_1,y_1,z_1\rangle$ as

            \begin{equation*}
                \overrightarrow{v}=x_1\textbf{i}+y_1\textbf{j}+z_1\textbf{k}
            \end{equation*}

            \noindent Using this concept, any vector in $\mathbb{R}^3$ can be written in terms of the
            standard basis vectors $i, j$, and $k$. For instance,

            \begin{equation*}
                \langle 1,-2,6 \rangle= \textbf{i} - 2\textbf{j} + 6\textbf{k}
            \end{equation*}

            \noindent \color{blue} Example 3 \color{black} If $\overrightarrow{v}=\langle3,-9,1\rangle$
            and $\overrightarrow{w}=-i+8k$ then compute $2\overrightarrow{v}-3\overrightarrow{w}.$\\

            \emph{Solution}

            \begin{align*}
                2\overrightarrow{v}-3\overrightarrow{w} &= 2\langle3,-9,1\rangle-3\langle-1,0,8\rangle\\
                &= \langle6,-18,2\rangle-\langle-3,0,24\rangle\\
                &= \langle9,-18,-22\rangle
            \end{align*}

            \noindent \color{blue} Example 4 \color{black} Find the unit vector in the direction of
            the vector $2i-j-2k$.\\

            \noindent \emph{Solution} The magnitude of the given vector is\\
            $|2i-j-2k|=\sqrt{2^2+(-1)^2+(-2)^2}=\sqrt{9}=3$\\
            Hence, the unit vector with the same direction is\\
            $\frac{1}{3}(2i-j-2k)=\frac{2}{3}i-\frac{1}{3}j-\frac{2}{3}k$.\\

            \noindent \color{purple} \textbf{Vector Properties} \color{black} If
            $\overrightarrow{v}$, $\overrightarrow{w}$, and $\overrightarrow{u}$ are vectors with the
            same number of components and $a$ and $b$ are two numbers, then

            \begin{center}
                \begin{tabular}{c|c}
                    $(1)\overrightarrow{v}+\overrightarrow{w}=\overrightarrow{w}+\overrightarrow{v}$
                    & $\overrightarrow{v}+(\overrightarrow{w}+\overrightarrow{u})=(\overrightarrow{v}
                    +\overrightarrow{w})+\overrightarrow{u}(2)$ \\
                    $(3)\overrightarrow{v}+0=\overrightarrow{v}$
                    & $\overrightarrow{v}+(-\overrightarrow{v})=0(4)$\\
                    $(5)a(\overrightarrow{v}+\overrightarrow{w})=a\overrightarrow{v}+a\overrightarrow{w}$
                    & $(a+b)\overrightarrow{v}=a\overrightarrow{v}+b\overrightarrow{v}(6)$\\
                    $(7)(ab)\overrightarrow{v}=a(b\overrightarrow{v})$
                    & $1\overrightarrow{v} = \overrightarrow{v}(8)$
                \end{tabular}
            \end{center}


        \subsection{The Dot Product}
            A vector can be \emph{multiplied} by another vector but not \emph{divided} by another.
            There are two types of products of vectors: one that produces a scalar quantity
            (the dot product) and one that produces a vector quantity (the cross product). \\\\

            \noindent \textbf{The Dot Product} Given two vectors $\overrightarrow{a}=\langle x_1, y_1, z_1\rangle$
            and $\overrightarrow{b}=\langle x_2, y_2, z_2\rangle$, the dot product is given by

            \begin{equation*}
                \overrightarrow{a}\bullet\overrightarrow{b}=x_1x_2+y_1y_2+z_1z_2
            \end{equation*}

            \noindent Alternatively, where $\theta$ is the angle between $\overrightarrow{a}$
            and $\overrightarrow{b}$,

            \begin{equation*}
                \overrightarrow{a}\bullet\overrightarrow{b}=ab\cos{\theta}
            \end{equation*}

            \noindent The dot product is also referred to as the \emph{scalar product} and is a type
            of an \emph{inner product}. Intuitively, the dot product of $\overrightarrow{a}$ and
            $\overrightarrow{b}$ is the product of $|b|$ with $|a_b|$ where $a_b$ is the
            \textbf{projection} of $\overrightarrow{a}$ onto $\overrightarrow{b}$.

            \begin{figure}[h]
                \centering
                \includegraphics[scale=0.4]{Resources/Unit3Vectors/dotproduct1.PNG}
                \caption*{Figure 6}
            \end{figure}

            \noindent Now taking any two vectors $\overrightarrow{a}$ and $\overrightarrow{b}$, we
            can decompose them into horizontal and vertical components. Then
            $\overrightarrow{a}=a_xi+a_yj$ and $\overrightarrow{b}=b_xi+b_yj$.

            \begin{figure}[h]
                \centering
                \begin{minipage}{.5\textwidth}
                    \centering
                    \includegraphics[scale=0.3]{Resources/Unit3Vectors/dotproduct2.PNG}
                    \captionOf{Figure 7}
                \end{minipage}%
                \begin{minipage}{.5\textwidth}
                    \centering
                    \includegraphics[scale = 0.3] {dotproduct3.PNG}
                    \captionOf{Figure 8}
                \end{minipage}
            \end{figure}

            \noindent Hence, $\overrightarrow{a}\bullet\overrightarrow{b}=(a_xi+a_yj)*(b_xi+b_yj)$.
            And since the perpendicular components have a dot product of zero,
            $\overrightarrow{a}\bullet\overrightarrow{b}=a_xb_x+a_yb_y$.\\

            \noindent The second dot product formula involving $\theta$ can be derived first by
            sketching Figure 9.

            \begin{figure} [h]
                \centering
                \includegraphics[scale=0.75]{Resources/Unit3Vectors/Fig9.PNG}
                \caption*{Figure 9}
            \end{figure}

            \noindent The three vectors in the figure form $\Delta AOB$. Note that the length of
            each side of the triangle is just the magnitude of the vector forming that side.
            By the Law of Cosines,

            \begin{equation*}
                ||\overrightarrow{a}-\overrightarrow{b}||
                =||\overrightarrow{a}||^2+||\overrightarrow{b}||^2-2||\overrightarrow{a}||||\overrightarrow{b}
                ||\cos{\theta}
            \end{equation*}

            \pagebreak
            \noindent We can rewrite the left side as

            \begin{align*}
                ||\overrightarrow{a}-\overrightarrow{b}||^2 &
                = (\overrightarrow{a}-\overrightarrow{b})\bullet(\overrightarrow{a}-\overrightarrow{b})\\
                &= \overrightarrow{a}\bullet\overrightarrow{a}
                -\overrightarrow{a}\bullet\overrightarrow{b}-\overrightarrow{b}\bullet\overrightarrow{a}
                +\overrightarrow{b}\bullet\overrightarrow{b}\\
                &= ||\overrightarrow{a}||^2-2\overrightarrow{a}\bullet\overrightarrow{b}+||\overrightarrow{b}||^2
            \end{align*}

            \noindent We can substitute this for the left side of the first equation.

            \begin{align*}
                ||\overrightarrow{a}-\overrightarrow{b}||^2
                &= ||\overrightarrow{a}||^2+||\overrightarrow{b}||^2-2||\overrightarrow{a}
                ||||\overrightarrow{b}||\cos{\theta}\\
                ||\overrightarrow{a}||^2-2\overrightarrow{a}\bullet\overrightarrow{b}
                +||\overrightarrow{b}||^2 &= ||\overrightarrow{a}||^2+||\overrightarrow{b}||^2-2
                ||\overrightarrow{a}||||\overrightarrow{b}||\cos{\theta}\\
                -2\overrightarrow{a}\bullet\overrightarrow{b} &= -2||\overrightarrow{a}||||\overrightarrow{b}
                ||\cos{\theta}\\\overrightarrow{a}\bullet\overrightarrow{b} &= ||\overrightarrow{a}
                ||||\overrightarrow{b}||\cos{\theta}
            \end{align*}

            \noindent \color{red} \textbf{Dot Product Properties}\\ \color{black} If
            $\overrightarrow{u}, \overrightarrow{v},$ and $\overrightarrow{w}$ are vectors and
            $c$ is a scalar, then

            \begin{center}
                \begin{tabular} {c|c}
                    $\overrightarrow{u}\bullet(\overrightarrow{v}+\overrightarrow{w})
                    =\overightarrow{u}\bullet\overrightarrow{v}+\overrightarrow{u}\bullet\overrightarrow{w}$
                    & $(c\overrightarrow{v})\bullet\overrightarrow{w}
                    =\overrightarrow{v}\bullet(c\overrightarrow{w})
                    =c(\overrightarrow{v}\bullet\overrightarrow{w})$ \\
                    $\overrightarrow{v}\bullet\overrightarrow{w}
                    =\overrightarrow{w}\bullet\overrightarrow{v}$
                    & $\overrightarrow{v}\bullet\overrightarrow{0}=0$\\
                    $\overrightarrow{v}\bullet\overrightarrow{v}=||\overrightarrow{v}||^2$
                    & If $\overrightarrow{v}\bullet\overrightarrow{v}=0$ then
                    $\overrightarrow{v}=\overrightarrow{0}$
                \end{tabular}
            \end{center}

            \noindent \color{blue} Example 1 \color{black} Compute the dot product for
            $\overrightarrow{v}=5i-8j, \overrightarrow{w}=i+2j$.\\

            \noindent \textit{Solution} $\overrightarrow{v}\bullet\overrightarrow{w}=5-16=-11$.

            \noindent \color{blue} Example 2 \color{black} Compute the dot product for
            $\overrightarrow{a}=\langle0,3,-7\rangle, \overrightarrow{b}=\langle 2,3,1\rangle$\\

            \noindent \textit{Solution} $\overrightarrow{a}\bullet\overrightarrow{b}=0+9-7=2$

            \noindent \color{blue} Example 3 \color{black} Determine the angle between
            $\overrightarrow{a}=\langle 3,-4,1\rangle$ and $\overrightarrow{b}=\langle 0,5,2\rangle$.\\
            \textit{Solution} We need both the dot product and the magnitude to find the angle.

            \begin{align*}
                \overrightarrow{a}\bullet\overrightarrow{b} &= -22\\
                ||\overrightarrow{a}|| &= \sqrt{26}\\
                ||\overrightarrow{b}|| &= \sqrt{29}
            \end{align*}

            \noindent The angle is then given by

            \begin{align*}
                \cos{\theta} &= \frac{\overrightarrow{a}\bullet\overrightarrow{b}}{||\overrightarrow{a}||||\overrightarrow{b}||}=\frac{-22}{\sqrt26\sqrt29}=-0.8011927\\
                \theta &= \arccos(-0.8011927)=2.5 rad = 143.24\degree
            \end{align*}

        \pagebreak
        \subsection{Projections}
        \subsection{Direction Angles and Cosines}
        \subsection{The Cross Product}
        \subsection{Lines and Planes}
        \subsection{Cylinders and Quadric Surfaces}
        \pagebreak

        \subsection{Polar Coordinates}
            The \textbf{Polar Coordinate System} is a 2D coordinate system which defines each point
            in the form $(r,\theta)$, where $r$ is the distance from the reference point
            (usually radius) and $\theta$ is an angle from a reference direction. The
            \textbf{pole} is the reference point and the \textbf{polar axis} is the ray from the
            pole in the reference direction. \\

            \begin{figure} [hbt!]
                \centering
                \includegraphics[scale=0.8]{Resources/Unit3Vectors/polar.PNG}
                \caption*{The Polar System}
            \end{figure}

            \noindent To convert between Polar and Cartesian coordinates we use the trigonometric
            function definitions. \\
            $\cos\theta=\frac{x}{r}\implies x=r\cos\theta$ \\
            $\sin\theta=\frac{y}{r}\implies y=r\sin\theta$ \\
            $r^2=x^2+y^2$ \\
            $\tan\theta=\frac{y}{x}$ \\
            \noindent If either $x$ or $y$ is negative, we have to determine $\theta$ through
            observing which quadrant $\theta$ belongs to. We know that Quadrant I,II,III,IV refer to
            $\frac{\pi}{2}$, $\pi$, $\frac{3\pi}{2}$, and $2\pi$ respectively. \\

            \noindent \textbf{Transformations on Polar Curves}: \\
            \textbf{Rotations}: Replace the parameter $\theta$ with $(\theta-\phi)$ and the curve
            will be rotated anticlockwise $\phi$ radians. \\
            \textbf{Dilations:} Replace the parameter $r$ with $\frac{r}{s}$, where $s$ is the scale
            factor. \\
            \textbf{Reflections:} For a reflection about the line $\theta=\phi$, replace the parameter
            $\theta$ with $(2\phi-\theta)$. For a reflection about the pole, replace the parameter
            $\theta$ with $(\theta-\pi)$. \\

            \noindent \textbf{Line:} \\
            The general form is given by $\theta=\alpha$ where $\alpha$ is the angle between the line
            and the positive $x$-axis. Note that any line $\theta=\alpha+\pi k$ is the same as the
            line $\theta=\alpha$ for any integer $k$. \\

            \begin{figure} [hbt!]
                \centering
                \includegraphics[scale=0.4]{Resources/Unit3Vectors/line.PNG}
                \caption*{$\theta=\frac{\pi}{6}$}
            \end{figure}

            \noindent \textbf{Circle:} \\
            The general form is given by $r=\alpha$, where $\alpha$ is the radius of the circle. \\

            \begin{figure} [hbt!]
                \centering
                \includegraphics[scale=0.4]{Resources/Unit3Vectors/circle.PNG}
                \caption*{$r=2$}
            \end{figure}

        \pagebreak

            \noindent \textbf{Cardioid:} \\
            A heart-shaped curve with the general form $r=\alpha+\alpha\cos\theta$, where $\alpha$ is
            the radii of the circles being traced by the cardioid. \\

            \begin{figure} [hbt!]
                \centering
                \includegraphics[scale=0.4]{Resources/Unit3Vectors/cardioid.PNG}
                \caption*{$r=1+\cos\theta$}
            \end{figure}

            \noindent \textbf{Limacon:} \\
            Limacons are general forms of cardioids, formed from the path traced by any point fixed
            to a circle. The general form of a limacon is $r=a+b\cos\theta$ where $\frac{b}{a}$
            determines the shape of the limacon. If $\frac{b}{a}<1$ then the limacon will have a
            smoothed heart shape. If $\frac{b}{a}=1$ then the limacon will be a cardioid.
            If $\frac{b}{a}>1$ then the limacon will have an inner loop.

%   @TODO: Fix alignemnt of limacon images
            \begin{figure} [hbt!]
                \centering
                %%%
                \begin{subfigure}{.5\textwidth}
                    \includegraphics[width=.4\linewidth]{Resources/Unit3Vectors/limacon1.PNG}
                    \caption*{$r=1.5+\cos\theta$}
                \end{subfigure}
                %%%
                \begin{subfigure}{.5\textwidth}
                    \includegraphics[width=.4\linewidth]{Resources/Unit3Vectors/limacon2.PNG}
                    \caption*{$r=0.5+\cos\theta$}
                \end{subfigure}
            \end{figure}

            \noindent \textbf{Rose:} \\
            A rose curve is a sinusoidal curve graphed in polar coordinates. Its loops are called
            \textbf{petals}. The general form of a rose is $r=a+b\cos(k\theta)$ where $\alpha$ is
            the magnitude of each petal and $k$ is an integer that determines the number of petals.
            If $k$ is odd then the number of petals is $k$. If $k$ is even then the number of petals
            is $2k$. \\

            \begin{figure} [hbt!]
                \centering
                \includegraphics[scale=0.4]{Resources/Unit3Vectors/rose.PNG}
                \caption*{$r=\cos{3\theta}$}
            \end{figure}

            \pagebreak
            \noindent \textbf{Archimedian Spiral:} \\
            An Archimedian Spiral is a spiral-shaped curve extending infinitely outward from the pole.
            The general form is $r=a+b\theta$ where the parameter $a$ affects the initial position of
            the graph and the parameter $b$ affects the spacing of the turns of the spiral. \\

            \begin{figure} [hbt!]
                \centering
                \includegraphics[scale=0.4]{Resources/Unit3Vectors/spiral.PNG}
                \caption*{$r=\frac{\theta}{2\pi}$}
            \end{figure}

            \noindent \textbf{Lemniscate:} \\
            An lemniscate is shaped like a figure-eight and the infinity symbol, for which it is named for.
            It is the locus of points where the product of the distances to two points (loci) is a
            constant value. The general form is $r^2=a^2\cos{2\theta}$, where $a$ is the magntiude
            of one of the petals.

            \begin{figure} [hbt!]
                \centering
                \includegraphics[scale=0.4]{Resources/Unit3Vectors/lemniscate.PNG}
                \caption*{$r=\sqrt{\cos{2\theta}}$}
            \end{figure}


        \subsection{Cylindrical Coordinates}
            The Cylindrical Coordinate System is the extension of the Polar System to $\mathbb{R}^3$,
            where a point is represented by the ordered triple $(r,\theta,z)$. $(r,\theta)$ is the
            polar coordinate in $\mathbb{R}^2$ and $z$ is the usual $z-$coordinate in the Cartesian
            Coordinate System. \\

            \begin{figure} [hbt!]
                \centering
                \includegraphics[scale=0.5]{Resources/Unit3Vectors/cylindrical.png}
                \caption*{The Cylindrical Coordinate System}
            \end{figure}

            \noindent \textbf{Cylindrical $\rightarrow$ Rectangular:} \\
            $   x=r\cos\theta$ \\
            $   y=r\sin\theta$ \\
            $   z=z$ \\
            \textbf{Rectangular $\rightarrow$ Cylindrical:} \\
            $   r^2=x^2+y^2$ \\
            $   \tan\theta=\frac{y}{x}$ \\
            $   z=z$ \\

        \pagebreak
        \subsection{Spherical Coordinates}

            In the Spherical Coordinate System, a point in space, $P$, is represented by the ordered
            triple $(\rho, \theta, \phi)$, where \\
            $   \pho $ is the distance between $P$ and the origin $(\rho\not=0)$ \\
            $   \theta $ is the angle between $P$ and the reference direction \\
            $   \phi $ is the angle formed between the positive $z-$axis and the line segment
            $\overline{OP}$, where $O$ is the origin and $0\leq\phi\leq\pi$.

            \begin{figure} [hbt!]
                \centering
                \includegraphics[scale=0.6]{Resources/Unit3Vectors/relation.PNG}
                \caption*{Relation Between Spherical, Rectangular, and Cylindrical Systems}
            \end{figure}

            \noindent \textbf{Spherical $\rightarrow$ Rectangular:} \\
            $  x=\rho\sin\phi\cos\theta$ \\
            $  y=\rho\sin\phi\sin\theta$ \\
            $  z=\rho\cos\phi$ \\
            \textbf{Rectangular $\rightarrow$ Spherical:} \\
            $  \rho^2=x^2+y^2+z^2$ \\
            $  \tan\theta=\frac{y}{x}$ \\
            $  \phi=\arccos{(\frac{z}{\sqrt{x^2+y^2+z^2}})}$ \\
            \textbf{Spherical $\rightarrow$ Cylindrical:} \\
            $  r=\rho\sin\phi$ \\
            $  \theta=\theta$ \\
            $  z=\rho\cos\phi$ \\
            \textbf{Cylindrical $\rightarrow$ Spherical:} \\
            $  \rho=\sqrt{r^2+z^2}$ \\
            $  \theta=\theta$ \\
            $  \phi=\arccos{(\frac{z}{\sqrt{r^2+z^2}})}$

            %Adding captions screws up the figure alignment / need to fix
            \noindent \textbf{Common Spherical Surfaces:} \\

%       @TODO: Fix figure alignment
            \begin{figure} [hbt!]
                \centering
                \begin{subfigure}{.33\textwidth}
                    \includegraphics[scale=0.6]{Resources/Unit3Vectors/sphere.PNG}
                \end{subfigure}
                %%%
                \begin{subfigure}{.33\textwidth}
                    \includegraphics[scale=0.6]{Resources/Unit3Vectors/halfplane.PNG}
                \end{subfigure}
                %%%
                \begin{subfigure}{.33\textwidth}
                    \includegraphics[scale=0.6]{Resources/Unit3Vectors/halfcone.PNG}
                \end{subfigure}
            \end{figure}

            \noindent $(1)$ Sphere of Radius $\rho$, $\rho=c$ \\
            $(2)$ Half-Plane of Distance $\theta$, $\theta=c$ \\
            $(3)$ Half-Cone of Angle $\phi$, $\phi=0$

        \pagebreak
        \subsection{Functions Defined by Vectors}

            A \textbf{vector function} is a function that takes one or more variables and returns a
            vector. A single-variable vector function in $\mathbb{R}^2$ and $\mathbb{R}^3$ will
            have the following forms, respectively: \\

            \begin{equation*}
                \overrightarrow{r}(t)=\langle f(t), g(t)\rangle
                ,
                \overrightarrow{r}(t)=\langle f(t), g(t), h(t) \rangle
            \end{equation*}

            \noindent The domain of a vector function is the set of all $ts$ for which all the
            component functions are defined. \\

            \noindent \textit{Example 1: Determine the domain of
            $\overrightarrow{r}(t)=\langle \cos t, \ln{(4-t)}, \sqrt{t+1}\rangle$} \\
            The first component is defined for all $ts$. The second component is defined for $t<4$.
            The third component is defined for $t\geq-1$. Combining these, we get the domain $[-1,4)$. \\

            \noindent \textit{Example 2: Sketch the graph of the function
            $\overrightarrow{r}(t)=\langle t,t^3-10t+7\rangle$} \\
            Evaluating the function several times gives us \\
            $\overrightarrow{r}(-3)=\langle -3,10\rangle$ \\
            $\overrightarrow{r}(-1)=\langle -1,16 \rangle$ \\
            $\overrightarrow{r}(1)=\langle 1, -2 \rangle$ \\
            $\overrightarrow{r}(3)=\langle 3,4 \rangle$ \\

            \begin{figure} [hbt!]
                \centering
                \includegraphics[scale=0.6]{Resources/Unit3Vectors/vectorfunction.PNG}
            \end{figure}

    \pagebreak
%%%-------------------------------------------NEW SECTION--------------------------------------%%%

    \section{Matrices}

    \subsection{Fundamentals of Matrices}
        A \textbf{matrix} is an array of numbers. The \textbf{Identity Matrix} is a special matrix
        equivalent to the number 1 and of the form \\

        \noindent $I$ =

        \begin{bmatrix}
            1 & 0 & 0 \\
            0 & 1 & 0 \\
            0 & 0 & 1
        \end{bmatrix}. \\

        \noindent It has the same number of rows as columns and has 1s on the main diagonal and 0s
        everywhere elsewhere. When we multiply by $I$ the original is unchanged. \\

        \noindent The \textbf{determinant} is a number that can be calculated from a
        \textbf{square matrix} (same number of rows as columns) that gives us information regarding
        systems of linear equations and matrix inverses. The symbol for determinants are two pipes on
        either side of a matrix, much like the absolute value symbol. For example, $|A|$ is the
        determinant of $A$.

        \noindent \textbf{Addition/Subtraction:} add the values in the corresponding positions \\
        \noindent
        \begin{bmatrix}
            a & b \\
            c & d
        \end{bmatrix}
        +
        \begin{bmatrix}
            e & f \\
            g & h
        \end{bmatrix}
        =
        \begin{bmatrix}
            a+e & b+f \\
            c+g & d+h
        \end{bmatrix} \\

        \noindent \textbf{Negation:} Distribute the opposite sign \\

        \noindent -
        \begin{bmatrix}
            a & b \\
            c & d
        \end{bmatrix}
        =
        \begin{bmatrix}
            -a & -b \\
            -c & -d
        \end{bmatrix} \\

        \noindent \textbf{Constant Multiple:} Distribute the constant \\

        \noindent k
        \begin{bmatrix}
            a & b \\
            c & d
        \end{bmatrix}
        =
        \begin{bmatrix}
            ka & kb \\
            kc & kd
        \end{bmatrix} \\

        \noindent \textbf{Matrix Multiplication:} Take the dot product of the rows and columns
        of the matrices being multiplied together. When matrices are multiplied together,
        the number of columns in the $1^{st}$ matrix must equal the number of rows in the $2^{nd}$
        matrix. The resulting matrix will always have the same number of rows as the $1^{st}$ matrix
        and the same number of columns as the $2^{nd}$ matrix. \\

        \noindent \textit{Example:} \\

        \noindent
        \begin{bmatrix}
            1 & 2 & 3 \\
            4 & 5 & 6
        \end{bmatrix}
        $\times$
        \begin{bmatrix}
            7 & 8 \\
            9 & 10 \\
            11 & 12
        \end{bmatrix}
        =
        \begin{bmatrix}
            58 & 64 \\
            139 & 154
        \end{bmatrix} \\

        \noindent $1^{st}$ row and $1^{st}$ column: \\
        \begin{align*}
            (1,2,3) \bullet (7,9,11) &= 1\cdot 7 + 2\cdot 9 + 3\cdot 11 \\
            &= 58
        \end{align*}

        \noindent $1^{st}$ row and $2^{nd}$ column: \\
        \begin{align*}
            (1,2,3) \bullet (8,10,12) &= 1\cdot 8 +2\cdot 10+3\cdot 12 \\
            &= 64
        \end{align*}

        \noindent $2^{nd}$ row and $1^{st}$ column: \\
        \begin{align*}
            (4,5,6) \bullet (7,9,11) &= 4\cdot 7+5\cdot 9+6\cdot 11 \\
            &= 139
        \end{align*}

        \noindent $2^{nd}$ row and $2^{nd}$ column: \\
        \begin{align*}
            (4,5,6) \bullet (8,10,12) &= 4\cdot 8+5\cdot 10+6 \cdot 12 \\
            &= 154
        \end{align*}

    \subsection{Solving Higher-Order Systems Using Augmented Matrices}
    \subsection{The Gauss-Jordan Method}


    \pagebreak
%%%-------------------------------------------NEW SECTION--------------------------------------%%%

    \section{Complex Numbers}

        \subsection{Introduction to Complex Numbers}
            A \textbf{complex number} is any number that can be written in the form $a+bi$, where
            $a$ and $b$ are real numbers and $i$ is the imaginary unit defined by $i=\sqrt{-1}$.
            Complex numbers appear often in trigonometry and polar coordinates, making them
            particularly applicable in physics and engineering. \\

            \noindent Complex numbers can be geometrically represented by graphing them on the
            \textbf{complex plane}, where $a+bi$ is graphed just as the ordered pair $(a,b)$ would
            be graphed on Cartesian coordinates. The real axis corresponds to the $x$-axis and the
            imaginary axis corresponds to the $y$-axis.\\

            \begin{figure} [hbt!]
                \centering
                \includegraphics[scale=0.75]{Resources/Unit4Complex/complex.PNG}
                \caption*{Complex Numbers Graphed on the Complex Plane}
            \end{figure}

            \noindent \textbf{Properties of $i$:} \\
            $i=\sqrt{-1}$ \\
            $i^2=-1$ \\
            $i^3=i\cdot i^2=i(-1)=-i$ \\
            $i^4=i^2\cdot i^2=(-1)(-1)=1$ \\
            To simplify larger powers of $i$, take the last two digits of the power and divide it by
            4. Find the remainder, $k$. Then the value is $i^k$. \\
            $i^k+i^{k+1}+i^{k+2}+i^{k+3}=0$ where $k$ is an integer. (The sum of four consecutive
            powers of $i$ equals 0)


        \subsection{Complex Arithmetic and Conjugates}
            \textbf{Addition:} \\
            Given complex numbers $a+bi$ and $c+di$, their sum is \\
            \begin{equation*}
                (a+c) + (b+d)i
            \end{equation*}

            \noindent \textbf{Multiplication:} \\
            Given complex numbers $a+bi$ and $c+di$, their product is

            \begin{align*}
                (a+bi) \times (c+di) &= a(c+di)+bi(c+di) \\
                &= (ac) + (ad)i + (bc)i + (bd)i^2 \\
                &= (ac) + (ad+bc)i + (bd)(-1) \\
                &= (ac-bd) + (ad+bc)i
            \end{align*}

            \noindent \textbf{Division:} \\
            Given complex numbers $a+bi$ and $c+di$, the division of these two numbers is done by
            rationalizing the complex number or multiplying and dividing by the conjugate of the
            denominator. \\

            \noindent The \textbf{complex conjugate} of a complex number $a+bi$ is $a-bi$.
            Complex conjugates are highly useful for rationalizing denominators containing complex
            numbers. \\

            \noindent \textit{Example 1: Rationalize the denominator and write in standard form for
            $\frac{3+2i}{5-2i}$} \\

            \begin{align*}
                \frac{3+2i}{5-2i} &= \frac{(3+2i)(5+2i)}{(5-2i)(5+2i)} \\
                &= \frac{11+16i}{29} \\
                &= \frac{11}{29} + \frac{11}{29}i
            \end{align*}

            \noindent \textbf{Complex Conjugate Root Theorem:} If $a+bi$ is a root of a polynomial
            with rational coefficients, then $a-bi$ is also a root of that polynomial. \\
            \textit{Example 2:} The quadratic $x^2+bx+c$ has $1+i$ as a root, where $b$ and $c$ are
            integers. What is $b+c$? \\
            We are given that $b$ and $c$ are integers, hence the other root must be the conjugate
            of $1+i=1-i$. Writing the polynomial in factored form gives \\

            \begin{align*}
                x^2+bx+c &= (x-1-i)(x-1+i) \\
                &= x^2-2x+2
            \end{align*}

            \noindent $\implies b=-2,c=2$ \\
            $\therefore b+c=0$


        \subsection{Gaussian Integers}
            A \textbf{Gaussian Integer} is a complex number $a+bi$, where both $a$ and $b$ are integers.
            Note that Gaussian Integers are not actually integers unless the imaginary component equals 0. \\

            \noindent \textbf{Examples of Gaussian Integers:} \\
            $3+2i$ \\
            $7-8i$ \\
            $14$ \\
            $-92i$ \\
            \noindent \textbf{Examples of Non-Gaussian Integers:} \\
            $\frac{1}{2}+\frac{\sqrt{3}}{2}i$ \\
            $7-\frac{i}{3}$


        \subsection{Complex Modulus and Argument}

            The absolute value of a real number is defined as the positive distance from 0 to that
            number. The absolute value of a complex number is defined in the same way, except its
            distance oftentimes referred to as \textbf{modulus}, is measured on the complex plane.
            Since the segment between 0 and the complex number is a hypotenuse of a right triangle \\

            \noindent \textbf{Modulus of a Complex Number $a+bi$}: \\

            \begin{equation*}
                |a+bi| = \sqrt{a^2+b^2}
            \end{equation*}

            \begin{figure} [hbt!]
                \centering
                \includegraphics[scale=0.75]{Resources/Unit4Complex/complex.PNG}
            \end{figure}

            \noindent The angle that the positive real axis makes with the ray connecting 0 to a
            complex number is called the \textbf{argument} of a complex number. Using triangle
            relationships, we can determine \\

            \noindent \textbf{Argument of a Complex Number $a+bi$}: \\
            \begin{equation*}
                \tan\theta=\frac{b}{a}
            \end{equation*}

            \noindent When attempting to determine $\theta$, one should take into account which
            quadrant the complex number is located in. \\
            \textit{Example: Determine the argument of $-3+4i$} \\
            We have \\

            \begin{equation*}
                \tan\theta=-\frac{4}{3}
            \end{equation*}

            \noindent Recall that the inverse tangent function has a range of $(-\frac{\pi}{2},\frac{\pi}{2})$.
            Taking the inverse tangent will give the angle that is in $Q4.$ The complex number,
            however, is located in $Q2$, so we add $\pi$ to give the correct argument. \\

            \begin{equation*}
                \therefore \theta = \arctan{-\frac{4}{3}}+\pi
            \end{equation*}


        \subsection{Complex Roots}
            By the \textbf{Fundamental Theorem of Algebra}, every polynomial of degree $n$ has exactly
            $n$ roots, counting for multiplicity. Occasionally, these roots are complex numbers.
            For example, $x^2+1=0\implies x^2=-1\implies x=\pm i$. \\

            \noindent \textit{Example 1: Find the roots of $2x^2+1=0$} \\
            \begin{align*}
                2x^2 + 1 &= 0 \\
                x^2 &= -\frac{1}{2} \\
                x &= \pm \sqrt{-\frac{1}{2}} \\
                &= \pm \sqrt{-1} \sqrt{\frac{1}{2}} \\
                &= \pm i\sqrt{\frac{1}{2}} \\
                &= \pm \frac{i}{\sqrt{2}}
            \end{align*}

            \noindent \textit{Example 2: Factor $x^2+6x+10$} \\
            Computing the discriminant $D$, we get \\

            \begin{equation*}
                D = b^2-4ac = 6^2-4(1)(10) = -4
            \end{equation*}

            \noindent Since $D<0$, we conclude that the quadratic has a pair of complex roots with
            imaginary components. To find the roots, we use the quadratic formula. \\

            \begin{align*}
                x &= \frac{-b\pm\sqrt{D}}{2a} \\
                &= \frac{-6\pm\sqrt{-4}}{2(1)} \\
                &= \frac{-6\pm 2\sqrt{-1}}{2} \\
                &= -3 \pm i
            \end{align*}

            \noindent Factoring the quadratic, we get \\

            \begin{align*}
                x^2+6x+10 &= (x-(-3+i))(x-(-3-i)) \\
                &= (x+3-i)(x+3+i)
            \end{align*}


        \subsection{Euler's Formula}
            \textbf{Euler's Formula} allows us to express a complex number in exponential form. \\
            \noindent Given a complex number $z$ with modulus $r$ and argument $\theta$, \\
            \begin{equation*}
                z=re^{i\theta}=r(\cos\theta+i\sin\theta)
            \end{equation*}
            \noindent This is usually wrote in simpler terms as given below. \\
            \begin{equation*}
                e^{i\theta}=\cos\theta+i\sin\theta
            \end{equation*}
            \noindent Oftentimes a shorthand notation is used where the $cis$ function refers to
            $cis=\cos\theta+i\sin\theta$.

            \noindent \textit{Example 1: Express $3e^{\frac{\pi i}{2}}$ in standard form} \\
            \begin{align*}
                3e^{\frac{\pi i}{2}} &= 3(\cos{\frac{\pi}{2}}+i\sin{\frac{\pi}{2}}) \\
                &= 3i
            \end{align*}

            \noindent \textit{Example 2: Express $\frac{1}{2}-i\frac{\sqrt{3}}{2}$ in exponential form} \\
            The modulus is given by \\
            \begin{align*}
                \left| \frac{1}{2}-i\frac{\sqrt{3}}{2}\right|
                &= (\frac{1}{2})^2+(\frac{\sqrt{3}}{2})^2 \\
                &= 1
            \end{align*}
            \noindent The argument is then \\
            \begin{equation*}
                \tan\theta=-\sqrt{3} \\
            \end{equation*}
            The complex number is in $Q4$ so \\
            \begin{equation*}
                \theta &= -\frac{\pi}{3}
            \end{equation*}

            \noindent With the modulus and argument known, we can directly stick them into
            Euler's Formula, giving us \\
            \begin{equation*}
                \frac{1}{2}-i\frac{\sqrt{3}}{2} = e^{\frac{-\pi i}{3}}
            \end{equation*}


            \noindent \textbf{Euler's Identity} is a special case of Euler's Formula in which
            $r=1$ and $\theta=\pi$. It is incredibly notable because it combines several important
            mathematical constants ($0,1,\pi,e,$ and $i$) into one equation. It is given by \\

            \begin{equation*}
                e^{\pi i}+1=0
            \end{equation*}

            \noindent \textbf{Finding Trig Identities with Euler's Formula}: \\
            For example, let's find the $\cos$ and $\sin$ functions' sum formulas. \\
            \begin{align*}
                e^{i\theta} &= \cos\theta + i\sin\theta \\
                e^{i(\alpha+\beta)} &= \cos{(\alpha+\beta)}+i\sin{(\alpha+\beta)} \\
            \end{align*}
            Using exponent laws, we can rewrite the left side. \\
            \begin{align*}
                e^{i\alpha}e^{\i\beta} &= (\cos\alpha+i\sin\alpha)(\cos\beta+i\sin\beta) \\
                &= \cos\alpha\cos\beta-\sin\alpha\sin\beta+i(\cos\alpha\sin\beta+\sin\alpha\cos\beta)
            \end{align*}
            Setting equal the real and imaginary parts of the second and third equations, we get \\
            \begin{equation*}
                \cos(\alpha+\beta) &= \cos\alpha\cos\beta-\sin\alpha\sin\beta
            \end{equation*}
            \noindent and \\
            \begin{equation*}
                \sin(\alpha+\beta) &= \cos\alpha\sin\beta + \sin\alpha\cos\beta
            \end{equation*}

            \noindent Euler's Formula is also useful for writing trigonometric functions in terms
            of exponentials. Knowing that $\cos{(-\theta)}=\cos{\theta}$ and $\sin{(-\theta)}=-\sin{(\theta)}$,
            \begin{equation*}
                e^{i\theta}=\cos\theta+i\sin\theta
            \end{equation*}
            and
            \begin{equation*}
                e^{-i\theta}=\cos\theta-i\sin\theta
            \end{equation*}
            $\implies$
            \begin{equation*}
                \cos\theta=\frac{e^{i\theta}+e^{-i\theta}}{2}
            \end{equation*}
            and
            \begin{equation*}
                \sin\theta=\frac{e^{i\theta}-e^{-i\theta}}{2i}
            \end{equation*}

            \noindent We can apply these identities to Calculus, for example, we can rewrite the
            following integrand as follows. \\
            \begin{align*}
                \sin^2{x} &= \left(\frac{e^{ix}-e^{-ix}}{2i}\right)^2 \\
                &= -\frac{1}{4}(e^{i2x}+e^{-i2x}-2e^0) \\
                &= -\frac{1}{4}(2\cos{(2x)}-2)
            \end{align*}


        \subsection{De Moivre's Theorem}

            Following Euler's Formula directly, De Moivre's Theorem allows us to raise a complex number
            to any real number power. \\

            \noindent Given a complex number $z=re^{i\theta}$ and a real number $n$, \\
            \begin{equation*}
                z^n = (re^{i\theta})^n=r^n[\cos{(n\theta)}+i\sin{(n\theta)}]
            \end{equation*}

            \noindent \textit{Example: Compute $(\sqrt{2}+i\sqrt{2})^4$} in standard form \\
            The modulus of the base complex number is \\
            \begin{equation*}
                |\sqrt{2}+i\sqrt{2}| = \sqrt{(\sqrt(2))^2+(\sqrt{2})^2} = 2
            \end{equation*}



            \subsection{Roots of Unity}

            The \textbf{$n^{th}$ roots of unity} are the complex solutions to an equation of the form
            $x^n=1$, where $n$ is a positive integer. Such equations can be solved by applying
            Euler's Formula and De Moivre'S Theorem. \\

            \noindent \textit{Example 1: Find the complex solutions to the equation $x^6=1$} \\
            We have \\
            \begin{align*}
                1 &= e^{2k\pi i},k\in\mathbb{Z} \\
                x^6 &= e^{2k\pi i} \\
                x &= e^{\frac{k\pi i}{3}}
            \end{align*}
            \noindent Below are 6 values of $k$ that give distinct complex numbers.
            All other values of $k$ give arguments that are co-terminal with these solutions,
            called the \textit{$6^{th}$ roots of unity}. \\

            \begin{equation*}
                k=0: x=e^0 = 1
            \end{equation*}
            \begin{equation*}
                k=1: x=e^{\frac{\pi i}{3}} = \frac{1}{2}+i\frac{\sqrt{3}}{2}
            \end{equation*}
            \begin{equation*}
                k=2: x=e^{\frac{2\pi i}{3}} = -\frac{1}{2}+i\frac{\sqrt{3}}{2}
            \end{equation*}
            \begin{equation*}
                k=3: x=e^{\pi} = -1
            \end{equation*}
            \begin{equation*}
                k=4: x=e^{\frac{4\pi i}{3}} = -\frac{1}{2}-i\frac{\sqrt{3}}{2}
            \end{equation*}
            \begin{equation*}
                k=5: x=e^{\frac{5\pi i}{3}} = \frac{1}{2} - i\frac{\sqrt{3}}{2}
            \end{equation*}


        \subsection{Complex Numbers in Geometry}
            Because of the circular relationships associated with complex numbers, they are useful
            for many geometrical problems. For example, the rotation of a point or rigid figure
            can be performed with complex numbers much more simply than it can be done with trigonometry. \\

            \noindent To rotate a point $\theta$ radians anticlockwise about the origin, \\
            1. Convert the ordered pair to the corresponding complex number \\
            2. Multiply the complex number by $e^{i\theta}$ \\
            3. Convert the result to the corresponding ordered pair \\

            \noindent \textit{Example: Find the image of rotation when the point (2,5) is rotated
            $30^\circ$ anticlockwise about the origin} \\
            \begin{equation*}
                (2,5)\implies 2+5i
            \end{equation*}
            \noindent The angle of rotation in radians is $\frac{\pi}{6}$, giving the complex number \\
            \begin{equation*}
                e^{\frac{\pi i}{6}}=\frac{\sqrt{3}}{2}+\frac{i}{2}
            \end{equation*}
            \noindent Multiplying these complex numbers together, we obtain the image of rotation \\
            \begin{equation*}
                \left ( 2+5i\right )\left (\frac{\sqrt{3}}{2}+\frac{i}{2}\right)
                =
                \sqrt{3}-\frac{5}{2}+i \left (1+\frac{5\sqrt{3}}{2} \right )
            \end{equation*}

            \noindent The corresponding ordered pair is then
            $\left(\sqrt{3}-\frac{5}{2},1+\frac{5\sqrt{3}}{2}\right)$

    \pagebreak
%%%-------------------------------------------NEW SECTION--------------------------------------%%%

    \section{Infinity}

        \subsection{Introduction to Infinity}
            \textbf{Infinity} is the \textit{concept} of a value greater than any number.
            Infinity is \textit{not} a number. \textbf{Infinitesimal} refers to the concept of
            infinitely small. Infinity, represented by the \textbf{lemniscate symbol ($\infy$)}, is
            often used to describe the limiting behavior of some functions. For example, a function
            "approaching infinity" translates to said function growing without bound.

        \subsection{Arithmetic with Infinity}
            For any real number $a$, \\
            $a+\infty=\infty$ \\
            $a-\infty=-\infty$ \\
            $a\cdot\infty=\infty$ \\
            $-a\cdot\infty=-\infty$ \\
            $\frac{a}{\infty}=-\frac{a}{\infty}=0$ \\
            $\frac{\infty}{a}=\infty$ \\
            $\frac{\infty}{-a}=-\infty$ \\
            \\
            $0\cdot\infty=$UND \\
            $\frac{\infty}{\infty}=$UND \\
            $\infty-\infty =$UND \\


\end{document}