\section{Foundations}

    \subsection{Representations}

        Visual representations are a vital part of understanding a problem and developing a model. Reprsentations may include pictures, sketches, diagrams, graphs, and a multitude of context-specific procedures.
        To avoid cluttering a reprsentation, it is best to oversimplify a real-life situation and gradually construct less idealized models. It our oversimplified model reproduces the main features of its real-world
        counterpart, then we know we have chosen adequate essential attributes. \\

        \noindent In this respect, mathematical symbols also act as representations of more complex phrases. Take, for example, the statement below.

        \begin{quote}
            "The magnitude of the acceleration of an object is directly proportional to the magnitude of the vector sum of the forces exerted on the object and inversely proportional to the object's inertia.
            The direction of the acceleration is the same as the direction of the vector sum of the forces."
        \end{quote}

        \noindent This can be expressed concisely and more clearly with the mathematic equation,

        \begin{equation*}
            \overrightarrow{a} = \frac{\sum \overrightarrow{F}}{m}
        \end{equation*}


    \subsection{Physical quantities and units}

        \textbf{Physical quantities and their symbols}:

        \begin{center}
            \begin{tabular}{|c|c|}
                \hline
                \textbf{Physical Quantity}  & \textbf{Symbol} \\
                \hline
                length                      & $l$ \\
                \hline
                time                        & $t$ \\
                \hline
                mass                        & $m$ \\
                \hline
                speed                       & $v$ \\
                \hline
                volume                      & $V$ \\
                \hline
                energy                      & $E$ \\
                \hline
                temperature                 & $T$ \\
                \hline
            \end{tabular}
        \end{center}

        \noindent Physical quantities are expressed as the product of a number and a unit of measurement. For example, the length $l$ of an object that is $1.2 m$ long can be expressed as $l=1.2m$. The global unit system
        used in science and engineering is the \textbf{Syst\'eme International (SI)}. There are seven base units in the SI system from which all other units can be derived. \\

        \noindent \textbf{The Seven SI Base Units:}

        \begin{center}
            \begin{tabular}{|c|c|c|}
                \hline
                \textbf{Name of Unit}   & \textbf{Abbreviation} & \textbf{Physical Quantity} \\
                \hline
                meter                   & m                     & length \\
                \hline
                kilogram                & kg                    & mass \\
                \hline
                second                  & s                     & time \\
                \hline
                ampere                  & A                     & electric current \\
                \hline
                kelvin                  & K                     & thermodynamic temperature \\
                \hline
                mole                    & mol                   & amount of substance \\
                \hline
                candela                 & cd                    & luminous intensity \\
                \hline
            \end{tabular}
        \end{center}

        \noindent To work conveniently with very large or very small numbers, we modify the unit name with prefixes representing integer powers of ten, conventionally powers of ten that are multiples of 3. For example,
        a billionth of a second is denoted by 1 ns and pronounced "one nanosecond", where 1 ns $=$ $10^{-9}$s. \\

        \noindent \textbf{SI Prefixes:}

        \begin{center}
            \begin{tabular}{|c|c|c|}
                \hline
                $\bm{10^n}$ & \textbf{Prefix}   & \textbf{Abbreviation} \\
                \hline
                $10^{24}$   & yotta-            & Y \\
                \hline
                $10^{21}$   & zetta-            & Z \\
                \hline
                $10^{18}$   & exa-              & E \\
                \hline
                $10^{15}$   & peta-             & P \\
                \hline
                $10^{12}$   & tera-             & T \\
                \hline
                $10^9$      & giga-             & G \\
                \hline
                $10^6$      & mega-             & M \\
                \hline
                $10^3$      & kilo-             & k \\
                \hline
                $10^0$      & \sout{     }      & \sout{  } \\
                \hline
                $10^{-3}$   & milli-            & m \\
                \hline
                $10^{-6}$   & micro-            & $\mu$ \\
                \hline
                $10^{-9}$   & nano-             & n \\
                \hline
                $10^{-12}$  & pico-             & p \\
                \hline
                $10^{-15}$  & femto-            & f \\
                \hline
                $10^{-18}$  & atto-             & a \\
                \hline
                $10^{-21}$  & zepto-            & z \\
                \hline
                $10^{-24}$  & yocto-            & y \\
                \hline
            \end{tabular}
        \end{center}

        \noindent A \textbf{mole} is currently defined as the number of atoms in $12\times 10^{-3}$ kg of carbon-12. This number is referred to as \textbf{Avogrado's number ($N_A$)}, where the currently accepted
        measurement of Avogradro's number is

        \begin{equation*}
            N_A = 6.0221413 \times 10^{23}
        \end{equation*}

        \noindent \textbf{Density} is the physical quantity measuring how much of some substance exists in a given volume. \textbf{Number density} is the number of objects per unit volume. If there are $N$ objects in a
        volume $V$, then the number density $n$ of these objects is

        \begin{tbhtheorem}{Number Density}
            $n = \frac{N}{V}$
        \end{tbhtheorem}

        \noindent \textbf{Mass density ($\rho$)} is the amount of mass $m$ per unit volume:

        \begin{tbhtheorem}{Mass Density}
            $\rho = \frac{m}{V}$
        \end{tbhtheorem}

        \noindent The easiest way to convert measurements to other units is to write the \textbf{conversion factor} as a fraction. Then multiply what you're to express by the conversion ratio, cancelling out the units.
        Below is an example of converting 4.5 inches to millimeters:

        \begin{equation*}
            4.5 \text{ in. } = (4.5 \cancel{\text{ in. }})\left(\frac{25.4\text{ mm}}{1 \cancel{\text{ in.}}}\right) = 4.5 \times 25.4 \text{ mm } = 1.1 \times 10^2 \text{ mm}
        \end{equation*}


    \subsection{From Reality to Model}

        If the position of an object is not changing, it is said to be \textbf{at rest}. A \textbf{Position-Time Graph} describes the position of an object as a function of a unit of time. Below is a position-time graph
        for walking.

        \begin{center}
            \begin{tikzpicture}[scale= ]
                \begin{axis}[
                    axis lines = center,
                    axis equal image,
                    xlabel = {Distance from origin (mm)},
                    ylabel = {Seconds},
                    xlabel near ticks,
                    ylabel near ticks,
                    xmin = 0,
                    xmax = 20,
                    ymin = 0,
                    ymax = 15,
                    xtick = {0,5,10,15,20},
                    xticklabels = {0,5,10,15,20},
                    ytick = {5,10,15},
                    yticklabels = {5,10,15}
                ]
                \draw[blue](1,2.5) -- (7.5,12.5);
                \draw[red](7.5,12.5) -- (12.5,12.5);
                \draw[ForestGreen](12.5,12.5) -- (19,8);
                \end{axis}
            \end{tikzpicture}
        \end{center}

        \noindent Here, the blue section of the graph describes when the person is walking forwards, the red section is when the person is pausing, and the green section is when the person is walking backwards.