\section{Angular Momentum}

    \subsection{Angular Momentum}

        The \textbf{rotational kinetic energy} of an object is given by

        \[
            K_{\text{rot}} = \frac{1}{2}I\omega^2
        \]

        The \textbf{angular momentum} of an object, denoted by $L$, is

        \[
            L_{\text{script } \theta} = I \omega_{\text{script} \theta} = rmv_t
        \]

        \textit{An object moving in a straight line may still have angular momentum}. \textbf{Rotational momentum} is the radius of the circle for which the \textbf{line of action} of the momentum is a tangent. \\
        The distance $r_{\bot}$ is called the \textbf{lever arm distance}, a.k.a. the \textbf{level arm} of the momentum relative to the axis of rotation. The magnitude of the angular momentum of a particle that moves
        along a straight line is thus

        \[
            L = r_{\bot} mv \text{ (particle)}
        \]

        \textbf{Law of Conservation of Angular Momentum:} Angular momentum can be transferred from one object to another, but it cannot be created or destroyed


    \subsection{Rotational Inertia of Extended Objects}

        The concept of rotational inertia can be applied to extended rigid objects. If you imagine breaking down the object into many small segments of equal inertia $\delta m$, each segment has a different velocity
        $\vec{v}$, but all move in circles about the axis at the rotational speed $\omega$. The rotational kinetic energy of the rotating object is the sum of the kinetic energies of the segments:

        \[
            K_{\text{rot}} = \frac{1}{2}I\omega^2,
        \]

        where

        \[
            I = \lim_{\delta m_n \rightarrow 0} \sum_n \delta m_n r^2_n = \int r^2 dm \text{ (extended object)}
        \]

        \begin{figure*}[hbt!]
            \centering
            \includegraphics[scale=0.75]{Resources/rotational_coords2}
        \end{figure*}

        \blue{\textbf{Uniform 1D objects:}} \\
        Place the object along a positive $x$ axis and introduce a new quantity: \textit{inertia per unit length} $\lambda = \frac{dm}{dx}$.

        \[
            I = \lambda \int x^2 dx \text{ (uniform 1D object)}
        \]

        \blue{\textbf{Uniform 2D objects:}} \\
        Convert $dm$ to a surface area by defining an \textit{inertia per unit area} $\sigma = \frac{dm}{dA}$.

        \[
            I = \sigma \int r^2 dA \text{ (uniform 2D object)},
        \]

        \blue{\textbf{Uniform 3D objects:}} \\
        Introduce an \textit{inertia per unit volume} $\rho = \frac{dm}{dV}$.

        \[
            I = \rho \int r^2 dV \text{ (uniform 3D object)}
        \]

        The \textbf{parallel-axis theorem} describes the rotational inertia about the parallel axis:

        \[
            I = I_{\text{cm}} + md^2
        \]





