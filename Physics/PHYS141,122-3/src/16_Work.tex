\section{Impulse and Work}

    \subsection{Force displacement}

        \textbf{Work}: the change in energy of a system due to external forces, requires that a displacement be made \\
        \textbf{Force displacement:} the displacement of the point of application of the force \\


    \subsection{Positive and negative work}

        \textit{The work done by a force on a system is positive when the work force and force displacement point are in the same direction and negative when they point in opposite directions.}


    \subsection{Energy diagrams}

        Bar diagrams are extended to include work done on a system. As with free-body diagrams, the first step in drawing enerby bar diagrams is to define the system. Below are some examples of energy bar diagrams for
        work:

        \begin{figure*}[hbt!]
            \centering
            \includegraphics[]{Resources/Work}
        \end{figure*}


    \subsection{Choice of system}

        Below are some examples of how the choice of system affects the energy bar diagrams:

        \begin{figure*}[hbt!]
            \centering
            \includegraphics[]{Resources/System_work}
        \end{figure*}

        \textit{Gravitational potential energy always refers to the relative position of various parts within a system, never to the relative positions of one component of the system and its environment.} \\

        \color{blue}
        \begin{quote}
            When drawing an energy diagram, do not choose a system for which friction occurs at the boundary of the system.
        \end{quote}
        \color{black}
