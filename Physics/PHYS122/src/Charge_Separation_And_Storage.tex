\section{Charge Separation and Storage}

    \subsection{Charge Separation}  % 26.1

        Don't confuse \textit{potential difference} and \textit{electric potential energy}:
        $\bullet$ the system's electric potential energy depends on the configuration of the positive and negative charge carriers in the system \\
        $\bullet$ the potential difference between points on the rod and the fur is a measure of the electrostatic work done on a particle carrying a unit of charge (not part of the system) while moving between those
        points \\

        \textit{Positive work must be done on a system to cause a charge separation of the positive and negative charge carriers in the system. This work increases the system's electric potential energy}. \\

        \textbf{Charge separating device (charging device):} has some mechanism that moves charge carriers \textit{against} an electric field

    \subsection{Capacitors}     % 26.2

        \textbf{Capacitor:} a system for storing electric potential energy that consists of two conductors

        \begin{figure*}[hbt!]
            \centering
            \caption*{A parallel-plate capacitor}
            \includegraphics[scale = 0.75]{Resources/26.2_Parallel_Plate_Capacitor}
        \end{figure*}

        The picture on the right illustrates a simple method for charging such a capacitor. Each plate is connected by a wire to a terminal of a battery, which maintains a fixed potential difference between its terminals.
        The figure below shows what happens when the connection is made between the battery and the capacitor. If the capacitor plates are far enough away from the battery, the potential difference between the plates
        initially is zero (Figure a). Immediately after the wires are connected, there is a potential difference between the ends of each wire. This difference in potential causes electrons (which are mobile in metal) in
        the wires to flow as indicated by the arrows in Figure b. A positive charge builds up on the plate connected to the positive terminal, and a negative charge of equal magnitude builds up on the other plate. As
        electrons leave one plate and accumulate on the other, the potential difference between the plates changes. This process continues until the potential is the same at both ends of each wire - that is, when the
        potential difference between the plates is equal to that between the terminals of the battery. Because there is no longer any potential difference from one end of the wire to the other, the flow of electrons
        stops and the capacitor is said to be \textit{fully charged}. \\

        In this process, the battery has done work on the electrons and this work has now become electric potential energy stored in the capacitor. When a capacitor is not connected to anything, it is \textbf{isolated}.
        For an isolated capacitor, the \textit{quantity of charge on each plate} is fixed because the charge carriers have nowhere to go. \\

        Effect of plate separation in relation to plate area on the field of a parallel-polate capacitor

        \begin{figure*}[hbt!]
            \centering
            \includegraphics[scale = 0.75]{Resources/26.2_Separation}
        \end{figure*}

        Effect of doubling plate separation of a parallel-plate capacitor:

        \begin{figure*}[hbt!]
            \centering
            \includegraphics[scale = 0.75]{Resources/26.2_Doubling_Separation}
        \end{figure*}

        \textit{For a given potential difference between the plates of a parallel-plate capacitor, the amount of charge stored on its plates increases with increasing plate area and decreases with increasing plate
        separation.} \\

        Howver, we cannot increase the amount of charge stored on a parallel-plate capacitor indefinitely by making teh plate separation infinitesimally small. This is because if plate spacing is decreased while the
        potential difference between the capacitor plates is fixed, the charge on each plate increases and thus the magnitude of the electric field in the capacitor increases. When the electric field is about
        $3 \times 10^6$ V/m, the air molecules between the plates become \textit{ionized} and the air becomes conducting, allowing a direct transfer of charge carriers between the plates. Once such a so-called
        \textbf{electric breakdown} occurs, the capacitor loses all its stored energy in the form of a spark. \\

        The electric field at which the electrical breakdown occurs is called the \textbf{breakdown threshold}. The breakdown threshold can be raised by inserting a nonconducting material between the capacitor plates.
        This is illustrated in the images below.

        \begin{figure*}[hbt!]
            \centering
            \includegraphics[scale = 0.75]{Resources/26.2_Metal_Slab}
        \end{figure*}

        \begin{figure*}[hbt!]
            \centering
            \includegraphics[scale = 0.75]{Resources/26.2_Metal_Slab_2}
        \end{figure*}

    \subsection{Dielectrics}    % 26.3

        \textbf{Dielectric:} a nonconducting material \\
        $\bullet$ \textbf{Polar dielectric:} consists of molecules that have a permanent electric dipole moment; each molecule is electrically neutral, but the centers of its positive and negative charge distributions do
        not coincide \\
        $\bullet$ \textbf{Nonpolar dielectric:} atoms or molecules in a nonpolar dielectric have no dipole moment in the absence of an electric field

        \begin{figure*}[hbt!]
            \centering
            \caption*{Polarization of Nonpolar and Polar Molecules in an Electric Field}
            \includegraphics[scale = 0.75]{Resources/26.3_Polarization}
        \end{figure*}

        The surface charge on either side of the polarized dielectric is \textbf{bound} because the charge carriers that cause it are not free to roam around in the material. \\
        The charge on the capacitor plates is \textbf{free} because the charge carriers that cause it can move around freely.

        \begin{figure*}[hbt!]
            \centering
            \caption*{Reason Why Polarized Dielectric Exhibits Macroscopic Polarization}
            \includegraphics[scale = 0.75]{Resources/26.3_Polarization2}
        \end{figure*}

        For practical purposes, the polarization induced on a dielectric in a parallel-plate capacitor is equivalent to two thin sheets carrying opposite charges.

        \begin{figure*}[hbt!]
            \centering
            \includegraphics[scale = 0.75]{Resources/26.3_Dielectric_Sandwhich}
        \end{figure*}

        The presence of a polarized dielectric reduces the strength of the electric field between the plates of a capacitor:

        \begin{figure*}[hbt!]
            \centering
            \includegraphics[scale = 0.75]{Resources/26.3_Dielectric_Polarization}
        \end{figure*}

    \subsection{Voltaic Cells and Batteries}    % 26.4

        \textbf{Voltaic cells:} a way to generate electric potential energy \\
        $\bullet$ chemical reactions turn chemical energy into electric potential energy by accumulating electrons on one side of the cell (the negative terminal) and removing electrons from the other side (the positive
        terminal) \\
        $\bullet$ two conducting terminals, or \textbf{electrodes}, are submerged in an \textbf{electrolyte}, a solvent containing mobile ions. \\
        $\bullet$ one electrode is usually made from an oxidized metal; the oxidized metal reacts by accepting positive ions from the electrolyte and electrons from the electrode \\
        $\bullet$ the other electrode is generally metallic; it oxidizes by taking in negative ions and giving up electrons \\
        $\bullet$ Because of these reactions, a surplus of electrons builds up on the metallic terminal and a deficit of electrons builds up on the oxidized-metal terminal, causing a potential difference between the two. \\
        $\bullet$ The reactions stop when the potential difference between the electrodes reaches a certain value called the \textbf{cell potential difference}, determined by the type of chemicals in the cell \\
        \textbf{Batteries:} assemblies of voltaic cells \\
        $\bullet$ a standard 9-V alkaline battery consists of six 1.5 V cells connected together

        \begin{figure*}[hbt!]
            \centering
            \includegraphics[scale = 0.75]{Resources/26.4_Voltaic_Cell}
        \end{figure*}

        \textbf{emf (electromotive force)}: of a charge-separating device, the work per unit charge done by nonelectrostatic interactions in separating positive and negative charge carriers inside the device.

    \subsection{Capacitance}    % 26.5

        \textbf{Capacitance:} the ratio of the magnitude of the charge on one of the objects to the magnitude of the potential difference across them, represents the capacitor's capacity to store charge

        \[
            C = \frac{q}{V_{\text{cap}}}
        \]

        where $Q$ is the magnitude of the charge on each conducting object and $V_{\text{cap}}$ is the magnitude of the potential difference between the conducting objects. Because both these quantities are positive, C
        is always positive. The value of C depends on the size, shape, and separation of the conductors. \\

        Capacitance has units of \textbf{farads} (F):

        \[
            1 \text{ F } = 1 \text{ C/V}
        \]



    \subsection{Electric Field Energy and emf}  % 26.6
    \subsection{Dielectric Constant}    % 26.7
    \subsection{Gauss's Law in Dielectrics}     % 26.8