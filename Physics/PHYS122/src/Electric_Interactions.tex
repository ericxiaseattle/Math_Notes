\section{Electric Interactions}


    \subsection{Static Electricity}     % 22.1

        \blue{Examples of Static Electricity (\textbf{Electric Interactions}):} \\
        $\bullet$ Plastic wrap is attracted to anything that gets close \\
        $\bullet$ Styrofoam peanuts are attracted to you when you open a box of them \\
        $\bullet$ If you rub a balloon against a wool sweater then the balloon will then be attracted to the wall if held close to the wall \\

        Objects involved in electric interactions exert \textbf{electric forces} on each other.Electric forces are \textit{field forces}, hence the magnitude of the electric force depends on distance, decreasing as you
        increase the separation of the involved objects. This relation can be expressed mathematically by \textbf{Coulomb's Law}, given below and covered later in more detail.

        \[
            F^E \propto \frac{1}{r^2}
        \]

        where $r$ is the distance. \\

        Strips of tape just pulled out of a dispenser will repel each other. The repulsive force is great enough to keep the strips apart even when they are weighed down by paper clips:

        \begin{figure*}[hbt!]
            \centering
            \includegraphics[scale=0.75]{Resources/22.1_Tape}
        \end{figure*}

    \subsection{Electrical Charge}      % 22.2

        \textbf{Electrical charge}: the attribute objects have that are responsible for the electric interaction \\
        \textbf{Charge carrier}: any microscopic objects (e.g. electrons and ions) that carry an electrical charge \\
        \textbf{Negative charge}: the state of having a surplus of electrons, hence a lower electric potential \\
        $\bullet$ a plastic comb passed through hair several times will be negatively charged
        \textbf{Electrically neutral}: objects with equal amounts of positive and negative charge; \textit{discharged} objects fall under this term \\
        $\bullet$ When two neutral objects touch, the result is that both of the objects are no longer neutral. Without further information, we \textit{cannot tell whether positive charge has been transferred between
        objects, negative charge has been transferred between objects, or a combination of the two.} \\

        Objects that carry like charges repel each other whereas objects that carry opposite charges attract each other.

    \subsection{Mobility of Charge Carriers}        % 22.3

        Charge can be transferred from one object to another by bringing the two into contact. \\

        \textbf{Electrical insulators:} materials through which charge carriers cannot flow easily or do not flow at all \\
        $\bullet$ any charge transferred to an insulator remains near the spot at which it was deposited \\
        $\bullet$ glass, rubber, wood, plastic \\
        $\bullet$ air, particularly dry air, is an insulator, though the presence of large amounts of charge can cause charge carriers to "jump" between objects, causing sparks \\

        If a charged rod is brought into contact with an uncharged metal rod, then \textit{all} points on the surface of the metal rod interact electrically with other objects, indicating that the charge has spread out
        on the object—a phenomenon that can be observed with a \textbf{electroscope}. \\

        \textbf{Electrical conductors:} materials through which charge carriers can flow \\
        $\bullet$ the human body due to its water content
        $\bullet$ metals are the only solid materials that are conductors at room temperature \\
        $\bullet$ \textit{not pure water}, but water with minute amounts of impurities \\
        $\bullet$ earth except for the outer layer of soil

        \textbf{Grounding:} discharging objects by connecting them to Earth by a wire \\
        \textbf{Conduction:} the flow of charge through conductors \\

        Charge is an inherent property of subatomic particles, thus, all electrical charge comes in whole number multiples of the electrical charge on the electron and there is no such thing as a discharged electron. \\

        \textbf{Elementary charge:} the magnitude of the charge on the electron, designated by $e$. \\
        $\bullet$ Charge on an electron: $-e$ \\
        $\bullet$ Charge on a proton: $+e$ \\

        \textbf{Ions:} charged atoms/molecules \\
        $\bullet$ always immobile in solids but can move freely in liquids \\

        Electrons that can move freely inside some object, e.g. metal, are called \textbf{free electrons} and are responsible for the easy flow of charge through a metal. \\

        When making charged pieces of tape from neutral pieces of tape, it is important to rub vigorously or separate strips of take quickly because with friction comes the breaking of chemical bonds, and if the breaking
        of chemical bonds occurs slowly, then the electrons migrate back and no surplus charge builds up. Another point is that any two dissimilar materials become charged when brought into contact with each other. When
        they are separated rapidly, small amounts of opposite charge may be left behind on each material. \\

        \begin{theorem}{Principle of Conservation of Charge}
            Electrical charge can be created or destroyed only in identical positive-negative pairs such that the charge of a closed system always remains constant.
        \end{theorem}

        \textit{\blue{When an electron (charge $-e$ collides with a subatomic particle known as a \textit{positron} (charge $+e$), both particles are destroyed, leaving only a flash of highly energetic radiation.)}}

    \subsection{Charge Polarization}        % 22.4

        \textbf{Polarization:} any separation of charge carriers in an object \\
        $\bullet$ explains why charged and neutral have some reaction (repel/attract) \\

        The below image shows how polarization can be applied to charge neutral conducting objects in a process called \textbf{charging by induction}.

        \begin{figure*}[hbt!]
            \centering
            \includegraphics[scale=0.75]{Resources/22.4_Induction}
        \end{figure*}

    \subsection{Coulomb's Law}      % 22.5

        \textbf{Electric force (electrostatic force):} the attractive or repulsive interaction between charged bodies \\
        $\bullet$ is sometimes referred to as electrostatic force because the interaction between charged particles becomes more complicated when particles are not at rest
        \textbf{Coulomb's Law:} gives the magnitude of the electric force exerted by two charged particles separated by a distance $r_{12}$ and carrying charges $q_1$ and $q_2$ as:

        \[
            F^E_{12} = k\frac{|q_1||q_2|}{r^2_{12}}
        \]

        \textbf{Coulomb (C):} the derived SI unit for charge, defined as the quantity of electrical charge transported in one second by a current of one ampere. \\
        $\bullet$ 1 C $=6.24\cdot 10^{18}$ electrons. \\
        $\bullet$ $e = \frac{1}{6.24}\cdot 10^{18}$ C $=1.60 \cdot 10^{-19}$ C. \\

        Experientially, the value of $k$ can be found to be

        \[
            k = 9.0 \cdot 10^9 \text{ N }\cdot \text{m}^2/\text{C}^2
        \]

        The electric force is \textit{central}, meaning that its line of acceleration is along the line connecting the two interacting charged particles. We can define a unit vector pointing from particle 1 to particle 2
        as follows:

        \[
            \hat{r}_{12} = \frac{\vec{r}_2 - \vec{r}_1}{r_{12}}
        \]

        \textbf{Charge distribution:} the manner in which a collection of charge carriers is spread over a macroscopic object. \\
        $\bullet$ two charged spheres separated by a small distance have uniform charge distributions with the center of each charge distribution coinciding with the center of the sphere and so $r_{12}$ is well-defined \\
        $\bullet$ when the spheres are brought closer together, the like charge carriers repel one another and move to the far side of each sphere

    \subsection{Force Exerted By Distributions of Charge Carriers}      % 22.6

        Coulomb's Law only deals with \textit{pairs} of charged objects, and so to calculate the force exerted by an assembly of objects carrying charges $q_2, q_3, q_4,\dots$ on an object that is carrying a charge $q_1$,
        we take a vector sum of all the forces exerted on an object 1 by each of the other charged objects independently:

        \[
            \sum \vec{F}_{1}^E = k\frac{q_2 q_1}{r^2_{21}}\hat{r}_{21} + k\frac{q_3 q_1}{r^2_{31}}\hat{r}_{31} + k\frac{q_4 q_1}{r^2_{41}} + \dots
        \]

        Note that the above law is valid only for charged \textit{particles}. \\

    \subsection{Chapter Glossary}

        \textbf{Charge:} (electrical) $q$ (C) A scalar that represents the attribute responsible for electromagnetic interactions, including electric interactions. There are two types of charge: \textit{positive}
        ($q > 0$) and \textit{negative} ($q < 0$). Two objects that carry the same type of charge exert repulsive forces on each other; objects that carry different types of charge exert attractive forces on each other \\
        \textbf{Charge carrier:} Any microscopic object that carries an electrical charge \\
        \textbf{Charge distribution:} The way in which a collection of charge carriers is distributed in space \\
        \textbf{Charge polarization:} A spatial separation of the positive and negative charge carriers in an object. The polarization of neutral objects induced by the presence of exernal charged objects is responsible
        for the electric interaction between charged and neutral objects. \\
        \textbf{Charging by induction:} a method of charging a neutral object using a charged object, with no physical contact between them. \\
        \textbf{Conduction:} The flow of charge carriers through a material. \\
        \textbf{Conservation of charge:} The principle that the charge of a closed system cannot change. Thus charge can be transferred from one object to another and can be created or destroyed only in identical
        positive-negative pairs. \\
        \textbf{Coulomb} (C): The derived SI unit of charge equal to the magnitude of the charge on about $6.24 \cdot 10^{18}$ electrons. (The coulomb is defined as the quantity of electrical charge transported in one
        second by a current of one ampere, a unit defined later.) \\
        \textbf{Coulomb's Law:} The force law that gives the direction and magnitude of the electric force between two particles at rest carrying charges $q_1$ and $q_2$ separated by a distance $r_{12}$:

        \[
            \vec{F}^E_{12} = k \frac{q_1 q_2}{r^2_{12}} \hat{r}_{12}
        \]

        The constant $k$ has the value $k=9.0\cdot 10^9$ N$\cdot\text{m}^2/\text{C}^2$. \\
        \textbf{Electric force} $\vec{F}^E$ (N): The force that charge carriers (and macroscopic objects that carry a surplus electrical charge) exert on each other. The magnitude and direction of this force are given
        by Coulomb's law. \\
        \textbf{Electric interaction:} a long-ranged interaction between charged particles or objects that carry a surplus electrical charge and that are at rest relative to the observer. \\
        \textbf{Elementary charge:} The smallest observed quantity of charge, corresponding to the magnitude of the charge of the electron: $e = 1.60\cdot 10^{-19}$ C. \\
        \textbf{Grounding:} The process of electrically connecting an object to Earth ("ground"). Grounding permits the exchange of charge carriers with Earth, a huge reservoir of charge carriers. A charged, conducting
        object that is grounded will retain no surplus of either type of charge, assuming no other nearby electrical influences. \\
        \textbf{Insulator} (electrical): Any material or object through which charge cannot flow easily. \\
        \textbf{Ion:} An atom or molecule that contains unequal numbers of electrons and protons and therefore carriers a surplus charge. \\
        \textbf{Negative charge:} The type of charge acquired by a plastic comb that has been passed through hair a few times. \\
        \textbf{Neutral:} The electrical state of objects whose charge is zero. Electrically neutral macroscopic objects contain the same number of positively and negatively charged particles (protons and electrons). \\
        \textbf{Positive charge:} The type of charge acquired by hair after a plastic comb has been passed through it a few times.