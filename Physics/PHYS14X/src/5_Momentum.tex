\section{Momentum}

    \subsection{Friction}

        \textbf{Friction}: the resistance to motion that one surface or object encounters when moving over another. \\
        In the absence of friction, objects moving along a horizontal track keep moving without slowing down.


    \subsection{Inertia}

        \textbf{Inertia}: a measure of an object's tendency to resist any change in its velocity. \\
        $\bullet$ The motion of larger objects made of the same material is harder to change than the motion of smaller objects \\
        $\bullet$ The ratio of the inertias of the two carts is equal to the inverse of the ratio of their velocity changes

    \subsection{What Determines Inertia?}

        \textit{The inertia of an object is determined entirely by the type of material of which the object is made and by the amount of that material contained in the object.}


    \subsection{Systems}

        \textbf{System}: any object or group of objects that we can separate, in our minds, from the surrounding environment \\
        \textbf{Extensive quantities:} quantities whose value is proportional to the size or "extent" of the system \\
        $\bullet$ Only four values can change the value of an extensive quantity: input, output, creation, and destruction \\
        $\bullet$ If we can divide the system into a number of pieces then the sum of an extensive quantity for all the separate pieces is equal to the value of that quantity for the entire system \\
        $\bullet$ The number of trees in a park is extensive because if we divide the park into two parts and add the number of trees in each part then we obtain the number of trees in the park \\
        $\bullet$ The price per gallon of gasoline is not an extensive quantity because we can divide a tankful of gas into two parts and add the price per gallon for the two parts, thus obtaining twice the price per
        gallon for the entire tank \\

        \textbf{Intensive Quantities:} quantities that do not depend on the extent of the system \\
        \textbf{System Diagrams:} diagrams that show a system's intitial and final conditions \\

        The change in the number of trees over a certain time interval is not

        \[
            \text{change } = \text{ final tree count } - \text{ initial tree count}
        \]

        but rather

        \[
            \text{change } = \text{ input - output + creation - destruction}
        \]

        \textbf{Conserved}: any extensive quantity that cannot be created nor destroyed \\
        Hence, the value of a conserved quantity can only change by

        \[
            \text{change } = \text{ input - output}
        \]



    \subsection{Inertial Standard}

        \text{Inertia:} \\
        $\bullet$ A scalar quantity \\
        $\bullet$ Denoted by $m$ \\
        $\bullet$ Represented in kilograms \\

        Recall that the ratio of the inertias of two colliding objects is the inverse of the ratio of the magnitude of their velocity changes. Hence,

        \[
            \frac{m_u}{m_s} = -\frac{\Delta v_{s,x}}{\DElta v_{u,x}}
        \]

        therefore

        \[
            m_u = -\frac{\Delta v_{s,x}}{\Delta v_{u,x}} m_s.
        \]

        Because in the case of colliding objects, the velocities are in opposite directions, the minus sign cancels out and \textit{inertia is then always a positive quantity}. We can substitute the standard quantity
        $m_s=1$ kg into the equation to get

        \[
            m_u = -\frac{\Delta v_{s,x}}{\Delta v_{u,x}} \text{ kg}
        \]



        \subsection{Momentum}

            \textbf{Momentum}: the product of the inertia and the velocity of an object, represented by $\overrightarrow{p}$ and the SI units "kg $\cdot$ m/s":

            \[
                \overrightarrow{p} = m\overrightarrow{v}
            \]

            The $x$ component of the momentum is the product of the inertia and the $x$ component of the velocity:

            \[
                p_x = mv_x
            \]

            Inertia is an intrinsic property of an object (inertia cannot be changed without changing the object) but the value of the momentum of an object can change. With the definition of momentum, we have also
            the equation below, which describes how the change in momentum for an object is always the negative of the change in momentum for another object in a collision.

            \[
                \Delta p_{u,x} + \Delta p_{s,x} = 0
            \]

            This can also be written in vector form:

            \[
                \Delta \overrightarrow{p}_u + \Delta \overrightarrow{p}_s = \overrightarrow{0}
            \]



        \subsection{Isolated Systems}

            The momentum of a system of two moving carts is the sum of the momenta of the two individual carts:

            \[
                \overrightarrow{p} = \overrightarrow{p}_1 + \overrightarrow{p}_2.
            \]

            \textbf{Interaction:} two objects acting on each other in such a way that one or both are accelerated \\
            \textbf{External Interactions:} interactions across the boundary of a system \\
            \textbf{Internal Interactions:} interactions between two objects inside the system \\
            \textbf{Isolated System:} a system for which there are no external interactions \\

