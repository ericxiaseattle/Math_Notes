\section{Circular Motion}

    \subsection{Circular Motion at Constant Speed}

        \textbf{Translational motion} involves no change in an object's orientation; i.e. all the particles in the object move along identical parallel trajectories. \\
        \textbf{Rotational Motion} involves changes in an object's orientation where particles in an object follow different circular paths centered on a straight line called the \textbf{axis of rotation}. \\

        \textbf{Revolve}: to move in circular motion around an \textit{external} center. \\
        \textbf{Rotate}: to move in circular motion around an \textit{internal} center. \\

        \textit{The instantaneous velocity $\vec{v}$ of an object in circular motion is always perpendicular to the object's position $\vec{r}$ measured from the center of the circular trajectory.} Directions of objects
        moving in circular motion are described in \textbf{rotational coordinates}, unitless numbers that increase by $2\pi$ for each circle completed by the revolving object. Rotational coordinates are denoted by
        a cursive script of the Greek letter theta. The \textbf{rotational velocity} is denoted by $\vec{omega}$. The time interval it takes an object in circular motion at constant speed to complete one revolution is
        called the \textbf{period}, denoted by the letter $T$. \\

        \textit{An object executing circular motion at constant speed has an acceleration of constant magnitude that is directed toward the center of its circular path.} This acceleration is known as
        \textbf{centripetal acceleration}. The \textbf{radial axis}, denoted by $r$, points in the direction of the radius of the circular trajectory, away from the axis of rotation. The \textbf{tangential axis},
        denoted by $t$, points in the direction of increasing [script $\theta$], tangent to the trajectory. Since the tangent to a circle is always perpendicular to the radius drawn to the point of contact of the
        tangent, the $r$ and $t$ axes are always perpendicular to each other. There is a third axis denoted by $z$ that describes the direction of forces exerted on the object. 

        \begin{figure*}[hbt!]
            \centering
            \includegraphics[scale=0.5]{Resources/rotational_coords}
        \end{figure*}

    \subsection{Forces and Circular Motion}

        \textit{An object that executes circular motion at constant speed is subject to a force (or a vector sum of forces) of constant magnitude directed toward the center of the circular trajectory.} \\

        \textit{Avoid analyzing forces from a rotating frame of reference because such a frame is accelerating and therefore noninertial.} \\

        \textit{The inward force required to make an object move in circular motion increases with increasing speed and decreases with increasing radius.}