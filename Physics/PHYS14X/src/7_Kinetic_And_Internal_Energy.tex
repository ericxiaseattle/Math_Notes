\section{Kinetic and Internal Energy}

    \subsection{Classification of Collisions}


        \textbf{Relative velocity:}
        \begin{align*}
            \overrightarrow{v}_{12} = \overrightarrow{v}_2 - \overrightarrow{v}_1 \text{ is velocity of cart 2 relative to cart 1} \\
            \overrightarrow{v}_{21} = \overrightarrow{v}_1 - \overrightarrow{v}_2 \text{ is velocity of cart 1 relative to cart 2}
        \end{align*}

        $v_{12}$ is velocity of cart 2 and $v_{21}$ is velocity of cart 2. \textbf{Relative speed} is the magnitude of the relative velocity.

        \textbf{Elastic Collision}: a collision in which the relative speed before the collision is the same as the relative speed after the collision \\
        $\bullet$ collisions between hard objects are usually elastic \\

        \textbf{Inelastic Collision:} a collision for which the relative speed after the collision is lower than that before the collision \\
        \textbf{Totally Inelastic Collision:} a special case of an inelastic collision in which the two objects move together after the collision so that their relative speed is reduced to zero \\



    \subsection{Kinetic Energy}

        \textbf{Kinetic energy}: energy associated with a single object's motion, extensive property (depend on amount of matter being measured), given by

        \[
            K = \frac{1}{2} mv^2
        \]

        \textit{In a typical elastic collision, the sum of the kinetic energies of the objects before the collision is the same as the sum of the kinetic energies after the collision.} \\

        Because kinetic energy is a scalar extensive quantity, bar diagrams are a good way to visually represent changes in this quantity:

        \begin{figure*}[hbt!]
            \centering
            \includegraphics[]{Resources/Kinetic_Energy}
        \end{figure*}



    \subsection{Internal Energy}

        \textbf{State}: the condition of an object as specified by some complete set of physical parameters: shape temperature, whatever - \textit{every possible physical variable that defines the object}. \\
        In inelastic collisions, objects deform and heat up. A \textbf{process} is a transformation of a system from an initial state to a final state. Inelastic collisions are \textbf{irreversible processes}.
        Elastic collisions are \textbf{reversible processes}. \\

        The energy of a system is given by the sum of the \textit{kinetic} and \textit{internal} energies:

        \[
            E = E_K + E_{int}
        \]

        \begin{figure*}[hbt!]
            \centering
            \includegraphics[scale=0.5]{Resources/Collisions}
        \end{figure*}

        \textit{In any inelastic collision, the states of the colliding objects change and the sum of their internal energies increases by an amount equal to the decrease in the sum of their kinetic energies. The energy
        of a system of two colliding objects does not change during the collision.} \\

        \textbf{Law of Conservation of Energy:} Energy can be transferred from one object to another or converted from one form to another, but it cannot be destroyed or created.

