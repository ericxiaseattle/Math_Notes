\section{Gravitational Potential Energy}

    \subsection{Interaction Range}

        Though the mechanics of interactions are poorly understood and nobody can answer "How do objects interact?", we have determined that each type of interaction only occurs between certain kinds of objects. So we
        assign these objects certain \textbf{attributes} like magnetic strength and color that are responsible for the interaction. It is important to acknowledge that these attributes are defined by interactions. \\

        Consider a magnet. Is it a magnet because it behaves like one, or does it behave like a magnet because it \textit{is} a magnet? Because of this confusion, we must \textit{define} the properties of matter
        according to the interactions the matter takes part in. \\

        \textbf{Physical contact:} on the macroscopic level, this means two objects are touching each other. On the atomic level, two atoms attract/repel each other even when separated by distances several times their
        size and hence physical contact has no meaning on the atomic level. \\
        \textbf{Strength:} a function of the distance separating the objects in the sense of the magnitude of the acceleration caused by the interaction \\
        \textbf{Interaction range:} the distance over which the interaction is appreciable \\
        $\bullet$ The \textbf{field} is a model often used to picture long-range interactions




    \subsection{Fundamental Interactions}

        In an alternative model, interactions are explained in terms of an exchange of fundamental particles called \textbf{gauge particles}. An interaction is \textbf{fundamental} if it cannot be explained in terms of
        other interactions. The interaction between two colliding carts is not fundamental because it can be explained as the result of the interactions between the atoms that make up the carts. Nor are the atomic
        interactions fundamental because they are the result of the interactions between the electrons and atomic nuclei that make up the atoms. And inside each atomic nucleus we observe yet another interaction
        responsible for holding the components of the nucleus together. This last interaction, the \textit{strong interaction} is today classified as fundamental. \\

        \blue{\textbf{Fundamental Interactions:}}
        \begin{center}
            \begin{tabular}{|c|c|c|c|c|c|}
                \hline
                \textbf{Type}   & \textbf{Required Attribute}   & \textbf{Relative Strength}    & \textbf{Range}    & \textbf{Gauge Particle}   & \textbf{Propagation Speed} \\
                \hline
                Gravitational   & Mass                          & 1                             & $\infty$          & graviton?                 & $c$? \\
                \hline
                Weak            & Weak Charge                   & $10^{25}$                     & $10^{-18}$m       & vector bosons             & varies \\
                \hline
                Electromagnetic & Electrical Charge             & $10^{36}$                     & $\infty$          & photon                    & $c$ \\
                \hline
                Strong          & Color Charge                  & $10^{38}$                     & $10^{-15}$m       & gluon                     & $c$ \\
                \hline
            \end{tabular}
        \end{center}

        \textbf{Gravitational Interactions:} \\
        $\bullet$ a long-range interaction manifested as an attraction between all objects that have \textit{mass} \\
        $\bullet$ meditated by a still undetected gauge particle called the \textit{graviton} \\
        $\bullet$ the weakest fundamental interaction, only for very massive objects does this interaction become appreciable \\

        \textbf{Electromagnetic Interactions:} \\
        $\bullet$ is responsible for most of what happens around us \\
        $\bullet$ is responsible for the structure of atoms and molecules, for all chemical and biological processes, for the cohesion of matter into liquids and solids, for the repulsive interaction between objects,
        as well as light and other electromagnetic radiation \\
        $\bullet$ the attribute of matter responsible for this interaction is called \textbf{electrical charge}, which is either \textit{negative} or \textit{positive} \\
        $\bullet$ the gauge particle associated with this interaction is the \textbf{photon} \\
        $\bullet$ is most noticeable at the atomic scale \\
        $\bullet$ \textbf{residual electromagnetic interaction:} the magnetic interaction left over from the incomplete canceling of the magnetic effect \\

        \textbf{Weak Interaction:} \\
        $\bullet$ a repulsive interaction responsible for some radioactive decay processes and for the conversion of hydrogen to helium in stars \\
        $\bullet$ acts inside the nucleus between subatomic particles that posses an attribute called \textbf{weak charge} and causes most subatomic particles to be unstable \\
        $\bullet$ \textbf{vector bosons:} the gauge particles associated with the weak interaction \\
        $\bullet$ \textbf{electroweak interactions;} the manifestation of both the electromagnetic and weak interactions \\

        \textbf{Strong Interaction:} \\
        $\bullet$ acts between \textbf{quarks}, the building blocks of protons, neutrons, and other particles \\
        $\bullet$ is so strong that it overwhelms all other interactions \\
        $\bullet$ the \textbf{residual strong interaction} is responsible for holding the nucleus of an atom together

    \pagebreak
    \subsection{Interactions and Accelerations}

        The ratio of the $x$-components of the accelerations of two interacting objects of constant inertia is equal to the negative inverse of the ratio of their inertias:

        \[
            \frac{a_{1x}}{a_{2x}} = - \frac{m_2}{m_1}
        \]



    \subsection{Nondissipative Interactions}

        The energy of any \textit{closed system} remains unchanged such that

        \[
            \Delta x = \Delta K + \Delta U + \Delta E_s + \Delta E_{th} = 0 \text{ (closed system)}
        \]

        For nondissipative interactions, there are no changes in either the source energy or the thermal energy of the system ($\Delta E_s=0$ and $\Delta E_{th}=0$). Thus the only energy conversions allowed in
        nondissipative interactions are reversible transformations between kinetic energy $K$ and potential energy $U$:

        \[
            \Delta E = \Delta K + \Delta U = 0 \text{ (closed system, nondissipative interaction)}
        \]

        If we introduce the \textbf{mechanical energy of a system}

        \[
            \bm{E_{\text{mech}}} = K + U,
        \]

        we can rewrite the energy conservation as

        \[
            \Delta E_{\text{mech}} = 0 \text{ (closed system, nondissipative interaction)}
        \]

        \blue{\textit{The parts of any closed system always tend to accelerate in the direction that lowers the system's potential energy.}}


    \subsection{Potential Energy Near Earth's Surface}

        Because a falling object and the Earth constitute a closed system and because the gravitational interaction is nondissipative, we can relate the change in the ball's kinetic energy to a change in the
        gravitational potential energy $U^G$:

        \[
            \Delta U^G + \Delta K_b = 0,
        \]

        where $\Delta U^G$ is the change in the gravitational potential energy of the Earth-ball system and $\Delta K_b$ is the change in the ball's kinetic energy. \\

        Because all objects falling near Earth's surface the same acceleration, if the vertical coordinate of any object of inertia $m$ changes by $\Delta x$, the change in the gravitational potential energy of the
        Earth-object system is

        \[
            \Delta U^G = mg\Delta x \text{ (near Earth's surface)}
        \]

        or as a function of position,

        \[
            U^G (x)  = mgx \text{ (near Earth's surface)}
        \]


