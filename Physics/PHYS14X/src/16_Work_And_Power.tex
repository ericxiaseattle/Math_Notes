\section{Impulse and Work}

    \subsection{Force displacement}

        \textbf{Work}: the change in energy of a system due to external forces, requires that a displacement be made \\
        \textbf{Force displacement:} the displacement of the point of application of the force \\


    \subsection{Positive and negative work}

        \textit{The work done by a force on a system is positive when the work force and force displacement point are in the same direction and negative when they point in opposite directions.}


    \subsection{Energy diagrams}

        Bar diagrams are extended to include work done on a system. As with free-body diagrams, the first step in drawing enerby bar diagrams is to define the system. Below are some examples of energy bar diagrams for
        work:

        \begin{figure*}[hbt!]
            \centering
            \includegraphics[]{Resources/Work}
        \end{figure*}


    \subsection{Choice of system}

        Below are some examples of how the choice of system affects the energy bar diagrams:

        \begin{figure*}[hbt!]
            \centering
            \includegraphics[]{Resources/System_work}
        \end{figure*}

        \textit{Gravitational potential energy always refers to the relative position of various parts within a system, never to the relative positions of one component of the system and its environment.} \\

        \color{blue}
        \begin{quote}
            When drawing an energy diagram, do not choose a system for which friction occurs at the boundary of the system.
        \end{quote}
        \color{black}

    \subsection{Work Done on a Single Particle}

        \textbf{Energy Law:}

        \[
            \Delta E = W
        \]

        It follows then that work is zero in a closed system. \\

        A \textbf{particle} is any object that has no internal structure and no extent in space. Because it doesn't have extent and internal structure, a particle's internal energy is fixed and so

        \[
            \Delta E = \Delta K \text{ (particle)}
        \]

        For motion in one dimension, we have the \textbf{work equation:}

        \[
            W = \sum F_x \Delta x_F \text{ (constant force exerted on particle, one dimension)}
        \]

        \begin{figure*}[hbt!]
            \centering
            \includegraphics[]{Resources/Energy_Momentum_Changes}
        \end{figure*}

        \begin{figure*}[hbt!]
            \centering
            \includegraphics[]{Resources/Kinetic_Energy_Falling_Ball}
        \end{figure*}

    \subsection{Work Done on a Many-Particle System}

        Because

        \[
            \Delta K_{\text{cm}} = ma_{\text{cm x}}\Delta x_{\text{cm}} = \left(\sum F_{\text{ext x}}\right)\Delta x_{\text{cm}} \text{ (constant forces, one dimension)}
        \]

        it follows that for a many-particle system,

        \[
            \Delta K_{\text{cm}} \not = W \text{ (many-particle system)}
        \]

        Thus the \textbf{Work equation for a two-particle system where an external force is exerted on one of the particles}:

        \[
            W \sum_n \left(F_{\text{ext nx}} \Delta x_{Fn}\right) \text{ (constant nondissipative forces, one dimension)}
        \]

        where $x_F$ is the $$-component of the displacement of the \textit{point the application of the force}, not the centre of mass.

        \begin{figure*}[hbt!]
            \centering
            \includegraphics[]{Resources/Equation_Of_Motion}
        \end{figure*}



    \subsection{Variable and Distributed Forces}

        For a nondissipative force in one dimension,

        \[
            W = \int_{x_i}^{x_f} F_x (x)dx \text{ (nondissipative force, one dimension)}
        \]

        The change in the \textbf{potential energy of a spring} as its free end is displaced from its relaxed position $x_0$ to any position $x$ is

        \[
            \Delta U_{\text{spring}} = \frac{1}{2}k(x-x_0)^2
        \]

        The $\Delta E_{\text{th}}$ for a block sliding across a table then coming to rest is equal to $K_i$ because $K_f$ is zero, so

        \[
            \Delta K_{\text{cm}} = F^f_{\text{sb x}} \Delta x_{\text{cm}}
        \]

        and

        \[
            \Delta E_{\text{th}} = -F^f_{\text{sb x}} \Delta x_{\text{cm}}  = \F^f_{\text{sb}} d_{\text{path}} \text{ (constant dissipative force, one dimension)}
        \]



    \subsection{Power}


        \textbf{Power:} the rate at which energy is transferred or converted, measured in \textbf{watts (W)} which are equal to 1 J/s

        \[
            P_{\text{avg}} = \frac{\Delta E}{\Delta t}
        \]

        The instantaneous power is

        \[
            P = \lim_{\Delta t \to 0} \frac{\Delta E}{\Delta t} = \frac{dE}{dt}
        \]

        and the power at which thermal energy is generated is given by

        \[
            P = \frac{dE_{\text{th}}}{dt}
        \]

        It follows then that the instantaneous power in one dimension is

        \[
            P - F_{\text{ext x}} v_x \text{ (one dimension)}
        \]

