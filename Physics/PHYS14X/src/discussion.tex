Something I found really interesting while reading this chapter was that, when a projectile is at the top of its trajectory, though its velocity is zero, its acceleration has a nonzero value.
This is because acceleration does not depend on instantaneous velocity. I personally found the verbal explanation to be confusing, and only understood after visually representing it. Consider the graph below:

\begin{center}
    \begin{tikzpicture}
        \begin{axis}[
            axis lines = middle,
            axis equal image,
            ylabel = {$v_x$ (m/s)},
            ylabel style = {above left},
            xlabel = {$t$(s)},
            xmin = 0,
            xmax = 8,
            ymin = -2,
            ymax = 2,
            ticks = none
        ]
        \draw[red] (4,0);
        \node[fill=red, draw, circle, minimum width=4pt, inner sep=0pt, pin={[fill=white, outer sep=1pt]60:{\color{red}Peak of Trajectory}\color{black}}] at (4,0) {};
        \addplot[
            color = blue,
            domain = 0:8
        ]
        {-0.5*x+2};
        \end{axis}
    \end{tikzpicture}
\end{center}

We know acceleration to be defined as the gradient of the velocity curve:

\[
    a = \frac{dv}{dt}
\]

Since the $v_x (t)$ curve is linear for scenario of projectile motions, $\frac{dv}{dt}$ is constant for the entire domain of our $v_x (t)$ curve and it happens to be -9.81 m/$\text{s}^2$. This tells us that not only
is the magnitude of acceleration nonzero throughout the entire trajectory, including the peak, but the acceleration is always in the direction of decreasing $x$.