\section{Relativity}

    \subsection{Relativity of Motion}

        \textbf{Observer:} the person doing the measuring in a discussion of motion: \\
        $\bullet$ The velocity measured for an object depends on the motion of the observer \\
        \textbf{Reference Frame:} the collective term for the axis and origin \\
        \textbf{Earth Reference Frame:} a reference frame at rest relative to the surface of Earth \\

        If an object moves at constant velocity in the Earth reference frame, its motion observed from \textit{any reference frame moving at constant velocity relative to the Earth} is also at constant velocity.

    \subsection{Inertial Reference Frames}

        \textbf{Inertial Reference Frame:} any reference frame moving at constant velocity relative to Earth \\
        $\bullet$ every inertial reference frame satisfies the \textbf{Law of Inertia}:

        \begin{tbhtheorem}[colback=red!10,colframe=red]{Law of Inertia}
            "In an inertial reference frame, any isolated object that is at rest remains at rest, and any isolated object in motion keeps moving at a constant velocity.
        \end{tbhtheorem}

        Example of Inertial Reference Frames:

        \begin{figure*}[hbt!]
            \centering
            \includegraphics[]{Resources/Reference_Frames}
        \end{figure*}

        \textbf{Non-inertial Reference Frame:} any reference frame that is not inertial



    \subsection{Principle of Relativity}

        \textit{The kinetic energy of a system of two elastically colliding objects does not change in any inertial reference frame.}  \\

        \begin{tbhtheorem}{Principle of Relativity}
            The laws of the universe are the same in all inertial reference frames moving at constant velocity relative to each other
        \end{tbhtheorem}

        To determine your velocity relative to Earth's surface, you need to take measurements in two reference frames. \textit{A consequence of the principle of relativity is that it is not possible to deduce from
        measurements taken entirely in one reference frame the motion of that reference frame relative to other reference frames.}
