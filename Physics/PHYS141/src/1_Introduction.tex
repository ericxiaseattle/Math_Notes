\section{Course Structure}

    Before each lecture: \\
    $\bullet$ Do the assigned reading \\
    $\bullet$ Complete the lecture reading discussion \\
    $\bullet$ Make notes about questions you have \\

    \noindent During Lecture: \\
    $\bullet$ Professor will assume I have done reading and covered selected points only \\
    $\bullet$  There will be frequent "Clicker questions" via LearningCatalytics \\
    $\bullet$ Ask questions in the Chat when I am uncertain/confused/curious \\

    \noindent After Lecture: \\
    $\bullet$ Notes & lecture recordings will be posted \\
    $\bullet$ Don't rely on these as a substitute for attending class \\

    \noindent Course Overview: \\
    $\bullet$ Lectures, labs, and tutorials all require \textit{interaction} \\
    $\bullet$ All components involve constructing and applying \textit{models} \\
    $\bullet$ All components are mutually reinforcing, but: \\
    - Note every topic is covered in every component \\
    - Labs generally emphasize observations, tutorials emphasize concepts, homework emphasizes applying models \\
    $\bullet$ Homework is not just for practice \\
    $\bullet$ Usually a topic is introduced in lecture first, but there are exceptions \\

    \noindent Difference between honors and normal physics: \\
    $\bullet$ Honors has live lectures, normal is prerecorded \\
    $\bullet$ The topics: \\
    - Cover some material more quickly (e.g., motion in 1D) \\
    - Cover some topics in much more depth (e.g., special relativity) \\
    - Deviate more from the textbook \\
    $\bullet$ The grade distribution \\
    $\bullet$ The students in honors physics have a stronger physics background and are more likely to consider a physics major/minor \\
    $\bullet$ Students in honors are more inquisitive, like to talk about important ideas, enjoy friendly competition, are great at collaborating \\

    \noindent Major Differences with physics courses you may have taken before: \\
    $\bullet$ Everything is in 1D first (e.g. no projectile motion until Chapter 10) \\
    $\bullet$ Conservation laws are emphasized (Momentum and energy are covered before forces and Newton's laws) \\
    $\bullet$ Inductive reasoning is emphasized (Figuring out how to explain observations, not just figuring out how to apply known laws) \\
    $\bullet$ Every chapter covers concepts first, then quantitative tools \\

    \noindent In-class quizzes aka "clicker" questions: \\
    $\bullet$ Questions asked a few times each class \\
    $\bullet$ Change your answer as often as you want \\
    $\bullet$ Discuss in the zoom chat if you want\\
    $\bullet$ 1 point for any answer (correct or not) \\
    $\bullet$ If you get 80\% of the possible points, you get a perfect class participation score