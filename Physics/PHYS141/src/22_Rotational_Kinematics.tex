\section{Rotational Kinematics}

    \subsection{Rotational Inertia}

        \textbf{Rotational inertia} is an object's tendency to resist a change in rotational velocity \\
        $\bullet$ not simply given by the object's inertia \\
        $\bullet$ denoted by $I$ \\
        $\bullet$ also known as the \textbf{moment of inertia}

        For a single body of mass $m$, rotating at radius $r$ from the axis of rotation the rotational inertia is

        \[
            I = mr^2
        \]

    \subsection{Rotational Kinematics}

        The \textbf{rotational coordinate} of an object moving along a circle of radius $r$ is defined as the length of the arc $s$ over which the object has moved divided by the radius:

        \[
            [\text{script } \theta] = \frac{s}{r}
        \]

        The \textbf{rotational velocity} $\omega_{\text{script }\theta}$ is obtained by letting the time interval during which the change in the rotational coordinate is measured approach zero:

        \[
            \omega_{\text{script }\theta} = \lim_{\Delta t\rightarrow 0} \frac{\Delta \text{ script } \theta}{\Delta t} = \frac{d\text{ script }\theta}{dt}
        \]

        \textbf{Centripetal acceleration} is given by

        \[
            a_c = \frac{v^2}{r} \text{ (circular motion)}
        \]

        When an object's speed is not constant, we have a \textbf{radial component of acceleration} and a \text{tangential component of acceleration}, where

        \[
            a_r = -\frac{v^2}{r} \text{ (any motion along arc of radius $r$)}
        \]

        and

        \[
            a_t = \frac{dv}{dt}
        \]

        The \textbf{period} $T$ (the time interval it takes the object to complete one revolution) is given by

        \[
            T = \frac{2\pi r}{v} = \frac{2\pi}{\omega} \text{ (circular motion at constant speed)}
        \]

        \begin{figure*}[hbt!]
            \centering
            \includegraphics[scale=0.5]{Resources/translational_vs_rotational}
        \end{figure*}