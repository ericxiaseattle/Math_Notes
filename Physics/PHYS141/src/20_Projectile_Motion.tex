\section{Projectile Motion}

    \subsection{Projectile Motion in Two Dimensions}

        The concept of decomposing vectors into components allows us to separate motion in a plane into two one-dimensional problems. Hence, for a projectile launched with initial velocity $\vec{v}_i$ from point
        $\left(x_i, y_i\right)$, we have

        \[
            x_f = x_i + v_{x,i} \Delta t \text{ (constant velocity)}
        \]

        and

        \[
            y_f = y_i + v_{y,i} \Delta t - \frac{1}{2} g\left(\Delta t\right)^2
        \]



    \subsection{Collisions and Momentum in Two Dimensions}

        The momentum equations for one dimension are the same for two dimensions. For a collision in an isolated system of two objects of inertia $m_1$ and $m_2$, we have

        \[
            \Delta p_x = \Delta p_{1x} + \Delta p_{2x} = m_1(v_{1x,f} - v_{1x,i}) + m_2(v_{2x,f} - v_{2x,i}) = 0
        \]

        and

        \[
            \Delta p_y = \Delta p_{1y} + \Delta p_{2y} = m_1 (v_{1y,f} - v_{1y,i}) + m_2 (v_{2y,f} - v_{2y,i}) = 0
        \]





    \subsection{Work as the Product of Two Vectors}

        For an object on an inclined plane, the weight can be split into 2 components: \\
        $\bullet$ $mg\sin{\theta}$, which acts parallel to the plane and downhill \\
        $\bullet$ $mg\cos{\theta}$, which acts perpendicular and into the plane \\

        It follows that the magnitude of the downward acceleration of an object on an inclined plane is given by

        \[
            a_x = \frac{\sum F_x}{m} = \frac{F^G_{Eb x}}{m} = \frac{+mg\sin{\theta}}{m} = +g\sin{\theta}
        \]

        In order for a block on an inclined plane to drop a vertical distance $h$, its displacement along the incline must be

        \[
            \Delta x = +\frac{h}{\sin{\theta}}
        \]

        The time the block takes to cover this distance is obtained from the following equation:

        \[
            \Delta x = v_{x,i} \Delta t + \frac{1}{2}a_x \left(\Delta t\right)^2 = 0 + \frac{1}{2}g\sin{\theta}\left(\Delta t\right)^2
        \]

        Simplifying and rearranging, it follows that

        \[
            \left(\Delta t\right)^2 = \frac{2h}{g\sin^2{\theta}}
        \]

        Using the formula for kinetic energy and the work-energy theorem, we have that

        \[
            W = \Delta K = K_f - K_i = \frac{1}{2}mv^2_{x,f} - 0 = mgh
        \]

        This can be stated more compactly by definining a \textbf{scalar product} of two vectors, where $\phi$ is the angle between them:

        \[
            \vec{A} \cdot \vec{B} = AB\cos{\phi}
        \]

        Using this definition of a scalar product, work can be written as a scalar product like so:

        \[
            W = \vec{F} \cdot \Delta \vec{r}_F \text{ (constant nondissipative force)}
        \]

        Since $\cos{90} = 0$, for a block sliding down an incline, the normal force does no work and the work done by the force of gravity on the block is

        \[
            W = \vec{F}^G_{Eb} \cdot \Delta \vec{r}_F = (mg)\left(\frac{h}{\sin{\theta}}\right)\cos{\phi},
        \]

        where $\theta$ is the angle of the incline and $\phi$ is the angle between $\vec{F}^G_{Eb}$ and $\Delta \vec{r}_F$.

        Additionally,

        \[
            W = \int_{\vec{r}_i}^{\vec{r}_f} \vec{F}\left(\vec{r}\right)\cdot d\vec{r} \text{ (variable nondissipative force)}
        \]

        and

        \[
            \Delta E_{\text{th}} = -\int_{\vec{r}_i}^{\vec{r}_f} \vec{F} \left(\vec{r}_{\text{cm}}\right)\cdot d\vec{r}_{\text{cm}} \text{ (variable dissipative force)}
        \]

\pagebreak

    \subsection{Coefficients of Friction}

        \textit{The maximum force of static friction exerted by a surface on an object is proportional to the force with which the object presses on the surface and does not depend on the contact area.} For any two
        surfaces 1 and 2, we can write

        \[
            \left(F^s_{12}\right)_{\text{max}} = \mu_s F^n_{12},
        \]

        where $\mu_s$ is the unitless proportionality constant called the \textbf{coefficient of static friction}. We have

        \[
            \mu_s = \frac{F^G_{Eb x}}{F^G_{Eb y}}
        \]

        and

        \[
            \mu_s = \tan{\theta_{\text{max}}}
        \]

        In any given situation in which the static force is present but has not reached its maximum value is that the magnitude of the force of static friction exerted by surface 1 on surface 2 must obey the condition

        \[
            F^s_{12} \lewq \mu_s F^n_{12}
        \]

        From experimental data, we have that

        \[
            F^k_{12} = \mu_k F^n_{12},
        \]

        where $\mu_k$ is called the \textbf{coefficient of kinetic friction}. The coefficient of kinetic friction is always smaller than the coefficient of static friction because otherwise an object would never slip
        because it would stop slipping the instant it starts slipping.

