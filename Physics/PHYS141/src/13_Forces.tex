\section{Forces}

    \subsection{Dissipative Interactions}

        \textbf{Dissipative Interaction:} one in which there are changes in thermal energy. \\
        $\bullet$ are \textit{irreversible} \\

        Because the sum of the change in kinetic and thermal energies in a dissipative interaction must add up to 0, we have that

        \[
            \Delta K = - \Delta E_{\text{th}}
        \]

        Since no energy is dissipated in elastic collisions, i.e. $\Delta E_{\text{th}}=0$, we have that

        \[
            \Delta K = 0 \text{ (elastic collision)}
        \]

        For inelastic collisions,

        \[
            \Delta E_{\text{th}} = -\Delta K = \frac{1}{2}\mu v^2_{12i} (1-e^2),
        \]

        where the reduced inertia $\mu$ is

        \[
            \frac{m_1 m_2}{m_1 + m_2}
        \]

        and the coefficient of restitution $e$ is

        \[
            \frac{v_{12f}}{v_{12i}}
        \]

        It follows then that the amount of kinetic energy converted to thermal energy is determined by the coefficient of restitution. \textit{The smaller the value of $e$, the larger the amount of energy dissipated
        from coherent energy to incoherent energy.} \\

        For any dissipative interaction,

        \[
            \Delta K + \Delta U + \Delta E_s + \Delta E_{\text{th}} = 0 \text{ (closed system, dissipative interaction)}
        \]

        In general, $\Delta E_s < 0$ because source energy is converted to another form of energy and $\Delta E_{\text{th}}>0$ because dissipation increases the amount of thermal energy.





    \subsection{Momentum and Force}

        The longer the interaction time of an interval, the smaller the force of impact. This is why hitting concrete feels worse than hitting a mattress; the momentum change occurs at a slower rate for the mattress. \\

        The \textbf{force} exerted on an object is the time rate of change in the object's momentum. It is a vector quantity. The vector sum of forces is called the \textbf{net force}. Forces obey the
        \textbf{principle of superposition}, which states that \textit{the vector sum of all forces exerted on an object equals the time rate of change in the momentum of the object.} \\

        Consider the situation of pushing a crate resting on a surface. When you push this crate along a surface, two opposing forces are exerted on the crate, one by your pushing, the other by the surface as friction.
        If these forces are equal in magnitude, then the crate moves at constant velocity. If the force exerted by the surface on the crate is smaller than the force you exert on it, the crate speeds up; if the force
        exerted by the surface is larger, the crate slows down.



    \subsection{The Reciprocity of Forces}

        \textbf{Whenever two objects interact, they exert on each other forces that are equal in magnitude but opposite in direction.} The pair of forces that two interacting objects exert on each other is called an
        \textbf{interaction pair}.


    \subsection{Identifying Forces}

        \textbf{Contact forces:} forces that arise when objects physically touch each other. This category of forces includes forces due to pushing, pulling, and rubbing. \\
        \textbf{Field forces:} forces associated with what is called "action at a distance". In this case, the objects exerting forces on each other do not need to be physically touching. For any object larger than atoms,
        gravitational and electromagnetic forces are the only field forces.


    \subsection{Translational Equilibrium}

        \textbf{Equilibrium:} describes when an object or system's motion or state is not changing, e.g. an object at rest or moving at constant velocity. \textbf{Translational equilibrium:} when an object's momentum
        is constant and its acceleration is zero.


    \pagebreak

    \subsection{Free-body Diagrams}

        If the vector sum of the forces exerted on an object is known, the time rate of change in the object's momentum is also known, and so the object's subsequent motion can be calculated.

        \begin{figure*}[hbt!]
            \centering
            \includegraphics[]{Resources/FBDs}
        \end{figure*}



    \subsection{Springs and Tension}

        A spring at \textbf{relaxed length} is neither stretched nor compressed. In general, whenever a spring holds an object in place, the spring must exert an upward force on the object, called a \textbf{support force},
        that is equal in magnitude to the downward gravitational force exerted by Earth on the object. As forces exerted by springs tend to return the spring to its relaxed length (i.e. when spring is stretched
        a pull is exerted back toward the spring, when it is compressed, a push away from the spring is exerted), forces exerted by springs are called \textbf{restoring forces}. The \textbf{elastic range} of a spring is
        the certain range over which the compression or stretching of the spring is reversible. Once stretched beyond a certain point known as the \textbf{elastic limit}, the spring no longer reverts to its original
        length when the load is removed. \\

        \textit{Soft and stiff springs exert exactly the same support force on the load, even though soft springs compress more.} This can be demonstrated by sitting on a bed versus sitting on the floor. The bed
        compresses a lot more than the floor when you sit on it even though it provides the same support force. An \textbf{elastic force} is the force exerted by a compressed or stretched material when the deformation
        is reversible. \\

        \textit{The force exerted on one end of a rope, spring, or thread is transmitted undiminished to the other end, provided the force of gravity ont he rope, spring, or thread is much smaller than the forces that
        cause the stretching.} \textbf{Tension} is the stress caused by the pair of outward forces that are exerted on each end of a rope being used to pull an object. The forces causing the tension are called
        \textbf{tensile forces}. Tension is a scalar and is represented by $T$.

    \subsection{Equation of Motion}

        The vector sum of the forces of an object is equal to the rate of change of the momentum with respect to time s.t.

        \[
            \sum \vec{F} = \frac{d\vec{p}}{dt} = m\frac{d\vec{v}}{dt}
        \]

        It follows then that

        \[
            \sum \vec{F} = m\vec{a}
        \]

        The \textbf{equation of motion} of an object is as follows.

        \[
            \vec{a} = \frac{\sum \vec{F}}{m}
        \]

        The unit of a force is the Newton (N).

        \[
            1 \text{N } = 1 \text{ kg } \cdot \text{ m/}\text{s}^2
        \]

        The property of adding forces vectorially is called \textbf{superposition of forces}.

        \begin{tbhtheorem}{Newton's Laws of Motion}
            \textbf{Newton's 1st Law of Motion (The Law of Inertia)}:
            \begin{quote}
                In an inertial reference frame, any isolated object that is at rest remains at rest, and any isolated object that is in motion keeps moving at a constant velocity.
            \end{quote}

            \textbf{Newton's 2nd Law of Motion (The Definition of Force)}
            \begin{quote}
                The vector sum of the forces exerted on an object is equal to the time rate of change in the momentum of that object.
            \end{quote}

            \textbf{Newton's 3rd Law of Motion (Law of Conservation of Momentum)}
            \begin{quote}
                Whenever two objects interact, they exert on each other forces that are equal in magnitude but opposite in direction.
            \end{quote}
        \end{tbhtheorem}

        When two objects interact, they exert forces on each other that are equal in magnitude and opposite in direction. These forces form an \textbf{interaction pair}.


    \subsection{Force of Gravity}

        Ignoring air resistance, all objects dropped from a certain height near Earth's surface have a downward acceleration given by $a_x = -g$ when upwards is defined as the positive $x$ direction. For an object in
        free fall, the only force exerted on the object is the gravitational force $\vec{F}^G_{Eo}$ exerted by Earth on the object. Thus, we have

        \[
            \sum F_x = F^G_{Eox} = -mg \text{ (near Earth's surface, $x$ axis vertically upward)}
        \]


    \subsection{Hooke's Law}

        The force exerted on a spring is always in the same direction as the displacement s.t.

        \[
            (F_{\text{by load on spring}})_x = k(x-x_0) \text{ (small displacement)},
        \]

        where $k$ is known as the \textbf{spring constant}, which represents the inverse of the slope of the linear curve through data points relating displacement of a spring and the force exerted on a spring. The
        spring constant is hence a measure of the stiffness of the spring and is always positive. The spring constant has the units $\text{N}/\text{m}$. The above equation is known as \textbf{Hooke's Law}. \\

        The \textbf{linear restoring force} is the force that tends to return a stretched or compressed spring or material to its relaxed position.


    \subsection{Impulse}

        The \textbf{impulse equation} is given below, where $\vec{J} = \Delta \vec{p}$:

        \[
            \vec{J} = \left(\sum\vec{F}\right) \Delta t \text{ (constant force)}
        \]

        The relationship between impulse and area under the $F_x (t)$ curve holds not only when $\sum\vec{F}$ is constant s.t.

        \[
            \vec{J} = \int^{t_f}_{t_i}\sum\vec{F}(t) dt \text{ (time-varying force)}
        \]

        Manipulating this equation, it follows that the momentum of an object in translational does not change, i.e.

        \[
            \Delta \vec{p} = \vec{0} \text{ (translational equilibrium)}
        \]


    \subsection{Systems of Two Interacting Objects}

        \textbf{Internal forces:} forces exerted on the system from inside the system \\
        \textbf{External forces:} forces exerted on the system from outside \\

        The external force of a system is given by

        \[
            \vec{F}_{\text{ext 1}} = m\vec{a}_{\text{cm}}
        \]

        and the equation of motion for the centre of mass of a system is given by

        \[
            \vec{a}_{\text{cm}} = \frac{\vec{F}_{\text{ext 1}}}{m}
        \]

        In other words, \textit{the centre of mass of a two-object system accelerates as though both objects were located at the centre of mass and the external force were exerted at that point.}



    \subsection{Systems of Many Interacting Objects}

        The equations for the external force and equation of motion for the centre of mass of a system composed of many interacting objects is pretty much identical to those for systems of two objects. \\

        External force:

        \[
            \sum \vec{F}_{\text{ext}} = m\vec{a}_{\text{cm}}
        \]

        Equation of motion for centre of mass of a nonisolated system:

        \[
            \vec{a}_{\text{cm}} = \frac{\sum\vec{F}_{\text{ext}}}{m}
        \]

        A \textbf{rigid object} is an ideal object that does not become deformed. Note that no object is truly rigid.


    \subsection{Chapter Glossary}

        \textbf{Contact force:} the force that an object exerts on a second object when the two are in physical contact with each other \\
        \textbf{Elastic force:} the force caused by the reversible compression or stretching of an object \\
        \textbf{Equation of motion}: the equation that relates the acceleration of an object to the vector sum of the forces exerted on it:

        \[
            \vec{a} = \frac{\sum\vec{F}}{m}
        \]

        For a system of more than one object, the equation of motion of the centre of mass is

        \[
            \vec{a}_{\text{cm}} = \frac{\sum\vec{F}_{\text{ext}}}{m}
        \]

        \textbf{Equilibrium:} any object or system whose motion or state is not changing in equilibrium \\
        \textbf{Field force:} the force that an object exerts on a second object without the two being in physical contact with each other. The three types of field forces we consider are gravitational, electric, and
        magnetic forces. \\
        \textbf{Force $\bm{\vec{F}}$ (N)}: When an object participates in an interaction, the force exerted on the object is given by the time rate of change caused in the object's momentum by that interaction.
        Forces obey the \textit{principle of superposition}, which states that the vector sum of the forces exerted on an object is equal to the vector sum of the rate of change caused in the object's momentum by the
        individual forces, and so

        \[
            \sum\vec{F} = \frac{d\vec{p}}{dt}
        \]

        \textbf{Free-body diagram:} a sketch showing a single object represented by its centre of mass and all forces exerted \textit{on} it. \\
        \textbf{Hooke's law}: the relationship between the force exerted on a spring (or other elastic object) and the displacement of the end of the spring from its relaxed value $x_0$ caused by that force:

        \[
            (F_{\text{by load on spring}})_x = k(x-x_0) \text{ (small displacement)}
        \]

        The constant $k$ is the \textit{spring constant}. \\
        \textbf{Impulse equation:} the equation that allows us to calculate the change in a system's momentum caused by forces exerted on the system during a time interval $\Delta t$. For a constant force,

        \[
            \vec{J} = left(\sum\vec{F}\right) \Delta t \text{ (constant force)}
        \]

        For a time-varying force,

        \[
            \vec{J} = \int^{t_f}_{t_i} \sum\vec{F}(t) dt \text{ (time-varying force)}
        \]

        \textbf{Interaction pair}: the forces that two interacting objects exert on each other. Conservation of momentum requires these two forces to be equal in magnitude and to point in opposite directions:

        \[
            \vec{F}_{12} = - \vec{F}_{21}
        \]

        \textbf{Newton (N)}: A derived SI unit of force, defined as 1 N $=$ 1 kg $\cdot$ m/$\text{s}^2$ \\
        \textbf{Newton's Laws of Motion:} Three fundamental principles which govern forces and their effects on the motion of objects \\
        \textbf{Rigid object:} an object whose deformation can be ignored when one or more forces exerted on it. For such an object the distance between two points on the object remains fixed. \\
        \textbf{Spring constant $\bm{k}$ (N/m):} A scalar defined as the ratio of the magnitude of the force exerted on a spring (or other elastic object) and the displacement of the end of the spring from its related
        $x_0$ caused by that force. \\
        \textbf{Tension $T$ (N)}: A scalar representing the stress caused in an object subject to a pair of forces, called \textit{tensile forces}, exerted so they stretch the object. In an object subject to two equal
        tensile forces of magnitude $F$, one on each end of the object, the value of the tension is $T=F$. \\
        \textbf{Translational equilibrium:} Any object whose momentum is not changing


