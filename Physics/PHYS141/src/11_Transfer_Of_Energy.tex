\section{Transfer of Energy}

    \subsection{The Effects of Interactions}

        \textbf{Interactions:} mutual influences between two objects that produce either physical change or a change in motion. An interaction that causes objects to accelerate can be either repulsive or attractive.
        A \textbf{repulsive interaction} is one in which the interacting bodies accelerate away from each other; an \textbf{attractive interaction} is one in which the interacting bodies accelerate toward each other. \\

        Whenever two objects interact, the ratio of the $x$ components of their accelerations is equal to the negative inverse of the ratio of their inertias. \\

        Below are some figures describing the conservation of momentum and kinetic energy in an elastic collision between two carts on a low-friction track. The ienrtia of cart 1 is 0.12 kg, and that of cart 2 is 0.24 kg.

        \begin{figure*}[hbt!]
            \centering
            \includegraphics[scale=1.1]{Resources/Interactions}
        \end{figure*}

        Even in an elastic collision between two objects, there is an instant at which the carts have the same velocity, which means that at that instant their relative velocity is zero. For a tennis ball hitting a wall,
        at the instant the ball has zero velocity, all its kinetic energy has been used to deform the ball (and to a lesser extent, the wall). Thus, this energy is temporarily stored as \textit{elastic potential energy}.
        \\

        \textbf{Summary of the Characteristics of an Interaction:} \\
        1. \textit{Two objects} are needed \\
        2. The \textit{momentum} of a system of interacting objects is the same before, during, and after the interaction (provided the system is isolated) \\
        For Interactions that affect the motion of objects: \\

        1. The ratio of the $x$ components of the acceleration of the interacting objects is the negative inverse ratio of their inertias. Because the velocities of both objects change in an interaction, the individual
        momenta and kinetic energies change. \\
        2. The system's \textit{kinetic energy changes during the interactions}. Part of it is converted to (or from) some internal energy. In an elastic collision, all of the converted energy reappears as kinetic energy
        \textit{after} the collision. In an inelastic collision, some of the converted kinetic energy reappears as kinetic energy; in an totally inelastic collision, none of the converted kinetic energy reappears.


    \pagebreak
    \subsection{Potential Energy}

        \textbf{Potential Energy}: a form of internal energy, represented by $U$. Potential energy is stored in reversible changes in the \textbf{configuration state} of the system in the context of the spatial
        arrangement of the system's interacting components. \\
        $\bullet$ can be converted entirely to kinetic energy \\

        \textbf{Gravitational Potential Energy:} the potential energy an object has because of its position in a gravitational field, most commonly Earth's
        gravitational atmosphere.


    \subsection{Energy Dissipation}

        The part of the converted kinetic energy that does not reappear after an inelastic collision is said to be \textbf{dissipated}, i.e. irreversibly converted. Deformation can take place in a \textbf{coherent}
        manner, meaning that at the atomic level, there is a pattern in the displacements of the atoms: they move orderly in rows, with each successive row experiencing a small displacement in the same direction as
        adjacent rows. \textbf{Incoherent} manner refers to how atoms are displaced in random directions. \\

        \textbf{Mechanical (coherent) energy}: the sum of a system's kinetic energy and potential energy of a system. An important part of a system's incoherent energy is its \textbf{thermal energy}.
        \textbf{Internal Energy} is the sum of the system's incoherent energy and its potential energy.


    \subsection{Source Energy}

        \textbf{Source energy}: the energy obtained from fossil and mineral fuels, nuclear fuel, biomass fuel, water reservoirs, solar radiation, and wind \\

        The four types of source energy:
        1. \textbf{chemical energy}: energy associated with the configuration of atoms inside molecules released in such chemical reactions as the burning of oil, coal, gas, and wood and the metabolizing of food \\
        2. \textbf{nuclear energy}: (energy associated with the configuration of the nuclei of atoms) released in nuclear reactions \\
        3. \textbf{solar energy:} delivered by radiation from the Sun \\
        4. \textbf{stored solar energy:} wind and hydroelectric energy

        All energy can be divided into four categories: kinetic energy $K$, potential energy $U$, source energy $E_s$, and thermal energy $E_{th}$. \\

        \textbf{Nondissipative} interactions are reversible, whereas \textbf{dissipative} interactions are irreversible.


