\section{Conservation of Energy}

    \subsection{Closed Systems}

        \textbf{Closed system}: any system to or from which no energy is transferred \\
        $\bullet$ a closed system need not be isolated \\

        Below are some energy bar diagrams for initial and final conditions:

        \begin{figure*}[hbt!]
            \centering
            \includegraphics[]{Resources/Energy_Bar_Closed_System}
        \end{figure*}



    \subsection{Elastic Collisions}

        For elastic collisions, the objects \textit{relative speed},

        \[
            v_{12} = |\overrightarrow{v}_2 - \overrightarrow{v}_1|
        \]

        is the same before and after the collision. For two objects moving along the $x$ axis, we can write this as

        \[
            v_{2x,i} = v_{1x,i} = -(v_{2x,f} - v{1x,f})
        \]

        Recall that \texxtbf{kinetic energy} is given by

        \[
            K = \frac{1}{2} mv^2
        \]

        Thus, for elastic collisions,

        \[
           K_i = K_F \equiv \Delta K = 0
        \]

        Kinetic energy is represented in \textbf{joules}, where

        \[
            1 \text{ kg } \cdot \text{ m}^2\text{/s}^2 = 1 \text{ J}
        \]



    \subsection{Inelastic Collisions}

        The majority of collisions are between the two extremes of elastic and totally inelastic. For these cases, we can define the ratio of relative speeds as the \textbf{coefficient of restitution ($e$)}:

        \[
            e = \frac{v_{12f}}{v_{12i}}
        \]

        Because $e$ is a ratio of speeds which are always positive, $e$ is also always positive. We generally write $e$ with a negative value because the relative velocity changes sign after the collision:

        \[
            e = - \frac{v_{2x,f} - v_{1x,f}}{v_{2x,i} - v_{1x,i}} = - \frac{v_{12,f}}{v_{12x,i}}
        \]

        \begin{figure*}[hbt!]
            \centering
            \includegraphics[scale=0.75]{Resources/Coeff_Of_Restitution}
        \end{figure*}



    \subsection{Conservation of Energy}

        The combined kinetic and internal energies of a system is given by

        \[
            E = K + E_{int}
        \]

        For a closed system, we have

        \[
            \Delta E_{int} = -\Delta K
        \]


    \subsection{Explosive Separations}

        \textbf{Explosive separation}: where objects separate or break apart from each other \\
        $\bullet$ kinetic energy increases and internal energy decreases in these cases

    \subsection{Topic Summary}

        \textbf{Closed system:} a system to or from which no energy is transferred \\
        \textbf{Coefficient of restitution ($\bm{e}$)} (unitless): A scalar equal to the ratio of relative speeds after and before a collision of two objects:

        \[
            e = \frac{v_{12f}}{v_{12i}}
        \]

        \textbf{Conservation of energy:} energy can be transferred from one object to another or converted from one form to another, but it cannot be created or destroyed. The energy of a closed system cannot change:

        \[
            \Delta E = 0 \text{ (closed system)}
        \]

        \textbf{Elastic, inelastic, totally inelasic collisions:} collisions between two objects are classified according to what happens to the relative speed

        \[
            v_{12} = |\overrightarrow{v}_2 - \overrightarrow{v}_1|
        \]

        of the two objects. \\

        \textbf{Energy, E(J)}: A scalar that provides a quantitative measure of the state or motion of an object or system. Energy appears in many different forms. The energy of an object or system always refers to the
        sum of all forms of energy in that object or system. \\
        \textbf{Explosive separation:} a process in which objects break apart from one another and the relative speed of the objects increases. \\
        \textbf{Internal Energy, $\bm{E}_{int}$ (J)}: any energy not associated with the motion of an object or system. Internal energy is a quantitative measure of the state of the object or system. \\
        \textbf{Irreversible process:} a process involving changes that cannot undo themselves spontaneously. \\
        \textbf{Joule (J)}: The derived SI unit of energy, defined as

        \[
            1 J = 1 \text{ kg } \cdot \text{ m}^2\text{/s}^2
        \]

        \textbf{Process}: the transformation of a system from an intial state to a final state \\
        \textbf{Relative Velocity, $\bm{\overrightarrow{v}_{12}}$ (m/s)}: The velocity of one object relative to another:

        \[
            \overrightarrow{v}_{12} = \overrightarrow{v}_2 - \overrightarrow{v}_1
        \]

        The magnitude of this velocity is called the \textbf{relative speed}:

        \[
            v_{12} = |\overrightarrow{v}_2 - \overrightarrow{v}_1}
        \]

        \textbf{Reversible process}: a process that can run backward so that the initial state is restored \\
        \textbf{State}: The condition of an object (or a system) as specified by a complete set of variables